
\ourpar{Signatures of Knowledge from SE-SNARKs.}  \markulf{22.04}{wrong place} The work of Groth and Maller
\cite{C:GroMal17} shows how to construct a signature of knowledge for messages in
$\{0, 1\}^*$ from target collision-resistant hash-function and simulation-extractable
SNARKs. This construction can be found in \Cref{sec:SoKconstruction}.
These schemes inherit the same properties as the underlying SNARK schemes and
may rely on a structured reference string, which can be updatable or universal, and they are
succinct.  In the following, our goal will be to show that recent efficient SNARKs are
simulation extractable, and therefore, they are a perfect candidate to build better
signatures of knowledge.

\section{SNARKY Signature Definition}


%\subsection{Signatures of Knowledge}
\markulf{22.04}{Moved here from preliminaries. Intro and maybe a final short section on SoK, definition could go to appendix..}

Signatures of Knowledge (SoKs) \cite{C:ChaLys06} are closely related to
simulation-extractable SNARKs: A signer can generate a valid signature for a message
only if she has a valid witness for the statement.
%
The notion of SoKs mimics digital signatures with strong existential unforgeability:
even if the adversary has seen many signatures on arbitrary messages under arbitrary
statement, it cannot create a new signature (not seen before) without knowing the
witness for the statement in question.
%
To define Signatures of Knowledge we use a recent version \cite{C:GroMal17} that
implicitly considers only compact (succinct) signatures. \COMMENT{We omit the
  description of SoK's algorithms and security properties and defer them to
  \cref{sec:sok}.} Intuitively, a SoK behaves like a non-interactive proof system
that ``binds'' the message being signed to the proof.
     
%\subsection{Signatures of Knowledge}
\label{sec:sok}
A $\SoK$ for an efficiently decidable binary relation $\REL$ is defined as a tuple of PPT algorithms $\SoK = (\signsetup,  \sign, \allowbreak \verify,  \simsetup, \simsign)$:

\begin{description}
    \item[$\signsetup(1^\secpar, \REL) \rightarrow  \param$:]
	The setup algorithm takes a security parameter $\secpar$ and a binary relation $\REL$
	and returns public parameters $ \param$.  The input $ \param$ is implicit to all subsequent algorithms. 

    \item[$\sign(\mesage, \inp, \wit)  \rightarrow \signature$:]
The signing algorithm takes as input a message $ \mesage \in \{0,1\}^{*}$, 
a statement $\inp$, and a witness $\wit$.
	Outputs a signature $\signature$.

    \item[$\verify(\mesage, \inp, \signature) \rightarrow 1/ 0$:]
The verification algorithm takes as input 
 a message $\mesage$,  a statement $\inp$ 
 together with a signature $\signature$,
	outputs $1$ if the the signature is valid, $0$ otherwise.
	
    \item[$\simsetup(\REL) \rightarrow (\param, \td)$:]
    	A simulated setup algorithm which takes as input a relation $\REL$ and returns public parameters $\param$ and a trapdoor $\td$. 
    	
   \item[$\simsign(\td, \mesage, \inp) \rightarrow \signature'$:]
   	A simulated signing  that takes as input  a trapdoor $\td$, a message $\mesage$ and a statement $\inp$ and returns a simulated signature $\signature'$.
\end{description}
 
A SoK scheme should satisfy correctness, extractability and simulatability:
   
\begin{description}
\item[Perfect Correctness.] This guarantees that a signer with a valid witness can
  always produce a signature that will convince the verifier. For all
  $\secpar\in \NN$, for all efficiently decidable binary relation $\REL$, for all
  $(\inp, \wit) \in \REL$, and for all $ \mesage \in \{0,1\}^{*}$:
   \[
  \condprob{
	  \begin{matrix}
~\verify(\mesage, \inp, \signature) = 1   
	 \end{matrix}
}{
	  \begin{matrix}
~(\param, \td) \gets \signsetup(1^\secpar, \REL)\\
~ \signature \gets  \sign(\mesage, \inp, \wit)
 \end{matrix} }  =1. 
\]

%

\markulf{22.04}{Because of the trapdoor these definitions are not very suitable for us.}
\item[Simulation Extractability:] This guarantees that an adversary is not able to issue a new signature
unless it knows a witness. This should hold even if the adversary gets to see signatures on
arbitrary messages under arbitrary statements. We model this notion in a strong sense, by
letting the adversary see simulated signatures for arbitrary messages and statements, which
potentially include false statements. Even under this strong attack model, we require that
whenever the adversary outputs a valid signature not queried before, it is possible to extract a
witness for the signature. More formally,  for any PPT adversary $\adv$ there exists a PPT extractor $\ext_\adv$ such that:
   \[
  \condprob{
	  \begin{matrix}
~ \verify(\mesage, \inp, \signature) = 1   \\
\land (\inp, \wit) \notin \REL \\
\land (\mesage, \inp, \signature) \notin Q_{\simsign_\td}
	 \end{matrix}
}{
	  \begin{matrix}
~(\param, \td) \gets \simsetup(1^\secpar, \REL),\\
~ (\mesage, \inp, \signature) \gets \adv^{\simsign_\td} (\param; r),\\
\wit \gets \ext_\adv(\param, Q_{\simsign_\td}, r)
 \end{matrix}
} \leq \negl
\]
where the adversary has access to simulated signatures via the oracle
$\simsign_\td(\inp_i, \mesage_i) \gets \simsign(\td, \inp_i, \mesage_i)$ and the
extractor $\ext_\adv$ takes as input public parameters $\param$, list
$Q_{\simsign_\td}$ of all queries made by $\adv$, and $\adv$'s randomness $r$.

%\vspace{4pt}
%
\item[Perfect Simulatability:] The verifier should not learn anything about the
  witness from the signature.  The secrecy of the witness is modelled by the ability
  to simulate signatures without the witness. More precisely, we say the signatures
  of knowledge are simulatable if there is a simulator that can create public
  parameters together with an associated trapdoor that enables producing signatures
  without a witness that are indistinguishable from real ones.  More formally, for
  any number of queried $\mesage_i\in \{0,1\}^{*}$ and $(\inp_i, \wit_i) \in \REL$:
	\begin{equation}
	\begin{split}
\left|	\condprob{  
  \begin{matrix}
		\adv(\param) = 1 
  \end{matrix}
		}{
		\begin{matrix}
      \param \gets  \signsetup(1^\secpar, \REL),\\
      ~\signature_i \gets \sign(\mesage_i, \inp_i, \wit_i)
		\end{matrix}
		}\right. \\
	 - \left.\condprob{
		\begin{matrix}
		\adv (\param) = 1 
		\end{matrix}
}{
		\begin{matrix}
      ~(\param,\td) \gets  \simsetup(1^\secpar, \REL), \\
      ~\signature_i \gets \simsign(\td,\mesage_i, \inp_i)
		\end{matrix}} \right| \leq \negl.
\end{split}
\end{equation}
\end{description}


\section{SNARKY Signatures Construction}\label{sec:SoKconstruction}

We will recall the construction of 
a signature of knowledge scheme for messages in $\{0, 1\}^*$
from an SE-SNARK as shown in \cite{C:GroMal17}. 
 
Consider a target collision-resistant hash function also known as universal one-way hash function $H:\{0, 1\}^{\ell_k} \times \{0, 1\}^* \to \{0, 1\}^{\ell_h}$ with $\ell_K, \ell_h$ polynomials in the security parameter $\secpar$. 

The target collision-resistance of $H$ requires that for any $\ppt$ adversary $\adv$ : 
 
	\[  \condprob{
	m_0 \gets \adv(1^\secpar); K \gets \{0, 1\}^{\ell_K}; m_1 \gets \adv(K)
}{
	m_0 \neq m_1 ~\land  ~H_K(m_0)=H_K(m_1)
} \leq \negl.
	\]
	
For any given relation $\REL'$, we consider the  following relation $\REL$:
 	\[\REL= \{((K,h,\inp),\wit) : K\in  \{0, 1\}^{\ell_K} ~\land ~h \in  \{0, 1\}^{\ell_h} ~\land ~(\inp, \wit) \in \REL'\}.
 	\]
 	
Let $\RELGEN(\secparam)$ be the relation generator that runs $\REL' \gets \RELGEN'(\secparam)$ and returns $\REL$ defined as above. 
Let $H$ be a target collision-resistant hash function and let $\ps = (\kgen, \prover, \verifier, \simulator)$ be a SE-NIZK argument for $\RELGEN$. 
We build a signature of knowledge  $\SoK$ for $\RELGEN'$ that works as follows: 

\begin{description}
    \item[$\signsetup(1^\secpar, \REL') \rightarrow  \param$:]
 Run $(\srs,\rho) \leftarrow \kgen(\REL)$. 
 Return $ \param = \srs$.

    \item[$\sign(\mesage, \inp, \wit)  \rightarrow \signature$:]
Sample $K \gets \{0, 1\}^{\ell_K}$.  Compute
$\zkproof \gets  \prover(\srs, (K, H_K(\mesage), \inp), \wit) $.
Return $\signature = (K, \zkproof)$.

    \item[$\verify(\mesage, \inp, \signature) \rightarrow b$:]
Parse $\signature = (K, \zkproof)$.
Return $b \gets \verifier(\srs, (K, H_K(\mesage), \inp), \zkproof)$.
	
    \item[$\simsetup(\REL') \rightarrow (\param, \td)$:]
  Run   $(\srs,\rho) \leftarrow \kgen(\REL)$. 
 Return $( \param = \srs,  \td = \rho)$.

    	
   \item[$\simsign(\td, \mesage, \inp) \rightarrow \signature'$:]
   Sample $K \gets \{0, 1\}^{\ell_K}$.  Compute
$\zkproof \gets \simulator(\srs, \rho, (K, H_K(\mesage), \inp)) $.
Return $\signature' = (K, \zkproof)$.

\end{description} 


\begin{theorem}[Signature of Knowledge] 
  \label{thm:SoK}
If $H:\{0, 1\}^{\ell_k} \times \{0, 1\}^* \to \{0, 1\}^{\ell_h}$ is a target collision-resistant hash function 
and $\ps = (\kgen, \prover, \verifier, \simulator)$ be is a SE-NIZK for $\RELGEN(\secparam)$, then the scheme
 $\SoK = (\signsetup,  \sign, \allowbreak \verify,  \simsetup, \simsign)$ above is a signature of knowledge for $\RELGEN'(\secparam)$ for  the message space $\{0, 1\}^*$.
\end{theorem}
 The proof can be found in 
 \cite{C:GroMal17}.
	

%%% Local Variables:
%%% mode: latex
%%% TeX-master: "main"
%%% End:
