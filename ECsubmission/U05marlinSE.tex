% !TEX root = main.tex
% !TEX spellcheck = en-US

\section{Simulation extractability of $\marlinprotfs$}
We show that $\marlin$ is \COMMENT{forking }simulation-extractable. To that end, we show
that $\marlin$ has all the required properties: has unique response property, is
forking special sound, and its simulator can provide indistinguishable proofs
without a trapdoor, just by programming the random oracle.

\subsection{$\marlin$ protocol rolled-out}
$\marlin$ uses R1CS as arithmetization method. That is, the prover given
instance $\inp$ and witness $\wit$ and $|\HHH| \times |\HHH|$ matrices $\vec{A},
\vec{B}, \vec{C}$ shows that $\vec{A} (\inp^\top, \wit^\top)^\top \circ \vec{B}
(\inp^\top, \wit^\top)^\top = \vec{C} (\inp^\top, \wit^\top)^\top$. (Here
$\circ$ is a entry-wise product.)

We assume that the matrices have at most $|\KKK|$ non-zero entries. Obviously,
$|\KKK| \leq |\HHH|^2$. Let $b = 3$, the upper-bound of polynomial evaluations
the prover has to provide for each of the sent polynomials.  Denote by $\dconst$
an upper-bound for $\smallset{|\HHH| + 2b -1, 2 |\HHH| + b - 1, 6 |\KKK| - 6}$.

The idea of showing that the constraint system is fulfilled is as
follows. Denote by $\vec{z} = (\inp, \wit)$. The prover computes polynomials
$\p{z_A} (X), \p{z_B} (X), \p{z_C} (X)$ which encode vectors
$\vec{A} \vec{z}, \vec{B} \vec{z}, \vec{C} \vec{z}$ and have degree $<
|\HHH|$. Importantly, when constrains are fulfilled,
$ \p{z_A} (X) \p{z_B} (X) - \p{z_C} (X) = \p{h_0} (X) \ZERO_\HHH (X)$, for some
$\p{h_0} (X)$ and vanishing polynomial $\ZERO_\HHH (X)$. The prover sends
commitments to these polynomials and shows that they have been computed
correctly. More precisely, it shows that
\begin{equation}
  \label{eq:marlin_eq_2}
\forall \vec{M} \in \smallset{\vec{A}, \vec{B}, \vec{C}},  \forall \kappa \in \HHH,
\p{z_M} (\kappa) = \sum_{\iota \in \HHH} \vec{M}[\kappa, \iota] \p{z}(\iota).
\end{equation}

The ideal verifier checks the following equalities
\begin{equation}
  \label{eq:marlin_ver_eq}
  \begin{aligned}
    \p{h}_3 (\beta_3) \ZERO_\KKK (\beta_3) & = \p{a} (\beta_3) - \p{b} (\beta_3)
    (\beta_3 \p{g_3} (\beta_3) + \sigma_3 / |\KKK|)\\
    \p{r}(\alpha, \beta_2) \sigma_3 & = \p{h_2} (\beta_2) \ZERO_\HHH (\beta_2) +
    \beta_2 \p{g2} (\beta_2) + \sigma_2/|\HHH|\\
    \p{s}(\beta_1) + \p{r}(\alpha, \beta_1) (\sum_M \eta_M \p{z_M} (\beta_1)) -
    \sigma_2 \p{z} (\beta_1) & = \p{h_1} (\beta_1) \ZERO_\HHH (\beta_1) +
    \beta_1
    \p{g_1} (\beta_1) + \sigma_1/|\HHH| \\
    \p{z_A} (\beta_1) \p{z_B} (\beta_1) - \p{z_C} (\beta_1) & = \p{h_0}
    (\beta_1) \ZERO_\HHH (\beta_1)
  \end{aligned}
\end{equation}
where $\p{g_i} (X), \p{h_i} (X)$, $i \in \range{1}{3}$,
$\p{a} (X), \p{b} (X), \sigma_1, \sigma_2, \sigma_3$ are polynomials and
variables required by the sumcheck protocol which allows verifier to efficiently
verify that \cref{eq:marlin_eq_2} holds.
                         

\subsection{Unique response property}
\begin{lemma}\label{lem:marlinprot_ur}
  Let $\PCOM$ be a commitment of knowledge with security $\epsk(\secpar)$,
  $\epsbind(\secpar)$ and has unique response property with security
  $\epsop(\secpar)$. Then probability that a $\ppt$ adversary $\adv$ breaks
  $\marlinprotfs$ $\ur{1}$ property is at most
  $6 \cdot (\epsbind + \epsop + \epsk)$ \michals{8.9}{Do we need to add
    probability that the idealized verifier fails $\epsid$?}
\end{lemma}
\begin{proof}
  As in previous proofs, we show the property by game hops. Let
  $N = \p{g_1}, \p{h_1}, \p{g_2}, \p{h_2}, \p{g_3}, \p{h_3}$. That is, $M$ is a
  set of all polynomials which commitments are send during the protocol after
  Round 1.

  \ncase{Game 0} In this game the adversary wins if it breaks evaluation
  binding, unique opening property, or knowledge soundness of one of commitments
  for polynomials in $N$.

  Probability that a $\ppt$ adversary wins in Game 0, is upper bounded by $6
  \cdot (\epsbind + \epsop + \epsk)$.

  \ncase{Game 1} In this game the adversary additionally wins if it breaks the
  $\ur{2}$ property of the protocol

  \ncase{Game 0 to Game 1} Probability that the adversary wins in
  Game 1 but not in Game 0 is $0$. This is since the polynomials in $N$ are
  uniquely determined. W.l.o.g.~we analyse probability that adversary is able to
  produce two (different) pairs of polynomials $(\p{h_2}, \p{g_2})$ and $(\p{h'_2},
  \p{g'_2})$ such that
  \begin{align*}
    \p{h_2} (X) \ZERO_{\HHH} (X) + X \p{g_2} (X) & = \p{h_2} (X) \ZERO_{\HHH} (X) +
                                                   X \p{g_2} (X)\\
    (\p{h_2} (X) - \p{h'_2} (X)) \ZERO_{\HHH} (X) & = X (\p{g'_2} (X) - \p{g'_2}
    (X)).
  \end{align*}
  Since $\p{h_2}, \p{g_2} \in \FF^{< |\HHH| - 1} [X]$ and
  $\ZERO \in \FF^{|\HHH|} [X]$, LHS has different degree than RHS unless both
  sides have degree $0$. This happens when $\p{h_2} (X) = \p{h'_2} (X)$ and
  $\p{g_2} (X) - \p{g'_2} (X)$.
\end{proof}

\subsection{Forking special soundness}
\begin{lemma}\label{lem:marlinprot_ss}
	Assume that an idealised $\marlinprot$ verifier fails with probability at most
	$\epsid(\secpar)$ and probability that a $\ppt$ adversary breaks dlog is
	bounded by $\epsdlog(\secpar)$. Then $\marlinprot$ is
	$(\epsid (\secpar) + \epsdlog (\secpar), 2, d + 1)$-forking special
	sound.
\end{lemma}
\begin{proof}
	% \michals{8.9}{Need to check the degrees}
	The proof goes similarly to the respective proofs for $\plonk$ and
	$\sonic$. That is, let $\srs$ be $\marlinprot$'s finalised SRS and denote by $\srs_1$
	all SRS's $\GRP_1$-elements. Let $\tdv$ be an algebraic adversary that
	produces a statement $\inp$ and a $(1, \dconst + 1, 1, 1)$-tree of
	acceptable transcripts $\tree$. Note that in all transcripts the instance
	$\inp$, proof elements
	$\sigma_1, \gone{\p{w}(\chi), \p{z_A}(\chi), \p{z_B}(\chi), \p{z_C}(\chi),
		\p{h_0}(\chi), \p{s}(\chi)}, \gone{\p{g_1}(\chi), \p{h_1}(\chi)}$
	and challenges $\alpha, \eta_1, \eta_2, \eta_3$ are common as the transcripts
	share the first $3$ messages. The tree branches after the third message of the
	protocol where the challenge $\beta_1$ is presented, thus tree $\tree$ is
	build using different values of $\beta_1$.
	
	We consider the following games.
	
	\ncase{Game 0} In this game the adversary wins if all the transcripts it
	produced are acceptable by the ideal verifier,
	i.e.~$\vereq_{\inp, \zkproof}(X) = 0$, cf.~\cref{eq:marlin_ver_eq}, yet the extractor
	fails to extract a valid witness out of them.
	
	Probability of $\tdv$ winning this game is $\epsid(\secpar)$ as the protocol
	$\marlinprot$, instantiated with the idealised verification equation, is
	perfectly sound except with negligible probability of the idealised verifier
	failure $\epsid(\secpar)$. Hence for a valid proof $\zkproof$ for a statement
	$\inp$ there exists a witness $\wit$, such that $\REL(\inp, \wit)$ holds. Note
	that since the $\tdv$ produces $(\dconst + 1)$ acceptable transcripts for
	different challenges $\beta_1$, it obtains the same number of different
	evaluations of polynomials $\p{z_A}, \p{z_B}, \p{z_C}$.
	
	Since the transcripts are acceptable by an idealised verifier, the equality
	$\p{z_A} (X) \p{z_B} (X) - \p{z_C} (X) = \p{h_0} (X) \ZERO_\HHH (X)$ holds and
	each of $\p{z}_M$, $M \in \smallset{A, B, C}$, has been computed
	correctly. Hence, $\p{z_A}, \p{z_B}, \p{z_C}$ encodes the valid witness for
	the proven statement. Since $\p{z_A}, \p{z_B}, \p{z_C}$ are of degree at most
	$\dconst$ and there is more than $(\dconst + 1)$ their evaluations
	known, $\extt$ can recreate their coefficients by interpolation and reveal the
	witness with probability $1$. Hence, the probability that extraction fails in
	that case is upper-bounded by probability of an idealised verifier failing
	$\epsid(\secpar)$, which is negligible.
	
	\ncase{Game 1} In this game the adversary additionally wins if it produces a
	transcript in $\tree$ such that $\vereq_{\inp, \zkproof}(\chi) = 0$, but
	$\vereq_{\inp, \zkproof}(X) \neq 0$. That is, the ideal verifier does not
	accept the proof, but the real verifier does.
	
	\ncase{Game 0 to Game 1} Assume the adversary wins in Game 1, but
	does not win in Game 0. We show that such adversary may be used to break the
	$\dlog$ assumption. More precisely, let $\tdv$ be an adversary that for
	relation $\REL$ and randomly picked $\srs \sample \kgen(\REL)$ produces a tree
	of acceptable transcripts such that the winning condition of the game
	holds. Let $\rdvdlog$ be a reduction that gets as input an
	$(\dconst, 1)$-dlog instance $\gone{1, \ldots, \chi^\dconst}, \gtwo{1, \chi}$ and
	is tasked to output $\chi$. The reduction proceeds as follows---it builds $\adv$'s SRS $\srs$ in the updatable setting using the input $\dlog$ instance. Namely it answers $\adv$'s queries for SRS updates and sets the honest update of the SRS to be $\srs$. Let $\srs'$ be the finalised SRS and $(1, \tree)$ be the output
	returned by $\adv$. Let $\inp$ be a relation proven in $\tree$.  Consider a
	transcript $\zkproof \in \tree$ such that $\vereq_{\inp, \zkproof}(X) \neq 0$,
	but $\vereq_{\inp, \zkproof}(\chi) = 0$. Since the adversary is algebraic, all
	group elements included in $\tree$ are extended by their representation as a
	combination of the input $\GRP_1$-elements. Hence all coefficients of the
	verification equation polynomial $\vereq_{\inp, \zkproof}(X)$ are known and
	$\rdvdlog$ can find its zero points. Since
	$\vereq_{\inp, \zkproof}(\chi) = 0$, the targeted discrete log value $\chi'$ is
	among them.  Let $\chi_1, \ldots, \chi_\ell$ be the partial trapdoors of $\adv$'s SRS updates,  extracted by the reduction from the update proofs given by $\adv$. Now $\rdvdlog$ returns $\chi = \chi' (\chi_1 \chi_2 \ldots \chi_\ell)^{-1}$ and breaks the $\dlog$ assumption. Hence, the probability that this event happens is upper-bounded
	by $\epsdlog(\secpar)$.
	
\end{proof}

\subsection{Trapdoor-less simulatability of Marlin}
\begin{lemma}
  \label{lem:marlin_hvzk}
  $\marlinprot$ is 2-programmable trapdoor-less simulatable.
\end{lemma}
\begin{proof}
The simulator follows the protocol except it picks the challenges $\alpha,
\eta_A, \eta_B, \eta_C, \beta_1, \beta_2, \beta_3$ before it picks polynomials
it sends.

First, it picks $\p{\tilde{z}}_A (X)$, $\p{\tilde{z}}_B (X)$ at random and
$\p{\tilde{z}}_C (X)$ such that
$\p{\tilde{z}}_A (\beta_1) \p{\tilde{z}}_B (\beta_1) = \p{\tilde{z}}_C
(\beta_1)$.  Given the challenges and polynomials $\p{\tilde{z}}_A (X)$,
$\p{\tilde{z}}_B (X)$, $\p{\tilde{z}}_C (X)$ the simulator computes
$\sigma_1 \gets \sum_{\kappa \in \HHH} \p{s}(\kappa) + \p{r}(\alpha, X) (\sum_{M
  \in \smallset{A, B, C}}\eta_M \p{\tilde{z}}_M(X)) - \sum_{M \in \smallset{A,
    B, C}} \eta_M \p{r}_M (\alpha, X) \p{\tilde{z}} (X)$.

Then the simulator starts the protocol and follows it, except it programs the
random oracle that on partial transcripts it returns the challenges picked by
$\simulator$.
\end{proof}

\subsection{From forking special soundness and unique response property to\COMMENT{ forking}
  simulation extractability of $\marlinprotfs$}
\begin{corollary}
  Assume that $\marlinprot$ is $\ur{1}$ with security
  $\epsur(\secpar) = 6 \cdot (\epsbind + \epsop + \epsk)$, and forking special sound
  with security $\epsfor (\secpar)$. Let $\ro\colon \bin^* \to \bin^\secpar$ be a
  random oracle. Let $\advse$ be an algebraic adversary that can make up to $q$
  random oracle queries, up to $S$ simulation oracle queries, and outputs an
  acceptable proof for $\marlinprotfs$ with probability at least
  $\accProb$. Then $\marlinprotfs$ is \COMMENT{forking }simulation-extractable with
  extraction error $\eta = \epsur(\secpar)$. The extraction probability
  $\extProb$ is at least
\[
  \extProb \geq \frac{1}{q^{\dconst}} (\accProb - \epsur(\secpar))^{\dconst + 1}
  - \eps(\secpar).
	\]
	for some negligible $\eps(\secpar)$, $\dconst$ being, the upper bound of
  constrains of the system.
  \[
    \extProb \geq q^{-(\dconst - 1)} (\accProb - 6 \cdot (\epsbind + \epsop +
    \epsk))^{\dconst} -\eps(\secpar)\,.
\]
\end{corollary}

%\section{Further work}
%We identify a number of problems which we left as further work. First of all,
%the generalised version of the forking lemma presented in this paper can be
%generalised even further to include protocols where forking soundness holds for
%protocols where $\extt$ extracts a witness from a $(n_1, \ldots, n_\mu)$-tree of
%acceptable transcripts, where more than one $n_j > 1$. I.e.~to include
%protocols that for witness extraction require transcripts that branch at more
%than one point.
%
%Although we picked $\plonk$ and $\sonic$ as examples for our framework, it is
%not limited to SRS-based NIZKs. Thus, it would be interesting to apply it to
%known so-called transparent zkSNARKs like Bulletproofs \cite{SP:BBBPWM18},
%Aurora \cite{EC:BCRSVW19} or AuroraLight \cite{EPRINT:Gabizon19a}.
%
%Since the rewinding technique and the forking lemma used to show simulation
%extractability of $\plonkprotfs$ and $\sonicprotfs$ come with security loss,
%it would be interesting to show SE of these protocols directly in the
%algebraic group model.
%
%Although we focused here only on zkSNARKs, it is worth to
%investigating other protocols that may benefit from our framework, like
%e.g.~identification schemes.
%
%Last, but not least, this paper would benefit greatly if a more tight version
%of the generalised forking lemma was provided. However, we have to note here
%that some of the inequalities used in the proof are already tight, i.e.~for
%specific adversaries, some of the inequalities are already equalities.

%%% Local Variables:
%%% mode: latex
%%% TeX-master: "main"
%%% End:
