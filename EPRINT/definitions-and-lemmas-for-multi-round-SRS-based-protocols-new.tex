% !TEX root = main.tex
% !TEX spellcheck = en-US


\begin{figure}
	\centering
		\begin{pcvstack}[center,boxed]
		\begin{pchstack}
			\procedure{$\simOH (x)$}
			{
			\pcif H[x] = \bot \pcthen \\
			\pcind H[x] \sample \mathsf{Im}(\ro) \\
		%	Q_\ro \gets \{(x,H[x])\}\\
			\pcreturn H[x]
			  }
			%
			\pchspace
			%
			\procedure{$\simulator\oracleo.\prog(x, h)$}
			{ 
				\pcif H[x] = \bot \pcthen \\ 
				\pcind H[x] \gets h \\
				\pcind \Qprog \gets \Qprog \cup \{x\}\\
		%		Q_\ro \gets \{(x,H[x])\}\\
				\pcreturn H[x]
			}
			\pchspace
			%
			\procedure{$\boxed{\simOP(\inp, \wit)} \ \simOP'(\inp)$}
			{ 
				\boxed{\pcassert (\inp,\wit)\in \REL} \\ 
				\pi \gets \simulator^{\simOH,\simulator\oracleo.\prog}(\srs,\inp)\\
				Q \gets Q \cup \{(\inp,\pi)\}\\
				\pcreturn \pi
			}
		\end{pchstack}
	\end{pcvstack}
	\caption{Simulation oracles: $\srs$ is the finalized SRS, only $\simOP'$ allows for simulation of false statements}
	\label{fig:real_simulator_oracles}
\end{figure}

\subsection{Trapdoor-Less Zero-Knowledge}



%We model $\simulator$ as a stateful
%algorithm that runs in two modes. The first mode,
%$(h, st') \gets \simulator (1, st, \srs, q)$ answers random oracle calls to $\ro$ on input
%$q$. The second mode $(\zkproof, st') \gets \simulator (2, st, \srs, \inp)$ simulates the
%actual argument for instance $\inp$.  



% \begin{itemize}
% \item $\simOH (\srs,q)$ denotes an oracle that returns the first output of
%   $\simulator (1, st, \srs, q)$;
% \item $\simOP (\srs,\inp, \wit)$ denotes an oracle that returns the first output of
%   $\simulator (2, st, \srs, \inp)$ if $(\inp, \wit) \in \REL$, and returns $\bot$ otherwise;
% \end{itemize}

% We call a proof system $\proofsystem$ \emph{zero-knowledge} if for any
% $\REL \in \RELGEN(\secparam)$, and adversary $\adv$ there exists a $\ppt$ simulator
% $\simulator$ such that for any $(\inp, \wit) \in \REL$, $\eps_0 \approx \eps_1$,
% where
% \[
%   \eps_b = \condprob{\adv^{\oracleo_b} (\srs)}{\srs \gets \kgen (\REL)}\,,
% \]
% \changedm{where $\oracleo_0$ on input $(1, q)$ responds with $h$ such that
%   $(h, st') \gets \simOH (1, st, q)$, where $st$ and $st'$ are old and new
%   states of the simulator $\simulator$ and on input $(2, \inp, \wit)$ returns
%   $\zkproof \gets \simOP (2, st, \srs, \inp)$. Alternatively, $\oracleo_1$ on
%   input $(1, q)$ responds with $h \gets \ro (q)$ and on input $(2, \inp, \wit)$
%   returns $\zkproof \gets \prover (\srs, \inp, \wit)$.}
% 	%
% We call zero knowledge \emph{perfect} if the distributions are equal and
% \emph{computational} if they are indistinguishable for any $\ppt$ distinguisher.
\iffalse
We call a proof system $\proofsystem$ \emph{zero-knowledge} if for any
$\REL \in \RELGEN(\secparam)$, and adversary $\adv$ there exists a $\ppt$ simulator
$\simulator$ with oracles $(\simOH,\simOP)$ such that for any $(\inp, \wit) \in \REL$, $\eps_0 \approx \eps_1$,
where
\[
  \eps_0(\secpar) = \condprob{\adv^{\simOH,\simOP} (\srs)}{\srs \gets \kgen
    (\REL)},\,  \eps_1 (\secpar) = \condprob{\adv^{\ro,\prover} (\srs)}{\srs \gets \kgen (\REL)},
\]

We call zero knowledge \emph{perfect} if the distributions are equal and
\emph{computational} if they are indistinguishable for any $\ppt$ distinguisher.


% \end{description}
Alternatively, zero-knowledge can be defined by allowing the simulator to use the
trapdoor $\td$ that is generated along the $\srs$. In this paper we distinguish
simulators that requires a trapdoor to simulate and those that do not. We call the
former \emph{SRS-simulators}. We say that a protocol is \emph{trapdoor-less
  zero-knowledge} (TLZK) if its simulator does not require the trapdoor, cf.~\cref{def:TLZK}.

  \markulf{22.04}{We don't consider SRS-simulators in this paper. I would simplify and move the above after the Updatable SRS scheme section, or maybe the start of Section 3.}
\fi

In this paper we distinguish between simulators that require a trapdoor to simulate and those that do not. We call a protocol \emph{trapdoor-less
 zero-knowledge} (TLZK) if there exists a simulator does not require the trapdoor, and works by programming the random oracle.
 Moreover, it may be not allowed to program every random oracle challenge but only challenges that come after the $k$-th prover's message. We call protocols which allow for such a simulation $k$-\emph{programmable trapdoor-less zero-knowledge}, and define this in~\cref{def:TLZK}.

 Our definition of zero-knowledge for non-interactive
arguments is in the explicitly programmable random oracle model where the simulator
$\simulator$ can program the random oracle. 
We model this using the oracles from Fig.~\ref{fig:real_simulator_oracles} that provide a stateful wrapper around $\simulator$.
$\simOH (x)$ simulates $\ro$ using lazy sampling, $\simulator\oracleo.\prog(x, h)$ allows for programming the simulated $\ro$ and is available only to $\simulator$. $\simOP(x, w)$ and $\simOP'(x)$ call the simulator. The former is used in the zero-knowledge definition and requires the statement and witness to be in the relation, the latter is used in the simulation extraction definition and does not require a witness input.

\begin{definition}[Updatable ($k$-programmable) Trapdoor-Less Zero-Knowledge]
  \label{def:TLZK}
  Let 
  $\psfs = (\SRScer, \prover, \verifier, \simulator)$ be a NIZK proof system with an updatable SRS setup compiled via the $(2\mu + 1)$-message Fiat--Shamir transformation. Let $\ro$ be the random oracle. 
  %$\simulator_{\fs}$ takes as input $\srs$ and instance $\inp$, programs $\ro$, and outputs a proof $\zkproof_\simulator$.  
  We call $\psfs$ \emph{trapdoor-less zero-knowledge} with security $\epszk$ if for any
  adversary $\adv$, $\abs{\eps_0(\secpar) - \eps_1(\secpar)} \leq \epszk(\secpar)$, where
  \begin{align*}
    \eps_0 (\secpar) = \Pr\left[ \adv^{\initU, \ro, \prover} (\secparam) \right],\,
    \eps_1 (\secpar)=  \Pr \left[\adv^{\initU, \simOH, \simOP} (\secparam) \right].
  \end{align*}
  
  If $\epszk(\secpar)$ is negligible, we say $\ps_{\fs}$ is trapdoor-less zero-knowledge. Additionally, we say that $\ps_{\fs}$ is $k$-programmable, if  $\simulator$ before returning a proof $\pi$ only calls $\simulator\oracleo.\prog$ on $(\tzkproof[0..k],h)$. That is, it only programs the $k$-th message.
  \end{definition}

  
\begin{remark}[TLZK vs HVZK]
  We note that TLZK notion is closely related to honest-verifier zero-knowledge in the
  standard model. That is, if we consider an interactive proof system $\proofsystem$
  that is HVZK in the standard model then $\proofsystem_\fs$ is TLZK. This comes as the simulator $\simulator$ in
  $\proofsystem$ produces a valid simulated proof by picking verifier's challenges
  according to a predefined distribution and $\proofsystem_\fs$'s simulator
  $\simulator_\fs$ produces its proofs similarly by picking the challenges and
  additionally programming the random oracle to return the picked
  challenges. Importantly, in both $\proofsystem$ and $\proofsystem_\fs$ success of
  the simulator does not depend on access to an SRS trapdoor.
\end{remark}

We note that $\plonk$ is $2$-programmable TLZK, $\sonic$ is $1$-programmable TLZK,
and $\marlin$ is $1$-programmable TLZK. This follows directly from the proofs of
their standard model zero-knowledge property in
\cref{lem:plonk_tlzk,lem:sonic_hvzk,lem:marlin_hvzk}.

\subsection{Simulation Extractability}
We note that the zero-knowledge property is only guaranteed for statements in the
language.
%; for simulator $\simulator = (\simOH, \simOP)$, $\simOP$
%answers with simulated proofs only for true statements.  This is, however, not sufficient
For \emph{simulation extractability} where the simulator
should be able to provide simulated proofs for false statements as well, we thus use the oracle $\simOP'$
\footnote{Note,
  that simulation extractability property where the simulator is required to give
  simulated proofs for true statements only is called \emph{true simulation
    extractability.}}. 
	
%Therefore, we introduce a wrapper oracle around the simulator
%called $\simOP'$ that on input $(\srs, \inp)$ always returns the first output of
%$\simulator (2, st, \srs, \inp)$, regardless of whether $\inp$ is in the language. We
%define \emph{simulation-extractability} with respect to oracle $\simOP'$; that is,
%simulation-extractability is with respect to a simulator
%$\simulator' = (\simOH, \simOP')$.
%



\begin{definition}[Updatable Simulation Extractable NIZK]
	\label{def:updsimext}
  \label{def:simext}
	Let $\psni = (\SRScer, \prover, \verifier, \simulator)$ be a NIZK proof system with an updatable SRS setup. 
	We say that
  $\psni$ is \emph{updatable simulation-extractable} with security loss $\epsse(\secpar,\accProb, q)$ if for
  any $\ppt$ adversary $\adv$ that is given oracle access to setup oracle
  $\initU$ and simulation oracle $\simulator\oracleo$ and that produces an accepting
  proof for $\psni$ with probability $\accProb$, where
	\[
	\accProb = \condprob{
	\begin{matrix}
	  \verifier(\srs, \inp, \zkproof) = 1  \\
	  \wedge
	(\inp, \zkproof) \not\in Q
	\end{matrix}
}{
	\begin{matrix}
	  r \sample \RND{\advse}\\
	(\inp, \zkproof) \gets \advse^{\initU, \simOH, \simOP'
		} (1^\secpar; r)
	\end{matrix}
}
	\]
	there exists an expected PPT extractor $\extse$ such that
	\[
	 \condprob{
	\begin{matrix}
  \verifier(\srs, \inp, \zkproof) = 1, \\
   (\inp, \zkproof) \not\in Q,  \\
	   \REL(\inp, \wit) = 0
	\end{matrix}
}{
	\begin{aligned}
	& r \sample \RND{\advse},
	(\inp, \zkproof) \gets \advse^{\initU, \simOH, \simOP'
		} (1^\secpar; r) \\
	& \wit \gets \ext_\se (\srs, \advse, r,
	\Qsim, \Qro, \Qsrs) 
	\end{aligned}
} \leq \epsse(\secpar,\accProb, q)
	\]
	%is at least 
	%\[
	%\extProb \geq \frac{1}{\poly} (\accProb - \nu)^d - \eps(\secpar)\,,
	%\]
	%for some polynomial $\poly$, constant $d$ and negligible $\eps(\secpar)$ whenever
	%$\accProb \geq \nu$. 
	Here, $\srs$ is the finalized SRS, list $Q$ contains all $(\inp, \zkproof)$ pairs where 
	$\inp$ is an instance queried to $\simOP'$ by the adversary and
	$\zkproof$ is the simulator's answer. List $\Qro$ contains all $\advse$'s
	queries to $\simOH$ and the (simulated) random oracle's answers, $\abs{\Qro}\leq q$, and list $\Qsrs$ contains all $(\srs, \rho)$ of update SRSs and their proofs.
  % \hamid{17.10}{shouldn't be "List $Q_\ro$ contains all $\advse$'s queries to $\simOH$"? Also, I think we need to remove $\ro$ from the statement of the definition!}
\end{definition}

\subsection{Unique Response Protocols}
A technical hurdle identified by Faust et al.~\cite{INDOCRYPT:FKMV12} for proving
simulation extraction via the Fiat--Shamir transformation is that the transformed
proof system satisfies a unique response property. The original formulation by Fischlin, although suitable for applications presented in
\cite{C:Fischlin05,INDOCRYPT:FKMV12}, does not suffice in our case. First, the
property assumes that the protocol has three messages, with the second being the
challenge from the verifier. That is not the case we consider here. Second, it is not
entirely clear how to generalize the property. Should one require that after the
first challenge from the verifier, the prover's responses are fixed?  That does not
work since the prover needs to answer differently on different verifier's challenges,
as otherwise the protocol could have fewer messages.  Another problem is that the
protocol could have a message, beyond the first prover's message, which is
randomized. Unique response cannot hold in this case. Finally, the protocols we
consider here are not in the standard model, but use an SRS.

We work around these obstacles by providing a generalized notion of the unique
response property. More precisely, we say that a $(2\mu + 1)$-message protocol
has \emph{unique responses from $k$}, and call it a $\ur{k}$-protocol, if it
follows the definition below:

\begin{definition}[Updatable $(\ur{k})$-protocol]
Let $\psfs = (\SRScer, \prover, \verifier, \simulator)$ be a NIZK proof system with an updatable SRS setup compiled via the $(2\mu + 1)$-message Fiat--Shamir transformation. Let $\ro$ be the random oracle. 
We say that $\psfs$ has \emph{unique responses for $k$} with security $\epsur(\secpar)$ if for any $\ppt$ adversary $\advur$:
  \[
	\Pr\left[
		\left.
	\begin{aligned}
	& \zkproof \neq \zkproof', \tzkproof[0..k] = \tzkproof'[0..k],  \\
	& \verifier' (\srs, \inp, \zkproof,c) =
	\verifier' (\srs, \inp, \zkproof',c) = 1  \\
	\end{aligned}
	\,\right|\,
	\begin{aligned}
		& (\inp, \zkproof, \zkproof', c) \gets \advur^{\initU,\ro}(1^\secpar) 
	\end{aligned}
	\right] \leq \epsur(\secpar) 
	\]
	where $\srs$ is the finalized SRS and $\verifier'(\srs,\inp,\zkproof=(a_1, \dots, a_\mu,a_{\mu+1}))$ behaves as $\verifier (\srs, \inp, \zkproof)$ except for the $k$-th challenge it uses $c$ instead of calling $\ro(\tzkproof[0..k]) $. That is, it allows the $k$-th challenge to be programmed by $\adv$. 
	If $\epsur(\secpar)$ is negligible, we simply say $\psfs$  is $\ur{k}$.
\end{definition}

Intuitively, a protocol is $\ur{k}$ if it is infeasible for a $\ppt$ adversary to produce a pair of accepting proofs $\zkproof \neq \zkproof'$ that are the same on the first $k$ messages of the prover.  
%We note that the definition above is also meaningful for protocols without an SRS. Intuitively in that case $\srs$ is the empty string.

We note that the definition above can be easily generalized to allow the adversary to program the oracle on more than just a single point. However, since all the protocols we analyze in this paper require only single-point programming, we opted for a simplified version of the definition.
  

\subsection{Rewinding-Based Knowledge Soundness}

Before giving the definition of rewinding-based knowledge soundness for NIZK proof systems compiled via the $2\mu + 1$-message FS transformation, we first recall the notion of a tree of transcripts.
\begin{definition}[Tree of accepting transcripts, cf.~{\cite{EC:BCCGP16}}]
	\label{def:tree_of_accepting_transcripts}
%	Consider a $(2\mu + 1)$-message proof system. \markulf{24.04}{Is this meaningful for interactive or also non-interactive prove systems obtained via FS?} 
	A $(n_1,
  \ldots, n_\mu)$-tree of accepting transcripts is a tree where each node on
  depth $i$, for $i \in \range{1}{\mu + 1}$, is an $i$-th prover's message in an
  accepting transcript; edges between the nodes are labeled with
  challenges, such that no two edges on the same depth have the same
  label; and each node on depth $i$ has $n_{i} - 1$ siblings and $n_{i +
    1}$ children. The tree consists of $N = \prod_{i = 1}^\mu n_i$
  branches, where $N$ is the number of accepting transcripts. We require $N = \poly$. We refer to a $(1, \ldots, n_k=n, 1, \ldots, 1)$-tree as a $(k,n)$-tree.
\end{definition}

\iffalse
		
	\begin{figure}[t]
		\centering
		\fbox{
		\procedure{$\genforking_{\zdv}^{m, m'} (y,h_1^{1}, \ldots, h_{q}^{1})$}		
		{
		\rho \sample \RND{\zdv}\\
		(i, s_1) \gets \zdv(y, h_1^{1}, \ldots, h_{q}^{1}; \rho)\\
    i_1 \gets i\\
		% \pcif i = 0\ \pcreturn (0, \bot)\\
		\pcfor j \in \range{2}{m'}\\
		\pcind h_{1}^{j}, \ldots, h_{i - 1}^{j} \gets h_{1}^{j - 1}, \ldots,
		h_{i - 1}^{j - 1}\\
		\pcind h_{i}^{j}, \ldots, h_{q}^{j} \sample H\\
		\pcind (i_j, s_j) \gets \zdv(y, h_1^{j}, \ldots, h_{i - 1}^{j}, h_{i}^{j},
		\ldots, h_{q}^{j}; \rho)\\
		%\pcind \pcif i_j = 0 \lor i_j \neq i\ \pcreturn (0, \bot)\\
   % \pcif \exists (j, j') \in \range{1}{m}^2, j \neq j' : (h_{i}^{j} = h_{i}^{j'})\
    \pcif \exists (j_1, \ldots, j_m) \in \range{1}{m'}^m, \text{ s.t. }\\
    \pcind j_k \neq j_{k'} \text{ for } k \neq k' \land \\
    \pcind i_{j_k} = i_{j_k'} \land \\
    \pcind 
		\pcelse \pcreturn (0, \bot)
	}}
\caption{Generalized forking algorithm $\genforking_{\zdv}^{m, m'}$
  \michals{5.11}{This forking lemma version is not as general as in Attema et al to
    be exact -- they allow tree of acceptable transcripts to branch at multiple
    places. however, that is not necessary in our case. Later, we should allow ``full
    generality'' but that would also require modification of the forking soundness
    definition (which now also relies on a fact that the tree branches at a single
    point)}}
	\label{fig:genforking_lemma}
\end{figure}


\fi



\iffalse
Note that the special soundness property (as usually defined) holds for
all---even computationally unbounded---adversaries. Unfortunately, since a
simulation trapdoor for $\plonkprot$, $\sonicprot$ and $\marlinprot$ exists, the protocols
cannot be special sound in that regard. This is because an unbounded adversary
can recover the trapdoor and build an arbitrary number of simulated proofs for a fake
statement. Hence, we provide a weaker, yet sufficient, definition that somehow lies between 
computational \emph{knowledge soundness} and \emph{special soundness} for unbounded adversaries. Differently from the standard definition of special soundness, we do not require from the extractor to be able to extract the witness from \emph{any} tree of acceptable transcripts. Similar to knowledge soundness, we require that the extractor fails to reconstruct the witness with some small probability only. 

More precisely, in proofs of rewinding-based knowledge soundness we show later that either the extractor extracts a witness from a tree of transcripts or the adversary who was used to produce the tree can be used to break some computationally hard problem. 
\fi
The existence of simulation trapdoor for $\plonkprot$, $\sonicprot$ and $\marlinprot$ means that they are not
special sound in the standard sense. We therefore put forth the notion of rewinding-based knowledge soundness that is a computational notion. 
Note that in the definition below, it is implicit that each transcript in the tree is accepting with respect to a ``local programming'' of the random oracle. However, the verification of the proof output by the adversary is with respect to a non-programmed random oracle.

\begin{definition}[Updatable Rewinding-based knowledge soundness]
	Let $n_1, \ldots, n_\mu \in \NN$. 
	%
	Let $\psfs = (\SRScer, \prover, \verifier, \simulator)$ be a NIZK proof system for relation $\REL$ with an updatable SRS setup compiled via the $(2\mu + 1)$-message Fiat--Shamir transformation. Let $\ro$ be the random oracle.
	%
	We require existence of an expected PPT tree builder $\tdv$ that eventually outputs a $\tree$ which is either a $(n_1, \ldots, n_\mu)$-tree of accepting transcript or $\bot$ and a PPT extractor $\extcss$. Let  adversary $\advcss$ be a PPT algorithm, that outputs a valid proof with probability at least $\accProb$, 
	where
	\[
	\accProb = \condprob{
	\begin{matrix}
	  \verifier(\srs, \inp, \zkproof) = 1  \\
	  \wedge
	(\inp, \zkproof) \not\in Q
	\end{matrix}
}{
	\begin{matrix}
	  r \sample \RND{\advcss}\\
	(\inp, \zkproof) \gets \advcss^{\initU, \ro
		} (1^\secpar; r)
	\end{matrix}
}.
	\]
	We say that $\psfs$ is $(n_1, \ldots, n_\mu)$-rewinding-based knowledge sound with security loss $\epscss(\secpar, \accProb, q)$ if
	\begin{align*}
	\Pr\left[
		\begin{matrix}
			\verifier(\srs, \inp, \zkproof) = 1,  \\
			\REL(\inp, \wit) = 0
		  \end{matrix}
	\,\left|\,
	\begin{aligned}
	& 	r \sample \RND{\advcss}, \\
	& 	(\srs, \inp, \cdot) \gets \advcss^{\initU, \ro} (1^\secpar; r)\\
	&  	\tree \gets \tdv(\srs, \advcss, r,\Qro, \Qsrs),
	\wit \gets \extcss(\tree)
	\end{aligned}\right.
	\right] \leq \epscss(\secpar,\accProb, q).
	\end{align*}
	Here, $\srs$ is the finalized SRS. List $\Qro$ contains all of the adversaries
	queries to $\ro$ and the random oracle's answers, $\abs{\Qro}\leq q$, and list $\Qsrs$ contains all $(\srs, \rho)$ of updated SRSs and their proofs.
%	 We call the protocol $(n_1, \ldots, n_\mu)$ rewinding-based knowledge sound when $\tdv$ builds a $(n_1, \ldots, n_\mu)$-tree of transcripts.
% \hamid{1.5}{We still need this sentence:  
We call the protocol $(k, n)$-rewinding-based knowledge sound if $\tree$ is a $(k, n)$-tree of accepting transcripts.
% }
\end{definition}
	



%%% Local Variables:
%%% mode: latex
%%% TeX-master: "main"
%%% End:
