% !TEX root = main.tex
% !TEX spellcheck = en-US
\section{Definitions and lemmas for multi-round SRS-based protocols}
\label{sec:se_definitions}


  \ourpar{Signatures of Knowledge from SE-SNARKs.}
The work of Groth and Maller  \cite{C:GroMal17} shows
how to construct a signature of knowledge for messages in $\{0, 1\}^*$
from  target collision-resistant hash-function and simulation-extractable SNARKs. 
These schemes inherit the same properties as the underlying SNARK schemes, 
they rely on a structured reference string, which can be updatable or universal and they are succinct. 
In the following, our goal will be to show that recent efficient SNARKs are simulation extractable,  and therefore, they are a perfect candidate to build better signatures of knowledge. 
 

  \ourpar{Simulation-Extractability for multi-round protocols.} 
Most recent SNARK schemes are following the same blueprint: First, an interactive (multi-round) information-theoretic proof system is considered. 
%However, this initial system is idealized and also inefficient.  
Second,  this initial proof system is compiled into an efficient, non-interactive and computationally sound scheme using cryptographic tools such as polynomial commitments and the Fiat-Shamir transformation.

We remark that existing results on the Fiat-Shamir transform regarding existential unforgeability (for signatures) and simulation extraction (for proof systems and signatures of knowledge) work for $3$-round protocols without reference string that require two transcripts for standard model extraction, e.g., \cite{JC:PoiSte00,INDOCRYPT:FKMV12,C:RotSeg21}.
In this section we prepare our analysis for multi-round protocols with a universal or updatable SRS.
In order to prove simulation-extractability for such protocols,  we require more than
just two transcripts for extraction.  Moreover, in the updatable setting we consider protocols that rely on an SRS where the adversary gets to contribute to the SRS. This makes out analysis more challenging.
We first give a definition of simulation-extractability which we base on~\cite{INDOCRYPT:FKMV12} adapted to the updatable SRS setting. Then, to support multi-round SRS-based protocols compiled using the Fiat-Shamir transform we generalize the unique response property, replace honest-verifier zero-knowledge property with trapdoor-less simulation and special soundness with a forking special soundness property that links up with our generalized forking lemma.
\michals{4.10}{Restructure to introduce here the purpouse and structure of the section}

\subsection{Simulation extractability}

We note that the zero-knowledge property is only guaranteed for statements
in the language. The simulator $\simulator = (\simulator_1, \simulator_2)$, where
$\simulator_2$ answers with simulated proofs only for true statements.  This is, however, not sufficient for
\emph{simulation extractability} where the simulator that the adversary has access to
should be able to provide simulated proofs for false statements as well\footnote{Note, that
  simulation extractability property where the simulator is required to give
  simulated proofs for true statements only is called \emph{true simulation
    extractability.}}. Therefore, we introduce a wrapper oracle around the simulator called $\simulator'_2$ that on
  input $(\srs,\inp)$ always returns the first output of $\simulator (2, st, \srs, \inp)$, regardless of whether $\inp$ is in the language. We define \emph{simulation-extractability} with respect to oracle $\simulator_2'$; that is,
simulation-extractability is with respect to a simulator $\simO = (\simulator_1, \simulator_2')$.
%
\begin{definition}[Simulation-extractable NIZK]
	\label{def:updsimext}
  \label{def:simext}
	Let $\ps = (\kgen, \prover, \verifier, \simulator)$ be a NIZK proof system. We say
  that $\ps$ is \emph{updatable simulation-extractable} with respect to $\simO$ with
  \emph{extraction error} $\nu$ if for any $\ppt$ adversary $\adv$ that is given
  oracle access to an updatable SRS setup $\initU$, a simulation oracle $\simO$,
  cf.~\cref{fig:upd}, and a random oracle $\ro$, and that produces an accepting proof
  for $\ps$ with probability $\accProb$, where
	\[
	\accProb = \condprob{
	\begin{matrix}
	  \verifier(\srs, \inp_{\advse}, \zkproof_{\advse}) = 1  \\
	  \wedge
	(\inp_{\advse}, \zkproof_{\advse}) \not\in Q
	\end{matrix}
}{
	\begin{matrix}
	  r \sample \RND{\advse}\\
	(\inp_{\advse}, \zkproof_{\advse}) \gets \advse^{\initU, \simO
		} (1^\secpar; r)
	\end{matrix}
}
	\]
	there exists an extractor $\extse$ such that
	\[
	\extProb = \condprob{
	\begin{matrix}
  \verifier(\srs, \inp_{\advse}, \zkproof_{\advse}) = 1 \\
 \wedge  ~(\inp_{\advse}, \zkproof_{\advse}) \not\in Q   \\
	 \wedge  ~\REL(\inp_{\advse}, \wit_{\advse}) = 1
	\end{matrix}
}{
	\begin{aligned}
	& r \sample \RND{\advse},
	(\inp_{\advse}, \zkproof_{\advse}) \gets \advse^{\initU, \simO
		} (1^\secpar; r) \\
	& \wit_{\advse} \gets \ext_\se (\srs, \advse, r, \inp_{\advse}, \zkproof_{\advse},
	Q, Q_\ro, Q_\srs) 
	\end{aligned}
}
	\]
	is at at least 
	\[
	\extProb \geq \frac{1}{\poly} (\accProb - \nu)^d - \eps(\secpar)\,,
	\]
	for some polynomial $\poly$, constant $d$ and negligible $\eps(\secpar)$ whenever
	$\accProb \geq \nu$. 
	Here, $\srs$ is the finalized SRS, list $Q$ contains all $(\inp, \zkproof)$ pairs where 
	$\inp$ is an instance queried to $\simO$ by the adversary and
	$\zkproof$ is the simulator's answer. List $Q_\ro$ contains all $\advse$'s
	queries to $\ro$ and the random oracle's answers.\hamid{17.10}{shouldn't be "List $Q_\ro$ contains all $\advse$'s queries to $\simulator_1$"? Also, I think we need to remove $\ro$ from the statement of the definition!}
\end{definition}

\chaya{13.10}{later, we should show $\simulator_2'$ for plonk/sonic/marlin. I think
  it will follow from TLS. But we should emphasize there how to define the above
  oracles so it is clear that $\simO$ can produce accepting proofs for false
  statements.}
\michals{19.10}{Re showing that Plonk's et al simulators can produce acceptable
  proofs for false statements, I am not convinced that's what we need -- we define SE
  wrt concrete simulators, hence if someone defines a proof system with a simulator
  that outputs a trapdoor when queried on a false statement, then such a proof
  systems is simply not SE. For our main theorem we use trapdoor-less simulators
  which cannot do harm, however I think we should discuss how to show that
  possibility of RO programming doesn't do harm as well. }

\subsection{Unique-response protocols}
A technical hurdle identified by Faust et al.~\cite{INDOCRYPT:FKMV12} for
proving simulation extraction via the Fiat-Shamir transformation is that the
transformed prove system satisfies a unique response property. The original
Fischlin's formulation, although suitable for applications presented in
\cite{C:Fischlin05,INDOCRYPT:FKMV12}, does not suffice in our case. First, the
property assumes that the protocol has three messages, with the second being the
challenge from the verifier. That is not the case we consider here. Second, it
is not entirely clear how to generalize the property. Should one require that
after the first challenge from the verifier, the prover's responses are fixed?
That does not work since the prover needs to answer differently on different
verifier's challenges, as otherwise the protocol could have fewer rounds.
Another problem is that the protocol could have a round beyond the
first round where the message of the prover is randomized. Unique response cannot hold in
this case. Finally, the protocols we consider here are not in the standard
model, but use an SRS.

We work around these obstacles by providing a generalised notion of the unique
response property. More precisely, we say that a $(2\mu + 1)$-message protocol
has \emph{unique responses from $i$}, and call it a $\ur{i}$-protocol, if it
follows the definition below:


\begin{definition}[$\ur{i}$-protocol in the updatable setting]
	\label{def:wiuru}
	Let $\proofsystem$ be a $(2\mu + 1)$-message public coin proof system
  $\ps = (\kgen, \prover, \verifier, \simulator)$. Let
  $\proofsystem_\fs = (\kgen_\fs, \prover_\fs, \verifier_\fs, \simulator_\fs)$ be
  $\proofsystem$ after the Fiat--Shamir transform and $\ro$ the random oracle. Denote
  by $\zkproof_1, \ldots, \zkproof_{\mu}, \zkproof_{\mu + 1}$ protocol messages
  output by the prover and by $\zkproof_i.\ch$ random oracle responses on a partial
  transcript
  $(\inp, \zkproof_1, \zkproof_1.\ch, \ldots, \zkproof_\mu.\ch, \zkproof_{\mu + 1})$,
  where $\inp$ is the proven statement.We say that $\proofsystem$ has \emph{unique
    responses from $i$ on} if for any $\ppt$ adversary $\adv$:
  \[
	\Pr\left[
	\begin{aligned}
	& \vec{\zkproof} \neq \vec{\zkproof'}, (\zkproof_1, \ldots, \zkproof_{i}) = (\zkproof'_1,
	\ldots, \zkproof'_{i}), \\
	& \verifier_\fs (\srs, \inp, \zkproof) =
	\verifier_\fs(\srs, \inp, \zkproof') = 1  \\
	\end{aligned}
	\,\left|\,
	\begin{aligned}
	& \inp, \zkproof, \zkproof'  \gets \adv^{\ro, \initU}(1^\secpar) \\
& \zkproof = (\zkproof_1, \zkproof_1.\ch, \ldots, \zkproof_{\mu}.\ch, \zkproof_{\mu + 1}), \zkproof' = (\zkproof'_1, \ldots,
	\zkproof'_{\mu}.\ch, \zkproof'_{\mu + 1})
	\end{aligned}
	\right.\right]
	\]
	is upper-bounded by some negligible function $\negl$.
\end{definition}

Note that in the above definition, $\srs$ is the SRS that $\advse$ finalised using
the update oracle $\initU$, defined in~\cref{fig:upd}.

Intuitively, a protocol is $\ur{i}$ if it is infeasible for a $\ppt$ adversary to
produce a pair of acceptable and different proofs $\zkproof$, $\zkproof'$ that are
the same on the first $i$ prover messages.  We note that the definition above is also
meaningful for protocols without an SRS. Intuitively in that case $\srs$ is the empty
string.

 \subsection{Trapdoor-Less Simulation}
In security reductions in this paper it is sometimes needed to produce
simulated NIZK proofs without knowing the trapdoor, just by
programming the random oracle. We call protocols which allow for such a
simulation \emph{trapdoor-less simulatable} (TLS). 

\begin{definition}[($k$-programmable) Trapdoor-Less Simulatable Proof System]
  Let $\ps$ be a $(2\mu + 1)$-message ZK proof system. Let
  $\ps_\fs= (\kgen_{\fs}, \prover_{\fs}, \verifier_{\fs}, \simulator_{\fs})$ be its
  Fiat--Shamir variant and $\ro$ the random oracle. $\simulator_{\fs}$ takes as input
  $\srs$ and instance $\inp$, programs $\ro$, and outputs a proof
  $\zkproof_\simulator$.  We call $\ps$ \emph{trapdoor-less simulatable} if for any
  adversary $\adv$, $\eps_0 \approx \eps_1$, where
  \begin{align*}
    \eps_b = \Pr\left[
    \begin{aligned}
      \adv^{\initU, \oracleo_b} = 0
    \end{aligned}
    \right]
  \end{align*}
  On $\adv$'s query $(1, q)$ $\oracleo_b$ responds with $\ro (q)$ if $b = 0$ and with
  $\simulator_1 (1, st, q)$ if $b = 1$. On query $(2, \inp, \wit)$, oracle
  $\oracleo_{b}$ responds with a real proof
  $\zkproof_\prover \gets \prover_{\fs} (\srs, \inp, \wit)$ if $b = 0$ or a simulated
  proof $\zkproof_\simulator \gets \simulator_{\fs} (\srs, \inp)$ if $b = 1$.

  Additionally, we say that $\ps_{\fs}$ is $k$-programmable, if $\simulator_\fs$
  programs the random oracle \emph{only} for challenges from round $k$ to $\mu + 1$.
  \end{definition}

  
\begin{remark}[TLS vs HVZK]
  We note that TLS notion is closely related to honest-verifier zero knowledge in the
  standard model. That is, if we consider an interactive proof system $\proofsystem$
  that is HVZK in the standard model then its Fiat--Shamir compiled version
  $\proofsystem_\fs$ is TLS. This comes as the simulator $\simulator$ in
  $\proofsystem$ produces a valid simulated proof by picking verifier's challenges
  according to a predefined distribution and $\proofsystem_\fs$'s simulator
  $\simulator_\fs$ produces its proofs similarly by picking the challenges and
  additionally programming the random oracle to return the picked
  challenges. Importantly, in both $\proofsystem$ and $\proofsystem_\fs$ success of
  the simulator does not depend on access to an SRS trapdoor.
\end{remark}

We note that $\plonk$ is $2$-programmable TLS, $\sonic$ is $1$-programmable TLS,
and $\marlin$ is $1$-programmable TLS. This follows directly from the proofs of
their standard model zero-knowledge property in
\cref{lem:plonk_hvzk,lem:sonic_hvzk,lem:marlin_hvzk}. 





\subsection{Generalised forking lemma and forking special soundness}
%\label{sec:forking_lemma}
Although dubbed ``general'', the forking lemma of~\cite{CCS:BelNev06} is not
general enough for our purpose as it is useful only for protocols where a witness
can be extracted from just two transcripts. To be able to extract a witness
from, say, an execution of $\plonkprot$ we need at least
$(3 \numberofconstrains + 1)$ valid proofs, and $(\multconstr + \linconstr + 1)$ for $\sonicprot$. We
propose a generalisation of the general forking lemma that given a probability of
producing an accepting transcript, $\accProb$, lower-bounds the probability of
generating a \emph{tree of accepting transcripts} $\tree$, which allows to
extract a witness.

\begin{definition}[Tree of accepting transcripts, cf.~{\cite{EC:BCCGP16}}]
	\label{def:tree_of_accepting_transcripts}
	Consider a $(2\mu + 1)$-message proof system. A $(n_1,
  \ldots, n_\mu)$-tree of accepting transcripts is a tree where each node on
  depth $i$, for $i \in \range{1}{\mu + 1}$, is an $i$-th prover's message in an
  accepting transcript; edges between the nodes are labeled with verifier's
  challenges, such that no two edges on the same depth have the same
  label; and each node on depth $i$ has $n_{i} - 1$ siblings and $n_{i +
    1}$ children. The tree consists of $N = \prod_{i = 1}^\mu n_i$
  branches, where $N$ is the number of accepting transcripts. We require $N = \poly$.
\end{definition}


\begin{lemma}[General forking lemma II]
	\label{lem:generalised_forking_lemma}
	Fix $q \in \ZZ$ and set $H$ of size $h \geq m$. Let $\zdv$ be a $\ppt$
  algorithm that on input $y, h_1, \ldots, h_q$ returns $(i, s)$ where $i \in
  \range{0}{q}$ and $s$ is called a side output. Denote by $\ig$ a randomised
  instance generator. We denote by $\accProb$ the probability
	\[
		\condprob{i \neq 0}{ y \gets \ig;\ h_1, \ldots, h_q \sample H;\ (i, s)
		\gets \zdv(y, h_1, \ldots, h_q)}\,.
	\]
	Let $\genforking_{\zdv}^{m}$ denote the algorithm described in
  \cref{fig:genforking_lemma} then the probability $\frkProb := \condprob{b =
    1}{y \gets \ig;\ h_1, \ldots, h_{q} \sample H;\ (b, \vec{s}) \gets
    \genforking_{\zdv}^{m}(y, h_1, \ldots, h_q)}$ is at least
	\[
		\frac{\accProb^m}{q^{m - 1}} - \accProb \cdot \left(1 -
    \frac{h!}{(h - m)! \cdot h^{m}}\right).
	\]
		
	\begin{figure}[t]
		\centering
		\fbox{
		\procedure{$\genforking_{\zdv}^{m} (y,h_1^{1}, \ldots, h_{q}^{1})$}		
		{
		\rho \sample \RND{\zdv}\\
		(i, s_1) \gets \zdv(y, h_1^{1}, \ldots, h_{q}^{1}; \rho)\\
    i_1 \gets i\\
		\pcif i = 0\ \pcreturn (0, \bot)\\
		\pcfor j \in \range{2}{m}\\
		\pcind h_{1}^{j}, \ldots, h_{i - 1}^{j} \gets h_{1}^{j - 1}, \ldots,
		h_{i - 1}^{j - 1}\\
		\pcind h_{i}^{j}, \ldots, h_{q}^{j} \sample H\\
		\pcind (i_j, s_j) \gets \zdv(y, h_1^{j}, \ldots, h_{i - 1}^{j}, h_{i}^{j},
		\ldots, h_{q}^{j}; \rho)\\
		\pcind \pcif i_j = 0 \lor i_j \neq i\ \pcreturn (0, \bot)\\
    \pcif \exists (j, j') \in \range{1}{m}^2, j \neq j' : (h_{i}^{j} = h_{i}^{j'})\
	\pcreturn (0, \bot)\\
		\pcelse \pcreturn (1, \vec{s})
	}}
	\caption{Generalised forking algorithm $\genforking_{\zdv}^{m}$}
	\label{fig:genforking_lemma}
\end{figure}
\end{lemma}
The proof is along similar lines as~\cite[Lemma 1]{CCS:BelNev06} with modifications required
by the fact that the protocol has more than $3$ rounds and the number of
transcripts required is larger. \COMMENT{We defer the proof to
  \cref{sec:forking_proof}.}

% \subsection{Proof of the generalized forking lemma (\cref{lem:generalised_forking_lemma})}
% \label{sec:forking_proof}
\begin{proof}
First denote by $\accProb(y)$ and $\frkProb(y)$ the following probabilities
\begin{align*}
\accProb(y) & =  \condprob{i \neq 0}{h_1, \ldots, h_q \sample H;\ (i, s)
\gets \zdv(y, h_1, \ldots, h_q)}\,.\\
	\frkProb(y) & = \condprob{b = 1}{(b, \vec{s}) \gets
\genforking_{\zdv}^{m}(y, h_1, \ldots, h_q)}\,.
\end{align*}

We start by claiming that for all $y$ 
\begin{equation}\label{eq:frkProb_y}
	\frkProb(y) \geq 
	\frac{\accProb(y)^m}{q^{m - 1}} - \accProb(y) \cdot \left(1 -
  \frac{h!}{(h - m)! \cdot h^{m}}\right)
	\end{equation}
Then with the expectation taken over $y \sample \ig$, we have
\begin{align}
	\frkProb & = \expected{\frkProb(y)} \geq
	\expected{\frac{\accProb(y)^m}{q^{m - 1}} -  \accProb(y) \cdot \left(1 -
  \frac{h!}{(h - m)! \cdot h^{m}}\right)} \label{eq:use_eq1}\\
	& \geq \frac{\expected{\accProb(y)}^m}{q^{m - 1}} -
	\expected{\accProb(y)} \cdot \left(1 - \frac{h!}{(h - m)! \cdot
  h^{m}}\right) \label{eq:by_lemma_jensen}\\
	& = \frac{\accProb^m}{q^{m - 1}} -  \accProb \cdot \left(1 -
  \frac{h!}{(h - m)! \cdot h^{m}}\right)\label{eq:by_accProb}\,.
\end{align}
Where \cref{eq:use_eq1} comes from \cref{eq:frkProb_y};
\cref{eq:by_lemma_jensen} comes from linearity of expected value and \cref{lem:jensen}; and
\cref{eq:by_accProb} holds by the fact that $\expected{\accProb(y)} =
\accProb$.

We now show \cref{eq:frkProb_y}.
Denote by $J = \range{1}{m}^2 \setminus \smallset{(j, j)}_{j \in \range{1}{m}}$. 
For any input $y$, with probabilities taken over the coin tosses of
$\genforking_{\zdv}^{m}$ we have
\begin{align*}
	\frkProb (y) & = \prob{i_j = i_{j'} \land i_j \geq 1 \land
h_{i_j}^{j} \neq h_{i_{j'}}^{j'} \text{ for } (j, j') \in J}	\\
	& \geq \prob{i_j = i_{j'} \land i_j \geq 1 \text{ for } (j, j') \in J} %\\
   - \prob{i_j \geq 1 \land h_{i_j}^{j} = h_{i_{j'}}^{j'} \text{ for some } (j, j') \in J}\\
	& = \prob{i_j = i_{j'} \land i_j \geq 1 \text{ for } (j, j') \in J} -
	\prob{i_j \geq 1} \cdot 
  \left(1 - \frac{h!}{(h - m)! \cdot h^{m}}\right) \\ 
	& = \prob{i_j = i_{j'} \land
	i_j \geq 1 \text{ for } (j, j') \in J} - \accProb(y) \cdot \left(1 -
\frac{h!}{(h - m)! \cdot h^{m}}\right)\,.
\end{align*}

Probability that for some $(j, j') \in J$ and $i_j = i_{j'}$ holds
$h_{i_j}^{j} \neq h_{i_{j'}}^{j'}$ equals 
\[
  \frac{h \cdot (h - 1)
\cdot \ldots \cdot (h - m - 1)}{h^m} = \frac{h!}{(h - m)! \cdot h^m}.
\]
That is, it equals the number
of all $m$-element strings where each element is different divided by
the number of all $m$-element strings, where elements are taken from a
set of size $h$. 

It remains to show that $\prob{i_j = i_{j'} \land i_j \geq 1 \text{ for } (j,
  j') \in J} \geq \infrac{\accProb(y)^m}{q^{m - 1}}$. Let $\RND{\zdv}$ denote
the set from which $\zdv$ picks its coins at random. For each $\iota \in
\range{1}{q}$ let $X_\iota \colon \RND{\zdv} \times H^{\iota - 1} \to [0, 1]$ be
defined by setting $X_\iota(\rho, h_1, \ldots, h_{\iota - 1})$ to
\[
  \condprob{i = \iota}{h_\iota, \ldots, h_q \sample H; (i, s) \gets \zdv(y, h_1,
    \ldots, h_q; \rho)}
\]
for all $\rho \in \RND{\zdv}$ and $h_1, \ldots, h_{\iota - 1} \in H$. Consider
$X_\iota$ be a random variable over the uniform distribution on its domain. Then
\begin{align*}
	& \prob{i_j = i_{j'} \land i_j \geq 1 \text{ for } (j, j') \in J} 
	 = \sum_{\iota = 1}^{q} \prob{i_1 = \iota \land \ldots \land i_m = \iota} \\
	& = \sum_{\iota = 1}^{q} \prob{i_1 = \iota} \cdot \condprob{i_2 = \iota}{i_1 = \iota} \cdot \ldots \cdot \condprob{i_m = \iota}{i_1 = \ldots = i_{m - 1} = \iota} \\
	& = \sum_{\iota = 1}^{q} \sum_{\rho, h_1, \ldots, h_{\iota - 1}} X_{\iota}
   (\rho, h_1, \ldots, h_{\iota - 1})^{m} \cdot \frac{1}{\abs{\RND{\zdv}} \cdot \abs{H}^{\iota - 1}}
   = \sum_{\iota = 1}^{q} \expected{X_\iota^m} \,.
\end{align*}
Importantly, $\sum_{\iota = 1}^q \expected{X_{\iota}} = \accProb(y)$.

By \cref{lem:jensen} we get
\[
	\sum_{\iota = 1}^{q} \expected{X_\iota^m} \geq \sum_{\iota = 1}^{q} \expected{X_\iota}^m\,.
\]
Note that for e.g.~$X_i = 1$, $i \in \range{1}{q}$ the inequality becomes equality, that is, it is tight.
 
We now use the H\"older inequality, cf.~\cref{lem:holder}, for $x_i = \expected{X_i}$, $y_i = 1$, $p = m$, and $q = m/(m - 1)$ obtaining
\begin{gather}
	\left(\sum_{i = 1}^{q} \expected{X_i}\right)^{m}  \leq \left(\sum_{i = 1}^{q} \expected{X_i}^m\right) \cdot q^{m - 1}\\
	\frac{1}{q^{m - 1}} \cdot \accProb(y)^{m} \leq \sum_{i = 1}^{q} \expected{X_i}^m\,.
\end{gather}
Finally, we get
\[
	\frkProb(y) \geq \frac{\accProb(y)^m}{q^{m - 1}} - 
	 \accProb(y) \cdot \left(1 - \frac{h!}{(h - m)! \cdot h^m}\right)\,.
\]
\qed
\end{proof}
\begin{lemma}\label{lem:jensen}
	Let $\RND{\zdv}$ denote the set from which $\zdv$ picks its coins at random.
	For each $\iota \in \range{1}{q}$ let $X_\iota \colon \RND{\zdv} \times
	H^{\iota - 1} \to [0, 1]$ be defined by setting $X_\iota(\rho, h_1, \ldots,
h_{\iota - 1})$ to 
\[
  \condprob{i = \iota}{h_\iota, \ldots, h_q \sample H; (i, s) \gets \zdv(y, h_1,
    \ldots, h_q; \rho)}
	\] 
	for all $\rho \in \RND{\zdv}$ and $h_1, \ldots, h_{\iota - 1} \in H$. Consider
  $X_\iota$ as a random variable over the uniform distribution on its domain.
  Then $\expected{X_\iota^m} \geq \expected{X_\iota}^m$.
\end{lemma}
\begin{proof}
	First we recall the Jensen inequality \cite{W:Weissten20}, if for some random
  variable $X$ holds $\abs{\expected{X}} \leq \infty$ and $f$ is a Borel convex
  function then
	\[
		f(\expected{X}) \leq \expected{f(X)}\,.
	\] 
	Finally, we note that $\abs{\expected{X}} \leq \infty$ and taking to the
  $m$-th power is a Borel convex function on $[0, 1]$ interval. \qed
\end{proof}

\begin{lemma}[H\"older's inequality. Simplified.]\label{lem:holder}
	Let $x_i, y_i$, for $i \in \range{1}{q}$, and $p, q$ be real numbers such that
  $1/p + 1/q = 1$. Then
	\begin{equation}
    \label{eq:tightness}
		\sum_{i = 1}^{q} x_i y_i \leq \left(\sum_{i = 1}^{q}
      x_i^p\right)^{\frac{1}{p}} \cdot \left(\sum_{i = 1}^{q}
      y_i^p\right)^{\frac{1}{q}}\,.
	\end{equation}
\end{lemma}

\begin{remark}[Tightness of the H\"older inequality]
	In is important to note that Inequality (\ref{eq:tightness}) is tight. More
  precisely, for $\expected{X_i} = x$, $i \in \range{1}{q}$ we have
	\begin{gather*}
		\sum_{i = 1}^q x = \left(\sum_{i = 1}^{q} x^m\right)^\frac{1}{m} \cdot \left(\sum_{i = 1}^{q} 1^{\frac{m}{m - 1}}\right)^{\frac{m - 1}{m}} \\
		qx = \left(qx^m\right)^\frac{1}{m} \cdot q^{\frac{m - 1}{m}} \\
		(qx)^m = qx^m \cdot q^{m - 1} \\
		(qx)^m = (qx)^m\,.
	\end{gather*}
\end{remark}

\begin{lemma}
  \label{lem:root_prob}
  Let $\p{f}(X)$ be a random degree-$d$ polynomial over $\FF_p[X]$. Then the
  probability that $\p{f}(X)$ has roots in $\FF_p$ is at least $\infrac{1}{d!}$.
\end{lemma}
\begin{proof}
  First observe that there is $p^{d}$ canonical polynomials in $\FF_p[X]$.  Each
  of the polynomials may have up to $d$ roots. Consider polynomials which are
  reducible to polynomials of degree $1$, i.e.~polynomials that have all $d$
  roots. The roots can be picked in $\bar{C}^{p}_{d}$ ways, where
  $\bar{C}^{n}_{k}$ is the number of $k$-elements combinations with repetitions
  from $n$-element set. That is,
  \[
    \bar{C}^n_k = \binom{n + k - 1}{k}\,.
  \]
  Thus, the probability that a randomly picked polynomial has all $d$ roots is
  \begin{multline*}
    p^{-d} \cdot \bar{C}^p_d = p^{-d} \cdot \binom{p + d - 1}{d} =
    p^{-d} \cdot \frac{(p + d - 1)!}{(p + d - 1 - d)! \cdot d!} = \\
    p^{-d} \cdot \frac{(p + d - 1) \cdot \ldots \cdot p \cdot (p - 1)!}{(p - 1)!
      \cdot d!} = p^{-d} \cdot \frac{(p + d - 1)\cdot
      \ldots \cdot p}{d!}
    \geq p^{-d} \cdot {\frac{p^d}{d!}} = \frac{1}{d!}
  \end{multline*}
  \qed
\end{proof}


\oursubsub{Forking special soundness}
Note that the special soundness property (as usually defined) holds for
all---even computationally unbounded---adversaries. Unfortunately, since a
simulation trapdoors for $\plonkprot$ and $\sonicprot$ exist, the protocols
cannot be special sound in that regard. This is because an unbounded adversary
can recover the trapdoor and build an arbitrary number of simulated proofs for a fake
statement. Hence, we provide a weaker, yet sufficient, definition of
\emph{forking special soundness}. More precisely, we state that an
adversary that is able to answer correctly multiple challenges either knows the
witness or can be used to break some computational assumption.
%\chaya{a notion of computational special soundness has been used to mean exactly the above in prior works, like BBF19. we should clarify if forking soundness different from computational special soundness? seems to me like the difference is just that here it is tailored for the NI version.}
%\chaya{I now see that the diff is the access to the simulator that the adversary gets. might be helpful to clarify this, either here or in a tech overview section. I am not sure why this needs to be defined this way by giving access to the simulator.}
However, differently from the standard definition of special soundness, we do
not require from the extractor to be able to extract the witness from \emph{any}
tree of acceptable transcripts. We require that the tree be produced honestly,
that is, all challenges are picked randomly --- exactly as an honest verifier would pick.
Intuitively, the tree is as it would be generated by a $\genforking$
algorithm from the generalized forking lemma.

% \begin{definition}[Forking soundness]
%   % Let $\proofsystem = (\kgen, \prover, \verifier, \simulator)$ be an
%   % $(2 \mu + 1)$-message proof system for a relation $\REL$. Let $\tree$ be an
%   % $(n_1, \ldots, n_\mu)$-tree of acceptable transcripts output by a $\ppt$ tree
%   % building algorithm $\tdv$ which plays the role of the verifier $\ps.\verifier$
%   % against $\ppt$ adversary $\adv$ which it interacts with.
% %
%   % We say that $\proofsystem$ is
%   % $(\epsss, (n_1, \ldots, n_\mu))$-\emph{computationally special sound} if there
%   % exists an extractor $\extt$ that given an $ (n_1, \ldots, n_\mu)$-tree of
%   % acceptable transcripts $\tree$ for an instance $\inp \in \LANG_\REL$ output by
%   % a $\ppt$ tree building adversary $\tdv$ some $\ppt$ adversary
%   % $\adv(\srs)$, for $\srs \sample \kgen(\REL)$, outputs $\wit$ such that
%   % $\REL(\inp, \wit) = 1$ with probability at least $1 - \epsss$.
%   Let $\proofsystem = (\kgen, \prover, \verifier, \simulator)$ be an
%   $(2 \mu + 1)$-message proof system for a relation $\REL$. Let $\tree$ be an
%   $(n_1, \ldots, n_\mu)$-tree of transcripts for an in valid instance
%   $\inp$. We say that $\ps$ is $(\epsss, (n_1, \ldots, n_\mu))$-forking sound if there is an extractor $\extt$ that given $\tree$ extracts $\wit$
%   such that $\REL(\inp, \wit) = 1$ with probability at least $1 - \epsss.$
% \end{definition}

% \begin{definition}[$k$-round forking soundness]
%   % Let $\proofsystem = (\kgen, \prover, \verifier, \simulator)$ be an
%   % $(2 \mu + 1)$-message proof system for a relation $\REL$. Let $\tree$ be an
%   % $(n_1, \ldots, n_\mu)$-tree of acceptable transcripts output by a $\ppt$ tree
%   % building algorithm $\tdv$ which plays the role of the verifier $\ps.\verifier$
%   % against $\ppt$ adversary $\adv$ which it interacts with.
% %
%   % We say that $\proofsystem$ is
%   % $(\epsss, (n_1, \ldots, n_\mu))$-\emph{computationally special sound} if there
%   % exists an extractor $\extt$ that given an $ (n_1, \ldots, n_\mu)$-tree of
%   % acceptable transcripts $\tree$ for an instance $\inp \in \LANG_\REL$ output by
%   % a $\ppt$ tree building adversary $\tdv$ some $\ppt$ adversary
%   % $\adv(\srs)$, for $\srs \sample \kgen(\REL)$, outputs $\wit$ such that
%   % $\REL(\inp, \wit) = 1$ with probability at least $1 - \epsss$.
%   Let $\proofsystem = (\kgen, \prover, \verifier, \simulator)$ be an
%   $(2 \mu + 1)$-message proof system for a relation $\REL$. Let $\tree$ be the
%   algorithm below that rewinds $\advse^{\simulator_\fs,
%           \ro} (\srs; r)$ to produce a $(1, n_k, 1)$-tree of
%   transcripts such that none of the challenges in round $k$ were used in
%   simulated proofs. We say that $\ps$ is $(\epsss, ((1, n_k, 1)))$-forking
%   special sound if there is an extractor $\extt$ that given a tree produced by
%   $\tree$ extracts $\wit$ such that $\REL(\inp, \wit) = 1$ with probability at
%   least $1 - \epsss.$
% \end{definition}

% Since we do not utilise the classical special soundness (that holds for all,
% even unbounded, adversaries) all references to that property should be
% understood as references to its computational version.



%\begin{definition}[$(\eps(\secpar), k,n)$-forking soundness]
%  Let $\proofsystem = (\kgen, \prover, \verifier, \simulator)$ be an
%  $(2 \mu + 1)$-message proof system for a relation $\REL$.
%
%For any $\ppt$ adversary $\advse^{\simulator_\fs,
%  \ro} (\srs; r)$ we consider the procedure $\zdv$ that provided the transcript $(\srs, \adv, r, Q, Q_{H})$ and $h_1, \ldots, h_q$ runs $\adv$ by providing it with random oracle queries
%and simulated proofs. While $Q_{H}$ is consistent with $h_1, \ldots, h_q$, it replays the proofs of $Q$.
%%
%$\zdv$ returns the index $i$ of the
%  random oracle query made for challenge $k$ and the proof $\adv$ returns.
%
%  Consider the algorithm $\genforking_{\zdv}^{n}$
%  that rewinds $\zdv$ to produce a $(1,\dots, n,\dots, 1)$-tree of
%  transcripts such that none of the $n$ challenges in round $k$ were used in
%  simulated proofs.
%
%  We say that $\ps$ is $(\eps(\secpar), k,n)$-forking sound if
%  for any PPT adversary the probability that
%  \begin{align*}
%    \Pr\left[
%    \REL(\inp, \wit) = 0
%    \,\Biggl|\,
%    \begin{aligned}
%       & \srs \sample \kgen(\REL),
%        r \sample \RND{\advse},
%         (\inp_{\advse}, \zkproof_{\advse}) \gets \advse^{\simulator_\fs,\ro} (\srs; r), \\
%       &    (1, \tree) \gets \genforking_{\zdv}^{m}((\srs,\adv,r,Q, Q_{H}),Q_{H}),
%           \wit \gets \extt(\tree)
%    \end{aligned}
%    \right] \leq \eps(\secpar).
%  \end{align*}
%   List $Q$ contains all $(\inp, \zkproof)$ pairs where
%  $\inp$ is an instance provided to the simulator by the adversary and
%  $\zkproof$ is the simulator's answer. List $Q_\ro$ contains all $\advse$'s
%  queries to $\ro$ and $\ro$'s answers.
%\end{definition}

%\hamid{The definition in the updatable setting should look like this:}

\begin{definition}[$(k,n)$-forking special soundness in the updatable setting]
	Let $\proofsystem = (\kgen, \prover, \verifier, \simulator)$ be an
	$(2 \mu + 1)$-message proof system for a relation $\REL$.  Let $\proofsystem_\fs$ be
	$\proofsystem$ after the Fiat--Shamir transform.
	
	For any $\ppt$ adversary $\advse^{\ro, \initU} (1^\secpar; r)$ with access to oracles $\initU$ defined in~\cref{fig:upd}, and random oracle $\ro$, we consider the procedure $\zdv$ that provided the transcript $(\srs, \adv, r)$ and $h_1, \ldots, h_q$ runs $\adv$ by providing it with random oracle queries
	and update oracle queries.
	%
	$\zdv$ returns the index $i$ of the
	random oracle query made for challenge $k$ and the proof $\adv$ returns.
	
	Consider the algorithm $\genforking_{\zdv}^{n}$
	that rewinds $\zdv$ to produce a $(1,\dots, n, \dots, 1)$-tree of
	accepting transcripts.
	
	We say that $\psfs$ is $(k,n)$-forking special sound with security loss $\eps(\secpar)$, if
	for any PPT adversary the probability that
	\begin{align*}
	\Pr\left[
	\REL(\inp, \wit) = 0
	\,\Biggl|\,
	\begin{aligned}
	& r \sample \RND{\advse},
	(\inp_{\advse}, \zkproof_{\advse}) \gets \advse^{\ro, \initU} (1^\secpar; r), \\
	&    (1, \tree) \gets \genforking_{\zdv}^{m}((\srs,\adv,r),Q_{H}),
	\wit \gets \extt(\tree)
	\end{aligned}
	\right] \leq \eps(\secpar).
	\end{align*}
	Here, $\srs$ is the SRS that $\advse$ finalised using the update oracle $\initU$.
	List $Q_\ro$ contains all $\advse$'s
	queries to $\ro$ and the random oracle's answers.
\end{definition}


\paragraph{Importance of the general forking lemma.}
To highlight the importance of the generalised forking lemma, we outline 
how it is used in our \COMMENT{forking} simulation-extractability proof.  Let $\psfs$ be a
forking special sound proof system where for an instance $\inp$ the
corresponding witness can be extracted from a
$(1, \ldots, 1, n_k, 1, \ldots, 1)$-tree of accepting transcripts.  Let $\advse$
be the simulation-extractability adversary that outputs an accepting proof with
probability at least $\accProb$. From $\adv$ we build an adversary $\adv$ that runs $\simO$ inside. Although we use the same $\accProb$ to denote
probability of $\zdv$ outputing a non-zero $i$ and the probability of $\advse$ (and $\adv$)
outputing an accepting proof, we claim that these probabilities are exactly the
same by the way we define $\zdv$ and $\bdv$. In the following we give the intuition of the proof without the additional indirection via $\bdv$ whose goal is primarily to simplify the forking special soundness definition.

Let $\adv$ output an accepting instance-proof pair
$(\inp_{\advse},\zkproof_{\advse})$; $r$ be $\adv$'s
randomness; \COMMENT{$Q$ the list of simulator queries made by $\advse$ \markulf{20/09/2021}{Q is now hidden within the forking special soundness adversary.} along with
$\simulator$'s answers; }%\hamid{$Q_\srs$ be the list of proofs for SRS honest updates;} 
and $Q_\ro$ be the list of all random oracle
queries made by $\adv$.  All of these are given to the extractor $\ext$ that
internally runs the forking algorithm $\genforking_\zdv^{n_k}$.  Algorithm $\zdv$
takes $(\srs, \advse, \COMMENT{Q, }r)$ as input $y$ and $Q_\ro$ as input $h_1^1, \ldots,
h_q^1$. 
(For the sake of completeness, we allow $\genforking_\zdv^{n_k}$ to
pick $h^1_{l + 1}, \ldots, h^1_q$ responses if $Q_\ro$ has only $l < q$
elements.)  

Next, $\zdv$ internally runs $\adv(\srs; r)$ %\hamid{$\\advse(1^\secpar; r)$}
and responds to its random
oracle queries using $Q_\ro$\COMMENT{ and $Q$}. Note that $\adv$ makes
the same queries as it did before it output $(\inp_{\advse}, \zkproof_{\advse})$
as it is run on the same random tape and with the same answers from the
simulator and random oracle. Once $\adv$ outputs
$\zkproof_{\advse}$, algorithm $\zdv$ outputs $(i, \zkproof_{\advse})$, where
$i$ is the index of a random oracle query submitted by $\advse$ to receive the challenge after the
$k$-th message from the prover---a message where the tree of transcripts
branches.
Then, after the first run of $\advse$ is done, the extractor runs $\zdv$ again,
but this time it provides fresh random oracle responses $h^2_i, \ldots,
h^2_q$. Note that this is equivalent to rewinding $\advse$ to a point just
before $\advse$ is about to ask its $i$-th random oracle
query. The probability that the adversary produces an accepting transcript with the
fresh random oracle responses is at least $\accProb$. This continues until the
required number of transcripts is obtained. 

We note that in the original forking lemma, the forking algorithm $\forking$,
cf.~\cite{CCS:BelNev06}, gets as input only $y$ and elements $h^1_1, \ldots,
h^1_q$ are randomly picked from $H$ internally by $\forking$. However, assuming
that $h^1_1, \ldots, h^1_q$ are random oracle responses, and thus random, makes
the change only notational.

We also note that the general forking lemma from
\cref{lem:generalised_forking_lemma} works for protocols with an extractor that can obtain the
witness from a $(1, \ldots, 1, n_k, 1, \ldots, 1)$-tree of accepting
transcripts. This limitation however does not affect the main result of this
paper, i.e.~showing that both $\plonk$, $\sonic$, and $\marlin$ are \COMMENT{forking }simulation extractable.

%\hamid{next definition should be commented!}
%\begin{definition}[$(\eps(\secpar), k,n)$-forking soundness]
%  Let $\proofsystem = (\kgen, \prover, \verifier, \simulator)$ be an
%  $(2 \mu + 1)$-message proof system for a relation $\REL$.
%
%  Let $\tdv$, called tree creator, be the
%  algorithm below that rewinds the $\ppt$ adversary $\advse^{\simulator_\fs,
%          \ro} (\srs; r)$ to produce a $(1,\dots, n,\dots, 1)$-tree of
%  transcripts such that none of the $n$ challenges in round $k$ were used in
%  simulated proofs.
%
%  $\tdv$ has oracle access to $\adv$ and provides it with (oracle) access to
%  random oracle $\ro$ and simulator $\simulator_\fs$ -- more precisely $\tdv$
%  has an internal procedure $\bdv$ that provided $\srs$ and random oracle
%  queries' responses $h_1, \ldots, h_q$ gives $\adv$ access to the random oracle
%  and simulates proof for it. In the end, $\bdv$ returns the index $i$ of the
%  random oracle query made for challenge $k$, the set $Q$ of simulator random
%  oracle indexes, the instance $\inp$, and the proof $\adv$ returns. Eventually,
%  $\tdv$ returns a $(1, \ldots, n, \dots, 1)$ tree of acceptable
%  transcripts~$\tree$.
%
%
%  \begin{figure}
%    \centering
% 		\fbox{ \procedure{$\tdv(\adv, \srs \sample \kgen(\REL))$}
%      {h_1^{1}, \ldots,
%      h_{q}^1 \sample H \\
%      (i, Q, \inp, \zkproof_1) \gets \bdv(\adv, \srs, h_1^{1}, \ldots, h_{q}^{1})\\
%        % i_1 \gets i\\
%        \pcif i\in Q \lor \verifier(\srs, \inp, \zkproof_1) = 0\ \pcreturn (0, \bot)\\
%        \pcfor j \in \range{2}{m}\\
%        \pcind h_{1}^{j}, \ldots, h_{i - 1}^{j} \gets h_{1}^{j - 1}, \ldots,
%        h_{i - 1}^{j - 1}\\
%        \pcind h_{i}^{j}, \ldots, h_{q}^{j} \sample H\\
%        \pcind (i_j, Q_j, \inp_j, \zkproof_j) \gets \bdv(\adv, \srs, h_1^{j}, \ldots, h_{i -
%          1}^{j}, h_{i}^{j},
%        \ldots, h_{q}^{j})\\
%        \pcind \pcif i \neq i_j \lor i_j \in Q_j \lor \inp \neq \inp_j \lor \verifier(\srs, \inp_j, \zkproof_i) = 0\ \pcreturn (0, \bot)\\
%        %\pcind \pcif i_j = 0 \lor i_j \neq i\ \pcreturn (0, \bot)\\
%        % \pcif \exists (j, j') \in \range{1}{m}^2, j \neq j' : (h_{i}^{j} =
%        % h_{i}^{j'})\
%        % 
%        % pcreturn (0, \bot)\\
%        \pcelse \pcreturn (1, \tree = (\inp, \pmb{\pi}))}
%    }
%  \end{figure}
%%
%  We say that $\ps$ is $(\eps(\secpar), k,n)$-forking sound if
%  for any PPT adversary the probability that
%  \[
%    \Pr\left[
%    %  \begin{aligned}
%       % & \forall_{\zkproof \in \tree} \verifier(\srs, \inp, \zkproof) = 1, \\
%         \wit \gets \extt(\tree), 
%         \REL(\inp, \wit) = 0
%    %  \end{aligned}
%      \,\left|\,
%        %\begin{aligned}
%           \srs \sample \kgen(\REL), 
%           (1, \tree) \gets \tdv(\adv, \srs)
%     %   \end{aligned}
%      \right.\right] \leq \eps(\secpar).
%      \]
%\end{definition}


%%% Local Variables:
%%% mode: latex
%%% TeX-master: "main"
%%% End:
