% !TEX root = main.tex
% !TEX spellcheck = en-US

\section{Non-malleability of $\sonicprotfs$}
\label{sec:sonic}
\subsection{\sonic{} protocol rolled out}
In this section we present $\sonic$'s constraint system and algorithms. Reader
familiar with them may jump directly to the next section.

\hamid{we should put the following figure \ref{fig:pcoms} somewhere in this section.}
 \begin{figure}[h!]
 \centering
 	\begin{pcvstack}[center,boxed]
 		\begin{pchstack}
 			\procedure{$\kgen(\secparam, \maxdeg)$} {
 				\alpha, \chi \sample \FF^2_p \\ [\myskip]
 				\pcreturn \gone{\smallset{\chi^i}_{i = -\multconstr}^{\multconstr},
           \smallset{\alpha \chi^i}_{i = -\multconstr, i \neq
             0}^{\multconstr}},\\
         \pcind \gtwo{\smallset{\chi^i, \alpha \chi^i}_{i =
             -\multconstr}^{\multconstr}}, \gtar{\alpha}\\
 				%\markulf{03.11.2020}{} \\
 			%	\hphantom{\pcind \p{o}_i(X) \gets \sum_{j = 1}^{t_i} \gamma_i^{j - 1} \frac{\p{f}_{i,j}(X) - \p{f}_{i, j}(z_i)}{X - z_i}}
 				\hphantom{\hspace*{5.5cm}}
 		}

 			\pchspace

 			\procedure{$\com(\srs, \maxconst, \p{f}(X))$} {
 				\p{c}(X) \gets \alpha \cdot X^{\dconst - \maxconst} \p{f}(X) \\ [\myskip]
 				\pcreturn \gone{c} = \gone{\p{c}(\chi)}\\ [\myskip]
 				\hphantom{\pcind \pcif \sum_{i = 1}^{\abs{\vec{z}}} r_i \cdot
           \gone{\sum_{j = 1}^{t_j} \gamma_i^{j - 1} c_{i, j} - \sum_{j = 1}^{t_j}
             s_{i, j}} \bullet \gtwo{1} + } }
 		\end{pchstack}
 		% \pcvspace

 		\begin{pchstack}
 			\procedure{$\open(\srs, z, s, f(X))$}
 			{
 				\p{o}(X) \gets \frac{\p{f}(X) - \p{f}(z)}{X - z}\\ [\myskip]
 				\pcreturn \gone{\p{o}(\chi)}\\ [\myskip]
 				\hphantom{\hspace*{5.5cm}}
 			}

 			\pchspace

 			\procedure{$\verify(\srs, \maxconst, \gone{c}, z, s, \gone{\p{o}(\chi)})$}
       {
         \pcif \gone{\p{o}(\chi)} \bullet \gtwo{\alpha \chi} + \gone{s - z
         \p{o}(\chi)} \bullet \gtwo{\alpha} = \\ [\myskip] \pcind \gone{c}
         \bullet \gtwo{\chi^{- \dconst + \maxconst}} \pcthen  \pcreturn 1\\
         [\myskip]
         \rlap{\pcelse \pcreturn 0.} \hphantom{\pcind \pcif \sum_{i =
             1}^{\abs{\vec{z}}} r_i \cdot \gone{\sum_{j = 1}^{t_j} \gamma_i^{j -
               1} c_{i, j} - \sum{j = 1}^{t_j} s_{i, j}} \bullet \gtwo{1} + } }
 		\end{pchstack}
 	\end{pcvstack}

 	\caption{$\PCOMs$ polynomial commitment scheme.}
 	\label{fig:pcoms}
 \end{figure}



\oursubsub{The constraint system}
\label{sec:sonic_constraint_system}
\sonic's system of constraints composes of three $\multconstr$-long vectors
$\va, \vb, \vc$ which corresponds to left and right inputs to multiplication
gates and their outputs. It hence holds $\va \cdot \vb = \vc$.

There is also $\linconstr$ linear constrains of the form
\[
  \va \vec{u_q} + \vb \vec{v_q} + \vc \vec{w_q} = k_q,
\]
where $\vec{u_q}, \vec{v_q}, \vec{w_q}$ are vectors for the $q$-th linear
constraint with instance value $k_q \in \FF_p$. Furthermore define polynomials
\begin{equation}
  \begin{split}
    \p{u_i}(Y) & = \sum_{q = 1}^\linconstr Y^{q + \multconstr} u_{q, i}\,,\\
    \p{v_i}(Y) & = \sum_{q = 1}^\linconstr Y^{q + \multconstr} v_{q, i}\,,\\
  \end{split}
  \qquad
  \begin{split}
    \p{w_i}(Y) & = -Y^i - Y^{-i} + \sum_{q = 1}^\linconstr Y^{q +
      \multconstr} w_{q, i}\,,\\
    \p{k}(Y) & = \sum_{q = 1}^\linconstr Y^{q + \multconstr} k_{q}.
  \end{split}
\end{equation}

$\sonic$ constraint system requires that
\begin{align}
  \label{eq:sonic_constraint}
  \vec{a}^\top \cdot \vec{\p{u}} (Y) + \vec{b}^\top \cdot \vec{\p{v}} (Y) +
  \vec{c}^\top \cdot \vec{\p{w}} (Y) + \sum_{i = 1}^{\multconstr} a_i b_i (Y^i +
  Y^{-i}) - \p{k} (Y) = 0.
\end{align}

In \sonic{} we will use commitments to the following polynomials.
\begin{align*}
  \pr(X, Y) & = \sum_{i = 1}^{\multconstr} \left(a_i X^i Y^i + b_i X^{-i} Y^{-i}
              + c_i X^{-i - \multconstr} Y^{-i - \multconstr}\right) \\
  \p{s}(X, Y) & = \sum_{i = 1}^{\multconstr} \left( u_i (Y) X^{-i} +
                v_i(Y) X^i + w_i(Y) X^{i + \multconstr}\right)\\
  \pt(X, Y) & = \pr(X, 1) (\pr(X, Y) + \p{s}(X, Y)) - \p{k}(Y)\,.
\end{align*}

Polynomials $\p{r} (X, Y), \p{s} (X, Y), \p{t} (X, Y)$ are designed such that
$\p{t} (0, Y) = \vec{a}^\top \cdot \vec{\p{u}} (Y) + \vec{b}^\top \cdot
\vec{\p{v}} (Y) + \vec{c}^\top \cdot \vec{\p{w}} (Y) + \sum_{i =
  1}^{\multconstr} a_i b_i (Y^i + Y^{-i}) - \p{k} (Y) $. That is, the prover is
asked to show that $\p{t} (0, Y) = 0$, cf.~\cref{eq:sonic_constraint}.

Furthermore, the commitment system in $\sonic$ is designed such that it is
infeasible for a $\ppt$ algorithm to commit to a polynomial with non-zero
constant term.

\oursubsub{Algorithms rolled out}
\ourpar{$\sonic$ SRS generation $\kgen(\REL)$.} The SRS generating algorithm picks
randomly $\alpha, \chi \sample \FF_p$ and outputs
	\[
      \srs = \left( \gone{\smallset{\chi^i}_{i = -\dconst}^{\dconst},
          \smallset{\alpha \chi^i}_{i = -\dconst, i \neq 0}^{\dconst}},
        \gtwo{\smallset{\chi^i, \alpha \chi^i}_{i = - \dconst}^{\dconst}},
        \gtar{\alpha} \right)
	\]
\ourpar{$\sonic$ prover $\prover(\srs, \inp, \wit=\va, \vb, \vc)$.}
\begin{description}
\item[Round 1] The prover picks randomly randomisers
  $c_{\multconstr + 1}, c_{\multconstr + 2}, c_{\multconstr + 3}, c_{\multconstr
    + 4} \sample \FF_p$. Sets
  $\pr(X, Y) \gets \pr(X, Y) + \sum_{i = 1}^4 c_{\multconstr + i} X^{- 2
    \multconstr - i}$. Commits to $\pr(X, 1)$ and outputs
  $\gone{r} \gets \com(\srs, \multconstr, \pr(X, 1))$.  Then it gets challenge $y$ from
  the verifier.
\item[Round 2] $\prover$ commits to $\pt(X, y)$ and outputs
  $\gone{t} \gets \com(\srs, \dconst, \pt(X, y))$. Then it gets a challenge $z$ from
  the verifier.
\item[Round 3] The prover computes commitment openings. That is, it outputs
  \begin{align*}
    \gone{o_a} & = \open(\srs, z, \pr(z, 1), \pr(X, 1)) \\
    \gone{o_b} & = \open(\srs, yz, \pr(yz, 1), \pr(X, 1)) \\
    \gone{o_t} & = \open(\srs, z, \pt(z, y), \pt(X, y)) 
  \end{align*}
  along with evaluations $a' = \pr(z, 1), b' = \pr(y, z), t' = \pt(z, y)$.  Then it
  engages in the signature of correct computation playing the role of the
  helper, i.e.~it commits to $\p{s}(X, y)$ and sends the commitment $\gone{s}$, commitment opening
  \begin{align*}
    \gone{o_s} & = \open(\srs, z, \p{s}(z, y), \p{s}(X, y)), \\
  \end{align*} and $s'=\p{s}(z, y)$. 
%
  Then
  it obtains a challenge $u$ from the verifier.
\item[Round 4] In the next round the prover computes
  $\gone{c} \gets \com(\srs, \dconst, \p{s}(u, Y))$ and
  computes commitments' openings
  \begin{align*}
    \gone{w} & = \open(\srs, u, \p{s}(u, y), \p{s}(X, y)), \\
    \gone{q_y} & = \open(\srs, y,\p{s}(u, y), \p{s}(u, Y)),
  \end{align*}
  and returns $\gone{w}, \gone{q_y}, s = \p{s}(u, y)$. Eventually the prover gets the last challenge
  from the verifier---$z'$.
\item[Round 5] In the final round, $\prover$ computes opening
  $\gone{q_{z'}} = \open(\srs, z', \p{s}(u, z'), \p{s}(u, X))$ and outputs $\gone{q_{z'}}$.
\end{description}

\ourpar{$\sonic$ verifier $\verifier(\srs, \inp, \zkproof)$.} The verifier
in \sonic{} runs as subroutines the verifier for the polynomial commitment. That
is it sets $t' = a'(b' + s') - \p{k}(y)$ and checks the following:
\begin{equation*}
  \begin{split}
    &\PCOMs.\verifier(\srs, \multconstr, \gone{r}, z, a', \gone{o_a}), \\
    &\PCOMs.\verifier(\srs, \multconstr, \gone{r}, yz, b', \gone{o_b}),\\
    &\PCOMs.\verifier(\srs, \dconst, \gone{t}, z, t', \gone{o_t}),\\
    &\PCOMs.\verifier(\srs, \dconst, \gone{s}, z, s', \gone{o_s}),\\
  \end{split}
  \qquad
  \begin{split}
    &\PCOMs.\verifier(\srs, \dconst, \gone{s}, u, s, \gone{w}),\\
    &\PCOMs.\verifier(\srs, \dconst, \gone{c}, y, s, \gone{q_y}),\\
    &\PCOMs.\verifier(\srs, \dconst, \gone{c}, z', \p{s}(u, z'), \gone{q_{z'}}),
  \end{split}
\end{equation*}
and accepts the proof iff all the checks holds. Note that the value
$\p{s}(u, z')$ that is recomputed by the verifier uses separate challenges $u$
and $z'$. This enables the batching of many proof and outsourcing of this
part of the proof to an untrusted helper.

\subsection{Unique opening property of $\PCOMs$}
\begin{lemma}
\label{lem:pcoms_unique_op}
$\PCOMs$ has the unique opening property in the AGM. 
\end{lemma}
\begin{proof}
Let 
$z \in \FF_p$ be the attribute the polynomial is evaluated at,
$\gone{c} \in \GRP$ be the commitment,  
$s \in \FF_p$ the evaluation value, and 
$o \in \GRP$ be the commitment opening. 
We need to show that for every $\ppt$ adversary $\adv$ probability
\[
  \Pr \left[
    \begin{aligned}
      & \verify(\srs, \gone{c}, z, s, \gone{o}) = 1, \\
      & \verify(\srs, \gone{c}, z, \tilde{s}, \gone{\tilde{o}}) = 1
    \end{aligned}
    \,\left|\, \vphantom{\begin{aligned}
          & \verify(\srs, \gone{c}, z, s, \gone{o}),\\
          & \verify(\srs, \gone{c}, z, s, \gone{\tilde{o}}) \\
          &o \neq \tilde{o})
		\end{aligned}}
      \begin{aligned}
        %& \srs \gets \kgen(\secparam, \maxdeg), \\
        & (\gone{c}, z, s, \gone{o}, \gone{\tilde{o}}) \gets \adv^{\initU}(1^\secpar, \maxdeg)
      \end{aligned}
    \right.\right]
  % \leq \negl.
\]
is at most negligible.

As noted in \cite[Lemma 2.2]{EPRINT:GabWilCio19} it is enough to upper bound the
probability of the adversary succeeding using the idealised verification
equation---which considers equality between polynomials---instead of the real
verification equation---which considers equality of the polynomials' evaluations.

For a polynomial $f$, its degree upper bound $\maxconst$, evaluation point $z$,
evaluation result $s$, and opening $\gone{o(X)}$ the idealised check verifies that
\begin{equation}
  \alpha (X^{\dconst - \maxconst}f(X) \cdot X^{-\dconst + \maxconst} -  s) \equiv \alpha \cdot o(X) (X - z)\,,
\end{equation}
what is equivalent to 
\begin{equation}
	f(X) -  s \equiv o(X) (X - z)\,.
	\label{eq:pcoms_idealised_check}
\end{equation}
Since $o(X)(X - z) \in \FF_p[X]$ then from the uniqueness of polynomial
composition, there is only one $o(X)$ that fulfils the equation above.
\qed
\end{proof}


\subsection{Unique response property}
The unique response property of $\sonicprot$ follows from the unique opening
property of the used polynomial commitment scheme $\PCOMs$.
\begin{lemma}
\label{lem:sonicprot_ur}
If a polynomial commitment scheme $\PCOMs$ is evaluation binding with
parameter $\epsbind(\secpar)$ and has unique openings property with parameter
$\epsop(\secpar)$, then $\sonicprot$ is $\ur{1}$ with parameter $\epsur(\secpar) \leq
\epsbind(\secpar) + \epsop(\secpar)$.  
\end{lemma}
\begin{proof}
  Let $\adv$ be an adversary that breaks $\ur{1}$-ness of $\sonicprot$.  We
  consider two cases, depending on which round $\adv$ is able to provide at
  least two different outputs such that the resulting transcripts are
  acceptable.  For the first case we show that $\adv$ can be used to break the
  evaluation binding property of $\PCOMs$, while for the second case we show
  that it can be used to break the unique opening property of $\PCOMs$.

  The proof goes similarly to the proof of \cref{lem:plonkprot_ur} thus we
  provide only draft of it here.  In each Round $i$, for $i > 1$, the prover
  either commits to some well-defined polynomials (deterministically), evaluates
  these on randomly picked points, or shows that the evaluations were performed
  correctly.  Obviously, for a committed polynomial $\p{p}$ evaluated at point
  $x$ only one value $y = \p{p}(x)$ is correct. If the adversary was able to
  provide two different values $y$ and $\tilde{y}$ that would be accepted as an
  evaluation of $\p{p}$ at $x$ then the $\PCOMs$'s evaluation binding would be
  broken.  Alternatively, if $\adv$ was able to provide two openings $\p{W}$ and
  $\p{\tilde{W}}$ for $y = \p{p}(x)$ then the unique opening property would be
  broken.
%
Hence the probability that $\adv$ breaks $\ur{1}$-property of $\PCOMs$ is
upper-bounded by $\epsbind(\secpar) + \epsop(\secpar)$. 
\qed

\end{proof}

\subsection{Forking soundness}
\begin{lemma}
	\label{lem:sonicprot_ss}
	$\sonicprot$ is $(\epsss(\secpar), 2, \multconstr + \linconstr + 1)$-forking sound against
	algebraic adversaries with
	\[
	\epss(\secpar) \leq \epsid(\secpar) + \epsldlog(\secpar) \,,
	\]
	where $\epsid(\secpar)$ is a soundness error of the idealized verifier, and
	$\epsldlog(\secpar)$ is security of $(\dconst, \dconst)$-$\ldlog$ assumption.
\end{lemma}
\begin{proof}
	Similarly as in the case of $\plonk$, the main idea of the proof is to show
	that an adversary who breaks forking soundness can be used to break a $\dlog$ problem
	instance. The proof goes by game hops. Let $\tree$ be the tree produced by
	$\tdv$ by rewinding $\adv$. Note that since the tree branches after Round 2,
	the instance $\inp$, commitments
	$\gone{\p{r} (\chi, 1), \p{r} (\chi, y), \p{s} (\chi, y), \p{t} (\chi, y)}$, and challenge
	$y$ are the same. The tree branches after the second round
	of the protocol where the challenge $z$ is presented, thus tree $\tree$ is
	build using different values of $z$.
	%
	We consider the following games.
	
	\ncase{Game 0} In this game the adversary wins if all the transcripts it
	produced are acceptable by the ideal verifier,
	i.e.~$\vereq_{\inp, \zkproof}(X) = 0$, cf.~\cref{eq:ver_eq}, and none of
	commitments
	$\gone{\p{r} (\chi, 1), \p{r} (\chi, y), \p{s} (\chi, y), \p{t} (\chi, y)}$ use
	elements from a simulated proof, and the extractor fails to extract a valid
	witness out of the proof.
	
	\ncase{Probability that $\adv$ wins Game 0 is negligible} Probability of
	$\adv$ winning this game is $\epsid(\secpar)$ as the protocol $\sonicprot$,
	instantiated with the idealised verification equation, is perfectly
	knowledge sound except with negligible probability of the idealised verifier
	failure $\epsid(\secpar)$. Hence for a valid proof $\zkproof$ for a
	statement $\inp$ there exists a witness $\wit$, such that $\REL(\inp, \wit)$
	holds. Note that since the $\tdv$ produces $(\multconstr + \linconstr + 1)$
	acceptable transcripts for different challenges $z$. As noted in
	\cite{CCS:MBKM19} this assures that the correct witness is encoded in
	$\p{r} (X, Y)$. Hence $\extt$ can recreate polynomials' coefficients by
	interpolation and reveal the witness with probability $1$. Moreover, the
	probability that extraction fails in that case is upper-bounded by
	probability of an idealised verifier failing $\epsid(\secpar)$, which is
	negligible.
	
	\ncase{Game 1} In this game the adversary additionally wins if it produces a
	transcript in $\tree$ such that $\vereq_{\inp, \zkproof}(\chi) = 0$, but
	$\vereq_{\inp, \zkproof}(X) \neq 0$, and none of commitments
	$\gone{\p{r} (\chi, 1), \p{r} (\chi, y), \p{s} (\chi, y), \p{t} (\chi, y)}$
	use elements from a simulated proof.  The first condition means that the
	ideal verifier does not accept the proof, but the real verifier does.
	
	\ncase{Game 0 to Game 1} Assume the adversary wins in Game 1, but does not
	win in Game 0. We show that such adversary may be used to break an
	instance of a $\ldlog$ assumption. More precisely, let $\tdv$ be an
	algorithm that for relation $\REL$ and randomly picked
	$\srs \sample \kgen(\REL)$ produces a tree of acceptable transcripts such
	that the winning condition of the game holds. Let $\rdvdlog$ be a
	reduction that gets as input an
	$(\dconst, \dconst)$-ldlog instance
	$\gone{\chi^{-\dconst}, \ldots, \chi^{\dconst}}, \gtwo{\chi^{-\dconst},
		\ldots, \chi^{\dconst}}$ and is tasked to output $\chi$.
	
	The reduction $\rdvdlog$ proceeds as follows.
	\begin{enumerate}
		\item Pick a random $\alpha$ and compute
		$\gone{\alpha \chi^{- \dconst}, \ldots, \alpha \chi^{-1}, \alpha \chi,
			\ldots, \alpha \chi^{\dconst}}$,
		$\gtwo{\alpha \chi^{- \dconst}, \ldots, \alpha \chi^{-1}, \alpha \chi,
			\ldots, \alpha \chi^{\dconst}}$. Set a SRS $\srs$ to be the $(\dconst, \dconst)$-ldlog instance and its multiplication with $\alpha$ as computed above.\hamid{Is this clear?}
		\item Build $\sonicprot$'s SRS in the updatable setting by answering $\adv$'s queries for SRS updates and setting the honest update of the SRS to be $\srs$. Let $\srs'$ be the finalised SRS.
		\item Let $(1, \tree)$ be the output returned by $\tdv$. Let $\inp$ be a
		relation proven in $\tree$.  Consider a transcript $\zkproof \in \tree$ such
		that $\vereq_{\inp, \zkproof}(X) \neq 0$, but
		$\vereq_{\inp, \zkproof}(\chi') = 0$. Since $\adv$ is algebraic, all group
		elements included in $\tree$ are extended by their representation as a
		combination of the input $\GRP_1$-elements. Hence, all coefficients of the
		verification equation polynomial $\vereq_{\inp, \zkproof}(X)$ are known.
		\item Find $\vereq_{\inp, \zkproof}(X)$ zero points and find $\chi'$ among
		them.
		\item Let $\chi_1, \ldots, \chi_\ell$ be the partial trapdoors of $\adv$'s SRS updates, extracted by the reduction from the update proofs given by $\adv$.
		\item Return  $\chi = \chi' (\chi_1 \chi_2 \ldots \chi_\ell)^{-1}$.
	\end{enumerate}
	Hence, the probability that the adversary wins Game 1 is upper-bounded by
	$\epsldlog(\secpar)$.
\end{proof}

\subsection{Trapdoor-less simulatability of Sonic}
\begin{lemma}
\label{lem:sonic_hvzk}
$\sonic$ is trapdoor-less simulatable.
\end{lemma}
\begin{proof}
  The simulator proceeds as follows.
  \begin{enumerate}
  \item Pick randomly vectors $\vec{a}$, $\vec{b}$ and set
    \begin{equation}
      \label{eq:ab_eq_c}
      \vec{c} = \vec{a} \cdot \vec{b}. 
    \end{equation}
  \item Pick randomisers $c_{\multconstr + 1}, \ldots, c_{\multconstr + 4}$,
    honestly compute polynomials $\p{r}(X, Y), \p{r'}(X, Y), \p{s}(X, Y)$ and
    pick randomly challenges $y$, $z$.
  \item Output commitment $\gone{r} \gets \com(\srs, \multconstr, \p{r} (X,
    1))$ and challenge $y$. 
  \item Compute
    \begin{align*}
      & a' = \p{r}(z, 1),\\
      & b' = \p{r}(z, y),\\
      & s' = \p{s}(z, y).
    \end{align*} 
  \item Pick polynomial $\p{t}(X, Y)$ such that
    \begin{align*}
      & \p{t} (X, y) = \p{r} (X, 1) (\p{r}(X, y) + \p{s} (X, y)) - \p{k} (Y)\\
      & \p{t} (0, y) = 0
    \end{align*}
  \item Output commitment $\gone{t} = \com (\srs, \dconst, \p{t} (X, y))$ and
    challenge $z$.
  \item Continue following the protocol.
  \end{enumerate}

  We note that the simulation is perfect. This comes since, except polynomial
  $\p{t} (X, Y)$ all polynomials are computed following the protocol. For
  polynomial $\p{t} (X, Y)$ we observe that in a case of both real and simulated
  proof the verifier only learns commitment $\gone{t} = \p{t} (\chi, y)$ and
  evaluation $t' = \p{t} (z, y)$. Since the simulator picks $\p{t} (X, Y)$ such
  that 
  \begin{align*}
      \p{t} (X, y) = \p{r} (X, 1) (\p{r}(X, y) + \p{s} (X, y)) - \p{k} (Y)
  \end{align*}
  Values of $\gone{t}$ are equal in both proofs.
  Furthermore, the simulator picks its polynomial such that $\p{t}(0, y) = 0$,
  hence it does not need the trapdoor to commit to it. (Note that the proof
  system's SRS does not allow to commit to polynomials which have non-zero
  constant term). \qed
\end{proof}
\begin{remark} 
  As noted in \cite{CCS:MBKM19}, $\sonic$ is statistically subversion-zero
  knowledge (Sub-ZK). As noted in \cite{AC:ABLZ17}, one way to achieve
  subversion zero knowledge is to utilise an extractor that extracts a SRS
  trapdoor from a SRS-generator. Unfortunately, a NIZK made subversion
  zero-knowledge by this approach cannot achieve perfect Sub-ZK as one has to
  count in the probability of extraction failure. However, with the simulation
  presented in \cref{lem:sonic_hvzk}, the trapdoor is not required for the
  simulator as it is able to simulate the execution of the protocol just by
  picking appropriate (honest) verifier's challenges. This result transfers to
  $\sonicprotfs$, where the simulator can program the random oracle to provide
  challenges that fits it.
\end{remark}

\subsection{From forking soundness and unique response property to \COMMENT{forking
  }simulation extractability of $\sonicprotfs$}
Since \cref{lem:sonicprot_ur,lem:sonicprot_ss} hold, $\sonicprot$ is $\ur{1}$
and forking sound. We now make use
of \cref{thm:se} and show that $\sonicprotfs$ is \COMMENT{ forking }simulation-extractable as defined in \cref{def:simext}.

\begin{corollary}[\COMMENT{Forking s}Simulation extractability of $\sonicprotfs$]
  \label{thm:sonicprotfs_se}
  Assume that $\sonicprot$ is $\ur{1}$ with security
  $\epsur(\secpar) = \epsbind(\secpar) + \epsop(\secpar)$ -- where
  $\epsbind (\secpar)$ is polynomial commitment's binding security, $\epsop$ is
  polynomial commitment unique opening security -- and forking-sound with
  security $\epsss(\secpar)$. Let $\ro\colon \bin^* \to \bin^\secpar$ be a
  random oracle. Let $\advse$ be an algebraic adversary that can make up to $q$
  random oracle queries, up to $S$ simulation oracle queries, and outputs an
  acceptable proof for $\sonicprotfs$ with probability at least $\accProb$. Then
  $\sonicprotfs$ is \COMMENT{forking }simulation-extractable with extraction error
  $\eta = \epsur(\secpar)$. The extraction probability $\extProb$ is at least
\[
		\extProb  \geq \frac{1}{q^{\multconstr + \linconstr}} (\accProb - \epsur(\secpar))^{\multconstr +
		\linconstr + 1} - \eps(\secpar).
	\]
	for some negligible $\eps(\secpar)$, $\multconstr$ and $\linconstr$ being,
  respectively, the number of multiplicative and linear constrains of the system.
\end{corollary}

%%% Local Variables:
%%% mode: latex
%%% TeX-master: "main"
%%% End:
