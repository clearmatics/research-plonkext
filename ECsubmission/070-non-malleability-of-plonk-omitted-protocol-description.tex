\section{Non-malleability of \plonk{}, omitted protocol descriptions}
\label{sec:plonk_supp_mat}

\newcommand{\vql}{\vec{q_{L}}}
\newcommand{\vqr}{\vec{q_{R}}}
\newcommand{\vqm}{\vec{q_{M}}}
\newcommand{\vqo}{\vec{q_{O}}}
\newcommand{\vx}{\vec{x}}
\newcommand{\vqc}{\vec{q_{C}}}
\subsection{Plonk protocol description}
\label{sec:plonk_explained}
\oursubsub{The constrain system}
Assume $\CRKT$ is a fan-in two arithmetic circuit,
which fan-out is unlimited and has $\numberofconstrains$ gates and $\noofw$ wires
($\numberofconstrains \leq \noofw \leq 2\numberofconstrains$). \plonk's constraint
system is defined as follows:
\begin{itemize}
\item Let $\vec{V} = (\va, \vb, \vc)$, where $\va, \vb, \vc
  \in \range{1}{\noofw}^\numberofconstrains$. Entries $\va_i, \vb_i, \vc_i$ represent indices of left,
  right and output wires of circuits $i$-th gate.
\item Vectors $\vec{Q} = (\vql, \vqr, \vqo, \vqm, \vqc) \in
  (\FF^\numberofconstrains)^5$ are called \emph{selector vectors}:
  \begin{itemize}
  \item If the $i$-th gate is a multiplicative gate then $\vql_i = \vqr_i = 0$,
    $\vqm_i = 1$, and $\vqo_i = -1$. 
  \item If the $i$-th gate is an addition gate then $\vql_i = \vqr_i  = 1$, $\vqm_i =
    0$, and $\vqo_i = -1$. 
  \item $\vqc_i = 0$ always. 
  \end{itemize}
\end{itemize}

We say that vector $\vx \in \FF^\noofw$ satisfies constraint system if for all $i
\in \range{1}{\numberofconstrains}$
\[
  \vql_i \cdot \vx_{\va_i} + \vqr_i \cdot \vx_{\vb_i} + \vqo \cdot \vx_{\vc_i} +
  \vqm_i \cdot (\vx_{\va_i} \vx_{\vb_i}) + \vqc_i = 0. 
\]

\oursubsub{Algorithms rolled out}
\label{sec:plonk_explained}
\plonk{} argument system is universal. That is, it allows to verify computation
of any arithmetic circuit which has no more than $\numberofconstrains$
gates using a single SRS. However, to make computation efficient, for each
circuit there is allowed a preprocessing phase which extend the SRS with
circuit-related polynomial evaluations.

For the sake of simplicity of the security reductions presented in this paper, we
include in the SRS only these elements that cannot be computed without knowing
the secret trapdoor $\chi$. The rest of the SRS---the preprocessed input---can
be computed using these SRS elements thus we leave them to be computed by the
prover, verifier, and simulator.

\ourpar{$\plonk$ SRS generating algorithm $\kgen(\REL)$:}
The SRS generating algorithm picks at random $\chi \sample \FF_p$, computes
and outputs
\[
	\srs = \left(\gone{\smallset{\chi^i}_{i = 0}^{\numberofconstrains + 2}},
	\gtwo{\chi} \right).
\]

\ourpar{Preprocessing:}
Let $H = \smallset{\omega^i}_{i = 1}^{\numberofconstrains }$ be a
(multiplicative) $\numberofconstrains$-element subgroup of a field $\FF$
compound of $\numberofconstrains$-th roots of unity in $\FF$. Let $\lag_i(X)$ be
the $i$-th element of an $\numberofconstrains$-elements Lagrange basis. During
the preprocessing phase polynomials $\p{S_{id j}}, \p{S_{\sigma j}}$, for
$\p{j} \in \range{1}{3}$, are computed:
\begin{equation*}
  \begin{aligned}
    \p{S_{id 1}}(X) & = X,\vphantom{\sum_{i = 1}^{\noofc} \sigma(i) \lag_i(X),}\\
    \p{S_{id 2}}(X) & = k_1 \cdot X,\vphantom{\sum_{i = 1}^{\noofc} \sigma(i) \lag_i(X),}\\
    \p{S_{id 3}}(X) & = k_2 \cdot X,\vphantom{\sum_{i = 1}^{\noofc} \sigma(i) \lag_i(X),}
  \end{aligned}
  \qquad
\begin{aligned}
  \p{S_{\sigma 1}}(X) & = \sum_{i = 1}^{\noofc} \sigma(i) \lag_i(X),\\
  \p{S_{\sigma 2}}(X) & = \sum_{i = 1}^{\noofc}
  \sigma(\noofc + i) \lag_i(X),\\
  \p{S_{\sigma 3}}(X) & =\sum_{i = 1}^{\noofc} \sigma(2 \noofc + i) \lag_i(X).
\end{aligned}
\end{equation*}
Coefficients $k_1$, $k_2$ are such that $H, k_1 \cdot H, k_2 \cdot H$ are
different cosets of $\FF^*$, thus they define $3 \cdot \noofc$
different elements. \cite{EPRINT:GabWilCio19} notes that it is enough to set
$k_1$ to a quadratic residue and $k_2$ to a quadratic non-residue.

Furthermore, we define polynomials $\p{q_L}, \p{q_R}, \p{q_O}, \p{q_M}, \p{q_C}$
such that
\begin{equation*}
  \begin{aligned}
  \p{q_L}(X) & = \sum_{i = 1}^{\noofc} \vql_i \lag_i(X), \\
  \p{q_R}(X) & = \sum_{i = 1}^{\noofc} \vqr_i \lag_i(X), \\
  \p{q_M}(X) & = \sum_{i = 1}^{\noofc} \vqm_i \lag_i(X),
\end{aligned}
\qquad
\begin{aligned}
  \p{q_O}(X) & = \sum_{i = 1}^{\noofc} \vqo_i \lag_i(X), \\
  \p{q_C}(X) & = \sum_{i = 1}^{\noofc} \vqc_i \lag_i(X). \\
  \vphantom{\p{q_M}(X)  = \sum_{i = 1}^{\noofc} \vqm_i \lag_i(X),}
\end{aligned}
\end{equation*}

\ourpar{$\plonk$ prover
  $\prover(\srs, \inp, \wit = (\wit_i)_{i \in \range{1}{3 \cdot
      \noofc}})$.}
\begin{description}
\item[Round 1] Sample $b_1, \ldots, b_9 \sample \FF_p$; compute
  $\p{a}(X), \p{b}(X), \p{c}(X)$ as
	\begin{align*}
		\p{a}(X) &= (b_1 X + b_2)\p{Z_H}(X) + \sum_{i = 1}^{\noofc} \wit_i \lag_i(X) \\
		\p{b}(X) &= (b_3 X + b_4)\p{Z_H}(X) + \sum_{i = 1}^{\noofc} \wit_{\noofc + i} \lag_i(X) \\
		\p{c}(X) &= (b_5 X + b_6)\p{Z_H}(X) + \sum_{i = 1}^{\noofc} \wit_{2 \cdot \noofc + i} \lag_i(X) 
	\end{align*}
	Output polynomial commitments $\gone{\p{a}(\chi), \p{b}(\chi), \p{c}(\chi)}$.
	
	\item[Round 2]
	Get challenges $\beta, \gamma \in \FF_p$
	\[
		\beta = \ro(\zkproof[0..1], 0)\,, \qquad \gamma = \ro(\zkproof[0..1], 1)\,.
	\]
	Compute permutation polynomial $\p{z}(X)$
	\begin{multline*}
		\p{z}(X) = (b_7 X^2 + b_8 X + b_9)\p{Z_H}(X) + \lag_1(X) + \\
			+ \sum_{i = 1}^{\noofc - 1} 
			\left(\lag_{i + 1} (X) \prod_{j = 1}^{i} 
			\frac{
			(\wit_j +\beta \omega^{j - 1} + \gamma)(\wit_{\noofc + j} + \beta k_1 \omega^{j - 1} + \gamma)(\wit_{2 \noofc + j} +\beta k_2 \omega^{j- 1} + \gamma)}
			{(\wit_j+\sigma(j) \beta + \gamma)(\wit_{\noofc + j} + \sigma(\noofc + j)\beta + \gamma)(\wit_{2 \noofc + j} + \sigma(2 \noofc + j)\beta + \gamma)}\right)
	\end{multline*}
	Output polynomial commitment $\gone{\p{z}(\chi)}$
		
	\item[Round 3]
	Get the challenge $\alpha = \ro(\zkproof[0..2])$, compute the quotient polynomial 
	\begin{align*}
	& \p{t}(X)  = \\
	& (\p{a}(X) \p{b}(X) \selmulti(X) + \p{a}(X) \selleft(X) + 
	\p{b}(X)\selright(X) + \p{c}(X)\seloutput(X) + \pubinppoly(X) + \selconst(X)) 
	\frac{1}{\p{Z_H}(X)} +\\
	& + ((\p{a}(X) + \beta X + \gamma) (\p{b}(X) + \beta k_1 X + \gamma)(\p{c}(X) 
	+ \beta k_2 X + \gamma)\p{z}(X)) \frac{\alpha}{\p{Z_H}(X)} \\
	& - (\p{a}(X) + \beta \p{S_{\sigma 1}}(X) + \gamma)(\p{b}(X) + \beta 
	\p{S_{\sigma 2}}(X) + \gamma)(\p{c}(X) + \beta \p{S_{\sigma 3}}(X) + 
	\gamma)\p{z}(X \omega))  \frac{\alpha}{\p{Z_H}(X)} \\
	& + (\p{z}(X) - 1) \lag_1(X) \frac{\alpha^2}{\p{Z_H}(X)}
	\end{align*}
	Split $\p{t}(X)$ into degree less then $\noofc$ polynomials $\p{t_{lo}}(X), \p{t_{mid}}(X), \p{t_{hi}}(X)$, such that
	\[
		\p{t}(X) = \p{t_{lo}}(X) + X^{\noofc} \p{t_{mid}}(X) + X^{2 \noofc} \p{t_{hi}}(X)\,.
	\]
	Output $\gone{\p{t_{lo}}(\chi), \p{t_{mid}}(\chi), \p{t_{hi}}(\chi)}$.
	
	\item[Round 4]
	Get the challenge $\chz \in \FF_p$, $\chz = \ro(\zkproof[0..3])$.
	Compute opening evaluations
	\begin{align*}
      \p{a}(\chz), \p{b}(\chz), \p{c}(\chz), \p{S_{\sigma 1}}(\chz), \p{S_{\sigma 2}}(\chz), \p{t}(\chz), \p{z}(\chz \omega),
	\end{align*}
	Compute the linearisation polynomial
	\[
		\p{r}(X) = 
		\begin{aligned}
      & \p{a}(\chz) \p{b}(\chz) \selmulti(X) + \p{a}(\chz) \selleft(X) + \p{b}(\chz) \selright(X) + \p{c}(\chz) \seloutput(X) + \selconst(X) \\
      & + \alpha \cdot \left( (\p{a}(\chz) + \beta \chz + \gamma) (\p{b}(\chz) + \beta k_1 \chz + \gamma)(\p{c}(\chz) + \beta k_2 \chz + \gamma) \cdot \p{z}(X)\right) \\
      & - \alpha \cdot \left( (\p{a}(\chz) + \beta \p{S_{\sigma 1}}(\chz) + \gamma) (\p{b}(\chz) + \beta \p{S_{\sigma 2}}(\chz) + \gamma)\beta \p{z}(\chz\omega) \cdot \p{S_{\sigma 3}}(X)\right) \\
      & + \alpha^2 \cdot \lag_1(\chz) \cdot \p{z}(X)
		\end{aligned}
	\]
	Output $\p{a}(\chz), \p{b}(\chz), \p{c}(\chz), \p{S_{\sigma 1}}(\chz), \p{S_{\sigma 2}}(\chz), \p{t}(\chz), \p{z}(\chz \omega), \p{r}(\chz).$
	
	\item[Round 5]
	Compute the opening challenge $v \in \FF_p$, $v = \ro(\zkproof[0..4])$.
	Compute the openings for the polynomial commitment scheme 
	\begin{align*}
	& \p{W_\chz}(X) = \frac{1}{X - \chz} \left(
	\begin{aligned}
		& \p{t_{lo}}(X) + \chz^\noofc \p{t_{mid}}(X) + \chz^{2 \noofc} \p{t_{hi}}(X) - \p{t}(\chz)\\
		& + v(\p{r}(X) - \p{r}(\chz)) \\
		& + v^2 (\p{a}(X) - \p{a}(\chz))\\
		& + v^3 (\p{b}(X) - \p{b}(\chz))\\
		& + v^4 (\p{c}(X) - \p{c}(\chz))\\
		& + v^5 (\p{S_{\sigma 1}}(X) - \p{S_{\sigma 1}}(\chz))\\
		& + v^6 (\p{S_{\sigma 2}}(X) - \p{S_{\sigma 2}}(\chz))
	\end{aligned}
	\right)\\
	& \p{W_{\chz \omega}}(X) = \frac{\p{z}(X) - \p{z}(\chz \omega)}{X - \chz \omega}
\end{align*}
	Output $\gone{\p{W_{\chz}}(\chi), \p{W_{\chz \omega}}(\chi)}$.
\end{description}

\ncase{$\plonk$ verifier $\verifier(\srs, \inp, \zkproof)$}\ \newline
The \plonk{} verifier works as follows
\begin{description}
	\item[Step 1] Validate all obtained group elements.
	\item[Step 2] Validate all obtained field elements.
	\item[Step 3] Validate the instance
      $\inp = \smallset{\wit_i}_{i = 1}^\instsize$.
	\item[Step 4] Compute challenges $\beta, \gamma, \alpha, \chz, v,
      u$ from the transcript.
	\item[Step 5] Compute zero polynomial evaluation
      $\p{Z_H} (\chz) =\chz^\noofc - 1$.
	\item[Step 6] Compute Lagrange polynomial evaluation
      $\lag_1 (\chz) = \frac{\chz^\noofc -1}{\noofc (\chz - 1)}$.
	\item[Step 7] Compute public input polynomial evaluation
      $\pubinppoly (\chz) = \sum_{i \in \range{1}{\instsize}} \wit_i
      \lag_i(\chz)$.
	\item[Step 8] Compute quotient polynomials evaluations
	\begin{multline*}
    \p{t} (\chz) = \frac{1}{\p{Z_H}(\chz)} \Big(
    \p{r} (\chz) + \pubinppoly(\chz) - (\p{a}(\chz) + \beta \p{S_{\sigma 1}}(\chz) + \gamma) (\p{b}(\chz) + \beta \p{S_{\sigma 2}}(\chz) + \gamma) \\
    (\p{c}(\chz) + \gamma)\p{z}(\chz \omega) \alpha - \lag_1 (\chz) \alpha^2
    \Big) \,.
	\end{multline*}
	\item[Step 9] Compute batched polynomial commitment
	$\gone{D} = v \gone{r} + u \gone {z}$ that is
	\begin{align*}
		\gone{D} & = v
		\left(
		\begin{aligned}
          & \p{a}(\chz)\p{b}(\chz) \cdot \gone{\selmulti} + \p{a}(\chz)  \gone{\selleft} + \p{b}  \gone{\selright} + \p{c}  \gone{\seloutput} + \\
          & + (	(\p{a}(\chz) + \beta \chz + \gamma) (\p{b}(\chz) + \beta k_1 \chz + \gamma) (\p{c} + \beta k_2 \chz + \gamma) \alpha  + \lag_1(\chz) \alpha^2)  + \\
			% &   \\
          & - (\p{a}(\chz) + \beta \p{S_{\sigma 1}}(\chz) + \gamma) (\p{b}(\chz)
          + \beta \p{S_{\sigma 2}}(\chz) + \gamma) \alpha \beta \p{z}(\chz
          \omega) \gone{\p{S_{\sigma 3}}(\chi)})
		\end{aligned}
		\right) + \\
		& + u \gone{\p{z}(\chi)}\,.
	\end{align*}
	\item[Step 10] Computes full batched polynomial commitment $\gone{F}$:
	\begin{align*}
      \gone{F} & = \left(\gone{\p{t_{lo}}(\chi)} + \chz^\noofc \gone{\p{t_{mid}}(\chi)} + \chz^{2 \noofc} \gone{\p{t_{hi}}(\chi)}\right) + u \gone{\p{z}(\chi)} + \\
               & + v
                 \left(
		\begin{aligned}
			& \p{a}(\chz)\p{b}(\chz) \cdot \gone{\selmulti} + \p{a}(\chz)  \gone{\selleft} + \p{b}(\chz)   \gone{\selright} + \p{c}(\chz)  \gone{\seloutput} + \\
			& + (	(\p{a}(\chz) + \beta \chz + \gamma) (\p{b}(\chz) + \beta k_1 \chz + \gamma) (\p{c}(\chz)  + \beta k_2 \chz + \gamma) \alpha  + \lag_1(\chz) \alpha^2)  + \\
			% &   \\
			& - (\p{a}(\chz) + \beta \p{S_{\sigma 1}}(\chz) + \gamma) (\p{b}(\chz) + \beta \p{S_{\sigma 2}}(\chz) + \gamma) \alpha  \beta \p{z}(\chz \omega) \gone{\p{S_{\sigma 3}}(\chi)})
		\end{aligned}
		\right) \\
		& + v^2 \gone{\p{a}(\chi)} + v^3 \gone{\p{b}(\chi)} + v^4 \gone{\p{c}(\chi)} + v^5 \gone{\p{S_{\sigma 1}(\chi)}} + v^6 \gone{\p{S_{\sigma 2}}(\chi)}\,.
	\end{align*}
	\item[Step 11] Compute group-encoded batch evaluation $\gone{E}$
	\begin{align*}
		\gone{E}  = \frac{1}{\p{Z_H}(\chz)} & \gone{
		\begin{aligned}
			& \p{r}(\chz) + \pubinppoly(\chz) +  \alpha^2  \lag_1 (\chz) + \\
			& - \alpha \left( (\p{a}(\chz) + \beta \p{S_{\sigma 1}} (\chz) + \gamma) (\p{b}(\chz) + \beta \p{S_{\sigma 2}} (\chz) + \gamma) (\p{c}(\chz) + \gamma) \p{z}(\chz \omega) \right)
		\end{aligned}
           }\\
      + & \gone{v \p{r}(\chz) + v^2 \p{a}(\chz) + v^3 \p{b}(\chz) + v^4 \p{c}(\chz) + v^5 \p{S_{\sigma 1}}(\chz) + v^6 \p{S_{\sigma 2}}(\chz) + u \p{z}(\chz \omega) }\,.
	\end{align*}
\item[Step 12] Check whether the verification
 % $\vereq_\zkproof(\chi)$
  equation holds
	\begin{multline}
		\label{eq:ver_eq}
		\left( \gone{\p{W_{\chz}}(\chi)} + u \cdot \gone{\p{W_{\chz
                \omega}}(\chi)} \right) \bullet
		\gtwo{\chi} - %\\
		\left( \chz \cdot \gone{\p{W_{\chz}}(\chi)} + u \chz \omega \cdot
          \gone{\p{W_{\chz \omega}}(\chi)} + \gone{F} - \gone{E} \right) \bullet
        \gtwo{1} = 0\,.
	\end{multline}
  The verification equation is a batched version of the verification equation
  from \cite{AC:KatZavGol10} which allows the verifier to check openings of
  multiple polynomials in two points (instead of checking an opening of a single
  polynomial at one point).
\end{description}

\ncase{$\plonk$ simulator $\simulator_\chi(\srs, \td= \chi, \inp)$}\ \newline
The \plonk{} simulator proceeds as an honest prover would, except:
\begin{enumerate}
  \item In the first round, it sets $\wit = (\wit_i)_{i \in \range{1}{3 \noofc}}
    = \vec{0}$, and at random picks $b_1, \ldots, b_9$. Then it proceeds with
    that all-zero witness.
  \item In Round 3, it computes polynomial $\pt(X)$ honestly, however uses
    trapdoor $\chi$ to compute commitments
    $\p{t_{lo}}(\chi), \p{t_{mid}}(\chi), \p{t_{hi}}(\chi)$.
  \end{enumerate}

%  \subsection{Trapdoor-less simulatability of Plonk}
%\label{sec:plonk-tls-proof}


%%% Local Variables:
%%% mode: latex
%%% TeX-master: "main"
%%% End:
