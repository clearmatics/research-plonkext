% !TEX root = main.tex
% !TEX spellcheck = en-US


\section{Non-Malleability of $\plonkprotfs$} 
\label{sec:plonk}
In this section, we show that $\plonkprotfs$ is \COMMENT{forking }simulation-extractable. Towards this end, we first use the unique opening property to show that
$\plonkprot$ has the $\ur{2}$ property,
cf.~\cref{lem:plonkprot_ur}.
Next, we show that $\plonkprot$ is forking-special-sound. That is, given a
number of accepting transcripts whose messages match on the first $3$ rounds of the
protocol, we can either extract a witness for the proven statement or use
one of the transcripts to break the $\dlog$ assumption. This result is shown in
the AGM, cf.~\cref{lem:plonkprot_ss}.

%Given forking-soundness of $\plonkprot$, we use the fact that it is also
%$\ur{2}$ and show, in a similar fashion to \cite{INDOCRYPT:FKMV12}, that it is
%simulation-extractable. That is, we build reductions that given a simulation
%extractability adversary $\advse$ either break the protocol's unique response
%property or based on forking soundness break the $\dlog$ assumption, if extracting a valid witness from a
%tree of transcripts is impossible. See \cref{thm:plonkprotfs_se}.

Given forking-soundness and $\ur{2}$ of $\plonkprot$, we invoke \cref{thm:se} and conclude that $\plonkprot_\fs$ is \COMMENT{forking }simulation-extractable.
Due to page limit, we omit description of \plonk{} here and refer to
\cref{sec:plonk_explained}. 
%Unfortunately, we also have to move some of the
% proofs to the Supplementary Materials as well, cf.~\cref{sec:plonk_supp_mat}

\newcommand{\vql}{\vec{q_{L}}}
\newcommand{\vqr}{\vec{q_{R}}}
\newcommand{\vqm}{\vec{q_{M}}}
\newcommand{\vqo}{\vec{q_{O}}}
\newcommand{\vx}{\vec{x}}
\newcommand{\vqc}{\vec{q_{C}}}

\subsection{Plonk protocol description}
\label{sec:plonk_explained}
\oursubsub{The constrain system}
Assume $\CRKT$ is a fan-in two arithmetic circuit,
which fan-out is unlimited and has $\numberofconstrains$ gates and $\noofw$ wires
($\numberofconstrains \leq \noofw \leq 2\numberofconstrains$). \plonk's constraint
system is defined as follows:
\begin{itemize}
\item Let $\vec{V} = (\va, \vb, \vc)$, where $\va, \vb, \vc
  \in \range{1}{\noofw}^\numberofconstrains$. Entries $\va_i, \vb_i, \vc_i$ represent indices of left,
  right and output wires of circuits $i$-th gate.
\item Vectors $\vec{Q} = (\vql, \vqr, \vqo, \vqm, \vqc) \in
  (\FF^\numberofconstrains)^5$ are called \emph{selector vectors}:
  \begin{itemize}
  \item If the $i$-th gate is a multiplicative gate then $\vql_i = \vqr_i = 0$,
    $\vqm_i = 1$, and $\vqo_i = -1$. 
  \item If the $i$-th gate is an addition gate then $\vql_i = \vqr_i  = 1$, $\vqm_i =
    0$, and $\vqo_i = -1$. 
  \item $\vqc_i = 0$ always. 
  \end{itemize}
\end{itemize}

We say that vector $\vx \in \FF^\noofw$ satisfies constraint system if for all $i
\in \range{1}{\numberofconstrains}$
\[
  \vql_i \cdot \vx_{\va_i} + \vqr_i \cdot \vx_{\vb_i} + \vqo \cdot \vx_{\vc_i} +
  \vqm_i \cdot (\vx_{\va_i} \vx_{\vb_i}) + \vqc_i = 0. 
\]

\oursubsub{Algorithms rolled out}
\label{sec:plonk_explained}
\plonk{} argument system is universal. That is, it allows to verify computation
of any arithmetic circuit which has no more than $\numberofconstrains$
gates using a single SRS. However, to make computation efficient, for each
circuit there is allowed a preprocessing phase which extend the SRS with
circuit-related polynomial evaluations.

For the sake of simplicity of the security reductions presented in this paper, we
include in the SRS only these elements that cannot be computed without knowing
the secret trapdoor $\chi$. The rest of the SRS---the preprocessed input---can
be computed using these SRS elements thus we leave them to be computed by the
prover, verifier, and simulator.

\ourpar{$\plonk$ SRS generating algorithm $\kgen(\REL)$:}
The SRS generating algorithm picks at random $\chi \sample \FF_p$, computes
and outputs
\[
	\srs = \left(\gone{\smallset{\chi^i}_{i = 0}^{\numberofconstrains + 2}},
	\gtwo{\chi} \right).
\]

\ourpar{Preprocessing:}
Let $H = \smallset{\omega^i}_{i = 1}^{\numberofconstrains }$ be a
(multiplicative) $\numberofconstrains$-element subgroup of a field $\FF$
compound of $\numberofconstrains$-th roots of unity in $\FF$. Let $\lag_i(X)$ be
the $i$-th element of an $\numberofconstrains$-elements Lagrange basis. During
the preprocessing phase polynomials $\p{S_{id j}}, \p{S_{\sigma j}}$, for
$\p{j} \in \range{1}{3}$, are computed:
\begin{equation*}
  \begin{aligned}
    \p{S_{id 1}}(X) & = X,\vphantom{\sum_{i = 1}^{\noofc} \sigma(i) \lag_i(X),}\\
    \p{S_{id 2}}(X) & = k_1 \cdot X,\vphantom{\sum_{i = 1}^{\noofc} \sigma(i) \lag_i(X),}\\
    \p{S_{id 3}}(X) & = k_2 \cdot X,\vphantom{\sum_{i = 1}^{\noofc} \sigma(i) \lag_i(X),}
  \end{aligned}
  \qquad
\begin{aligned}
  \p{S_{\sigma 1}}(X) & = \sum_{i = 1}^{\noofc} \sigma(i) \lag_i(X),\\
  \p{S_{\sigma 2}}(X) & = \sum_{i = 1}^{\noofc}
  \sigma(\noofc + i) \lag_i(X),\\
  \p{S_{\sigma 3}}(X) & =\sum_{i = 1}^{\noofc} \sigma(2 \noofc + i) \lag_i(X).
\end{aligned}
\end{equation*}
Coefficients $k_1$, $k_2$ are such that $H, k_1 \cdot H, k_2 \cdot H$ are
different cosets of $\FF^*$, thus they define $3 \cdot \noofc$
different elements. \cite{EPRINT:GabWilCio19} notes that it is enough to set
$k_1$ to a quadratic residue and $k_2$ to a quadratic non-residue.

Furthermore, we define polynomials $\p{q_L}, \p{q_R}, \p{q_O}, \p{q_M}, \p{q_C}$
such that
\begin{equation*}
  \begin{aligned}
  \p{q_L}(X) & = \sum_{i = 1}^{\noofc} \vql_i \lag_i(X), \\
  \p{q_R}(X) & = \sum_{i = 1}^{\noofc} \vqr_i \lag_i(X), \\
  \p{q_M}(X) & = \sum_{i = 1}^{\noofc} \vqm_i \lag_i(X),
\end{aligned}
\qquad
\begin{aligned}
  \p{q_O}(X) & = \sum_{i = 1}^{\noofc} \vqo_i \lag_i(X), \\
  \p{q_C}(X) & = \sum_{i = 1}^{\noofc} \vqc_i \lag_i(X). \\
  \vphantom{\p{q_M}(X)  = \sum_{i = 1}^{\noofc} \vqm_i \lag_i(X),}
\end{aligned}
\end{equation*}

\ourpar{$\plonk$ prover
  $\prover(\srs, \inp, \wit = (\wit_i)_{i \in \range{1}{3 \cdot
      \noofc}})$.}
\begin{description}
\item[Round 1] Sample $b_1, \ldots, b_9 \sample \FF_p$; compute
  $\p{a}(X), \p{b}(X), \p{c}(X)$ as
	\begin{align*}
		\p{a}(X) &= (b_1 X + b_2)\p{Z_H}(X) + \sum_{i = 1}^{\noofc} \wit_i \lag_i(X) \\
		\p{b}(X) &= (b_3 X + b_4)\p{Z_H}(X) + \sum_{i = 1}^{\noofc} \wit_{\noofc + i} \lag_i(X) \\
		\p{c}(X) &= (b_5 X + b_6)\p{Z_H}(X) + \sum_{i = 1}^{\noofc} \wit_{2 \cdot \noofc + i} \lag_i(X) 
	\end{align*}
	Output polynomial commitments $\gone{\p{a}(\chi), \p{b}(\chi), \p{c}(\chi)}$.
	
	\item[Round 2]
	Get challenges $\beta, \gamma \in \FF_p$
	\[
		\beta = \ro(\zkproof[0..1], 0)\,, \qquad \gamma = \ro(\zkproof[0..1], 1)\,.
	\]
	Compute permutation polynomial $\p{z}(X)$
	\begin{multline*}
		\p{z}(X) = (b_7 X^2 + b_8 X + b_9)\p{Z_H}(X) + \lag_1(X) + \\
			+ \sum_{i = 1}^{\noofc - 1} 
			\left(\lag_{i + 1} (X) \prod_{j = 1}^{i} 
			\frac{
			(\wit_j +\beta \omega^{j - 1} + \gamma)(\wit_{\noofc + j} + \beta k_1 \omega^{j - 1} + \gamma)(\wit_{2 \noofc + j} +\beta k_2 \omega^{j- 1} + \gamma)}
			{(\wit_j+\sigma(j) \beta + \gamma)(\wit_{\noofc + j} + \sigma(\noofc + j)\beta + \gamma)(\wit_{2 \noofc + j} + \sigma(2 \noofc + j)\beta + \gamma)}\right)
	\end{multline*}
	Output polynomial commitment $\gone{\p{z}(\chi)}$
		
	\item[Round 3]
	Get the challenge $\alpha = \ro(\zkproof[0..2])$, compute the quotient polynomial 
	\begin{align*}
	& \p{t}(X)  = \\
	& (\p{a}(X) \p{b}(X) \selmulti(X) + \p{a}(X) \selleft(X) + 
	\p{b}(X)\selright(X) + \p{c}(X)\seloutput(X) + \pubinppoly(X) + \selconst(X)) 
	\frac{1}{\p{Z_H}(X)} +\\
	& + ((\p{a}(X) + \beta X + \gamma) (\p{b}(X) + \beta k_1 X + \gamma)(\p{c}(X) 
	+ \beta k_2 X + \gamma)\p{z}(X)) \frac{\alpha}{\p{Z_H}(X)} \\
	& - (\p{a}(X) + \beta \p{S_{\sigma 1}}(X) + \gamma)(\p{b}(X) + \beta 
	\p{S_{\sigma 2}}(X) + \gamma)(\p{c}(X) + \beta \p{S_{\sigma 3}}(X) + 
	\gamma)\p{z}(X \omega))  \frac{\alpha}{\p{Z_H}(X)} \\
	& + (\p{z}(X) - 1) \lag_1(X) \frac{\alpha^2}{\p{Z_H}(X)}
	\end{align*}
	Split $\p{t}(X)$ into degree less then $\noofc$ polynomials $\p{t_{lo}}(X), \p{t_{mid}}(X), \p{t_{hi}}(X)$, such that
	\[
		\p{t}(X) = \p{t_{lo}}(X) + X^{\noofc} \p{t_{mid}}(X) + X^{2 \noofc} \p{t_{hi}}(X)\,.
	\]
	Output $\gone{\p{t_{lo}}(\chi), \p{t_{mid}}(\chi), \p{t_{hi}}(\chi)}$.
	
	\item[Round 4]
	Get the challenge $\chz \in \FF_p$, $\chz = \ro(\zkproof[0..3])$.
	Compute opening evaluations
	\begin{align*}
      \p{a}(\chz), \p{b}(\chz), \p{c}(\chz), \p{S_{\sigma 1}}(\chz), \p{S_{\sigma 2}}(\chz), \p{t}(\chz), \p{z}(\chz \omega),
	\end{align*}
	Compute the linearisation polynomial
	\[
		\p{r}(X) = 
		\begin{aligned}
      & \p{a}(\chz) \p{b}(\chz) \selmulti(X) + \p{a}(\chz) \selleft(X) + \p{b}(\chz) \selright(X) + \p{c}(\chz) \seloutput(X) + \selconst(X) \\
      & + \alpha \cdot \left( (\p{a}(\chz) + \beta \chz + \gamma) (\p{b}(\chz) + \beta k_1 \chz + \gamma)(\p{c}(\chz) + \beta k_2 \chz + \gamma) \cdot \p{z}(X)\right) \\
      & - \alpha \cdot \left( (\p{a}(\chz) + \beta \p{S_{\sigma 1}}(\chz) + \gamma) (\p{b}(\chz) + \beta \p{S_{\sigma 2}}(\chz) + \gamma)\beta \p{z}(\chz\omega) \cdot \p{S_{\sigma 3}}(X)\right) \\
      & + \alpha^2 \cdot \lag_1(\chz) \cdot \p{z}(X)
		\end{aligned}
	\]
	Output $\p{a}(\chz), \p{b}(\chz), \p{c}(\chz), \p{S_{\sigma 1}}(\chz), \p{S_{\sigma 2}}(\chz), \p{t}(\chz), \p{z}(\chz \omega), \p{r}(\chz).$
	
	\item[Round 5]
	Compute the opening challenge $v \in \FF_p$, $v = \ro(\zkproof[0..4])$.
	Compute the openings for the polynomial commitment scheme 
	\begin{align*}
	& \p{W_\chz}(X) = \frac{1}{X - \chz} \left(
	\begin{aligned}
		& \p{t_{lo}}(X) + \chz^\noofc \p{t_{mid}}(X) + \chz^{2 \noofc} \p{t_{hi}}(X) - \p{t}(\chz)\\
		& + v(\p{r}(X) - \p{r}(\chz)) \\
		& + v^2 (\p{a}(X) - \p{a}(\chz))\\
		& + v^3 (\p{b}(X) - \p{b}(\chz))\\
		& + v^4 (\p{c}(X) - \p{c}(\chz))\\
		& + v^5 (\p{S_{\sigma 1}}(X) - \p{S_{\sigma 1}}(\chz))\\
		& + v^6 (\p{S_{\sigma 2}}(X) - \p{S_{\sigma 2}}(\chz))
	\end{aligned}
	\right)\\
	& \p{W_{\chz \omega}}(X) = \frac{\p{z}(X) - \p{z}(\chz \omega)}{X - \chz \omega}
\end{align*}
	Output $\gone{\p{W_{\chz}}(\chi), \p{W_{\chz \omega}}(\chi)}$.
\end{description}

\ncase{$\plonk$ verifier $\verifier(\srs, \inp, \zkproof)$}\ \newline
The \plonk{} verifier works as follows
\begin{description}
	\item[Step 1] Validate all obtained group elements.
	\item[Step 2] Validate all obtained field elements.
	\item[Step 3] Validate the instance
      $\inp = \smallset{\wit_i}_{i = 1}^\instsize$.
	\item[Step 4] Compute challenges $\beta, \gamma, \alpha, \chz, v,
      u$ from the transcript.
	\item[Step 5] Compute zero polynomial evaluation
      $\p{Z_H} (\chz) =\chz^\noofc - 1$.
	\item[Step 6] Compute Lagrange polynomial evaluation
      $\lag_1 (\chz) = \frac{\chz^\noofc -1}{\noofc (\chz - 1)}$.
	\item[Step 7] Compute public input polynomial evaluation
      $\pubinppoly (\chz) = \sum_{i \in \range{1}{\instsize}} \wit_i
      \lag_i(\chz)$.
	\item[Step 8] Compute quotient polynomials evaluations
	\begin{multline*}
    \p{t} (\chz) = \frac{1}{\p{Z_H}(\chz)} \Big(
    \p{r} (\chz) + \pubinppoly(\chz) - (\p{a}(\chz) + \beta \p{S_{\sigma 1}}(\chz) + \gamma) (\p{b}(\chz) + \beta \p{S_{\sigma 2}}(\chz) + \gamma) \\
    (\p{c}(\chz) + \gamma)\p{z}(\chz \omega) \alpha - \lag_1 (\chz) \alpha^2
    \Big) \,.
	\end{multline*}
	\item[Step 9] Compute batched polynomial commitment
	$\gone{D} = v \gone{r} + u \gone {z}$ that is
	\begin{align*}
		\gone{D} & = v
		\left(
		\begin{aligned}
          & \p{a}(\chz)\p{b}(\chz) \cdot \gone{\selmulti} + \p{a}(\chz)  \gone{\selleft} + \p{b}  \gone{\selright} + \p{c}  \gone{\seloutput} + \\
          & + (	(\p{a}(\chz) + \beta \chz + \gamma) (\p{b}(\chz) + \beta k_1 \chz + \gamma) (\p{c} + \beta k_2 \chz + \gamma) \alpha  + \lag_1(\chz) \alpha^2)  + \\
			% &   \\
          & - (\p{a}(\chz) + \beta \p{S_{\sigma 1}}(\chz) + \gamma) (\p{b}(\chz)
          + \beta \p{S_{\sigma 2}}(\chz) + \gamma) \alpha \beta \p{z}(\chz
          \omega) \gone{\p{S_{\sigma 3}}(\chi)})
		\end{aligned}
		\right) + \\
		& + u \gone{\p{z}(\chi)}\,.
	\end{align*}
	\item[Step 10] Computes full batched polynomial commitment $\gone{F}$:
	\begin{align*}
      \gone{F} & = \left(\gone{\p{t_{lo}}(\chi)} + \chz^\noofc \gone{\p{t_{mid}}(\chi)} + \chz^{2 \noofc} \gone{\p{t_{hi}}(\chi)}\right) + u \gone{\p{z}(\chi)} + \\
               & + v
                 \left(
		\begin{aligned}
			& \p{a}(\chz)\p{b}(\chz) \cdot \gone{\selmulti} + \p{a}(\chz)  \gone{\selleft} + \p{b}(\chz)   \gone{\selright} + \p{c}(\chz)  \gone{\seloutput} + \\
			& + (	(\p{a}(\chz) + \beta \chz + \gamma) (\p{b}(\chz) + \beta k_1 \chz + \gamma) (\p{c}(\chz)  + \beta k_2 \chz + \gamma) \alpha  + \lag_1(\chz) \alpha^2)  + \\
			% &   \\
			& - (\p{a}(\chz) + \beta \p{S_{\sigma 1}}(\chz) + \gamma) (\p{b}(\chz) + \beta \p{S_{\sigma 2}}(\chz) + \gamma) \alpha  \beta \p{z}(\chz \omega) \gone{\p{S_{\sigma 3}}(\chi)})
		\end{aligned}
		\right) \\
		& + v^2 \gone{\p{a}(\chi)} + v^3 \gone{\p{b}(\chi)} + v^4 \gone{\p{c}(\chi)} + v^5 \gone{\p{S_{\sigma 1}(\chi)}} + v^6 \gone{\p{S_{\sigma 2}}(\chi)}\,.
	\end{align*}
	\item[Step 11] Compute group-encoded batch evaluation $\gone{E}$
	\begin{align*}
		\gone{E}  = \frac{1}{\p{Z_H}(\chz)} & \gone{
		\begin{aligned}
			& \p{r}(\chz) + \pubinppoly(\chz) +  \alpha^2  \lag_1 (\chz) + \\
			& - \alpha \left( (\p{a}(\chz) + \beta \p{S_{\sigma 1}} (\chz) + \gamma) (\p{b}(\chz) + \beta \p{S_{\sigma 2}} (\chz) + \gamma) (\p{c}(\chz) + \gamma) \p{z}(\chz \omega) \right)
		\end{aligned}
           }\\
      + & \gone{v \p{r}(\chz) + v^2 \p{a}(\chz) + v^3 \p{b}(\chz) + v^4 \p{c}(\chz) + v^5 \p{S_{\sigma 1}}(\chz) + v^6 \p{S_{\sigma 2}}(\chz) + u \p{z}(\chz \omega) }\,.
	\end{align*}
\item[Step 12] Check whether the verification
 % $\vereq_\zkproof(\chi)$
  equation holds
	\begin{multline}
		\label{eq:ver_eq}
		\left( \gone{\p{W_{\chz}}(\chi)} + u \cdot \gone{\p{W_{\chz
                \omega}}(\chi)} \right) \bullet
		\gtwo{\chi} - %\\
		\left( \chz \cdot \gone{\p{W_{\chz}}(\chi)} + u \chz \omega \cdot
          \gone{\p{W_{\chz \omega}}(\chi)} + \gone{F} - \gone{E} \right) \bullet
        \gtwo{1} = 0\,.
	\end{multline}
  The verification equation is a batched version of the verification equation
  from \cite{AC:KatZavGol10} which allows the verifier to check openings of
  multiple polynomials in two points (instead of checking an opening of a single
  polynomial at one point).
\end{description}

\ncase{$\plonk$ simulator $\simulator_\chi(\srs, \td= \chi, \inp)$}\ \newline
The \plonk{} simulator proceeds as an honest prover would, except:
\begin{enumerate}
  \item In the first round, it sets $\wit = (\wit_i)_{i \in \range{1}{3 \noofc}}
    = \vec{0}$, and at random picks $b_1, \ldots, b_9$. Then it proceeds with
    that all-zero witness.
  \item In Round 3, it computes polynomial $\pt(X)$ honestly, however uses
    trapdoor $\chi$ to compute commitments
    $\p{t_{lo}}(\chi), \p{t_{mid}}(\chi), \p{t_{hi}}(\chi)$.
  \end{enumerate}
 
\subsection{Unique response property}
\begin{lemma}
	\label{lem:plonkprot_ur}
  Let $\PCOMp$ be commitment of knowledge with security $\epsk(\secpar)$,
  $\epsbind(\secpar)$-binding and has unique opening property with security
  $\epsop(\secpar)$, then probability that a $\ppt$ adversary $\adv$ breaks
  $\plonkprotfs$'s $\ur{2}$ property is at most
  $\epsop + 9 \cdot (\epsbind + \infrac{2}{\FF_p}) + \epss + \epsro$, where
  $\epsro$ is probability that a $\ppt$ adversary finds collision in a random
  oracle.
\end{lemma}
\begin{proof}
  Let
  $\adv$
  be an algebraic adversary tasked to break the $\ur{2}$-ness of
  $\plonkprotfs$. We show that the first 2 rounds of the protocol determines,
  along with the verifiers challenges, the rest of it.  This is done by game
  hops. In the games, the adversary outputs two proofs $\zkproof$ and
  $\zkproof'$ for the same statement.  To distinguish polynomials and
  commitments which an honest prover sends in the proof from the polynomials and
  commitments computed by the adversary we write the latter using indices $0$
  and $1$ (two indices as we have two transcripts), e.g.~to describe the
  quotient polynomial provided by the adversary we write $\p{t}^0$ and $\p{t}^1$
  instead of $\p{t}$ as in the description of the protocol.

  \ngame{0} In this game, the adversary wins if provides two
  transcripts that match on all $5$ messages sent by the prover or finds a
  collision in the random oracle. Since such two transcripts cannot break the
  unique response property, the adversary wins this game with probability
  $\epsro$ tops.

  \ngame{1} This game is identical to Game $\game{0}$ except that now the
  adversary additionally wins if it provides two transcripts that matches on the first four
  messages of the proof.

  \ncase{Game 0 to Game 1} We show that the probability that $\adv$
  wins in one game but does not in the other is negligible.  Observe that in
  Round 5 of the proof, the adversary is given a challenge $v$ and has to open
  the previously computed commitments. Since the transcripts match up to Round
  4, the challenge is the same in both. Hence, to be able to give two different
  openings in Round 5, $\adv$ has to break the unique opening property of the
  KZG commitment scheme which happens with probability $\epsop$ tops.
  % Since
  % there are two commitments that the adversary opens, by the union bound
  % probability that $\adv$ wins in one game but not the other is upper-bounded
  % by
  % $2 \cdot \epsop$.

  \ngame{2} This game is identical to Game $\game{1}$ except that now the
  adversary additionally wins if it provides two transcripts that matches on the
  first three messages of the proof.

  \ncase{Game 0 to Game 1} In Round 4 of the protocol the adversary
  has to provide evaluations
  $a_\chz = \p{a}(\chz), b_\chz = \p{b}(\chz), c_\chz = \p{c}(\chz), t_\chz =
  \p{t}(\chz), S_{1, \chz} = \p{S_{\sigma 1}}(\chz), s_{2, \chz} = \p{S_{\sigma
      2}}(\chz), z_\chz = \p{z}(\chz \omega)$ of previously committed
  polynomials, and compute and evaluate a linearization polynomial $\p{r}$.

  As before, the adversary cannot provide two different evaluations for the
  committed polynomials, since that would require breaking the evaluation
  binding property, which happens (by the union bound) with probability at most
  $7 \cdot (\epsbind + \infrac{2}{\abs{\FF_p}})$. The latter terms are since
  the adversary does not provide an opening for each of the commitment
  separately, but only in a batched way. That comes with $\infrac{1}{\FF_p}$ of
  security loss. Another $\infrac{1}{\FF_p}$ security loss comes from the fact
  that the verification of commitment openings are batched as well.

  The adversary cannot also provide two different linearization polynomials
  $\p{r^0}$ and $\p{r^1}$ evaluations $r^0_\chz$ and $r^1_\chz$ as the
  linearization polynomial is determined by values known to the verifier who
  also can compute a commitment to $\p{r}(X)$ equal $\gone{\p{r}(\chi)}$ by its
  own. The evaluation of $\p{r}$ provided by the adversary is later checked, as
  $\adv$ opens the commitment in Round 5. Hence, the probability that the
  adversary manages to build two convincing proofs that differ in evaluations
  $r_\chz$ and $r'_\chz$ is at most $\epsbind + \infrac{2}{\abs{\FF_p}}$.

  Hence, the probability that adversary wins in one game but does not in the
  other is upper-bounded by $8 \cdot (\epsbind + \infrac{2}{\FF_p})$

  \ngame{3} This game is identical to Game $\game{2}$ except that now the
  adversary additionally wins if it provides two transcripts that matches on the
  first two messages of the proof.

  \ncase{Game 2 to Game 3} In Round 3 the adversary computes the
  quotient polynomial $\pt(X)$ and provides its commitment that compounds of
  three separate commitments
  $\gone{\p{t_{lo}}(\chi), \p{t_{mid}}(\chi), \p{t_{hi}}(\chi)}$. Let
  $\gone{\p{t^0_{lo}}(\chi), \p{t^0_{mid}}(\chi), \p{t^0_{hi}}(\chi)}$ be the
  commitments output by the adversary in one transcript, and
  $\gone{\p{t^1_{lo}}(\chi), \p{t^1_{mid}}(\chi), \p{t^1_{hi}}(\chi)}$ the commitments
  provided in the other.
%
  Since the commitment scheme is deterministic, the adversary cannot come up
  with two different valid commitments for the same polynomial.

  If the adversary picks two different polynomials: $\p{t^0}(X)$, that is committed
  as $\gone{\p{t^0_{lo}}(\chi), \p{t^0_{mid}}(\chi), \p{t^0_{hi}}(\chi)}$, and
  $\p{t^1}(X)$ that is committed as
  $\gone{\p{t^1_{lo}}(\chi), \p{t^1_{mid}}(\chi), \p{t^1_{hi}}(\chi)}$, then one of
  them has to be computed incorrectly. 

  Importantly, polynomial $\p{t}(X)$ assures that the constraints of the system
  hold. Hence, the probability that one of $\p{t^0}(X)$, $\p{t^1}(X)$ is computed
  incorrectly, the adversary gives and opens acceptably a commitment to it, and
  the proof is acceptable, is upper bounded by the soundness of the proof system
  $\epss$. Alternatively, $\adv$ may compute a commitment to an invalid
  $\p{t^0}(X)$ (or $\p{t^1}(X)$) and later open the commitment at $\chz$ to
  $\p{t}(\chz)$. That is, give an evaluation from the correct polynomial
  $\p{t}(X)$. Since the commitment scheme is evaluation binding, probability of
  such event is upper bounded by $\epsbinding + \infrac{2}{\abs{\FF_p}}$.

  \ncase{Conclusion} Taking all the games together, probability that $\adv$ wins
  in Game 3 is upper-bounded by
  \[
    2 \cdot \epsop + 9 \cdot (\epsbind + \infrac{2}{\FF_p}) + \epsro + \epss.
  \]
  \qed
\end{proof}

\subsection{Forking special soundness}
\begin{lemma}
	\label{lem:plonkprot_ss}
	Let $\plonkprot$'s idealized verifier fails with probability $\epsid (\secpar)$, and
	$(\noofc + 2, 1)$-$\dlog$ problem be $\epsdlog (\secpar)$ hard. Then $\plonkprot$ is
	$(\epsid (\secpar) + \epsdlog (\secpar) , 3, 3 \noofc + 1)$-forking special sound against algebraic
	adversary $\adv$.
\end{lemma}

\begin{proof}
	The main idea of the proof is to show that an adversary who breaks forking special
	soundness can be used to break a $\dlog$ problem instance. The proof goes by game hops. Let $\tree$
	be the tree produced by $\tdv$ by rewinding $\adv$. Note that since the tree
	branches after Round 3, the instance $\inp$, commitments
	$\gone{\p{a} (\chi), \p{b} (\chi), \p{c} (\chi), \p{z} (\chi), \p{t_{lo}}
		(\chi), \p{t_{mid}} (\chi), \p{t_{hi}} (\chi)}$, and challenges
	$\alpha, \beta, \gamma$ are the same. The tree branches after the third round
	of the protocol where the challenge $\chz$ is presented, thus tree $\tree$ is
	build using different values of $\chz$. 
	%
	We consider the following games.
	
	\ncase{Game 0} In this game the adversary wins if
	% \begin{inparaenum}[(1)]
	% \item
	all the transcripts it produced are acceptable by the ideal verifier,
	i.e.~$\vereq_{\inp, \zkproof}(X) = 0$, cf.~\cref{eq:ver_eq}, and
	% \item
	none of commitments
	$\gone{\p{a} (\chi), \p{b} (\chi), \p{c} (\chi), \p{z} (\chi), \p{t_{lo}}
		(\chi), \p{t_{mid}} (\chi), \p{t_{hi}} (\chi)}$ use elements from a
	simulated proof, and
	% \item
	the extractor fails to extract a valid witness out of the proof.
	%\end{inparaenum}
	
	\ncase{Probability that $\adv$ wins Game 0 is negligible} Probability of
	$\adv$ winning this game is $\epsid(\secpar)$ as the protocol $\plonkprot$,
	instantiated with the idealised verification equation, is perfectly knowledge
	sound except with negligible probability of the idealised verifier failure
	$\epsid(\secpar)$. Hence for a valid proof $\zkproof$ for a statement $\inp$
	there exists a witness $\wit$, such that $\REL(\inp, \wit)$ holds. Note that
	since the $\tdv$ produces $(3 \noofc + 1)$ acceptable transcripts for
	different challenges $\chz$, it obtains the same number of different
	evaluations of polynomials
	$\p{a} (X), \p{b} (X), \p{c} (X), \p{z} (X), \p{t} (X)$. Since the transcripts
	are acceptable by an idealised verifier, the equality between polynomial
	$\p{t} (X)$ and combination of polynomials
	$\p{a} (X), \p{b} (X), \p{c} (X), \p{z} (X)$ described in Round 3 of the
	protocol holds. Hence, $\p{a} (X), \p{b} (X), \p{c} (X)$ encodes the valid
	witness for the proven statement. Since $\p{a} (X), \p{b} (X), \p{c} (X)$ are
	of degree at most $(\noofc + 2)$ and there is more than $(\noofc + 2)$ their
	evaluations known, $\extt$ can recreate polynomials' coefficients by interpolation
	and reveal the witness with probability $1$. Hence, the probability that
	extraction fails in that case is upper-bounded by probability of an idealised
	verifier failing $\epsid(\secpar)$, which is negligible.
	
	\ncase{Game 1} In this game the adversary additionally wins if
	%\begin{inparaenum}
	% \item
	it produces a transcript in $\tree$ such that
	$\vereq_{\inp, \zkproof}(\chi) = 0$, but $\vereq_{\inp, \zkproof}(X) \neq 0$,
	and
	% \item
	none of commitments
	$\gone{\p{a} (\chi), \p{b} (\chi), \p{c} (\chi), \p{z} (\chi), \p{t_{lo}}
		(\chi), \p{t_{mid}} (\chi), \p{t_{hi}} (\chi)}$ use elements from a
	simulated proof.
	% \end{inparaenum}
	The first condition means that the ideal verifier does not accept the proof,
	but the real verifier does.
	
	\ncase{Game 0 to Game 1} Assume the adversary wins in Game 1, but
	does not win in Game 0. We show that such adversary may be used to break the
	$\dlog$ assumption. More precisely, let $\tdv$ be an algorithm that for
	relation $\REL$ and randomly picked $\srs \sample \kgen(\REL)$ produces a tree
	of acceptable transcripts such that the winning condition of the game
	holds. Let $\rdvdlog$ be a reduction that gets as input an
	$(\noofc + 2, 1)$-dlog instance $\gone{1, \ldots, \chi^{\noofc+2}}, \gtwo{\chi}$ and is tasked to output $\chi$.
	
	The reduction $\rdvdlog$ proceeds as follows.
	\begin{enumerate}
			\item Build $\adv$'s SRS $\srs$ in the updatable setting using the input $\dlog$ instance by answering $\adv$'s queries for SRS updates and setting the honest update of the SRS to be $\srs$. Let $\srs'$ be the finalised SRS. Start $\tdv(\adv, \srs')$;
			\item Let $(1, \tree)$ be the output returned by $\tdv$. Let $\inp$ be a
			relation proven in $\tree$.  Consider a transcript $\zkproof \in \tree$ such
			that $\vereq_{\inp, \zkproof}(X) \neq 0$, but
			$\vereq_{\inp, \zkproof}(\chi') = 0$. Since $\adv$ is algebraic, all group
			elements included in $\tree$ are extended by their representation as a
			combination of the input $\GRP_1$-elements. Hence, all coefficients of the
			verification equation polynomial $\vereq_{\inp, \zkproof}(X)$ are known.
			\item Find $\vereq_{\inp, \zkproof}(X)$ zero points and find $\chi'$ among
			them.
			\item Let $\chi_1, \ldots, \chi_\ell$ be the partial trapdoors of $\adv$'s SRS updates. These trapdoors can be extracted by the reduction from the update proofs given by $\adv$.
			\item Return  $\chi = \chi' (\chi_1 \chi_2 \ldots \chi_\ell)^{-1}$.
      \end{enumerate}

      \ncase{Conclusion} Considering all the games together, probability that adversary wins in
      Game 1 is upper-bounded by Hence, the probability that the adversary wins Game 1 is
      upper-bounded by $\epsdlog(\secpar) = \epsid (\secpar)$.
      \qed
\end{proof}

\iffalse
\begin{proof}
	Let $\srs$ be $\plonkprot$'s SRS and denote by $\srs_1$ all SRS's
	$\GRP_1$-elements; that is,
	$\srs_1 = \gone{1, \chi, \ldots, \chi^{\noofc + 2}}$. Let $\tdv$ be an
	algebraic adversary that produces a statement $\inp$ and a
	$(1, 1, 3\noofc + 1, 1)$-tree of acceptable transcripts $\tree$.  Note that in
	all transcripts the instance $\inp$, proof elements
	$\gone{\p{a}(\chi), \p{b}(\chi), \p{c}(\chi), \p{z}(\chi), \p{t}(\chi)}$ and
	challenges $\alpha, \beta, \gamma$ are common as the transcripts share the
	first three rounds. The tree branches after the third round of the protocol
	where the challenge $\chz$ is presented, thus tree $\tree$ is build using
	different values of $\chz$.
	
	We consider two games.
	
	\ncase{Game 0} In this game the adversary wins if all the transcripts it
	produced are acceptable by the ideal verifier,
	i.e.~$\vereq_{\inp, \zkproof}(X) = 0$, cf.~\cref{eq:ver_eq}, yet the extractor
	fails to extract a valid witness out of them.
	
	Probability of $\tdv$ winning this game is $\epsid(\secpar)$ as the protocol
	$\plonkprot$, instantiated with the idealised verification equation, is
	perfectly sound except with negligible probability of the idealised verifier
	failure $\epsid(\secpar)$. Hence for a valid proof $\zkproof$ for a statement
	$\inp$ there exists a witness $\wit$, such that $\REL(\inp, \wit)$ holds. Note
	that since the $\tdv$ produces $(3 \noofc + 1)$ acceptable transcripts for
	different challenges $\chz$, it obtains the same number of different
	evaluations of polynomials $\p{a}, \p{b}, \p{c}, \p{z}, \p{t}$. Since the
	transcripts are acceptable by an idealised verifier, the equality between
	polynomial $\p{t}$ and combination of polynomials $\p{a}, \p{b}, \p{c}, \p{z}$
	described in Round 3 of the protocol holds. Hence, $\p{a}, \p{b}, \p{c}$
	encodes the valid witness for the proven statement. Since
	$\p{a}, \p{b}, \p{c}$ are of degree at most $(\noofc + 2)$ and there is more
	than $(\noofc + 2)$ their evaluations known, $\extt$ can recreate their
	coefficients by interpolation and reveal the witness with probability
	$1$. Hence, the probability that extraction fails in that case is
	upper-bounded by probability of an idealised verifier failing
	$\epsid(\secpar)$, which is negligible.
	
	\ncase{Game 1} In this game the adversary additionally wins if it produces a
	transcript in $\tree$ such that $\vereq_{\inp, \zkproof}(\chi) = 0$, but
	$\vereq_{\inp, \zkproof}(X) \neq 0$. That is, the ideal verifier does not
	accept the proof, but the real verifier does.
	
	\ncase{Game 0 to Game 1} Assume the adversary wins in Game 1, but
	does not win in Game 0. We show that such adversary may be used to break the
	$\dlog$ assumption. More precisely, let $\tdv$ be an adversary that for
	relation $\REL$ and randomly picked $\srs \sample \kgen(\REL)$ produces a tree
	of acceptable transcripts such that the winning condition of the game
	holds. Let $\rdvdlog$ be a reduction that gets as input an
	$(\noofc + 2, 1)$-dlog instance $\gone{1, \ldots, \chi^{\noofc}}, \gtwo{\chi}$
	and is tasked to output $\chi$. The reduction proceeds as follows---it gives
	the input instance to the adversary as the SRS. Let $(1, \tree)$ be the output
	returned by $\adv$. Let $\inp$ be a relation proven in $\tree$.  Consider a
	transcript $\zkproof \in \tree$ such that $\vereq_{\inp, \zkproof}(X) \neq 0$,
	but $\vereq_{\inp, \zkproof}(\chi) = 0$. Since the adversary is algebraic, all
	group elements included in $\tree$ are extended by their representation as a
	combination of the input $\GRP_1$-elements. Hence all coefficients of the
	verification equation polynomial $\vereq_{\inp, \zkproof}(X)$ are known and
	$\rdvdlog$ can find its zero points. Since
	$\vereq_{\inp, \zkproof}(\chi) = 0$, the targeted discrete log value $\chi$ is
	among them.  Hence, the probability that this event happens is upper-bounded
	by $\epsdlog(\secpar)$. \qed
\end{proof}
\fi

%\subsection{Forking soundness}
%\begin{lemma}
%\label{lem:plonkprot_ss}
%Let KZG be hiding with security $\epsh (\secpar)$, $\plonkprot$'s idealized
%verifier fail with probability $\epsid (\secpar)$, and $(\noofc + 2, 1)$-$\dlog$
%problem be $\epsdlog (\secpar)$ hard. Then $\plonkprot$ is
%$(\epsid (\secpar) + \epsdlog (\secpar) + 8 \cdot S \cdot \epsh(\secpar) +
%, 3, 3 \noofc + 1)$-forking sound against algebraic adversary
%$\adv$ who makes up to $S = \poly$ simulation oracle queries.\hamid{In the new definition, $\adv$ does not have access to the simulation oracle; so this should be changed!}
%\end{lemma}
%
%\begin{proof}
%  The main idea of the proof is to show that an adversary who breaks forking
%  soundness can be used to break hiding properties of the polynomial commitment
%  scheme or a $\dlog$ problem instance. The proof goes by game hops. Let $\tree$
%  be the tree produced by $\tdv$ by rewinding $\adv$. Note that since the tree
%  branches after Round 3, the instance $\inp$, commitments
%  $\gone{\p{a} (\chi), \p{b} (\chi), \p{c} (\chi), \p{z} (\chi), \p{t_{lo}}
%    (\chi), \p{t_{mid}} (\chi), \p{t_{hi}} (\chi)}$, and challenges
%  $\alpha, \beta, \gamma$ are the same. The tree branches after the third round
%  of the protocol where the challenge $\chz$ is presented, thus tree $\tree$ is
%  built using different values of $\chz$. 
%%
%  We consider the following games.
%
%  \ncase{Game 0} In this game the adversary wins if
% % \begin{inparaenum}[(1)]
% % \item
%  all the transcripts it produced are acceptable by the ideal verifier,
%    i.e.~$\vereq_{\inp, \zkproof}(X) = 0$, cf.~\cref{eq:ver_eq}, and
%    % \item
%    none of commitments
%    $\gone{\p{a} (\chi), \p{b} (\chi), \p{c} (\chi), \p{z} (\chi), \p{t_{lo}}
%      (\chi), \p{t_{mid}} (\chi), \p{t_{hi}} (\chi)}$ use elements from a
%    simulated proof, and
%    % \item
%    the extractor fails to extract a valid witness out of the proof.
%  %\end{inparaenum}
%
%  \ncase{Probability that $\adv$ wins Game 0 is negligible} Probability of
%  $\adv$ winning this game is $\epsid(\secpar)$ as the protocol $\plonkprot$,
%  instantiated with the idealised verification equation, is perfectly knowledge
%  sound except with negligible probability of the idealised verifier failure
%  $\epsid(\secpar)$. Hence for a valid proof $\zkproof$ for a statement $\inp$
%  there exists a witness $\wit$, such that $\REL(\inp, \wit)$ holds. Note that
%  since the $\tdv$ produces $(3 \noofc + 1)$ acceptable transcripts for
%  different challenges $\chz$, it obtains the same number of different
%  evaluations of polynomials
%  $\p{a} (X), \p{b} (X), \p{c} (X), \p{z} (X), \p{t} (X)$. Since the transcripts
%  are acceptable by an idealised verifier, the equality between polynomial
%  $\p{t} (X)$ and combination of polynomials
%  $\p{a} (X), \p{b} (X), \p{c} (X), \p{z} (X)$ described in Round 3 of the
%  protocol holds. Hence, $\p{a} (X), \p{b} (X), \p{c} (X)$ encodes the valid
%  witness for the proven statement. Since $\p{a} (X), \p{b} (X), \p{c} (X)$ are
%  of degree at most $(\noofc + 2)$ and there is more than $(\noofc + 2)$ their
%  evaluations known, $\extt$ can recreate polynomials' coefficients by interpolation
%  and reveal the witness with probability $1$. Hence, the probability that
%  extraction fails in that case is upper-bounded by probability of an idealised
%  verifier failing $\epsid(\secpar)$, which is negligible.
%
%  \ncase{Game 1} In this game the adversary additionally wins if
%  %\begin{inparaenum}
%  % \item
%  it produces a transcript in $\tree$ such that
%  $\vereq_{\inp, \zkproof}(\chi) = 0$, but $\vereq_{\inp, \zkproof}(X) \neq 0$,
%  and
%  % \item
%  none of commitments
%  $\gone{\p{a} (\chi), \p{b} (\chi), \p{c} (\chi), \p{z} (\chi), \p{t_{lo}}
%    (\chi), \p{t_{mid}} (\chi), \p{t_{hi}} (\chi)}$ use elements from a
%  simulated proof.
%  % \end{inparaenum}
%  The first condition means that the ideal verifier does not accept the proof,
%  but the real verifier does.
%
%  \ncase{Game 0 to Game 1} Assume the adversary wins in Game 1, but
%  does not win in Game 0. We show that such adversary may be used to break the
%  $\dlog$ assumption. More precisely, let $\tdv$ be an algorithm that for
%  relation $\REL$ and randomly picked $\srs \sample \kgen(\REL)$ produces a tree
%  of acceptable transcripts such that the winning condition of the game
%  holds. Let $\rdvdlog$ be a reduction that gets as input an
%  $(\noofc + 2, 1)$-dlog instance $\gone{1, \ldots, \chi^{\noofc}}, \gtwo{\chi}$ \hamid{shouldn't be $\gone{1, \ldots, \chi^{\noofc+2}}, \gtwo{\chi}$}
%  and is tasked to output $\chi$.
%
%  The reduction $\rdvdlog$ proceeds as follows.
%  \hamid{\begin{enumerate}
%  \item Build a SRS $\srs$ using the input $\dlog$ instance. Answer $\adv$'s queries for SRS updates and set the honest update of the SRS to be $\srs$. Let $\srs'$ be the finalised SRS. Start $\tdv(\adv, \srs')$;
%  \item Let $(1, \tree)$ be the output returned by $\tdv$. Let $\inp$ be a
%    relation proven in $\tree$.  Consider a transcript $\zkproof \in \tree$ such
%    that $\vereq_{\inp, \zkproof}(X) \neq 0$, but
%    $\vereq_{\inp, \zkproof}(\chi') = 0$. Since $\adv$ is algebraic, all group
%    elements included in $\tree$ are extended by their representation as a
%    combination of the input $\GRP_1$-elements. Hence, all coefficients of the
%    verification equation polynomial $\vereq_{\inp, \zkproof}(X)$ are known.
%  \item Find $\vereq_{\inp, \zkproof}(X)$ zero points and find $\chi'$ among
%    them.
%  \item Let $\chi_1, \ldots, \chi_\ell$ be the partial trapdoors of $\adv$'s SRS updates. These trapdoors can be extracted by the reduction from the update proofs given by $\adv$.
%  \item Return  $\chi = \chi' (\chi_1 \chi_2 \ldots \chi_\ell)^{-1}$.
%  \end{enumerate}}
%  Hence, the probability that the adversary wins Game 1 is upper-bounded by
%  $\epsdlog(\secpar)$.
%
%  \iffalse
%  \ncase{Game 0} In this game the adversary $\adv$ wins if $\tdv$ fails to
%  output tree $\tree$ such that $\extt$ is able to extract a witness out of it
%  and none of commitments
%  $\gone{\p{a} (\chi), \p{b} (\chi), \p{c} (\chi), \p{z} (\chi), \p{t_{lo}}
%    (\chi), \p{t_{mid}} (\chi), \p{t_{hi}} (\chi)}$ use elements from a
%  simulated proof.
%
%  \ncase{Probability that $\adv$ wins Game 0 is negligible}
%  \michals{10.09}{Reduction to knowledge soundness}
%  Let $\adv$ be considered forking soundness adversary. 
%  Let $\adv$ be an adversary that breaks forking soundness.
%  Note that if
%  $\p{a}(X), \p{b} (X), \p{c} (X)$ contain witness at their coefficients, then 
%  We use $\adv$ to
%  build a reduction $\rdv$ that breaks soundness of $\plonkprotfs$. Let $\inp$
%  be the instance the adversary output and $\zkproof$ its proof. The
%  reduction proceeds as follows:
%  \begin{enumerate}
%  \item 
%  \end{enumerate}
%  \fi
%
%  \ncase{Game 2} In this game the adversary additionally wins if at least one of
%  the commitments $\p{a} (\chi), \p{b} (\chi), \p{c} (\chi), \p{z} (\chi)$
%  utilizes a commitment that comes from a simulated proof; for example, $\adv$
%  could compute its commitment to $\p{c} (X)$ as follows: it picks a polynomial
%  $\p{p} (X)$, computes $\gone{\p{p} (\chi)}$, and outputs commitment
%  $\gone{\p{c} (\chi)} = \gone{\p{p} (\chi)} + c$, where $c$ is a commitment
%  output by a simulator. In the following, w.l.o.g, we assume that $\adv$ uses
%  some simulated element to compute commitment $\gone{\p{c} (\chi)}$.
%
%  \ncase{Game 1 to Game 2} Given adversary $\adv$ that wins in Game 2, but not
%  in Game 1, we show a reduction $\rdv$ that uses $\adv$ and $\tdv$ to break
%  hiding property
%  % \michals{9.9}{Define hiding property --
%  % w.r.t.~masking, adversary can get evaluation at number of points (here --
%  % 1)}
%  of the commitment scheme. $\rdv$ proceeds as follows:
%  \begin{enumerate}
%  \item \hamid{Given polynomial commitment SRS $\srs_{\PCOM}$, produce $\plonk$'s SRS
%    $\srs$.}
%  \item Pick random polynomials $\p{p} (X), \p{p'} (X) \in \FF^{< |\HHH|} [X]$,
%    hiding parameter $k = 2$ and send them to the polynomial commitment
%    challenger $\cdv$.
%  \item From the challenger get the challenge commitment $c$.
%  \item Let $S$ be the upper bound on the number of simulator oracles queries
%    the adversary can make. %\michals{9.9}{Note -- new bound!}
%  \item Guess which simulator's response is going to be used by $\adv$ in its proof. Let $s$ be the index of this response.\hamid{should be changed!}
%  \item Guess which of the simulated polynomials in response $s$ will be
%    used. Let $i$ be the index of this polynomial.\hamid{should be changed!}
%  \item Let $\tdv'$ be an algorithm that behaves exactly as $\tdv$, except when
%    $\adv$ asks for $s$-th simulated proof, $\tdv'$'s internal procedure $\bdv'$
%    provides $\adv$ with a simulated proof such that instead of randomly picked
%    commitment $\p{c} (\chi)$ it gives $c$.
%    \michals{9.9}{Alternatively, we can parametrize $\tdv$ by $\bdv$.}
%  \item Start $\tdv'(\adv, \srs)$ and get the tree $\tree$.
%  % \item If indices $s$ or $i$ have not been guessed correctly, rewind $\adv$ to
%  %   the beginning and pick new $s$ and $i$. Since $S = \poly$ probability that
%  %   the correct $s$ will be guessed in polynomial time is overwhelming. That is,
%  %   the reduction works in expected polynomial time. Similarly, $i$ takes values
%  %   from $\range{1}{4}$, hence probability that $\rdv$ guesses $i$ in polynomial
%  %   time is overwhelming. 
%  \item Since $\tree$ contains $\noofc + 1$ evaluations of $\p{c} (X)$, the
%    polynomial can be reconstructed. 
%  \item Since $\adv$ is algebraic, $\rdv$ learns composition of $\p{c} (X)$ in
%    the $\srs$ and simulated elements. 
%  \item Hence $\rdv$ learns whether $c$ is a commitment to $\p{p} (X)$ or
%    $\p{p'} (X)$.
%  \item $\rdv$ returns its guessing bit to $\cdv$.
%  \end{enumerate}
%
%  \ncase{Game 3} In this game the adversary additionally wins if at least one of
%  the commitments $\p{t_{lo}} (\chi), \p{t_{mid}} (\chi), \p{t_{hi}} (\chi)$
%  comes from a simulated proof.
%
%  \ncase{Game 2 to Game 3} Given adversary $\adv$ that wins in Game
%  3, but not in Game 2, we show a reduction $\rdv$ that uses $\adv$ and $\tdv$
%  to break hiding property
%  % \michals{9.9}{Define hiding property --
%  %   w.r.t.~masking, adversary can get evaluation at number of points (here --
%  %   1)}
%  of the commitment scheme. $\rdv$ proceeds as follows:
%  \begin{enumerate}
%  \item Guess the simulation query index $s$ the polynomial(s) come from and
%    whether the polynomial is $\p{t_{lo}} (X), \p{t_{mid}} (X)$, or
%    $\p{t_{hi}} (X)$. Denote by $i \in \range{1}{3}$ the index of the guessed
%    polynomial. W.l.o.g.~assume $i = 1$, i.e.~it is $\p{t_{lo}} (X)$.
%  \item Produce two random polynomials $\p{p_0} (X)$ and $\p{p_1} (X)$ and
%    send them to the challenger $\cdv$. Get commitment $c$.
%  \item Let $\tdv'$ be an algorithm that behaves exactly as $\tdv$, except when
%    $\adv$ asks for $s$-th simulated proof, $\tdv'$'s internal procedure $\bdv'$
%    provides $\adv$ with a simulated proof such that:
%    \begin{enumerate}
%    \item Start making the simulated proof as a trapdoor-less simulator would.
%    \item Before polynomials $\p{t_{lo}} (X), \p{t_{mid}}, \p{t_{hi}}$ are
%      computed, pick random $\chz$ and get evaluation $p_\chz = \p{p_b} (\chz)$,
%      i.e.~evaluate the polynomial in $c$ at $\chz$.
%      % \michals{9.9}{If we give
%      %   hiding adversary possibility to evaluate polynomials, we need to use
%      %   masking}
%    \item Let $\ev{t_{lo}}$ be the evaluation of the simulated $\p{t_{lo}} (\chz)$.
%    \item Pick $r$ such that $p_\chz + r = \p{t_{lo}} (\chz)$.
%    \item For the commitment of $\p{t_{lo}}$ output $c' = c + \gone{r}$.
%    \item Compute the rest of the simulated proof as a simulator would.
%    \end{enumerate}
%  \item Let $\tdv'$ be an algorithm that behaves exactly as $\tdv$, except when
%    $\adv$ asks for $s$-th simulated proof, $\tdv'$'s internal procedure $\bdv'$
%    provides $\adv$ with a simulated proof such that instead of a simulated
%    $\p{t_{lo}} (\chi)$  it gives $c'$.
%  \item Start $\tdv'(\adv, \srs)$ and get the tree $\tree$.
%  % \item If indices $s$ or $i$ have not been guessed correctly, rewind $\adv$ to
%  %   the beginning and pick new $s$ and $i$. Since $S = \poly$ probability that
%  %   the correct $s$ will be guessed in polynomial time is overwhelming. That is,
%  %   the reduction works in expected polynomial time. Similarly, $i$ takes values
%  %   from $\range{1}{3}$, hence probability that $\rdv$ guesses $i$ in polynomial
%  %   time is overwhelming. 
%  \item Since $\tree$ contains $\noofc + 1$ evaluations of $\p{t_{lo}} (X)$, the
%    polynomial can be reconstructed. 
%  \item Since $\adv$ is algebraic, $\rdv$ learns composition of $\p{t_{lo}} (X)$ in
%    the $\srs$ and simulated elements. 
%  \item Hence $\rdv$ learns whether $c$ is a commitment to $\p{p}$ or $\p{p'}$.
%  \item $\rdv$ returns its guessing bit to $\cdv$.
%  \end{enumerate}
%  % \ncase{Conclusion}
%  % Since probability 
%\end{proof}
%
%\iffalse
%\begin{proof}
%  Let $\srs$ be $\plonkprot$'s SRS and denote by $\srs_1$ all SRS's
%  $\GRP_1$-elements; that is,
%  $\srs_1 = \gone{1, \chi, \ldots, \chi^{\noofc + 2}}$. Let $\tdv$ be an
%  algebraic adversary that produces a statement $\inp$ and a
%  $(1, 1, 3\noofc + 1, 1)$-tree of acceptable transcripts $\tree$.  Note that in
%  all transcripts the instance $\inp$, proof elements
%  $\gone{\p{a}(\chi), \p{b}(\chi), \p{c}(\chi), \p{z}(\chi), \p{t}(\chi)}$ and
%  challenges $\alpha, \beta, \gamma$ are common as the transcripts share the
%  first three rounds. The tree branches after the third round of the protocol
%  where the challenge $\chz$ is presented, thus tree $\tree$ is build using
%  different values of $\chz$.
%
%  We consider two games.
%
%  \ncase{Game 0} In this game the adversary wins if all the transcripts it
%  produced are acceptable by the ideal verifier,
%  i.e.~$\vereq_{\inp, \zkproof}(X) = 0$, cf.~\cref{eq:ver_eq}, yet the extractor
%  fails to extract a valid witness out of them.
%
%  Probability of $\tdv$ winning this game is $\epsid(\secpar)$ as the protocol
%  $\plonkprot$, instantiated with the idealised verification equation, is
%  perfectly sound except with negligible probability of the idealised verifier
%  failure $\epsid(\secpar)$. Hence for a valid proof $\zkproof$ for a statement
%  $\inp$ there exists a witness $\wit$, such that $\REL(\inp, \wit)$ holds. Note
%  that since the $\tdv$ produces $(3 \noofc + 1)$ acceptable transcripts for
%  different challenges $\chz$, it obtains the same number of different
%  evaluations of polynomials $\p{a}, \p{b}, \p{c}, \p{z}, \p{t}$. Since the
%  transcripts are acceptable by an idealised verifier, the equality between
%  polynomial $\p{t}$ and combination of polynomials $\p{a}, \p{b}, \p{c}, \p{z}$
%  described in Round 3 of the protocol holds. Hence, $\p{a}, \p{b}, \p{c}$
%  encodes the valid witness for the proven statement. Since
%  $\p{a}, \p{b}, \p{c}$ are of degree at most $(\noofc + 2)$ and there is more
%  than $(\noofc + 2)$ their evaluations known, $\extt$ can recreate their
%  coefficients by interpolation and reveal the witness with probability
%  $1$. Hence, the probability that extraction fails in that case is
%  upper-bounded by probability of an idealised verifier failing
%  $\epsid(\secpar)$, which is negligible.
%
%  \ncase{Game 1} In this game the adversary additionally wins if it produces a
%  transcript in $\tree$ such that $\vereq_{\inp, \zkproof}(\chi) = 0$, but
%  $\vereq_{\inp, \zkproof}(X) \neq 0$. That is, the ideal verifier does not
%  accept the proof, but the real verifier does.
%
%  \ncase{Game 0 to Game 1} Assume the adversary wins in Game 1, but
%  does not win in Game 0. We show that such adversary may be used to break the
%  $\dlog$ assumption. More precisely, let $\tdv$ be an adversary that for
%  relation $\REL$ and randomly picked $\srs \sample \kgen(\REL)$ produces a tree
%  of acceptable transcripts such that the winning condition of the game
%  holds. Let $\rdvdlog$ be a reduction that gets as input an
%  $(\noofc + 2, 1)$-dlog instance $\gone{1, \ldots, \chi^{\noofc}}, \gtwo{\chi}$
%  and is tasked to output $\chi$. The reduction proceeds as follows---it gives
%  the input instance to the adversary as the SRS. Let $(1, \tree)$ be the output
%  returned by $\adv$. Let $\inp$ be a relation proven in $\tree$.  Consider a
%  transcript $\zkproof \in \tree$ such that $\vereq_{\inp, \zkproof}(X) \neq 0$,
%  but $\vereq_{\inp, \zkproof}(\chi) = 0$. Since the adversary is algebraic, all
%  group elements included in $\tree$ are extended by their representation as a
%  combination of the input $\GRP_1$-elements. Hence all coefficients of the
%  verification equation polynomial $\vereq_{\inp, \zkproof}(X)$ are known and
%  $\rdvdlog$ can find its zero points. Since
%  $\vereq_{\inp, \zkproof}(\chi) = 0$, the targeted discrete log value $\chi$ is
%  among them.  Hence, the probability that this event happens is upper-bounded
%  by $\epsdlog(\secpar)$. \qed
%\end{proof}
%\fi

\subsection{Trapdoor-less simulatability of Plonk}
\begin{lemma}
  \label{lem:plonk_hvzk}
  Let $\plonkprot$ be zero knowledge with security $\epszk(\secpar)$. Let
  $(\pR, \pS, \pT, \pf, 1)$-uber assumption for $\pR, \pS, \pT, \pf$ as defined
  in \cref{eq:uber} hold with security $\epsuber(\secpar)$. Then $\plonkprot$ 2-programmable trapdoor-less simulatable zero knowledge with simulator $\simulator$
  that does not require a SRS trapdoor with security
  $\epszk(\secpar) + \epsuber(\secpar)$.\footnote{The simulator works as a simulator for proofs
    that are zero-knowledge in the standard model. However, we do not say that
    $\plonk$ is HVZK in the standard model as proof of that \emph{requires} the
    SRS simulator.}
\end{lemma}

%Due to page limit, the proof has been moved to \cref{sec:plonk-tls-proof}
\begin{proof}
  As noted in \cref{def:upd-scheme}, subvertible zero knowledge implies updatable zero
  knowledge. Hence, here we show that Plonk is TLS even against algebraic adversaries who picks
  the SRS on its own.
  
  The proof goes by game-hopping. The environment that controls the games
  provides starts the adversary who sets a SRS $\srs$, then the adversary outputs an
  instance--witness pair $(\inp, \wit)$ and, depending on the game, is provided
  with either real or simulated proof for it. In the end of the game the
  adversary outputs either $0$ if it believes that the proof it saw was provided
  by the simulator and $1$ in the other case.

  \ngame{0} In this game $\adv(\secparam)$ picks an SRS $\srs$ and instance--witness pair
  $(\inp, \wit)$ and gets a real proof $\zkproof$ for it.

  \ngame{1} In this game for $\adv(\secparam)$ picks an SRS $\srs$ and an instance--witness pair
  $(\inp, \wit)$ and gets a proof $\zkproof$ that is simulated by a simulator
  $\simulator_\chi$ which utilises for the simulation the SRS trapdoor and
  proceeds as described in \cref{sec:plonk_explained}.

  \ncase{Game 0 to Game 1} Since $\plonk$ is (subvertible) zero-knowledge,
  probability that $\adv$ outputs a different bit in both games is negligible.
  Hence
  \(
	\abs{\prob{\game{0}} - \prob{\game{1}}} \leq \epszk(\secpar).
\)

\ngame{2} In this game $\adv(\secparam)$ picks an SRS $\srs$ and instance--witness pair
$(\inp, \wit)$ and gets a proof $\zkproof$ simulated by the simulator
$\simulator$ which proceeds as follows.

In Round 1 the simulator  picks randomly both the randomisers $b_1, \ldots, b_6$ and
sets $\wit_i = 0$ for $i \in \range{1}{3\noofc}$. Then $\simulator$
outputs $\gone{\p{a}(\chi), \p{b}(\chi), \p{c}(\chi)}$. For the first round
challenge, the simulator picks permutation argument challenges $\beta, \gamma$
randomly.

In Round 2, the simulator computes $\p{z}(X)$ from
the newly picked randomisers $b_7, b_8, b_9$ and coefficients of polynomials
$\p{a}(X), \p{b}(X), \p{c}(X)$. Then it evaluates $\p{z}(X)$ honestly and outputs
$\gone{\p{z}(\chi)}$. Challenge $\alpha$ that should be sent by the verifier
after Round 2 is picked by the simulator at random.

In Round 3 the simulator starts by picking at random a challenge $\chz$, which
in the real proof comes as a challenge from the verifier sent \emph{after} Round
3. Then $\simulator$ computes evaluations
\(\p{a}(\chz), \p{b}(\chz), \p{c}(\chz), \p{S_{\sigma 1}}(\chz), \p{S_{\sigma
    2}}(\chz), \pubinppoly(\chz), \lag_1(\chz), \p{Z_H}(\chz),\allowbreak
\p{z}(\chz\omega)\) and computes $\p{t}(X)$ honestly. Since for a random
$\p{a}(X), \p{b}(X), \p{c}(X), \p{z}(X)$ the constraint system is (with
overwhelming probability) not satisfied and the constraints-related polynomials
are not divisible by $\p{Z_H}(X)$, hence $\p{t}(X)$ is a rational function
rather than a polynomial. Then, the simulator evaluates $\p{t}(X)$ at $\chz$ and
picks randomly a degree-$(3 \noofc - 1)$ polynomial $\p{\tilde{t}}(X)$ such that
$\p{t}(\chz) = \p{\tilde{t}}(\chz)$ and publishes a commitment
$\gone{\p{\tilde{t}_{lo}}(\chi), \p{\tilde{t}_{mid}}(\chi),
  \p{\tilde{t}_{hi}}(\chi)}$. After this round the simulator outputs $\chz$ as a
challenge.

In the next round, the simulator computes polynomial $\p{r}(X)$ as an honest
prover would, cf.~\cref{sec:plonk_explained} and evaluates $\p{r}(X)$ at $\chz$.

The rest of the evaluations are already computed, thus $\simulator$ simply
outputs
\(
  \p{a}(\chz), \p{b}(\chz), \p{c}(\chz), \p{S_{\sigma 1}}(\chz), \p{S_{\sigma
      2}}(\chz), \p{t}(\chz), \p{z}(\chz \omega)\,.
\)
After that it picks randomly the challenge $v$, proceeds in the last round as an
honest prover would proceed and outputs the final challenge, $u$, by picking it
at random as well.

\ncase{Game 1 to Game 2} We now describe the reduction $\rdv$ which
relies on the $(\pR, \pS, \pT, \pF, 1)$-uber assumption, cf.~\cref{sec:uber_assumption}
where $\pR, \pS, \pT, \pF$ are polynomials over variables
$\vB = B_1, \ldots, B_9$ and are defined as follows. Let
$E = \smallset{\smallset{2}, \smallset{3, 4}, \smallset{5, 6}, \smallset{7, 8,
    9}}$ and $E' = E \setminus \smallset{2}$. Let
\begin{align}
\label{eq:uber}
\pF(\vB) & = \smallset{B_1} \cup \smallset{B_1B_i \mid i \in A,\ A \in E'} \cup
             \smallset{B_1B_iB_j \mid i \in A, j \in B,\ A, B \in E', B
             \neq A} \cup \notag\\
           & \smallset{B_1B_iB_jB_k \mid i \in A, j \in
             B, k \in C,\ A, B, C \in E', A \neq B \neq C \neq A}\notag\,,\\
  \pR(\vB) & = \smallset{B_i \mid i \in A,\ A \in E} \cup \smallset{B_i B_j \mid i \in
             A, j \in B,\ A \neq B, A, B \in E} \cup \\ 
           & \smallset{B_i B_j B_k \mid i \in A,\ j \in
             B,\ k \in C,\
             A, B, C \text{ all different and in } E} \cup \notag \\
           & \smallset{B_i B_j B_k B_l \mid i \in A,\ j \in B,\ k \in C,\ l \in D,\
             A, B, C, D \text{ all different and in } E} \notag \\
           & \setminus \pF(\vB)\,,\notag \\
  \pS(\vB) & = \emptyset, \qquad \pT(\vB) = \emptyset.
\end{align}
That is, the elements of $\pR$ are all singletons, pairs, triplets and
quadruplets of $B_i$ variables that occur in polynomial $\pt(\vB)$ except the
challenge element $\pf(\vB)$ which are all elements that depends on a variable
$B_1$. Variables $\vB$ are evaluated to randomly picked
$\vb = b_1, \ldots, b_9$.

The reduction $\rdv$ learns $\gone{\pR}$ and challenge
$\gone{\vec{w}} = \gone{w_1, \ldots, w_{12}}$ where $\vec{w}$ is either a vector
of evaluations $\pF(\vb)$ or a sequence of random values $y_1, \ldots, y_{12}$,
for the sake of concreteness we state $w_1 = b_1$ or $w_1 = y_1$ (depending on
the chosen random bit).

Then it starts $\adv$ and learns $\srs$, since $\adv$ is algebraic, the reduction also learns trapdoor $\chi$. Then $\rdv$ picks $\chz$. Elements $b_i$ are interpreted as polynomials in $X$ that are
evaluated at $\chi$, i.e. $b_i = b_i(\chi)$. Next, $\rdv$ sets for
$\xi_i, \zeta_i \sample \FF_p$
\(
  \gone{\p{\tb}_1(X)} =
(X - \chz)(X - \ochz) \gone{w_1}(X) + \xi_i (X - \chz) \gone{1} +
\zeta_i (X - \ochz) \gone{1}, % \text{ for } i \in % \range{1}{9}, u_1
\),
and
\(
  \gone{\p{\tb}_i(X)} =
(X - \chz)(X - \ochz) \gone{b_i}(X) + \xi_i (X - \chz) \gone{1} +
\zeta_i (X - \ochz) \gone{1}, % \text{ for } i \in % \range{1}{9}, u_1
\) 
for $i \in \range{2}{9}$.

Denote by $\tb_i$ evaluations of $\p{\tb}_i$ at $\chi$.  The reduction computes
all
$\gone{\tb_i \tb_j}, \gone{\tb_i \tb_j \tb_k}, \gone{\tb_i \tb_j \tb_k \tb_l}$
such that $\gone{B_i B_j, B_i B_j B_k, B_i B_j B_k B_l} \in \pR$.  This is
possible since $\rdv$ knows all singletons $\gone{w_1, b_2, \ldots, b_9}$ and pairs
$\gone{b_i b_j} \in \pR$ which can be used to compute all required pairs
$\gone{\tb_i \tb_j}$:
\begin{align*}
\gone{\tb_i \tb_j} 
& = ((\chi - \chz)(\chi - \ochz)\gone{b_i} + \xi_i (\chi - \chz)\gone{1} +
\zeta_i (\chi - \ochz) \gone{1}) 
\cdot \\
 & ((\chi - \chz)(\chi - \ochz)\gone{b_j} + \xi_j (\chi - \chz)\gone{1} +
\zeta_j (\chi - \ochz) \gone{1}) = \\
 & ((\chi - \chz)(\chi - \ochz))^2 \gone{b_i b_j} +  ((\chi - \chz)(\chi -
 \ochz)\gone{b_i} (\xi_j (\chi - \chz) \gone{1} + \zeta_j (\chi - \ochz)
 \gone{1}) + \\
 & ((\chi - \chz)(\chi -
 \ochz)\gone{b_j} (\xi_i (\chi - \chz) \gone{1} + \zeta_i (\chi - \ochz)
 \gone{1}) + \psi,
\end{align*}
where $\psi$ compounds of $\xi_i, \xi_j, \zeta_i, \zeta_j, \chz, \ochz, \chi$ which
are all known by $\rdv$ and no $b_i$ nor $b_j$.
Analogously for the triplets and quadruplets and elements dependent on~$\vec{w}$. 

Next the reduction obtains from $\adv$ an instance--witness pair $(\inp, \wit)$.  $\rdv$ now
prepares a simulated proof as follows:
\begin{compactdesc} 
\item[Round 1] $\rdv$ computes $\gone{\pa(\chi)}$ using as
randomisers $\gone{\tb_1}, \gone{\tb_2}$ and setting $\wit_i = 0$, for $i
\in \range{1}{3 \noofc}$. Similarly it computes
$\gone{\pb(\chi)}, \gone{\pc(\chi)}$.  $\rdv$ publishes the obtained values
and picks a Round 1 challenge $\beta, \gamma$ at random.  Note that regardless
$w_1 = b_1$ or a random element, $\gone{a(\chi)}$ is random. Thus $\rdv$'s
output has the same distribution as output of a real prover.  
\item[Round 2]
$\rdv$ computes $\gone{\pz(\chi)}$ using $\tb_7, \tb_8, \tb_9$ and publishes
it. Then it picks randomly the challenge $\alpha$. This round output is
independent on $b_1$ thus $\rdv$'s output is indistinguishable from the prover's. 
\item[Round 3] The reduction computes
  $\p{t_{lo}}(\chi), \p{t_{mid}}(\chi), \p{t_{hi}}(\chi)$, which all depend on
  $b_1$. To that end $\gone{\tb_1}$ is used. Note that if $\vec{w}$ is a vector
  of $\pF(b_1, \ldots, b_9)$ evaluations then
  $\gone{\p{t_{lo}}(\chi), \p{t_{mid}}(\chi), \p{t_{hi}}(\chi)}$ is the same as
  the real prover's. Alternatively, if $\vec{w}$ is a vector of random values,
  then $\p{t_{lo}}(\chi), \p{t_{mid}}(\chi), \p{t_{hi}}(\chi)$ are all random
  polynomials which evaluates at $\chz$ to the same value as the polynomials
  computed by the real prover. That is, in that case
  $\p{t_{lo}}(\chi), \p{t_{mid}}(\chi), \p{t_{hi}}(\chi)$ are as the simulator
  $\simulator$ would compute. Eventually, $\rdv$ outputs $\chz$.
\item[Round 4] The reduction outputs
  $\pa(\chz), \pb(\chz), \pc(\chz), \p{S_{\sigma 1}}(\chz), \p{S_{\sigma 2} (\chz)},
  \pt(\chz), \pz(\ochz)$.  For the sake of concreteness, denote by
  $S = \smallset{\pa, \pb, \pc, \pt, \pz}$. Although for a polynomial $\p{p} \in S$,
  reduction $\rdv$ does not know $\p{p}(\chi)$ or even do not know all the
  coefficients of $\p{p}$, the polynomials in $S$ was computed such that the
  reduction always knows their evaluation at $\chz$ and $\ochz$.
\item[Round 5] $\rdv$ computes the openings of the polynomial commitments assuring
  that evaluations at $\chz$ it provided were computed honestly.
\end{compactdesc}

If the adversary $\adv$'s output distribution differ in Game $\game{1}$ and
$\game{2}$ then the reduction uses it to distinguish between
$\vec{w} = \pF(b_1, \ldots, b_9)$ and $\vec{w}$ being random, thus
\( \abs{\prob{\game{1}} - \prob{\game{2}}} \leq \epsuber(\secpar).  \)

\ncase{Conclusion} Probability that the adversary's outputs differ in Game 0 and Game
2 is upper-bounded by $\leq \epszk(\secpar) + \epsuber(\secpar)$. \qed
\end{proof}

\begin{lemma}[Simulation for false statement]
  Let $\simulator$ be a trapdoor-less simulator from Game 2 above, then for every
  $\inp \not\in \LANG_\REL$ holds
  \[
    \Pr \left[ \verifier (\srs, \inp, \zkproof) = 1 \left|\, \srs \gets \adv
        (\secparam), \zkproof \gets \simulator_2' (\srs, \inp) \right.  \right] \geq 1 - \eps (\secpar),
  \]
  for some negligible $\eps (\secpar)$.\michals{15.10}{Could we have just ``$= 1$'' here?}
\end{lemma}

\subsection{Simulation extractability of~$\plonkprotfs$}
Since \cref{lem:plonkprot_ur,lem:plonkprot_ss} hold, $\plonkprot$ is $\ur{2}$ and
forking special sound. We now make use of \cref{thm:se} and show that
$\plonkprot_\fs$ is simulation-extractable as defined in \cref{def:updsimext}.

\begin{corollary}[Simulation extractability of $\plonkprot_\fs$]
\label{thm:plonkprotfs_se}
Assume an idealised $\plonkprot$ verifier fails at most with probability
$\epsid(\secpar)$, the discrete logarithm advantage is bounded by $\epsdlog(\secpar)$
and the $\PCOMp$ is a commitment of knowledge with security $\epsk(\secpar)$, binding
security $\epsbind(\secpar)$ and has unique opening property with security
$\epsop(\secpar)$. Let $\ro\colon \bin^* \to \bin^\secpar$ be a random oracle. Let
$\advse$ be an adversary that can make up to $q$ random oracle queries, up to $S$
simulation oracle queries, and outputs an acceptable proof for $\plonkprotfs$ with
probability at least $\accProb$. Then $\plonkprotfs$ is \COMMENT{forking
}simulation-extractable with extraction error $\eta = \epsur(\secpar)$. The
extraction probability $\extProb$ is at least
\[
	\extProb \geq \frac{1}{q^{3 (\epsid(\secpar)+\epsdlog(\secpar))}} (\accProb - \epsk(\secpar) - 2\cdot\epsbind(\secpar) -
  \epsop(\secpar))^{3\noofc + 1} -\eps(\secpar)\,,
\]
for some negligible $\eps(\secpar)$ and $\noofc$ being the number of
constraints in the proven circuit.
\end{corollary}
 
%%% Local Variables:
%%% mode: latex
%%% TeX-master: "main"
%%% End:
