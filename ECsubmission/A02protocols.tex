% !TEX root = main.tex
% !TEX spellcheck = en-US

\section{Polynomial Commitment Schemes}
\label{sec:pcom}

The following properties are expected of a secure polynomial commitment $\PCOM$. Note that since we are in the updatable setting, $\srs$ in the following definitions is the SRS that $\advse$ finalises using the update oracle $\initU$ (See~\cref{fig:upd}).

\begin{description}
\item[Evaluation binding:] A $\ppt$ adversary $\adv$ which outputs a commitment
  $\vec{c}$ and evaluation points $\vec{z}$ has at most negligible chances to open
  the commitment to two different evaluations $\vec{s}, \vec{s'}$. That is, let
  $k \in \NN$ be the number of committed polynomials, $l \in \NN$ number of
  evaluation points, $\vec{c} \in \GRP^k$ be the commitments, $\vec{z} \in
  \FF_p^l$ be the arguments the polynomials are evaluated at, $\vec{s},\vec{s}'
  \in \FF_p^k$ the evaluations, and $\vec{o},\vec{o}' \in \FF_p^l$ be the
  commitment openings. Then for every $\ppt$ adversary $\adv$
	\[
		\Pr
			\left[
			\begin{aligned}
				& \verify(\srs, \vec{c}, \vec{z}, \vec{s}, \vec{o}) = 1,  \\ 
				& \verify(\srs, \vec{c}, \vec{z}, \vec{s}', \vec{o}') = 1, \\
				& \vec{s} \neq \vec{s}'
			\end{aligned}
			\,\left|\,\vphantom{\begin{aligned}
                  & \\
                  & \\
                  &
                \end{aligned}}
			\begin{aligned}
%				& \srs \gets \kgen(\secparam, \maxdeg),\\
				& (\vec{c}, \vec{z}, \vec{s}, \vec{s}', \vec{o}, \vec{o}') \gets \adv^{\initU}(\secparam, \maxdeg)
			\end{aligned}
			\right.\right] \leq \negl\,.
	\]

\end{description}
	
We say that $\PCOM$ has the unique opening property if the following holds:
\begin{description}
\item[Opening uniqueness:] Let $k \in \NN$ be the number of committed
  polynomials, $l \in \NN$ number of evaluation points, $\vec{c} \in \GRP^k$ be
  the commitments, $\vec{z} \in \FF_p^l$ be the arguments the polynomials are
  evaluated at, $\vec{s} \in \FF_p^k$ the evaluations, and $\vec{o} \in \FF_p^l$
  be the commitment openings. Then for every $\ppt$ adversary $\adv$
	\[
		\Pr
			\left[
			\begin{aligned}
				& \verify(\srs, \vec{c}, \vec{z}, \vec{s}, \vec{o}) = 1,  \\ 
				& \verify(\srs, \vec{c}, \vec{z}, \vec{s}, \vec{o'}) = 1, \\
				& \vec{o} \neq \vec{o'}
			\end{aligned}
			\,\left|\, \vphantom{\begin{aligned}
                  & \\
                  & \\
                  &
                \end{aligned}}
			\begin{aligned}
%				& \srs \gets \kgen(\secparam, \maxdeg),\\
				& (\vec{c}, \vec{z}, \vec{s}, \vec{o}, \vec{o'}) \gets \adv^{\initU}(\secparam, \maxdeg)
			\end{aligned}
			\right.\right] \leq \negl\,.
	\]
\end{description}
Intuitively, opening uniqueness assures that there is only one valid opening
for the committed polynomial and given evaluation point. This property is
crucial in showing forking simulation-extractability of $\plonk$ and $\sonic$. We show
that the $\plonk$'s and $\sonic$'s polynomial commitment schemes satisfy this
requirement in \cref{lem:pcomp_op} and \cref{lem:pcoms_unique_op}
respectively.

We also formalize notion of $k$-hiding property of a polynomial commitment scheme
\begin{description}
\item[Hiding:] Let $\HHH$ be a set of size $\maxdeg + 1$ and $\ZERO_\HHH$ its
  vanishing polynomial. We say that a polynomial scheme is \emph{hiding} with
  security $\epsh(\secpar)$ if for every $\ppt$ adversary $\adv$, $k \in \NN$,
  probability
  \begin{align*}
    \Pr\left[
    \begin{aligned}
      & b' = b
      \end{aligned}
        \,\left|\,
        \begin{aligned}
    & (f_0, f_1, c, k, b') \gets \adv^{\initU, \oraclec}(\secparam, \maxdeg), f_0, f_1 \in \FF^{\maxdeg}
    [X]
    \end{aligned}
    \right.\right] \leq \frac{1}{2} + \eps(\secpar)
  \end{align*}
  Here, $\oraclec$ is a challenge oracle that
  \begin{compactenum}
    \item takes polynomials $f_0, f_1$ provided by the adversary and parameter $k$,
    \item samples bit $b$,
    \item samples vector $\vec{a} \in \FF^k$,
    \item computes polynomial,
      $f'_b (X) = f_b + \ZERO_\HHH (X) (a_0 + a_1 X + \ldots a_{k - 1} X^{k -
        1})$,
    \item outputs polynomial commitment $c = f'_b (\chi)$,
    \item on adversary's evaluation query $x$ it adds $x$ to initially empty set
      $Q_x$ and if $|Q_x| \leq k$, it provides $f'_b (x)$.
    \end{compactenum}
  \end{description}

\begin{description}
\item[Commitment of knowledge]  For every $\ppt$ adversary $\adv$ who produces
  commitment $c$, evaluation $s$ and opening $o$ there
  exists a $\ppt$ extractor $\ext$ such that
\[
  \Pr \left[
    \begin{aligned}
      & \deg \p{f} \leq \maxdeg\\
      & c = \com(\srs, \p{f}),\\
      & \verify(\srs, c, z, s, o) = 1
    \end{aligned}
    \,\left|\,
      \vphantom{
        \begin{aligned}
          & \\
          & \\
          &
        \end{aligned}
        }
    \begin{aligned}
      %
     % & \srs \gets \kgen(\secparam, \maxdeg),\\
      & c \gets \adv^{\initU}(\secparam, \maxdeg),
      z \sample \FF_p \\
      & (s, o) \gets \adv(c, z), \\
      & \p{f} = \ext_\adv(\srs, c)\\
    \end{aligned}
  \right.\right]
  \geq 1 - \epsk(\secpar).
\]
In that case we say that $\PCOM$ is $\epsk(\secpar)$-knowledge.
\end{description}

Intuitively when a commitment scheme is ``of knowledge'' then if an
adversary produces a (valid) commitment $c$, which it can open, then it also
knows the underlying polynomial $\p{f}$ which commits to that value.
\cref{fig:pcomp,fig:pcoms} present variants of KZG polynomial commitment schemes
used in \plonk{} and \sonic{}. The key generation algorithm $\kgen$ takes as
input a security parameter $\secparam$ and a parameter $\maxdeg$ which
determines the maximal degree of the committed polynomial. We assume that
$\maxdeg$ can be read from the output SRS. While the figures only describe trusted SRS setup, it is not hard to lift the SRS generation into the updatable setting by defining the extra algorithms $\upd$, $\verifyCRS$ as defined in~\cref{def:upd-scheme}.
\cite{CCS:MBKM19} shows, using AGM, that $\PCOMs$ is a commitment of knowledge.
The same reasoning could be used to show that property for $\PCOMp$.

\begin{figure}[h!]
\centering
	\begin{pcvstack}[center,boxed]
		\begin{pchstack}
			\procedure{$\kgen(\secparam, \maxdeg)$}
			{
			\chi \sample \FF^2_p \\ [\myskip]
			\pcreturn \gone{1, \ldots, \chi^{\numberofconstrains + 2}}, \gtwo{\chi}\\ [\myskip]
				\hphantom{\hspace*{5.5cm}}	
        %\hphantom{\pcind \p{o}_i(X) \gets \sum_{j = 1}^{t_i} \gamma_i^{j - 1}
        %\frac{\p{f}_{i,j}(X) - \p{f}_{i, j}(z_i)}{X - z_i}}
      }
			
			\pchspace
			
			\procedure{$\com(\srs, \vec{\p{f}}(X))$}
			{ 
				\pcreturn \gone{\vec{c}} = \gone{\vec{\p{f}}(\chi)}\\ [\myskip]
				\hphantom{\pcind \pcif 
					\sum_{i = 1}^{\abs{\vec{z}}} r_i \cdot \gone{\sum_{j = 1}^{t_j}
					\gamma_i^{j - 1} c_{i, j} - \sum{j = 1}^{t_j} s_{i, j}} \bullet
				\gtwo{1} + }
			}
		\end{pchstack}
		% \pcvspace
    
		\begin{pchstack}
			\procedure{$\open(\srs, \vec{\gamma}, \vec{z}, \vec{s}, \vec{\p{f}}(X))$}
			{
			\pcfor i \in \range{1}{\abs{\vec{z}}} \pcdo\\ [\myskip]
      \pcind \p{o}_i(X) \gets \sum_{j = 1}^{t_i} \gamma_i^{j - 1}
      \frac{\p{f}_{i,j}(X) - \p{f}_{i, j}(z_i)}{X - z_i}\\ [\myskip] \pcreturn
      \vec{o} = \gone{\vec{\p{o}}(\chi)}\\ [\myskip]
				\hphantom{\hspace*{5.5cm}}	
			}
			
			\pchspace
			
			\procedure{$\verify(\srs, \gone{c}, \vec{z}, \vec{s}, \gone{\p{o}(\chi)})$}
			{
				\vec{r} \gets \FF_p^{\abs{\vec{z}}}\\ [\myskip]
				\pcfor i \in \range{1}{\abs{\vec{z}}} \pcdo \\ [\myskip]
				\pcind \pcif 
          \sum_{i = 1}^{\abs{\vec{z}}} r_i \cdot \gone{\sum_{j = 1}^{t_j}
          \gamma_i^{j - 1} c_{i, j} - \sum{j = 1}^{t_j} s_{i, j}} \bullet
          \gtwo{1} + \\ [\myskip] \pcind \sum_{i = 1}^{\abs{\vec{z}}} r_i z_i
          o_i
          \bullet \gtwo{1} \neq \gone{- \sum_{i = 1}^{\abs{\vec{z}}} r_i o_i }
          \bullet \gtwo{\chi} \pcthen  \\
					\pcind \pcreturn 0\\ [\myskip]
					\pcreturn 1.
			}
		\end{pchstack}
	\end{pcvstack}
	\caption{$\PCOMp$ polynomial commitment scheme.}
	\label{fig:pcomp}
  \end{figure}

% \begin{figure}[h!]
% \centering
% 	\begin{pcvstack}[center,boxed]
% 		\begin{pchstack}
% 			\procedure{$\kgen(\secparam, \maxdeg)$} {
% 				\alpha, \chi \sample \FF^2_p \\ [\myskip]
% 				\pcreturn \gone{\smallset{\chi^i}_{i = -\multconstr}^{\multconstr},
%           \smallset{\alpha \chi^i}_{i = -\multconstr, i \neq
%             0}^{\multconstr}},\\
%         \pcind \gtwo{\smallset{\chi^i, \alpha \chi^i}_{i =
%             -\multconstr}^{\multconstr}}, \gtar{\alpha}\\
% 				%\markulf{03.11.2020}{} \\
% 			%	\hphantom{\pcind \p{o}_i(X) \gets \sum_{j = 1}^{t_i} \gamma_i^{j - 1} \frac{\p{f}_{i,j}(X) - \p{f}_{i, j}(z_i)}{X - z_i}}
% 				\hphantom{\hspace*{5.5cm}}
% 		}
%
% 			\pchspace
%
% 			\procedure{$\com(\srs, \maxconst, \p{f}(X))$} {
% 				\p{c}(X) \gets \alpha \cdot X^{\dconst - \maxconst} \p{f}(X) \\ [\myskip]
% 				\pcreturn \gone{c} = \gone{\p{c}(\chi)}\\ [\myskip]
% 				\hphantom{\pcind \pcif \sum_{i = 1}^{\abs{\vec{z}}} r_i \cdot
%           \gone{\sum_{j = 1}^{t_j} \gamma_i^{j - 1} c_{i, j} - \sum_{j = 1}^{t_j}
%             s_{i, j}} \bullet \gtwo{1} + } }
% 		\end{pchstack}
% 		% \pcvspace
%
% 		\begin{pchstack}
% 			\procedure{$\open(\srs, z, s, f(X))$}
% 			{
% 				\p{o}(X) \gets \frac{\p{f}(X) - \p{f}(z)}{X - z}\\ [\myskip]
% 				\pcreturn \gone{\p{o}(\chi)}\\ [\myskip]
% 				\hphantom{\hspace*{5.5cm}}
% 			}
%
% 			\pchspace
%
% 			\procedure{$\verify(\srs, \maxconst, \gone{c}, z, s, \gone{\p{o}(\chi)})$}
%       {
%         \pcif \gone{\p{o}(\chi)} \bullet \gtwo{\alpha \chi} + \gone{s - z
%         \p{o}(\chi)} \bullet \gtwo{\alpha} = \\ [\myskip] \pcind \gone{c}
%         \bullet \gtwo{\chi^{- \dconst + \maxconst}} \pcthen  \pcreturn 1\\
%         [\myskip]
%         \rlap{\pcelse \pcreturn 0.} \hphantom{\pcind \pcif \sum_{i =
%             1}^{\abs{\vec{z}}} r_i \cdot \gone{\sum_{j = 1}^{t_j} \gamma_i^{j -
%               1} c_{i, j} - \sum{j = 1}^{t_j} s_{i, j}} \bullet \gtwo{1} + } }
% 		\end{pchstack}
% 	\end{pcvstack}
%
% 	\caption{$\PCOMs$ polynomial commitment scheme.}
% 	\label{fig:pcoms}
% \end{figure}

\subsection{Unique opening property of $\PCOMp$}
Now, we show that the version of the KZG polynomial
commitment scheme that is used in \plonk{}, $\PCOMp$, has the unique opening
property.

\begin{lemma}
\label{lem:pcomp_op}
Let $\PCOMp$ be a batched version of a KZG polynomial commitment,
cf.~\cref{fig:pcomp}, then $\PCOMp$ has the unique opening property (see~\cref{sec:pcom}) in the AGM
with security
$\epsop(\secpar) \leq 2 \epsdlog(\secpar) + \infrac{1}{\abs{\FF_p}}$, where
$\epsdlog(\secpar)$ is security of the $(\noofc + 2, 1)$-dlog assumption and
$\FF_p$ is the field used in $\PCOMp$.\end{lemma}
\begin{proof}
  Let $\vec{z} = (z, z') \in \FF_p^2$ be the two points the polynomials are
  evaluated at, $k \in \NN$ be the number of the committed polynomials to be
  evaluated at $z$, $k' \in \NN$ be the number of the committed polynomials
  to be evaluated at $z'$, $\vec{c} \in \GRP^k, \vec{c'} \in \GRP^{k'}$ be the
  commitments, $\vec{s} \in \FF_p^k, \vec{s'} \in \FF_p^{k'}$ the evaluations,
  and $\vec{o} = (o, o') \in \FF_p^2$ be the commitment openings.  We need to
  show that the probability a $\ppt$ $\adv$ opens the same commitment in two
  different ways is at most $\epsop(\secpar)$, even when the commitment openings
  are verified in batches.

  The idealised verifier checks whether the following equality, for $\gamma, r'$
  picked at random, holds:
  \begin{multline}
    \label{eq:ver_eq_poly}
    \left(\sum_{i = 1}^{k} \gamma^{i - 1} \cdot \p{f}_i(X) - \sum_{i = 1}^{k}
      \gamma^{i - 1} \cdot s_i\right) + r' \left(\sum_{i = 1}^{k'} \gamma'^{i -
        1} \cdot \p{f'}_i(X) -
      \sum_{i = 1}^{k'} \gamma'^{i - 1} \cdot s'_i \right)\\
    \equiv \p{o}(X)(X - z) + r' \p{o}'(X)(X- z').
  \end{multline}
  Since $r'$ has been picked at random from $\FF$, probability that
  \cref{eq:ver_eq_poly} holds while either
  \[
    \sum_{i = 1}^{k} \gamma^{i - 1} \cdot \p{f}_i(X) - \sum_{i = 1}^{k}
    \gamma^{i - 1} \cdot s_i \not\equiv \p{o}(X)(X - z) \text{, or}
  \]
  \[
    \sum_{i = 1}^{k'} \gamma'^{i - 1} \cdot \p{f'}_i(X) - \sum_{i = 1}^{k'}
    \gamma'^{i - 1} \cdot s'_i \not\equiv \p{o'}(X)(X - z')
  \]
  is $\infrac{1}{\abs{\FF_p}}$~cf.~\cite{EPRINT:GabWilCio19}. 
  When \(
    \sum_{i = 1}^{k} \gamma^{i - 1} \cdot \p{f}_i(X) - \sum_{i = 1}^{k}
    \gamma^{i - 1} \cdot s_i = \p{o}(X)(X - z) 
  \)
  holds, polynomial $\p{o}(X)$ is uniquely determined from the uniqueness of
  polynomial composition. Similarly, $\p{o'}(X)$ is uniquely determined as well.

  Any discrepancy
  between the idealised verifier rejection and real verifier acceptance allows
  one to break the discrete logarithm problem.

  The reduction $\rdvdlog$ proceeds as follows: $\rdvdlog$ initializes the SRS $\srs$ using the input $\dlog$ instance, and then answers $\adv$'s queries for SRS updates. Let $\srs'$ be the finalised SRS. 
  Consider a proof $\zkproof$ such
      that $\vereq_{\inp, \zkproof}(X) \neq 0$, but
      $\vereq_{\inp, \zkproof}(\chi') = 0$. Since $\adv$ is algebraic, all proof
      elements are extended by their representation as a
      combination of the input $\GRP_1$-elements. Therefore, all coefficients of the
      verification equation polynomial $\vereq_{\inp, \zkproof}(X)$ are known.
   Now, $\rdvdlog$ computes the roots of $\vereq_{\inp, \zkproof}(X)$ and finds $\chi'$ among
      them.
    Let $\chi_1, \ldots, \chi_\ell$ be the partial trapdoors of $\adv$'s SRS updates that are extracted by $\rdvdlog$ from the update proofs given by $\adv$.
  $\rdvdlog$ returns $\chi = \chi' (\chi_1 \chi_2 \ldots \chi_\ell)^{-1}$.
  
  Since any discrepancy
  between the idealised verifier and real verifier rejection allows
  one to break the discrete logarithm problem, the probability that the real
  verifier accepts in one of the cases above is upper-bounded by
  $2 \epsdlog + \infrac{1}{\FF_p}$.

    \qed
\end{proof}
