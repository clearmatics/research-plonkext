% !TEX root = main.tex
% !TEX spellcheck = en-US

\section{Non-malleability of Plonk} 
\label{sec:plonk}
In this section, we show that $\plonkprotfs$ is simulation-extractable. Towards this end, we first use the unique opening property to show that
$\plonkprot$ has the $\ur{3}$ property,
cf.~\cref{lem:plonkprot_ur}.
Next, we show that $\plonkprot$ is rewind-based knowledge sound. That is, given a number of accepting transcripts whose messages match on the first $5$ messages of the protocol, we can either extract a witness for the proven statement or use one of the transcripts to break the $\dlog$ assumption. This result is shown in the AGM, cf.~\cref{lem:plonkprot_ss}. We then show that $\plonkprot$ is trapdoor-less ZK in the AGM, cf.~\cref{lem:plonk_tlzk}.

Given rewinding-based knowledge soundness, $\ur{3}$ and trapdoor-less zero knowledge of $\plonkprot$, we invoke \cref{thm:se} and conclude that $\plonkprot_\fs$ is simulation-extractable.

\newcommand{\vql}{\vec{q_{L}}}
\newcommand{\vqr}{\vec{q_{R}}}
\newcommand{\vqm}{\vec{q_{M}}}
\newcommand{\vqo}{\vec{q_{O}}}
\newcommand{\vx}{\vec{x}}
\newcommand{\vqc}{\vec{q_{C}}}

\subsection{Plonk Protocol Description}
\label{sec:plonk_explained}
\oursubsub{The constraint system}
Assume $\CRKT$ is a fan-in two arithmetic circuit,
which fan-out is unlimited and has $\numberofconstrains$ gates and $\noofw$ wires
($\numberofconstrains \leq \noofw \leq 2\numberofconstrains$). \plonk's constraint
system is defined as follows:
\begin{compactitem}
\item Let $\vec{V} = (\va, \vb, \vc)$, where $\va, \vb, \vc
  \in \range{1}{\noofw}^\numberofconstrains$. Entries $\va_i, \vb_i, \vc_i$ represent indices of left,
  right and output wires of circuits $i$-th gate.
\item Vectors $\vec{Q} = (\vql, \vqr, \vqo, \vqm, \vqc) \in
  (\FF^\numberofconstrains)^5$ are called \emph{selector vectors}:
  \begin{inparaenum}[(a)]
  \item If the $i$-th gate is a multiplicative gate then $\vql_i = \vqr_i = 0$,
    $\vqm_i = 1$, and $\vqo_i = -1$. 
  \item If the $i$-th gate is an addition gate then $\vql_i = \vqr_i  = 1$, $\vqm_i =
    0$, and $\vqo_i = -1$. 
  \item $\vqc_i = 0$ always. 
  \end{inparaenum}
\end{compactitem}

We say that vector $\vx \in \FF^\noofw$ satisfies constraint system if for all $i
\in \range{1}{\numberofconstrains}$
\[
  \vql_i \cdot \vx_{\va_i} + \vqr_i \cdot \vx_{\vb_i} + \vqo \cdot \vx_{\vc_i} +
  \vqm_i \cdot (\vx_{\va_i} \vx_{\vb_i}) + \vqc_i = 0. 
\]

\oursubsub{Algorithms rolled out}
\label{sec:plonk_explained}
\plonk{} argument system is universal. That is, it allows to verify computation
of any arithmetic circuit which has up to $\numberofconstrains$
gates using a single SRS. However, to make computation efficient, for each
circuit there is allowed a preprocessing phase which extends the SRS with
circuit-related polynomial evaluations.

For the sake of simplicity of the security reductions presented in this paper, we
include in the SRS only these elements that cannot be computed without knowing
the secret trapdoor $\chi$. The rest of the SRS---the preprocessed input---can
be computed using these SRS elements thus we leave them to be computed by the
prover, verifier, and simulator separately.

\ourpar{$\plonk$ SRS generating algorithm $\kgen(\REL)$:}
The SRS generating algorithm picks at random $\chi \sample \FF_p$, computes
and outputs
\(
	\srs = \left(\gone{\smallset{\chi^i}_{i = 0}^{\numberofconstrains + \changedm{5}}},
	\gtwo{\chi} \right).
\)

\ourpar{Preprocessing:}
Let $H = \smallset{\omega^i}_{i = 1}^{\numberofconstrains }$ be a
(multiplicative) $\numberofconstrains$-element subgroup of a field $\FF$
compound of $\numberofconstrains$-th roots of unity in $\FF$. Let $\lag_i(X)$ be
the $i$-th element of an $\numberofconstrains$-elements Lagrange basis. During
the preprocessing phase polynomials $\p{S_{id j}}, \p{S_{\sigma j}}$, for
$\p{j} \in \range{1}{3}$, are computed:
\begin{equation*}
  \begin{aligned}
    \p{S_{id 1}}(X) & = X,\\[\myskip]
    \p{S_{id 2}}(X) & = k_1 \cdot X,\\[\myskip]
    \p{S_{id 3}}(X) & = k_2 \cdot X,
  \end{aligned}
  \qquad
\begin{aligned}
  \p{S_{\sigma 1}}(X) & = {\textstyle{\sum_{i = 1}^{\noofc} \sigma(i) \lag_i(X)}},\\[\myskip]
  \p{S_{\sigma 2}}(X) & = {\textstyle \sum_{i = 1}^{\noofc}
  \sigma(\noofc + i) \lag_i(X)},\\[\myskip]
  \p{S_{\sigma 3}}(X) & ={\textstyle\sum_{i = 1}^{\noofc} \sigma(2 \noofc + i) \lag_i(X)}.
\end{aligned}
\end{equation*}
Coefficients $k_1$, $k_2$ are such that $H, k_1 \cdot H, k_2 \cdot H$ are
different cosets of $\FF^*$, thus they define $3 \cdot \noofc$
different elements. \cite{EPRINT:GabWilCio19} notes that it is enough to set
$k_1$ to a quadratic residue and $k_2$ to a quadratic non-residue.

Furthermore, we define polynomials $\p{q_L}, \p{q_R}, \p{q_O}, \p{q_M}, \p{q_C}$
such that
\begin{equation*}
  \begin{aligned}
  \p{q_L}(X) & = {\textstyle \sum_{i = 1}^{\noofc}} \vql_i \lag_i(X), \\
  \p{q_R}(X) & = \textstyle \sum_{i = 1}^{\noofc} \vqr_i \lag_i(X), \\
  \p{q_M}(X) & = \textstyle \sum_{i = 1}^{\noofc} \vqm_i \lag_i(X),
\end{aligned}
\qquad
\begin{aligned}
  \p{q_O}(X) & = \textstyle  \sum_{i = 1}^{\noofc} \vqo_i \lag_i(X), \\
  \p{q_C}(X) & =  \textstyle \sum_{i = 1}^{\noofc} \vqc_i \lag_i(X). \\
  \vphantom{\p{q_M}(X)  = \textstyle \sum_{i = 1}^{\noofc} \vqm_i \lag_i(X),}
\end{aligned}
\end{equation*}

\ourpar{Proving statements in $\plonkprotfs$} We show how prover's algorithm
$\prover(\srs, \inp, \wit = (\wit_i)_{i \in \range{1}{3 \cdot \noofc}})$ operates for
the Fiat--Shamir transformed version of Plonk.
\begin{description}
\item[Message 1] Sample $b_1, \ldots, b_9 \sample \FF_p$; compute
  $\p{a}(X), \p{b}(X), \p{c}(X)$ as
	\begin{align*}
		\p{a}(X) &= (b_1 X + b_2)\p{Z_H}(X) + \textstyle \sum_{i = 1}^{\noofc} \wit_i \lag_i(X) \\
		\p{b}(X) &= (b_3 X + b_4)\p{Z_H}(X) + \textstyle \sum_{i = 1}^{\noofc} \wit_{\noofc + i} \lag_i(X) \\
		\p{c}(X) &= (b_5 X + b_6)\p{Z_H}(X) + \textstyle \sum_{i = 1}^{\noofc} \wit_{2 \cdot \noofc + i} \lag_i(X) 
	\end{align*}
	Output polynomial commitments $\gone{\p{a}(\chi), \p{b}(\chi), \p{c}(\chi)}$.  
	
\item[Message 2] Compute challenges $\beta, \gamma \in \FF_p$ by querying random oracle
  on partial proof, that is,
	\(
		\beta = \ro(\tzkproof[0..1], 0)\,, \qquad \gamma = \ro(\tzkproof[0..1], 1)\,.
	\)
  
	Compute permutation polynomial $\p{z}(X)$
	\begin{multline*}
		\p{z}(X) = (b_7 X^2 + b_8 X + b_9)\p{Z_H}(X) + \lag_1(X) + \\
    + \sum_{i = 1}^{\noofc - 1} \left(\lag_{i + 1} (X) \prod_{j = 1}^{i} \frac{
        (\wit_j +\beta \omega^{j - 1} + \gamma)(\wit_{\noofc + j} + \beta k_1
        \omega^{j - 1} + \gamma)(\wit_{2 \noofc + j} +\beta k_2 \omega^{j- 1} +
        \gamma)} {(\wit_j+\sigma(j) \beta + \gamma)(\wit_{\noofc + j} + \sigma(\noofc
        + j)\beta + \gamma)(\wit_{2 \noofc + j} + \sigma(2 \noofc + j)\beta +
        \gamma)}\right)
	\end{multline*}
	Output polynomial commitment $\gone{\p{z}(\chi)}$
		
\item[Message 3] Compute the challenge $\alpha = \ro(\tzkproof[0..2])$, compute the quotient
  polynomial
	\begin{align*}
    & \p{t}(X)  = \\
    & (\p{a}(X) \p{b}(X) \selmulti(X) + \p{a}(X) \selleft(X) + 
      \p{b}(X)\selright(X) + \p{c}(X)\seloutput(X) + \pubinppoly(X) + \selconst(X)) 
      \frac{1}{\p{Z_H}(X)} +\\
    & + ((\p{a}(X) + \beta X + \gamma) (\p{b}(X) + \beta k_1 X + \gamma)(\p{c}(X) 
      + \beta k_2 X + \gamma)\p{z}(X)) \frac{\alpha}{\p{Z_H}(X)} \\
    & - (\p{a}(X) + \beta \p{S_{\sigma 1}}(X) + \gamma)(\p{b}(X) + \beta 
      \p{S_{\sigma 2}}(X) + \gamma)(\p{c}(X) + \beta \p{S_{\sigma 3}}(X) + 
      \gamma)\p{z}(X \omega))  \frac{\alpha}{\p{Z_H}(X)} \\
    & + (\p{z}(X) - 1) \lag_1(X) \frac{\alpha^2}{\p{Z_H}(X)}
	\end{align*}
	Split $\p{t}(X)$ into degree less then $\noofc$ polynomials
  $\p{t_{lo}}(X), \p{t_{mid}}(X), \p{t_{hi}}(X)$, such that
	\(
		\p{t}(X) = \p{t_{lo}}(X) + X^{\noofc} \p{t_{mid}}(X) + X^{2 \noofc}
    \p{t_{hi}}(X)\,.
	\)
	Output $\gone{\p{t_{lo}}(\chi), \p{t_{mid}}(\chi), \p{t_{hi}}(\chi)}$.
	
\item[Message 4] Get the challenge $\chz \in \FF_p$, $\chz = \ro(\tzkproof[0..3])$.
  Compute opening evaluations
	\(
    \p{a}(\chz), \p{b}(\chz), \p{c}(\chz), \p{S_{\sigma 1}}(\chz), \p{S_{\sigma 2}}(\chz), \p{t}(\chz), \p{z}(\chz \omega),
	\)
	Compute the linearisation polynomial
	\[
		\p{r}(X) =
		\begin{aligned}
      & \p{a}(\chz) \p{b}(\chz) \selmulti(X) + \p{a}(\chz) \selleft(X) + \p{b}(\chz) \selright(X) + \p{c}(\chz) \seloutput(X) + \selconst(X) \\
      & + \alpha \cdot \left( (\p{a}(\chz) + \beta \chz + \gamma) (\p{b}(\chz) + \beta k_1 \chz + \gamma)(\p{c}(\chz) + \beta k_2 \chz + \gamma) \cdot \p{z}(X)\right) \\
      & - \alpha \cdot \left( (\p{a}(\chz) + \beta \p{S_{\sigma 1}}(\chz) + \gamma) (\p{b}(\chz) + \beta \p{S_{\sigma 2}}(\chz) + \gamma)\beta \p{z}(\chz\omega) \cdot \p{S_{\sigma 3}}(X)\right) \\
      & + \alpha^2 \cdot \lag_1(\chz) \cdot \p{z}(X)
		\end{aligned}
	\]
	Output
  $\p{a}(\chz), \p{b}(\chz), \p{c}(\chz), \p{S_{\sigma 1}}(\chz), \p{S_{\sigma
      2}}(\chz), \p{t}(\chz), \p{z}(\chz \omega), \p{r}(\chz).$
	
\item[Message 5] Compute the opening challenge $v \in \FF_p$,
  $v = \ro(\tzkproof[0..4])$.  Compute the openings for the polynomial commitment
  scheme
	\begin{align*}
	& \p{W_\chz}(X) = \frac{1}{X - \chz} \left(
   \begin{aligned}
     & \p{t_{lo}}(X) + \chz^\noofc \p{t_{mid}}(X) + \chz^{2 \noofc} \p{t_{hi}}(X) - \p{t}(\chz)\\
     & + v(\p{r}(X) - \p{r}(\chz)) 
      + v^2 (\p{a}(X) - \p{a}(\chz))\\
     & + v^3 (\p{b}(X) - \p{b}(\chz))
      + v^4 (\p{c}(X) - \p{c}(\chz))\\
     & + v^5 (\p{S_{\sigma 1}}(X) - \p{S_{\sigma 1}}(\chz))
      + v^6 (\p{S_{\sigma 2}}(X) - \p{S_{\sigma 2}}(\chz))
   \end{aligned}
       \right)\\
    & \p{W_{\chz \omega}}(X) = \frac{\p{z}(X) - \p{z}(\chz \omega)}{X - \chz \omega}
  \end{align*}
	Output $\gone{\p{W_{\chz}}(\chi), \p{W_{\chz \omega}}(\chi)}$.
\end{description}

\ncase{$\plonk$ verifier $\verifier(\srs, \inp, \zkproof)$}\ \newline
The \plonk{} verifier works as follows
\begin{enumerate}
	\item Validate all obtained group elements.
	\item Validate all obtained field elements.
	\item Validate the instance
      $\inp = \smallset{\wit_i}_{i = 1}^\instsize$.
	\item Compute challenges $\beta, \gamma, \alpha, \chz, v, u$ from the transcript.
	\item Compute zero polynomial evaluation
      $\p{Z_H} (\chz) =\chz^\noofc - 1$.
	\item Compute Lagrange polynomial evaluation
      $\lag_1 (\chz) = \frac{\chz^\noofc -1}{\noofc (\chz - 1)}$.
	\item Compute public input polynomial evaluation
      $\pubinppoly (\chz) = \sum_{i \in \range{1}{\instsize}} \wit_i
      \lag_i(\chz)$.
	\item Compute quotient polynomials evaluations
	\begin{multline*}
    \p{t} (\chz) = \frac{1}{\p{Z_H}(\chz)} \Big(
    \p{r} (\chz) + \pubinppoly(\chz) - (\p{a}(\chz) + \beta \p{S_{\sigma 1}}(\chz) + \gamma) (\p{b}(\chz) + \beta \p{S_{\sigma 2}}(\chz) + \gamma) \\
    (\p{c}(\chz) + \gamma)\p{z}(\chz \omega) \alpha - \lag_1 (\chz) \alpha^2
    \Big) \,.
	\end{multline*}
	\item Compute batched polynomial commitment
	$\gone{D} = v \gone{r} + u \gone {z}$ that is
	\begin{align*}
		\gone{D} & = v
		\left(
		\begin{aligned}
          & \p{a}(\chz)\p{b}(\chz) \cdot \gone{\selmulti} + \p{a}(\chz)  \gone{\selleft} + \p{b}  \gone{\selright} + \p{c}  \gone{\seloutput} + \\
          & + (	(\p{a}(\chz) + \beta \chz + \gamma) (\p{b}(\chz) + \beta k_1 \chz + \gamma) (\p{c} + \beta k_2 \chz + \gamma) \alpha  + \lag_1(\chz) \alpha^2)  + \\
			% &   \\
          & - (\p{a}(\chz) + \beta \p{S_{\sigma 1}}(\chz) + \gamma) (\p{b}(\chz)
          + \beta \p{S_{\sigma 2}}(\chz) + \gamma) \alpha \beta \p{z}(\chz
          \omega) \gone{\p{S_{\sigma 3}}(\chi)})
		\end{aligned}
		\right) + \\
		& + u \gone{\p{z}(\chi)}\,.
	\end{align*}
	\item Computes full batched polynomial commitment $\gone{F}$:
	\begin{align*}
      \gone{F} & = \left(\gone{\p{t_{lo}}(\chi)} + \chz^\noofc \gone{\p{t_{mid}}(\chi)} + \chz^{2 \noofc} \gone{\p{t_{hi}}(\chi)}\right) + u \gone{\p{z}(\chi)} + \\
               & + v
                 \left(
		\begin{aligned}
			& \p{a}(\chz)\p{b}(\chz) \cdot \gone{\selmulti} + \p{a}(\chz)  \gone{\selleft} + \p{b}(\chz)   \gone{\selright} + \p{c}(\chz)  \gone{\seloutput} + \\
			& + (	(\p{a}(\chz) + \beta \chz + \gamma) (\p{b}(\chz) + \beta k_1 \chz + \gamma) (\p{c}(\chz)  + \beta k_2 \chz + \gamma) \alpha  + \lag_1(\chz) \alpha^2)  + \\
			% &   \\
			& - (\p{a}(\chz) + \beta \p{S_{\sigma 1}}(\chz) + \gamma) (\p{b}(\chz) + \beta \p{S_{\sigma 2}}(\chz) + \gamma) \alpha  \beta \p{z}(\chz \omega) \gone{\p{S_{\sigma 3}}(\chi)})
		\end{aligned}
		\right) \\
		& + v^2 \gone{\p{a}(\chi)} + v^3 \gone{\p{b}(\chi)} + v^4 \gone{\p{c}(\chi)} + v^5 \gone{\p{S_{\sigma 1}(\chi)}} + v^6 \gone{\p{S_{\sigma 2}}(\chi)}\,.
	\end{align*}
	\item Compute group-encoded batch evaluation $\gone{E}$
	\begin{align*}
		\gone{E}  = \frac{1}{\p{Z_H}(\chz)} & \gone{
		\begin{aligned}
			& \p{r}(\chz) + \pubinppoly(\chz) +  \alpha^2  \lag_1 (\chz) + \\
			& - \alpha \left( (\p{a}(\chz) + \beta \p{S_{\sigma 1}} (\chz) + \gamma) (\p{b}(\chz) + \beta \p{S_{\sigma 2}} (\chz) + \gamma) (\p{c}(\chz) + \gamma) \p{z}(\chz \omega) \right)
		\end{aligned}
           }\\
      + & \gone{v \p{r}(\chz) + v^2 \p{a}(\chz) + v^3 \p{b}(\chz) + v^4 \p{c}(\chz) + v^5 \p{S_{\sigma 1}}(\chz) + v^6 \p{S_{\sigma 2}}(\chz) + u \p{z}(\chz \omega) }\,.
	\end{align*}
\item Check whether the verification
 % $\vereq_\zkproof(\chi)$
  equation holds
	\begin{multline}
		\label{eq:ver_eq} 
		\left( \gone{\p{W_{\chz}}(\chi)} + u \cdot \gone{\p{W_{\chz
                \omega}}(\chi)} \right) \bullet
		\gtwo{\chi} - %\\
		\left( \chz \cdot \gone{\p{W_{\chz}}(\chi)} + u \chz \omega \cdot
          \gone{\p{W_{\chz \omega}}(\chi)} + \gone{F} - \gone{E} \right) \bullet
        \gtwo{1} = 0\,.
	\end{multline}
  The verification equation is a batched version of the verification equation
  from \cite{AC:KatZavGol10} which allows the verifier to check openings of
  multiple polynomials in two points (instead of checking an opening of a single
  polynomial at one point).
\end{enumerate}

\ncase{$\plonk$ simulator $\simulator_\chi(\srs, \td= \chi, \inp)$}\ 
We describe the simulator in \cref{lem:plonk_tlzk}.

\subsection{Unique Response Property}
\begin{lemma}
	\label{lem:plonkprot_ur}
	Let $\PCOMp$ be a polynomial commitment that is $\epsbind(\secpar)$-binding and has unique opening property with loss $\epsudlog (\secpar) + 2 \infrac{q}{\FF_p}$, let $(\noofc + 5, 1)$-$\udlog$ problem be $\epsdlog (\secpar)$ hard. Then $\plonkprotfs$ is $\ur{3}$ against algebraic adversaries, who makes up to $q$ random oracle queries, with security loss $8  \cdot \epsbinding (\secpar) + \epsudlog (\secpar) + 2 \infrac{q}{\FF_p}$.
\end{lemma}
\begin{proof}
  Let $\adv$ be an algebraic adversary tasked to break the $\ur{3}$-ness of
	$\plonkprotfs$. We show that the first three prover's messages determine, along with 	the verifiers challenges, the rest of it. We denote by $\zkproof^0$ and $\zkproof^1$ the two proofs that the adversary outputs. To distinguish polynomials and commitments which an honest prover would send in the proof from the polynomials and commitments computed by the adversary we write the latter using indices $0$ and $1$ (two indices as we have two transcripts), e.g.~to describe the quotient polynomial provided by the adversary we write $\p{t}^0$ and $\p{t}^1$ instead of $\p{t}$ as in the description of the protocol.

  We note that since the unique response property requires from $\zkproof^{0}$ and $\zkproof^{1}$ that the first place they possibly differ is the $4$-th prover's message, then the challenge $\chz$, that is picked by the adversary after the $3$-rd message is the same in both transcripts. This challenge determines the evaluation point of polynomials $\p{a}(X), \p{b}(X), \p{c}(X), \p{t}(X), \p{z}(X)$ which commitments are already sent.

  In its fourth message, the prover provides evaluations of the aforementioned polynomials, along with evaluations of publicly known polynomials $
  \p{S_{\sigma 1}} (\chz), \p{S_{\sigma, 2}} (\chz)$, and evaluation of a linearization polynomial $\p{r}(\chz)$.
  Note that the adversary can output two acceptable proofs that differ on their fourth message only if it either manages to break evaluation binding of one of the opening, or provides an unacceptable opening which is the accepted due to a batching error. Since the commitment scheme is evaluation binding with security loss $\epsbinding (\secpar)$, and the batched verification equation accepts an incorrect opening with probability at most $\infrac{q}{\abs{\FF_p}}$, cf.~\cite{EPRINT:GabWilCio19}, the adversary can make $\zkproof^{0}$ and $\zkproof^{1}$ differ on the fourth message with probability at most $8  \cdot \epsbinding (\secpar) + \infrac{q}{\abs{\FF_p}}$. 

  Next, assume that the transcripts are the same up to the fourth message, but differ at the fifth. In that message, the adversary provides openings of the evaluations. Since the unique opening property, the adversary can open the valid evaluation of a polynomial to two different values with probability at most $\epsop (\secpar)$, which is upper-bounded by $\epsudlog (\secpar) + \infrac{q}{\FF_p}$, cf.~\cref{lem:pcomp_op}.

  By the union bound, the adversary is able to break the unique response property with probabilty upper bounded by $8  \cdot \epsbinding (\secpar) + \epsudlog (\secpar) + 2 \infrac{q}{\FF_p}$.
  \end{proof}

\subsection{Updatable Rewinding-Based Knowledge Soundness}
\begin{lemma}
	\label{lem:plonkprot_ss}
  $\plonkprot$ is $(3, 3 \noofc + 1)$-updatable rewinding-based knowledge sound against algebraic adversaries who make up to $q$ random oracle queries with security loss 
  \[
    \epscss(\secpar,\accProb, q) \leq \left(1 - \frac{\accProb - (q + 1) \epsid (\secpar)}{1 - \epsid (\secpar)}\right) + \epsudlog (\secpar) + \epsid (\secpar)\,,
  \]
	Here $\accProb$ is a probability that the adversary outputs an acceptable proof, $q$ is the upper bound for a number of random oracle queries the adversary makes, $\epsid(\secpar)$ is a soundness error of the ideal verifier for $\plonkprot$, and $\epsudlog(\secpar)$ is security of $(\numberofconstrains + \changedm{5}, 1)$-$\udlog$ 
	assumption.
\end{lemma}

\begin{proof}
	Let $\adv^{\ro, \initU}(\secparam; r)$ be the adversary who outputs $(\inp, \zkproof)$ such that $\plonkprotfs.\verifier$ accepts the proof. Let $\tdv$ be a tree-building algorithm that runs $\zdv_1$ from \cref{fig:Attema-ext} and outputs a tree $\tree$, and let $\extcss$ be an extractor that given the tree output by $\tdv$ reveals the witness for $\inp$. The main idea of the proof is to show that an adversary who breaks updatable rewinding-based knowledge soundness can be used to break a $\udlog$-problem instance. The proof goes by game hops. Note that since the tree branches after $\adv$'s $3$-rd message, the instance $\inp$, commitments $\gone{\p{a} (\chi), \p{b} (\chi), \p{c} (\chi), \p{z} (\chi), \p{t_{lo}} (\chi), \p{t_{mid}} (\chi), \p{t_{hi}} (\chi)}$, and challenges $\alpha, \beta, \gamma$ are the same in all the transcripts. Also, the tree branches after the third adversary's message where the challenge $\chz$ is presented, thus tree $\tree$ is built using different values of $\chz$.	We consider the following games.

  \ncase{Game 0} %
  In this game the adversary wins if it outputs a valid instance--proof pair $(\inp, \zkproof)$, and the extractor $\extcss$ does not manage to output a witness $\wit$ such that $\REL (\inp, \wit)$ holds.

  \ncase{Game 1} %
  In this game the environment aborts the game if the tree building algorithm $\tdv$ fails in building a tree of acceptable transcripts $\tree$. 

  \ncase{Game 0 to Game 1} %
  By \cref{lem:attema} probability that Game 1 is aborted, while Game 0 is not, is at most 
  \[
    \frac{\accProb - (q + 1) \epsid (\secpar)} {1 - \epsid (\secpar)} \,.
  \]

  \ncase{Game 2} %
  In this game the environment additionally aborts if at least one of its proofs in $\tree$ is not acceptable by an ideal verifier.

  \ncase{Game 1 to Game 2} % 
  As usual, we show a reduction that breaks an instance of a $\udlog$ assumption when Game 2 is aborted, while Game 1 is not.

  Let $\rdvudlog$ be a reduction that gets as input an $(\noofc + 5, 1)$-$\udlog$ instance $\gone{1, \ldots, \chi^{\noofc + 5}}, \gtwo{\chi}$. Then it can update the instance to another one $\gone{1, \ldots, {\chi'}^{\noofc + 5}}, \gtwo{\chi'}$. Eventually, the reduction ouputs $\chi'$.
	%
	The reduction $\rdvudlog$ proceeds as follows.
	First, it builds $\adv$'s SRS $\srs$ using the input $\udlog$ instance. Then it processes the adversary's update query by adding it to the list $\Qsrs$ and passing it to its own update oracle getting instance $\gone{1, \ldots, {\chi'}^{\noofc + 5}}, \gtwo{\chi'}$. The updated SRS $\srs'$ is then computed and given to $\adv$. $\rdvdulog$ also takes care of the random oracle queries made by $\adv$. It picks their answers honestly and write them in $\Qro$. The reduction then starts $\tdv(\srs, \adv, r, \Qro, \Qsrs)$.
	
  Let $(1, \tree)$ be the output returned by $\tdv$. Let $\inp$ be a relation proven in $\tree$.  Consider a transcript $\zkproof \in \tree$ such that $\vereq_{\inp, \zkproof}(X) \neq 0$, but $\vereq_{\inp, \zkproof}(\chi') = 0$. Since $\adv$ is algebraic, all group elements included in $\tree$ are extended by their representation as a combination of the input $\GRP_1$-elements. Hence, all coefficients of the verification equation polynomial $\vereq_{\inp, \zkproof}(X)$ are known. 
  Eventually, the reduction finds $\vereq_{\inp, \zkproof}(X)$ zero points and returns $\chi'$ which is one of them.
    
  Hence, the probability that the adversary wins in Game 2 but does not win in Game 1 is upper-bounded by $\epsudlog (\secpar)$.

  \ncase{Conclusion}
  Note that the adversary can win in Game 2 only if $\tdv$ manages to produce a tree of acceptable transcript $\tree$, such that each of the transcripts in $\tree$ is acceptable by an ideal verifier. Note that since $\tdv$ produces $(\noofc + 6)$ acceptable transcripts for different challenges $\chz$, it obtains the same number of different evaluations of polynomials $\p{a} (X), \p{b} (X), \p{c} (X), \p{z} (X), \p{t} (X)$. Since the transcripts are acceptable by an idealised verifier, the equality between polynomial $\p{t} (X)$ and combination of polynomials $\p{a} (X), \p{b} (X), \p{c} (X), \p{z} (X)$ defined in prover's $3$-rd message description holds. Hence, $\p{a} (X), \p{b} (X), \p{c} (X)$ encodes the valid witness for the proven statement. Since $\p{a} (X), \p{b} (X), \p{c} (X)$ are of degree at most $(\noofc + 5)$ and there is more than $(\noofc + 5)$ their evaluations known, $\extcss$ can recreate polynomials' coefficients by interpolation and reveal the witness with probability $1$. Thus, the probability that extraction fails in that case is upper-bounded by probability of an idealised verifier failing $\epsid(\secpar)$.

  Hence, the probability that the adversary wins in Game 0 is upper-bounded by 
  \[
    \epscss(\secpar,\accProb, q) \leq \left(1 - \frac{\accProb - (q + 1) \epsid (\secpar)}{1 - \epsid (\secpar)}\right) + \epsudlog (\secpar) + \epsid (\secpar)\,. 
  \]
 \end{proof}
  

\subsection{Trapdoor-Less Zero-Knowledge of Plonk}
\begin{lemma}
  \label{lem:plonk_tlzk}
  Let $\plonkprot$ be zero knowledge with security $\epszk(\secpar)$. Then $\plonkprot$ is 3-programmable trapdoor-less zero-knowledge
  with security $\epszk(\secpar)$.
\end{lemma}

\begin{proof}
  As noted in \cref{def:upd-scheme}, subvertible zero knowledge implies updatable zero
  knowledge. Hence, here we show that Plonk is TLZK even against adversaries who picks
  the SRS on its own.

The adversary $\adv(\secparam)$ picks an SRS $\srs$ and instance--witness pair
$(\inp, \wit)$ and gets a proof $\zkproof$ simulated by the simulator
$\simulator$ which proceeds as follows.

For its $1$-st message the simulator  picks randomly both the randomisers $b_1, \ldots, b_6$ and
sets $\wit_i = 0$ for $i \in \range{1}{3\noofc}$. Then $\simulator$
outputs $\gone{\p{a}(\chi), \p{b}(\chi), \p{c}(\chi)}$. For the first
challenge, the simulator picks permutation argument challenges $\beta, \gamma$
randomly.

For its $2$-nd message, the simulator computes $\p{z}(X)$ from
the newly picked randomisers $b_7, b_8, b_9$ and coefficients of polynomials
$\p{a}(X), \p{b}(X), \p{c}(X)$. Then it evaluates $\p{z}(X)$ honestly and outputs
$\gone{\p{z}(\chi)}$. Challenge $\alpha$ that should be sent by the verifier
after the simulator's $2$ message is picked by the simulator at random.

In its $3$-rd message the simulator starts by picking at random a challenge $\chz$, which
in the real proof comes as a challenge from the verifier sent \emph{after} the prover
sends its $3$-rd message. Then $\simulator$ computes evaluations
\(\p{a}(\chz), \p{b}(\chz), \p{c}(\chz), \p{S_{\sigma 1}}(\chz), \p{S_{\sigma
    2}}(\chz), \pubinppoly(\chz), \lag_1(\chz), \p{Z_H}(\chz),\allowbreak
\p{z}(\chz\omega)\) and computes $\p{t}(X)$ honestly. Since for a random
$\p{a}(X), \p{b}(X), \p{c}(X), \p{z}(X)$ the constraint system is (with
overwhelming probability) not satisfied and the constraints-related polynomials
are not divisible by $\p{Z_H}(X)$, hence $\p{t}(X)$ is a rational function
rather than a polynomial. Then, the simulator evaluates $\p{t}(X)$ at $\chz$ and
picks randomly a degree-$(3 \noofc - 1)$ polynomial $\p{\tilde{t}}(X)$ such that
$\p{t}(\chz) = \p{\tilde{t}}(\chz)$ and publishes a commitment
$\gone{\p{\tilde{t}_{lo}}(\chi), \p{\tilde{t}_{mid}}(\chi),
  \p{\tilde{t}_{hi}}(\chi)}$. After that the simulator outputs $\chz$ as a
challenge.

For the next message, the simulator computes polynomial $\p{r}(X)$ as an honest
prover would, cf.~\cref{sec:plonk_explained} and evaluates $\p{r}(X)$ at $\chz$.

The rest of the evaluations are already computed, thus $\simulator$ simply outputs
\( \p{a}(\chz), \p{b}(\chz), \p{c}(\chz), \p{S_{\sigma 1}}(\chz), \p{S_{\sigma
    2}}(\chz), \p{t}(\chz), \p{z}(\chz \omega)\,.  \) After that it picks randomly
the challenge $v$, and prepares the the last message as an honest prover
would. Eventually, $\simulator$ and outputs the final challenge, $u$, by picking it
at random as well.

We argue about zero knowledge as usually. The property holds since the polynomials that has witness elements at their coefficients are randomized by at least two randomizers and are evaluated at at most two points; and the simulator computes all polynomials as an honest prover would.
\end{proof}

\subsection{Simulation Extractability of~$\plonkprotfs$}
Since \cref{lem:plonkprot_ur,lem:plonkprot_ss,lem:plonk_tlzk} hold, $\plonkprot$ is $\ur{3}$,
computational special sound and trapdoor-less zero knowledge. We now make use of \cref{thm:se} and show that
$\plonkprot_\fs$ is simulation-extractable as defined in \cref{def:updsimext}.

\begin{corollary}[Simulation extractability of $\plonkprot_\fs$]
\label{thm:plonkprotfs_se}
$\plonkprotfs$ is \emph{updatable simulation-extractable} against any $\ppt$ adversary $\advse$ who makes up to $q$ random oracle queries and returns an acceptable proof with probability at least $\accProb$ with extraction failure probability 
\[
  \epsse(\secpar, \accProb, q) \leq \left(1 - \frac{\accProb - \epsur (\secpar) - (q + 1) \epserr (\secpar)} {1 - \epserr (\secpar)}\right) + \epsudlog (\secpar) + \epsid (\secpar),
\]
where $\epserr (\secpar) = \left(1 - \frac{\noofc + 6}{\abs{\FF_p}} \right)$, $\abs{\FF_p}$ is the size of the field, and $\noofc$ is the number of constrains in the circuit. 
\end{corollary}
 
%%% Local Variables:
%%% mode: latex
%%% TeX-master: "main"
%%% End:
