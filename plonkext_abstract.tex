% !TeX spellcheck = en_GB
\let\accentvec\vec
\documentclass[runningheads,11pt]{llncs}
 % \documentclass[runningheads]{amsart}
\let\spvec\vec
\let\vec\accentvec
\usepackage{amssymb,amsmath}
\let\vec\spvec

\usepackage[T1]{fontenc}

\newcommand{\iflipics}[1] {}
\newcommand{\iflncs}[1] {#1}

\def\vec#1{\mathchoice{\mbox{\boldmath$\displaystyle#1$}}
	{\mbox{\boldmath$\textstyle#1$}}
	{\mbox{\boldmath$\scriptstyle#1$}}
	{\mbox{\boldmath$\scriptscriptstyle#1$}}}

\DeclareFontFamily{U}{mathx}{\hyphenchar\font45}
\DeclareFontShape{U}{mathx}{m}{n}{<-> mathx10}{}
\DeclareSymbolFont{mathx}{U}{mathx}{m}{n}
\DeclareMathAccent{\widebar}{0}{mathx}{"73}

% lncs size (as printed in books, with small margins):
% \usepackage[paperheight=23.5cm,paperwidth=15.5cm,text={13.2cm,20.3cm},centering]{geometry}


% \newcommand{\ifamsart}[1] {}
% \ifamsart{
% 	\newtheorem{theorem}{Theorem}%[section]
% 	\newtheorem{proposition}[theorem]{Proposition}
% 	\newtheorem{lemma}[theorem]{Lemma}
% 	\newtheorem{corollary}[theorem]{Corollary}
% 	\theoremstyle{definition}
% 	\newtheorem{definition}[theorem]{Definition}
% 	\newtheorem{example}[theorem]{Example}
% }
\usepackage{fullpage}
\usepackage{soul}
\usepackage{soulutf8}
\soulregister\cite7
\soulregister\ref7
\soulregister\pageref7
\usepackage{hyperref}
\usepackage[color=yellow]{todonotes}
\hypersetup{final}
\usepackage{mathrsfs}
\usepackage[advantage,asymptotics,adversary,sets,keys,ff,lambda,primitives,events,operators,probability,logic,mm,complexity]{cryptocode}
\usepackage[capitalise]{cleveref}
\usepackage{cite}
\usepackage{booktabs}
\usepackage{paralist}
\usepackage[innerleftmargin=5pt,innerrightmargin=5pt]{mdframed}

% \usepackage{fouriernc}

\newcommand{\newdefs}[1] {\setlength{\fboxsep}{1pt}\colorbox{gray!20}{\(#1\)}}

\newcommand{\COMMENT}[1]  {}

%general formatting
\newcommand{\pcvarstyle}[1]{\mathsf{#1}}
\newcommand{\comment}[1]{{\color{lightgray}#1}}
\newcommand{\continue}{{\Huge{\hl{$\cdots$}}}}

% General mathematics
\newcommand{\range}[2] {[#1 \, .. \, #2]}
\newcommand{\SD}{\Delta}
\newcommand{\smallset}[1] {\{#1\}}
\newcommand{\bigset}[1] {\left\{#1\right\}}
\newcommand{\GRP} {\mathbb{G}}
\newcommand{\pair} {\hat{e}}
\newcommand{\brak}[1] {\left(#1\right)}
\newcommand{\sbrak}[1] {(#1)}
\newcommand{\alg}[1] {\pcalgostyle{#1}}
\newcommand{\image} {\operatorname{im}}
\newcommand{\myland} {\,\land\,}
\newcommand{\mylor} {\,\lor\,}
\newcommand{\vect}[1] {\operatorname{vect}(#1)}
\newcommand{\w}{\omega}
\newcommand{\const}{\pcpolynomialstyle{const}}
\newcommand{\p}[1]{\pcpolynomialstyle{#1}}
\newcommand{\ev}[1]{\tilde{\pcpolynomialstyle{#1}}}
\newcommand{\numberofconstrains}{\pcvarstyle{n}}
\newcommand{\expected}[1]{\mathbb{E}\left[#1\right]}
\newcommand{\infrac}[2]{#1 / #2}

% bilinear maps

\newcommand{\bmap}[2] {\left[#1\right]_{#2}}
\newcommand{\gone}[1] {\bmap{#1}{1}}
\newcommand{\gtwo}[1] {\bmap{#1}{2}}
\newcommand{\gi} {\iota}
\newcommand{\gtar}[1] {\bmap{#1}{T}}
\newcommand{\grpgi}[1] {\bmap{#1}{\gi}}


% zero knowledge
\newcommand{\oracleo}{\mathsf{O}}
\newcommand{\crs}{\pcvarstyle{crs}}
\newcommand{\td}{\pcvarstyle{td}}
\newcommand{\ip}[2]{\left\langle #1, #2\right\rangle}
\newcommand{\zkproof}{\pi}
\newcommand{\proofsystem}{\mathrm{\Psi}}
\newcommand{\ps}{\proofsystem}
\newcommand{\nuppt}{\pcmachinemodelstyle{NUPPT}}
\newcommand{\ro}{\mathcal{H}}
\newcommand{\rof}[2]{\mathbf{\Omega}_{#1, #2}}
\newcommand{\trans}{\pcvarstyle{trans}}
\newcommand{\tr}{\pcvarstyle{tr}}
\newcommand{\instsize}{\pcvarstyle{n}}
\newcommand{\KG} {\mathsf{K}}
\newcommand{\kcrs} {\KG_{\crs}}
\renewcommand{\dist}{\ddv}
\newcommand{\fs}{\pcalgostyle{FS}}
\newcommand{\sigmaprot}{\pcalgostyle{\Sigma}}
\newcommand{\se}{\pcvarstyle{se}}
\newcommand{\snd}{\pcvarstyle{snd}}
\newcommand{\zk}{\pcvarstyle{zk}}
\newcommand{\advse}{\adv_\se}

%rewinding---tree of transcripts
\newcommand{\pcboolstyle}[1]{\mathtt{#1}}
\newcommand{\treebuild}{\pcalgostyle{TreeBuild}}
\newcommand{\tree}{\pcvarstyle{tree}}
\newcommand{\counter}{\pcvarstyle{counter}}


%PLONK related
\newcommand{\plonkprot}{\mathbf{P}}
\newcommand{\plonkprotfs}{\mathbf{P}_\fs}
\newcommand{\selector}[1]{\pcvarstyle{q_{#1}}}
\newcommand{\selmulti}{\selector{M}}
\newcommand{\selleft}{\selector{L}}
\newcommand{\selright}{\selector{R}}
\newcommand{\seloutput}{\selector{O}}
\newcommand{\selconst}{\selector{C}}
\newcommand{\chz}{\mathfrak{z}}
\newcommand{\reduction}{\rdv}

\newcommand{\game}[1]{\pcalgostyle{G}_{#1}}

\newcommand{\lag}{\p{L}}
\newcommand{\pubinppoly}{\p{PI}}

% general complexity theory
% \newcommand{\RND}[1]{\pcalgostyle{RND}(#1)}
\newcommand{\RND}[1]{\pcvarstyle{R}(#1)}
\newcommand{\RELGEN}{\mathcal{R}}
\newcommand{\REL}{\mathbf{R}}
\newcommand{\LANG}{\mathcal{L}}
\newcommand{\inp}{\pcvarstyle{x}}
\newcommand{\wit}{\pcvarstyle{w}}
\newcommand{\class}[1]{\mathfrak{#1}}
\newcommand{\ig}{\pcalgostyle{IG}}
\newcommand{\accProb}{\event{acc}}
\newcommand{\frkProb}{\event{frk}}
\newcommand{\FS}{\pcalgostyle{FS}} % Fiat-Shamir transform
\newcommand{\aux}{\pcvarstyle{aux}} %auxiliary input

%Plonk and Sonic
\newcommand{\plonk}{\ensuremath{\textsc{Plonk}}}
\newcommand{\plonkmod}{\ensuremath{\plonk^\star}}
\newcommand{\plonkint}{\ensuremath{\plonk^\star}}
\newcommand{\polyprot}{\pcalgostyle{poly}}
\newcommand{\plonkintpoly}{\plonkint_\polyprot}
\newcommand{\sonic}{\textsc{Sonic}}
\newcommand{\maxdegree}{\pcvarstyle{N}}

\newcommand{\dlog}{\pcvarstyle{dlog}}

\newcommand{\ur}[1]{{#1\text{-}\mathsf{ur}}}

%forking
\newcommand{\forking}{\pcalgostyle{F}}
\newcommand{\genforking}{\pcalgostyle{GF}}

%colors
\definecolor{darkmagenta}{rgb}{0.5,0,0.5}
\definecolor{lightmagenta}{rgb}{1,0.85,1}
\definecolor{lightmagenta}{rgb}{0.9,0.9,0.9}
\definecolor{darkred}{rgb}{0.7,0,0}
\definecolor{blueish}{rgb}{0.1,0.1,0.5}
\definecolor{pinkish}{rgb}{0.9,0.8,0.8}
\definecolor{darkgreen}{rgb}{0,0.6,0}
\definecolor{lightgreen}{rgb}{0.85,1,0.85}
\definecolor{skyblue}{rgb}{0.3,0.9,0.99}

%comments
\DeclareRobustCommand{\markulf}[2]  {{\color{darkmagenta}\hl{\scriptsize\textsf{Markulf #1:} #2}}}
\DeclareRobustCommand{\michals}[2]  {{\color{blueish}\sethlcolor{pinkish}\hl{\scriptsize\textsf{Michal #1:} #2}}}
\newcommand{\task}[2]{\todo[author=\textbf{Task},inline]{({\textit{#1}}) #2}}
% \newcommand{\task}[2] {\xcommenti{Task}{#1}{#2}}
% \DeclareRobustCommand{\task}[2]  {{\color{black}\sethlcolor{yellow}\hl{\textsf{TASK #1:} #2}}}

%%% Local Variables:
%%% mode: latex
%%% TeX-master: "plonkext"
%%% End:


% \title{On Simulation-Extractability of \textsc{Plonk}, abstract}
\title{On the Non-malleability of Real-world SNARKs}

\author{Markulf Kohlweiss\inst{1,2} \and Mikhail Volkhov\inst{1} \and Michał Zając\inst{3}}
\iflncs{
  \institute{University of Edinburgh, Edinburgh, UK \and IOHK \\ \email{mkohlwei@inf.ed.ac.uk}, \email{mikhail.volkhov@ed.ac.uk} \and Clearmatics, London UK \\ \email{m.p.zajac@gmail.com}}
}

\begin{document}
	\maketitle
	
	In the paper we focus on non-malleability properties of two popular zkSNARKs---\groth{}  \cite{EC:Groth16}, which is the most efficient zkSNARK for QAP, and \plonk{} \cite{EPRINT:GabWilCio19}, which is less efficient than \groth{} but is universal and updatable.
	We show that the former is (weakly) simulation-extractable and the latter is simulation-sound
	
	Importantly, the security proofs we show do not require any modification to the proof system. 
	Hence, there is no computational overhead compared to other works that took an existing zkSNARK and apply some existing transformation which inevitable lead to efficiency loss---either because the final proof is longer \cite{EPRINT:BowGab18}, the prover has to prove a more complicated statement \cite{ASIACCS:DerSla18,EPRINT:AbdRamSla20}), or, in the context of showing correctness of computation, the arithmetisation of choice is less efficient \cite{C:GroMal17}.	
	
	Additionaly, we show how to generalise these results to cover a broader class of protocols.  
	
	\subsubsection*{\groth{} as the most efficient zkSNARK.}
	Due to its simplicity, succinctness, and performance, \groth{} is
  currently the most widely deployed succinct (zero-knowledge) argument of
  knowledge (SNARK) system. \groth{} is known to be perfectly zero-knowledge and knowledge sound in the generic (and algebraic) group model. However, the existing security arguments for \groth{} are silent about the soundness of the proof system in the presence of simulated proofs---a common requirement for both the composable security and game-hopping style security analysis of protocols built using such argument systems. This important gap let to a line of work on simulation-extractable, also called simulation knowledge sound, succinct proof systems. \groth{} itself cannot satisfy the strongest
  notion of simulation-extractability that implies proof
  non-malleability---in fact proofs are perfectly randomizable.

  Surprisingly, in this short note we show that \groth{} does satisfy a
  weaker notion of simulation-extractability implying statement
  non-malleability. This property is often sufficient for typical
  applications that motivate the use of strong
  simulation-extractability. Notably, one can obtain UC security using
  efficient compilers.
	
	\subsubsection*{$\plonk$ as an efficient universal zkSNARK.}
	Although efficient, the \groth{} proof system has limitations. First of all,
  ts CRS is relation-dependent and must be regenerated whenever the relation
  changes. Such secure CRS generation is a troublesome task and requires
  either a trusted (by everyone) third party or computationally costly multi
  party computations.  In contract, $\plonk$ can be used to prove correct evaluation
  of any circuit (up to a given size). Hence, we call \plonk{} \emph{universal}.
	
	Similarly, \plonk{} relies on a milder trust assumption. 
	Although \groth{} can be shown to be \emph{subversion zero-knowledge} (i.e.~the zero knowledge property holds even if the CRS is subverted, e.g.~by the verifier), it is not sound if some untrusted party computed the CRS. 
	On the other hand, \plonk{} is \emph{updatable sound}. That is, the parties using the proof system can (verifiably) update the CRS; the security guarantee is that the system remains sound if at least one of the updating parties is honest.
	
	Last, but not least $\plonk$ is efficient. Although it does not use QAP for arithmetisation it is able to achieve similar proving times as \groth{}. (However the number of proof elements is considerably bigger). 
	It is arguably the most efficient updatable and universal zkSNARK up to date,
  especially when one considers implementations\footnote{We exclude \sonic{} with helpers
  from this comparison.}. 
	
	\subsubsection*{On the importance of simulation extractability.}
	Arguably, the strongest non-malleability property for SNARK systems are
  \emph{simulation soundness} (SS) and \emph{simulation extractability} (SE)
  \cite{FOCS:Sahai99,C:DDOPS01}---security notions that extends soundness (for
  SS) and knowledge soundness (for SE) by allowing the adversary to also access the simulation oracle and learn simulated proofs. 
	%
	One of the important properties of simulation extractability is that its
	straight-line, blackbox variant is necessary to achieve Universally Composable~(UC~\cite{FOCS:Canetti01}) security, as shown more generally in 	\cite{STOC:CLOS02,AC:Groth06,EC:GroOstSah06} for non-interactive zero-knowledge (NIZK) proof systems. Moreover, it is also needed in game-hopping style proofs~\cite{EPRINT:Shoup04} in which one game hop
	introduces the simulator and a subsequent game hop relies on knowledge
	soundness~\cite{SP:KMSWP16,CCS:CamDriDub17}.
	%
	Compared to just sound NIZKs, simulation sound NIZKs allow for more efficient transformation to black-box simulation-extractable and universally composable (UC) NIZKs \cite{AFRICACRYPT:Baghery19}. 

  This is crucial for efficiently realizing protocols such as Hawk and
  Crypsinous that employ the UC composition theorem in their analysis. However,
  non-malleability is important beyond UC security: As is well known, the
  malleability of bitcoin's ECDSA signatures, resulted in inconvenience and
  attacks\footnote{\url{https://bitcointalk.org/index.php?topic=8392.msg1245898\#msg1245898}.}
  In turn, the security properties of \texttt{Zerocash}, the precursor of
  \texttt{Zcash} requires transaction non-malleability. While ad-hoc solutions
  such as those employed by \texttt{Zcash} exist, simulation
  extractability is the gold standards for non-malleable proof systems that support both
  simulation (zero-knowledge) and extraction (proof of knowledge).

	
  % + implies both UC security, CCA security and strong signature unforgeability
  % for well known compilers

  % + weak simulation extractabilty is sufficient for many applications, RCCA,
  % stabndard signature unforgeability
		
	\subsubsection*{Preliminaries.}
	Before we continue we introduce definitions crucial to understand the following result. \michals{01.09}{Maybe we should remove it to gain some space?}
	\begin{description}
		\item[Algebraic Group Model] is a computational model introduced in \cite{C:FucKilLos18}, which assumes that each algorithm $\adv$ that takes a vector of group elements $\vec{g}$ and returns a vector of $n$ group elements $\vec{h}$ outputs also a matrix $\vec{A} = (\vec{a_1}, \ldots, \vec{a_n})$ such that $h_i = \vec{a_i}^\top \vec{g}$, for $i \in \range{1}{n}$.
		\item[zkSNARK] is a non-interactive zero-knowledge proof, which proof length is sublinear to the statement and witness size. Due to impossibility result of \cite{STOC:GenWic11}, zkSNARKs can be sound only under some non-falsifiable assumption, like the Algebraic Group Model.
		\item[Simulation soundness] is a strong notion of soundness, which states that no PPT adversary can break the soundness of the system even if it is given access to a simulator oracle that provides it simulated proofs for required statements.
		\item[Simulation extractability] is a strong notion of soundness, which states that for every adversary there exists an extractor able to extract a valid witness from a proven statement given an acceptable proof for that statement. This property holds even if the adversary has access to a simulator oracle that provides it simulated proofs for required statements. 
		\item[Unique response property] states that after some round (here, Round 3) prover's responses to verifier's challenges are deterministic. That is, no PPT adversary can produce two valid transcripts that have the same challenges yet different answers (after Round 3).
		\item[$q$-dlog assumption] states that it is infeasibly hard for a PPT adversary to output $\chi$ from $1, g^{\chi}, \ldots, g^{\chi^q}, h^{\chi}$, where $g \in \GRP_1$ and $h \in \GRP_2$.
	\end{description}
	
	\subsubsection*{Simulation-soundness of $\plonk$.}
	We were able to show simulation soundness of $\plonk$ by providing three reductions, each of which able to break some assumption if the proof system has not the desired security property. 
	More precisely, we show that if there exists a simulation extractability-breaking adversary $\advse$ then one can either break: 
	\begin{compactenum}
		\item \label{it:ks} knowledge soundness of the interactive protocol underlying $\plonk$; 
		\item \label{it:dlog} unique response property (after $3$-rd round) of $\plonk$; or 
		\item \label{it:ur} an $(\numberofconstrains + 2)$ discrete log assumption, for $\numberofconstrains$ being a number of constrain in the proven statement.
	\end{compactenum}
	
	We believe, although it is for the further work, that the result can be generalised to other protocols which relies on Fiat--Shamir transform and are shown to be knowledge sound in AGM.
	
	% \comment{
	\subsubsection*{Simulation-extractability of special-sound protocols. Generalised forking lemma}
	Independently on the previous results, we also show that a broad class of protocols is simulation extractable. 
	Following protocols like \cite{EPRINT:GabWilCio19,CCS:MBKM19,EC:CHMMVW20} we focus on \emph{interactive} proofs of knowledge that are made non-interactive using random oracle and the Fiat--Shamir transform. For our reduction to work we require the underlying interactive protocol to 
	\begin{compactenum}
		\item be special-sound, that is given a tree of acceptable transcripts, which share some root, it is feasible to extract a witness; and
		\item have unique response property.
	\end{compactenum}
	For that end, we prove a generalisation of the general forking lemma which applies to proofs that require more than two transcript for witness extraction.
	% }
	
	\bibliographystyle{abbrv}
	\bibliography{cryptobib/abbrev1,cryptobib/crypto,additional_bib}
\end{document}

