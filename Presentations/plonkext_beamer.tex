\documentclass[]{beamerikz}
% Package options:
% [] (none)	- compile everything, final version, without grid
% [final]	- the same as nothing
% [ready]	- as [] but add compilation info to the first plain frame
% [short]   - compile all the frames in preview mode (one slide per frame),
%				no grid visible
% [draft]	- compile all the frames,
%				those with parameter [show] in full step-by-step manner,
%				others in preview mode (one slide per frame),
%				show grid and slide number on all [show] frames
% [brief]   - compile only frames with [show] (in full step-by-step manner),
%				others are replaced by a single [DISABLED] slide,
%				show grid and slide number on all [show] frames

%\usepackage{helvet}
\usepackage{roboto} 
\usepackage{mathptmx}
\usepackage{PlayfairDisplay}
\renewcommand*\oldstylenums[1]{{\playfairOsF #1}}
\usepackage[T1]{fontenc}
\usefonttheme[stillsansseriftext]{serif}
% \usepackage{mathptmx}
% \DeclareSymbolFont{letters}{OML}{ztmcm}{m}{it}
% \DeclareSymbolFontAlphabet{\mathnormal}{letters}

% set title to appear at the bottom of frames
% you can access this value using \bzTitle
\sBzTitle{{What is non-malleability and why it matters?}}

% set author to appear at the bottom of frames
% you can access this value using \bzAuthor
\sBzAuthor{Markulf Kohlweiss and \underline{Michal Zajac}}

% You can set up colors for your presentation using the following macros
% Use some online palette generator to get nice colours :)

\definecolor{bzblack}{RGB}{0,0,0}
\definecolor{bzgreen}{RGB}{0,150,0}
\definecolor{bzblue}{RGB}{0,0,255}
\definecolor{bzred}{RGB}{255,0,0}
\definecolor{bzwhite}{RGB}{255,255,255}
\definecolor{bzgray}{RGB}{230,230,230}

\renewcommand{\bzCtext}{bzblack}
\renewcommand{\bzCemph}{bzblack}
\renewcommand{\bzChigh}{bzblack}
\renewcommand{\bzCname}{bzred}
\renewcommand{\bzCback}{bzgray}

\newcommand{\ftitle}[1]{\bzLeft{\textrm{\Large{#1}}}}

\usepackage[advantage,asymptotics,adversary,sets,keys,ff,lambda,primitives,events,operators,probability,logic,mm,complexity]{cryptocode}

\renewcommand{\prover}{\pcadvstyle{P}}
\renewcommand{\verifier}{\pcadvstyle{V}}
\renewcommand{\kgen}{\pcadvstyle{K}}
\renewcommand{\simulator}{\pcadvstyle{S}}
\newcommand{\pcvarstyle}[1]{\mathsf{#1}}
\newcommand{\srs}{\pcvarstyle{srs}}
\newcommand{\inp}{\pcvarstyle{x}}
\newcommand{\wit}{\pcvarstyle{w}}

\usepackage{anyfontsize}
\makeatletter
\newcommand\HUGE{\@setfontsize\Huge{100}{120}}
\makeatother   
% =============================================================================

\begin{document}

% The title frame that you can adjust as you wish (e.g. add logo in a corner).
% This frame has no footline.
\begin{bzPlainFrame}
\bzOn{
	\iBzH[6.0]
	\bzCenter{\textrm{\Large{What is non-malleability and why it matters?}}}

	%\iBzH[1.0]
	\bzCenter[scale=0.7]{Markulf Kohlweiss and \fbox{Michal Zajac}}

	% \iBzH[1.5]
	% \bzCenter[scale=0.7]{Conference 2018\\City}
}
\end{bzPlainFrame}

% =============================================================================
\begin{bzFrame}[show]
  \bzOn{
    \ftitle{Proof system}
    \bzLeft{3 parties:
      $(\kgen, \prover, \verifier)$}
    \iBzI[1.0]
    \bzLeft{\eb{SRS generator} $\kgen$ \hl{trusted} party that generates parameters}
    \bzLeft{\eb{Prover} $\prover$ that proves statements}
    \bzLeft{\eb{Verifier} $\verifier$ that verifies proofs}
    %\bzItem{Simulator $\simulator$ (special party needed for zero-knowledge)}
  }
\end{bzFrame}

\begin{bzFrame}[show]
  \bzOn{
    \ftitle{Proof system $\kgen, \prover, \verifier$}
    \iBzH[3.0]
    \bzCenter{\HUGE{$\kgen$}}
    \bzEvalInt{\kg}{\bzN}
    \iBzH[4.0]
    \iBzI
    \bzLeft{\HUGE{$\prover$}}
    \bzEvalInt{\pro}{\bzN}
    \bzLeft{$\inp, \wit$}
    \iBzH[-2.0]
    \bzRight{\HUGE{$\verifier$}}
    \bzEvalInt{\ver}{\bzN}
    \iBzI[1.0]
    \bzRight{$\inp$}
  }

  \bzOn{
    \draw
    (\bzLabel{\kg}.west)
    edge[Circle-Latex, bend right] node[anchor=east] {$\srs$}
    (\bzLabel{\pro}.north);

    \draw
    (\bzLabel{\kg}.east)
    edge[Circle-Latex, bend left] node[anchor=west] {$\srs$} (\bzLabel{\ver}.north);
  
  }

\bzOn{
	% always use \bzH instead of fixed vertical coordinates!
	\draw
		(\bzLabel{\pro}.east)
		edge[Circle-Latex] node[anchor=west] {}
		(\bzLabel{\ver}.west);
}
\end{bzFrame}

\begin{bzFrame}
  \bzOn{ \ftitle{Zcash -- an oversimplied overview}
    \iBzH[1.0]
    \iBzI
    \draw (0,\bzH) rectangle node{$cm_1$} (2,\bzH+1);
    \draw (2.5,\bzH) rectangle node{$cm_2$} (4.5,\bzH+1);
    \node at (5,\bzH+0.5) {...};
    \draw (5.5, \bzH) rectangle node{$cm_n$} (7.5, \bzH+1);
    %\iBzH[4.0]
    % \fill[red!30, text=black,rounded corners=3pt, align = left] (-10, \bzH + 1) rectangle node[align=left]{set
    %   of \bf{coins}} (-1, \bzH-1);
    % \fill[red!30, text=black,rounded corners=3pt] (-10, \bzH -1) rectangle
    % node{each coin hides:} (-1, \bzH-3);
    \bzLeft{Set of \bf{coins}}
    \bzLeft{Each coin $cm_i$ has}
    \iBzI
    \bzLeft{value $v_i$}
    \bzLeft{serial number $sn_i$}
  }
  \bzUnder{
    \fill[red!30, text=black,rounded corners=3pt] (-10, \bzH + 5) rectangle (-2, \bzH-1);
  }
\end{bzFrame}

\begin{bzFrame}[show]
  \bzOn{
    \bzLeft[bzBO]{Malleability attacks}
    \iBzH[4.0]
    \iBzI
    \bzLeft{\HUGE{$\adv$}}
    \bzEvalInt{\aaa}{\bzN}
    \iBzH[-1.0]
    \bzRight{\HUGE{$\bdv$}}
    \bzEvalInt{\bb}{\bzN}
  }

  \bzOn{
    \bzH[-3.0]
	  %\bzUnder{
		% \fill[red!30, rounded corners=3pt] (-6,\bzH+2.6) rectangle (6,\bzH+0.6) node
    % [pos=0.5] {\textcolor{black}$\adv$ makes proof that confirms that he
    %    posses \\ means to pay $\bdv$ \$100};
		
		\dBzH
    \iBzI[10.0]
    \node[draw, align=left] at (0,-4) {$\adv$ makes proof that confirms that he \\
      posses means to pay $\bdv$ \$100};
		% inside you can put all the content you like, it overlays one another
		% \bzCenter[scale=0.8]{$\adv$ makes proof that confirms that he posses means
    %   to pay $\bdv$ \$100}
	%}
}

  \end{bzFrame}
  
% The first content frame of the presentation.
% This frame has parameter [show] so will be compiled in the full step-by-step
% manner no matter what package options are used (except short).
% [show] option applies also to frames created using bzPlainFrame!
\begin{bzFrame}%[show]

% The flow of the presentation is controlled by the following variables:
%	\bzS - the number of the next slide, starting from 1
%	\bzH - the height of the next item (from 0 downward to -14)
%	\bzI - the indent of the next item (from -10 up to +10)
% 	\bzN - the number used to label the last displayed node
% 	\bzL - the number used to number the last displayed list element
% All these variables are automatically modified, as explained below.

% Here we discuss \bzH.

% \bzOn creates a new slide of the presentation (shortly slide).
% The content of the macro will be visible from that slide onward (On).
% It increases \bzS by one.

  \bzOn{
	\bzCenter{Non-highlighted title of the frame}
	\bzCenter[bzBO]{Boldfaced title}
	\bzCenter[bzHL]{Highlighted title}
	\bzCenter[bzHB]{Highlighted\&boldfaced title}
	\bzCenter[bzEL]{Emphasized title}
	\bzCenter[bzEB]{Emphasized\&boldfaced title}
	\bzLeft{Partially \bo{boldfaced}, \hl{highlighted}, \hb{bold-lighted} text, aligned to left}
	\bzText{\el{Emphasized} and \eb{bold-emphed} text, aligned to left with an indent}
	\bzRight{This text is aligned to the right}
	
	% We can introduce H-space by incrementing the \bzH paremeter using \iBzH
	\iBzH
	\bzText[bzEL]{The options bzBO, bzHL, bzHB, bzEL, and bzEB apply to all these elements}
}

% \iBzH has optional parameter saying how much space to introduce
\iBzH[2.0]

\bzOn{
	\bzLeft{\ldots and more content appears lower}
}

% \dBzH moves content up (by default by 1)
\dBzH[2.0+3.0]

\bzOn{
	\bzCenter[bzHL]{this content appears higher than expected}
}

% \zBzH resets \bzH to 0
\zBzH

\bzOn{
	\bzLeft{- HI!}
	\bzRight{!IH-}
}
\end{bzFrame}

% =============================================================================

% The second content frame of the presentation. It's layout depends on
% the class option used because the [show] parameter is commented-out here.
\begin{bzFrame}%[show]

% \bzCoords shows a coordinate system
% it is also always available in compilation modes draft and brief
\bzCoords

\bzOn{
	% \iBzI increases indent controlled by \bzI
	\iBzI
	\bzLeft{indented}
	
	% \iBzI[2.0] increases indent by two
	\iBzI[2.0]
	\bzLeft{indented more}
	
	% the value of \bzI influences right-aligned content as well
	\bzRight{right-indented more}
	
	% \dBzI[2.0] decreases indent by two
	\dBzI[2.0]
	\bzLeft{indented just a bit}
	
	% \zBzI resets \bzI back to -10
	\zBzI
	\bzLeft{not indented at all}
}

% let's store the current \bzH in a floating point variable
\bzEval{\itemsHeight}{\bzH}

\bzOn{
	% \bzItem creates itemized list (it is also affected by \bzI)
	\bzItem{item 1}
}

\bzOn{
	\bzItem{item 2}
}

\bzOn{
	\bzItem{item 3}
}

% lets increase index
\iBzI[3]

\bzOn{
	% \bzList creates itemized list (it is also affected by \bzI)
	\bzList{list item 1}
	
	% \bzL stores number of last displayed item
	\bzList{this list item has number \bzL}
	\bzList{list item 3}
}

% \zBzL resets item counter to 0
\zBzL

% let's zero indent
\zBzI

\bzOn{
	\bzList{new list item \bzL}
	\bzList{new list item \bzL}
	\bzList{new list item \bzL}
}

% move right up
\dBzH[\itemsHeight-\bzH]
\iBzI[10]

% \bzItemDot controls the shape of the item dot
\renewcommand{\bzItemDot}{{\bf X}}

\bzOn{
	\bzItem{new item 1}
	\bzItem{new item 2}
	\bzItem{new item 3}	
}

\iBzH[3]

% \zBzItemDot resets the shape of the item dot
\zBzItemDot

\bzOn{
	\bzItem{new item 1}
	\bzItem{new item 2}
	\bzItem{new item 3}	
}

\end{bzFrame}

% =============================================================================

\begin{bzFrame}%[show]
\bzOn{
	\bzTheorem{Authors}{You can also use \hb{Lemma}, \hb{Fact}, \hb{Conjecture}, \hb{Definition}, \hb{Question}, \hb{Problem}, \hb{Proposition}, \hb{Corollary}, \hb{Exercise}, and \hb{Example}.}
	
	\iBzH
}

\bzOn{
	\bzExample{}{If you skip the authors, they disappear :)}
	\bzText{\bzNames{abc} recognises its argument empty, like here: \bzNames{}.}
}

% \bzI influences these environments as well
\iBzI

\bzOn{
	\bzProof{Which is indented by default by $1$ but here by $2$ (because of the previous indentation that increased in by $1$).}
}

\iBzH[1.5]

% normal nodes do not admit displayed math inside...
\bzOn{
	\bzCenter{$\bzEq{\sin(\alpha) + \int_x^y 1\ \mathrm{d}z}$}
}

\zBzI

\bzOn{
	\bzQed
}

\bzOn{
	\bzBox{Shows text that can extend over multiple lines, like in this particular, slightly lengthy example, where multiple words are used but not much content is in fact delivered\ldots}
}

\iBzH[1]

% notice that 1cm = 1unit !
\bzOn{
	\bzLeft[bzHL]{\hspace{4cm}for multi-line nodes the height is not increased appropriately!}
	
	% and \bzLine shows a horizontal line
	\bzLine
}
\end{bzFrame}

% =============================================================================

\begin{bzFrame}%[show]

% The following content will be visible from the beginning of the frame until
% the 4'th slide.
% We put the \bzS+n everywhere here to allow automatical shifting - we can add
% some \bzOn before this part and everything will be appropriately delayed.
\bzIn{\bzS+0}{\bzS+3}{
	\bzLeft{Title that \hl{disappears} in\ }
	\bzEvalInt{\titNum}{\bzN}
	\bzNext{\ \ \ slides}
	
	\iBzH[13]
	\bzLeft[bzHL]{Sub-title aligned at the bottom of the slide}
	\dBzH[13]
}

% countdown (\bzS is not modified but used for the sake of shift-invariance)
\bzIn{\bzS+0}{\bzS+0}{
	\bzNext{4}[\titNum];
}

% again
\bzIn{\bzS+1}{\bzS+1}{
	\bzNext{3}[\titNum];
}

% again
\bzIn{\bzS+2}{\bzS+2}{
	\bzNext{2}[\titNum];
}

% and the last
\bzIn{\bzS+3}{\bzS+3}{
	\bzNext{1}[\titNum];
}

% We need to explicitly increment \bzS to let the title be visible alone in the first slide
\iBzS

% This slide will appear and then show that it has disappeared (one more step, without it).
% \bzOnly increases \bzS by two
\bzOnly{
	\bzTheorem{}{Unnamed theorem that will vanish\ldots}
	\bzLemma{Author One, Author Two [1986]}{Named lemma that will vanish\ldots}
}

\bzOn{
	\bzCenter{$\bzEq{\left\{\int_x^{x+1} f(x)\mathrm{d}x\mid\text{$x$ is displayed in inline-math style}\right\}}$}
}

\dBzH[7]

% increase \bzS to give some time
\iBzS[1]

% \bzOne increases \bzS by one
\bzOne{
	\bzLeft{Something that immediately disappears}
}

% \bzOne also increases \bzS by one, but lasts two slides
\bzTwo{
	\bzLeft{Something that takes two slides to disappear}
}


\bzOn{
	\bzText{Something visible forever}
}

% leaving the last argument empty shows that content until the end of a frame
\bzIn{\bzS-3}{}{
	\bzText{And a bit more}
}

\iBzS

% leaving the first argument empty shows that content from the beginning
\bzIn{}{\bzS-3}{
	\bzRight{And even more}
}
\end{bzFrame}

% =======================================================================================

% Frame with a title of a subsection (no footline nor frame number)
\begin{bzPlainFrame}%[show]
\bzOn{
	\iBzH[4.5]
	\bzCenter[scale=1.4, bzEB]{Part II}
}
\end{bzPlainFrame}

% =======================================================================================

% The third frame of the presentation. In this frame we draw around in TikZ.
\begin{bzFrame}%[show]

% We start one line below the top of the frame
\iBzH

\bzOn{
	% the empty space at the end here is on purpose...
	\bzLeft{The fancy formula:$\ $}
	
	% \bzNext puts the node next to the previously drawn one
	\bzNext{$e\to m\ c^{2+\varepsilon}$}
		
	% store the number of the formula's node in \formN
	\bzEvalInt{\formN}{\bzN}
	
	\bzNext{$\ $ will be finally explained.}
}

\bzOnly{
	% under-brace localised relatively to the formula node
	\path
		($(\bzLabel{\formN}.south west)+(0,-0.1)$)
		edge[bzBraceU] node {\hb{super}-quadratic in $c$ {\bf !!!}}
		($(\bzLabel{\formN}.south east)+(0,-0.1)$);
}

\iBzH[4]

\bzOn{
	\bzTheorem{Author [date]}{$<$triviality$>$}
}

% We put a nice pink rectangle UNDER the theorem.
% Notice that \bzOn is inside \bzUnder, not the other way round.
% This order is important!
\bzUnder{
	\bzOn{
		\fill[red!30, rounded corners=3pt] (-10.3,\bzH+2.6) rectangle (10.3,\bzH+0.6);
		
		\dBzH
		% inside you can put all the content you like, it overlays one another
		\bzCenter[scale=0.8]{hahaha}
	}
}

\bzOn{
	% always use \bzH instead of fixed vertical coordinates!
	\draw
		(-1.7,\bzH+2)
		edge[Circle-Latex, bend right] node[anchor=west] {implies that $\varepsilon<e^{-1}$}
		(\bzLabel{\formN}.south east);
}

\bzOn{
	\bzCenter{CENTER\hspace{3cm}CENTER}
	\bzEvalInt{\cenN}{\bzN}
}

\bzOn{
	\bzCenter{centered}
	\bzPrev{prev-}
	\bzNext{-next}
}

% \bzPrev does not label its node
% but \bzNext does, so here we have a problem:
\bzOnly{
	\bzCenter{...........}
	\bzNext{$|$next}
	\bzPrev{prev$|$}
}

% It's fine if you add explicit reference argument:
\bzOn{
	\bzNext{-NEXT}[\cenN]
	\bzPrev{PREV-}[\cenN]
}

\bzOn{
    \bzText{Condensed enumeration:}
}

\bzOn{
    \iBzI
    \bzItem{option 1,}
}

\bzOn{
	% continue in the same line
    \bzNext{ option 2,}
    \bzEvalInt{\lineToExtend}{\bzN} %store node number for further filling
}

\bzTwo{
    \bzNext{ yet another option}
}

\bzOnly{
    \draw[-] (\bzLabel{\bzN}.north east) -- (\bzLabel{\bzN}.south west);
}

\bzOn{
    \iBzI[2]
    \bzText{One can refer to the previously typeset objects (nodes)!}
}

% reset the indent
\zBzI
% and then decrease it even more
\dBzI

\bzOn{
    \bzNext{~or rather many different options.}[\lineToExtend]
    \bzText{Note that a meanwhile change of \hl{indent} does not affect line continuations!}
}

\end{bzFrame}

% =======================================================================================

\end{document}
