% !TEX root = main.tex
% !TEX spellcheck = en-US

\section{Simulation Soundness---definitions and the general result}
\noindent \textbf{Simulation sound NIZKs in the updatable setting.}
Another notion for non-malleable NIZKs is \emph{simulation soundness}. It allows the adversary to see simulated proof, however, in contrast to simulation
extractability it does not require an extractor to provide a witness for the
proven statement. Instead, it is only necessary, that an adversary who sees
simulated proofs cannot make the verifier accept a proof of an incorrect
statement. More precisely,


\begin{definition}[Simulation soundness in the updatable setting]
	\label{def:simsnd}
	Let $\ps = (\kgen, \prover, \verifier, \simulator)$ be a NIZK proof system. We say that
  $\ps$ is \emph{updatable simulation-sound} if for any $\ppt$ adversary $\adv$ that is
  given oracle access to an updatable SRS setup $\initU$, cf.~\cref{fig:upd}, a simulation oracle $\simulator
  = (\simulator_1, \simulator_2')$, and a random oracle $\ro$, probability
	\[
	\ssndProb = \condprob{
		\begin{matrix}
		\verifier(\srs, \inp_{\advse}, \zkproof_{\advse}) = 1 \\
		\wedge  ~(\inp_{\advse}, \zkproof_{\advse}) \not\in Q   \\
		\wedge \neg \exists \wit_{\adv}: \REL(\inp_{\adv}, \wit_{\adv}) = 1
		\end{matrix}
	}{
		\begin{aligned}
		& r \sample \RND{\advse},
		(\inp_{\advse}, \zkproof_{\advse}) \gets \advse^{\initU, \simulator_1, \simulator_2'} (1^\secpar; r) \\
		\end{aligned}
	}
	\]
	is at most negligible.  
	Here, $\srs$ is the finalized SRS, list $Q$ contains all $(\inp, \zkproof)$ pairs where 
	$\inp$ is an instance provided to the simulator by the adversary and
	$\zkproof$ is the simulator's answer. List $Q_\ro$ contains all $\advse$'s
	queries to $\ro$ and $\ro$'s answers.  
\end{definition}

\label{rem:simext_to_simsnd}
We note that the probability $\ssndProb$ in~\cref{def:simsnd} can be expressed in
terms of simulation-extractability. More precisely, the
condition $\neg \exists \wit: \REL(\inp_\adv, \wit_\adv) = 1$ can be substituted with
$\REL(\inp_\adv, \wit_\adv) = 0$, where $\wit_\adv$, returned by a possibly unbounded
extractor, is either a witness to $\inp_\adv$ (if there exists any) or $\bot$ (if
there is none). More precisely,
\[
\ssndProb = \condprob{
	\begin{matrix}
	\verifier(\srs, \inp_{\advse}, \zkproof_{\advse}) = 1 \\
	\wedge  ~(\inp_{\advse}, \zkproof_{\advse}) \not\in Q   \\
	\wedge  ~\REL(\inp_{\advse}, \wit_{\advse}) = 0
	\end{matrix}
}{
	\begin{aligned}
	& r \sample \RND{\advse},
	(\inp_{\advse}, \zkproof_{\advse}) \gets \advse^{\initU, \simulator_1, \simulator_2'} (1^\secpar; r) \\
	& \wit_{\advse} \gets \ext(\srs, \advse, r, \inp_{\advse}, \zkproof_{\advse},
	Q, Q_\ro, Q_\srs) 
	\end{aligned}
}
\]
The only necessary input to the unbounded extractor $\ext$ is the instance
$\inp_\adv$ (the rest is given for the consistency with the simulation extractability
definition). 
%
With the probabilities in \cref{def:simext} holding regardless of whether the extractor
is unbounded or not, we obtain the following equality
$ \ssndProb = \accProb - \extProb$.

\subsection{Simulation soundness---the general result}
\label{sec:general}
Equipped with definitional framework of \cref{sec:se_definitions}, we can also show the proof of simulation soundness of Fiat-Shamir NIZKs based on multi-round protocols.

%\begin{theorem}[Forking simulation-extractable multi-message protocols]
%	\label{thm:se}
%	Let $\ps = (\kgen, \prover, \verifier, \simulator)$ be an interactive $(2 \mu + 1)$-message
%	zero-knowledge proof system for $\RELGEN(\secparam)$, which is trapdoor-less simulatable, has
%	$\ur{k}$ property with security $\epsur(\secpar)$, and is $(\epss(\secpar), k, n)$-forking
%	sound.  Let $\ro\colon \bin^{*} \to \bin^{\secpar}$ be a random oracle.  Then $\psfs$ is
%	forking simulation-extractable with extraction error $\epsur(\secpar)$ against $\ppt$
%	algebraic adversaries that makes up to $q$ random oracle queries and returns an acceptable
%	proof with probability at least $\accProb$.  The extraction probability $\extProb$ is at
%	least
%	\( \extProb \geq \frac{1}{q^{n - 1}} (\accProb - \epsur(\secpar))^{n} -\eps(\secpar)\,, \)
%	for some negligible $\eps(\secpar)$.
%\end{theorem}

\begin{theorem}[Simulation soundness]
	\label{thm:simsnd}
	Let $\ps = (\kgen, \prover, \verifier, \simulator)$ be an interactive $(2 \mu + 1)$-message
	zero-knowledge proof system for $\RELGEN(\secparam)$, which is trapdoor-less simulatable, has
	$\ur{k}$ property with security $\epsur(\secpar)$. Let $\ro\colon \bin^{*} \to \bin^{\secpar}$ be a random oracle. Then, the
	probability that a $\ppt$ adversary $\adv$ breaks simulation soundness of
	$\ps$ is upper-bounded by
	\(
	\epsur(\secpar) + q_\ro^\mu  \epss(\secpar)\,,
	\)
	where $q$ is the total number of queries made by the adversary $\adv$ to $\ro$.
\end{theorem}

\begin{proof}
	\ngame{0} This is a simulation soundness game played between an adversary
	$\advse$ who is given access to an oracle $\initU$ that defines an updatable SRS setup, a random oracle $\ro$ and a simulation oracle
	$\simulator$. $\adv$ wins if it manages to produce an accepting proof
	for a false statement. In the following game hops, we upper-bound the
	probability that this happens.
	
	\ngame{1} This is identical to $\game{0}$ except that the game is aborted if
	there is a simulated proof $\zkproof_\simulator$ for $\inp_{\adv}$ such that
	$(\inp_{\adv}, \zkproof_\simulator[1..k]) = (\inp_{\adv},
	\zkproof_{\adv}[1..k])$. That is, the adversary in its final proof reuses at
	least $k$ messages from a simulated proof it saw before and the proof is
	accepting.  Denote this event by $\event{\errur}$.
	
	\ncase{Game 0 to Game 1} We have, \( \prob{\game{0} \land
		\nevent{\errur}} = \prob{\game{1} \land \nevent{\errur}} \) and, from the
	difference lemma, cf.~\cref{lem:difference_lemma},
	$ \abs{\prob{\game{0}} - \prob{\game{1}}} \leq \prob{\event{\errur}}\,$.
	Thus, to show that the transition from one game to another introduces only
	minor change in probability of $\adv$ winning it should be shown that
	$\prob{\event{\errur}}$ is small.
	
	We can assume that $\adv$ queried the simulator on the instance it wishes to
	output, i.e.~$\inp_{\adv}$. We show a reduction $\rdvur$ that utilises $\adv$
	to break the $\ur{k}$ property of $\ps$. Let $\rdvur$ run $\advse$ internally
	as a black-box:
	\begin{compactitem}
		\item The reduction answers $\adv$ update queries by asking the same query from the update oracle in the unique response experiment. The reduction finalises the same SRS $\srs$ as the one $\adv$ does.
		\item The reduction answers both queries to the simulator $\psfs.\simulator$
		and to the random oracle.  It also keeps lists $Q$, for the simulated
		proofs, and $Q_\ro$ for the random oracle queries.
		\item When $\adv$ makes a fake proof $\zkproof_{\adv}$ for $\inp_{\adv}$,
		$\rdvur$ looks through lists $Q$ and $Q_\ro$ until it finds
		$\zkproof_{\simulator}[0..k]$ such that
		$\zkproof_{\adv}[0..k] = \zkproof_{\simulator}[0..k]$ and a random oracle
		query $\zkproof_{\simulator}[k].\ch$ on $\zkproof_{\simulator}[0..k]$.
		\item $\rdvur$ returns two proofs for $\inp_{\adv}$:
		\begin{align*}
		\zkproof_1 = (\zkproof_{\simulator}[1..k],
		\zkproof_{\simulator}[k].\ch, \zkproof_{\simulator}[k + 1..\mu + 1])\\
		\zkproof_2 = (\zkproof_{\simulator}[1..k],
		\zkproof_{\simulator}[k].\ch, \zkproof_{\adv}[k + 1..\mu + 1])
		\end{align*}
	\end{compactitem}  
	If $\zkproof_1 = \zkproof_2$, then $\adv$ fails to break simulation soundness,
	as $\zkproof_2 \in Q$. On the other hand, if the proofs are not equal, then
	$\rdvur$ breaks $\ur{k}$-ness of $\ps$. This happens only with negligible
	probability $\epsur(\secpar)$, hence
	\( \prob{\event{\errur}} \leq \epsur(\secpar)\,. \)
	
	\ngame{2} This is identical to $\game{1}$ except that now the environment
	aborts if the instance the adversary proves is not in the language.
	
	\ncase{Game 1 to Game 2} 
	% REDUCTION TO INTERACTIVE SOUNDNESS:
	We show that
	$\abs{\prob{\game{1}} - \prob{\game{2}}} \leq q^{\mu} \cdot \epss(\secpar)$,
	where $\epss(\secpar)$ is the probability of breaking soundness of the underlying
	\emph{interactive} protocol $\ps$. Note that
	$\abs{\prob{\game{1}} - \prob{\game{2}}}$ is the probability that $\adv$
	outputs an acceptable proof for a false statement which does not break the
	unique response property (such proofs have been excluded by
	$\game{1}$). Consider a soundness adversary $\adv'$ who initiates a proof with
	$\ps$'s verifier $\ps.\verifier$, internally runs $\adv$ and proceeds as
	follows:
	\begin{compactitem}
		\item It guesses indices $i_1, \ldots, i_\mu$ such that random oracle queries
		$h_{i_1}, \ldots, h_{i_\mu}$ are the queries used in the $\zkproof_\adv$
		proof eventually output by $\adv$. This is done with probability at least
		$1/q^\mu$ (since there are $\mu$ challenges from the verifier in
		$\ps$).
		\item On input $h$ for the $i$-th,
		$i \not\in \smallset{{i_1}, \ldots, {i_\mu}}$, random oracle query, $\adv'$
		returns randomly picked $y$, sets $\ro(h) = y $ and stores $(h, y)$ in
		$Q_\ro$ if $h$ is sent to $\ro$ the first time. If that is not the case,
		$\adv$ finds $h$ in $Q_\ro$ and returns the corresponding $y$.
		\item On input $h_{i_j}$ for the $i_j$-th,
		$i_j \in \smallset{{i_1}, \ldots, {i_\mu}}$, random oracle query, $\adv'$
		parses $h_{i_j}$ as a partial proof transcript $\zkproof_\adv[1..j]$ and
		runs $\ps$ using $\zkproof_\adv[j]$ as a $\ps.\prover$'s $j$-th message to
		$\ps.\verifier$. The verifier responds with a challenge
		$\zkproof_\adv[j].\ch$. $\adv'$ sets $\ro(h_{i_j}) =
		\zkproof_\adv[j].\ch$. If we guessed the indices correctly we have that
		$h_{i_{j'}}$, for $j' \leq j$, parsed as $\zkproof_\adv[1..j']$ is a prefix
		of $\zkproof_\adv[1..j]$.
		\item On query $\inp_\simulator$ to $\simulator$, $\adv'$ runs the simulator
		$\ps.\simulator$ internally. Note that we require a simulator that only
		programs the random oracle for $j \geq k$.  If the simulator makes a
		previously unanswered random oracle query with input
		$\zkproof_\simulator[1..j]$, $1 \leq j < k$, and this is the $i_j$-th query,
		it generates $\zkproof_\simulator[j].\ch$ by invoking $\ps.\verifier$ on
		$\zkproof_\simulator[j]$ and programs
		$\ro(h_{i_j}) = \zkproof_\simulator[j].\ch$.  It returns
		$\zkproof_\simulator$.
		\item Answers $\ps.\verifier$'s final challenge $\zkproof_\adv[\mu].\ch$ using the
		answer given by $\adv$, i.e.~$\zkproof_\adv[\mu]$.
	\end{compactitem}
	That is, $\adv'$ manages to break soundness of $\ps$ if $\adv$ manages to
	break simulation soundness without breaking the unique response property and
	$\adv'$ correctly guesses the indices of $\adv$ random oracle queries. This
	happens with probability upper-bounded by $\abs{\prob{\game{1}} -
		\prob{\game{2}}} \cdot \infrac{1}{q^{\mu}}$. Hence $\abs{\prob{\game{1}} -
		\prob{\game{2}}} \leq q^{\mu} \cdot \epss(\secpar)$.
	
	Note that in $\game{2}$ the adversary cannot win. Thus the probability
	that $\advss$ is successful is upper-bounded by
	$\epsur(\secpar) + q^{\mu} \cdot \epss(\secpar)$.  \qed
\end{proof}


We conjecture that based on the recent results on state restoration soundness~\cite{cryptoeprint:2020:1351}, which effectively allows to query the verifier multiple times on different overlapping transcripts, the $q^{\mu}$ loss could be avoided. However, this would reduce the class of protocols covered by our results. 


\subsection{Simulation soundness of~$\plonkprotfs$}
Since \cref{lem:plonkprot_ur,lem:plonkprot_ss,lem:plonk_hvzk} hold, $\plonkprot$ is $\ur{2}$, computational special sound and trapdoor-less simulatable. We now make use of \cref{thm:simsnd} and show that
$\plonkprot_\fs$ is simulation sound as defined in
\cref{def:simsnd}.

 \begin{corollary}[Simulation soundness of $\plonkprot_\fs$]
   \label{cor:simsnd-P}
   Assume that $\plonkprot$ is $2$-programmable HVZK in the standard model, that
   is computational special sound with security $\epss(\secpar)$, and the $\PCOMp$ is a commitment of knowledge with
   security $\epsk(\secpar)$, binding security $\epsbind(\secpar)$ and has unique
   opening property with security $\epsop(\secpar)$. Then the probability that a
   $\ppt$ adversary $\adv$ breaks simulation soundness of $\plonkprot_{\fs}$ is
   upper-bounded by
   \( \epsk(\secpar) + 2\cdot\epsbind(\secpar) + \epsop(\secpar) + q_\ro^4
   \epss(\secpar)\,, \) where $q$ is the total number of queries made by the
   adversary $\adv$ to a random oracle $\ro\colon \bin^{*} \to \bin^{\secpar}$.
 \end{corollary}

\subsection{Simulation soundness of~$\sonicprotfs$}
The following corollary shows the simulation soundness of $\sonicprotfs$ based on~\cref{lem:sonicprot_ur,lem:sonicprot_ss,lem:sonic_hvzk} and~\cref{thm:simsnd}.
\begin{corollary}[Simulation soundness of $\sonicprot_\fs$]
	\label{cor:simsnd-S}
	Assume that $\sonicprot$ is $1$-programmable HVZK in the standard model, that
	is computational special sound with security $\epss(\secpar)$, and the $\PCOMs$ is a commitment of knowledge with
	security $\epsk(\secpar)$, binding security $\epsbind(\secpar)$ and has unique
	opening property with security $\epsop(\secpar)$. Then the probability that a
	$\ppt$ adversary $\adv$ breaks simulation soundness of $\sonicprot_{\fs}$ is
	upper-bounded by
	\( \epsk(\secpar) + 2\cdot\epsbind(\secpar) + \epsop(\secpar) + q_\ro^4
	\epss(\secpar)\,, \) where $q$ is the total number of queries made by the
	adversary $\adv$ to a random oracle $\ro\colon \bin^{*} \to \bin^{\secpar}$.
\end{corollary}

\subsection{Simulation soundness of~$\marlinprotfs$}
	The simulation soundness of $\marlinprot_\fs$ follows from \cref{thm:simsnd}, and \cref{lem:marlinprot_ur,lem:marlinprot_ss,lem:marlin_hvzk}.
\begin{corollary}[Simulation soundness of $\marlinprot_\fs$]
	\label{cor:simsnd-M}
	Assume that $\marlinprot$ is $1$-programmable HVZK in the standard model, that
	is computational special sound with security $\epss(\secpar)$, and the $\PCOM$ is a commitment of knowledge with
	security $\epsk(\secpar)$, binding security $\epsbind(\secpar)$ and has unique
	opening property with security $\epsop(\secpar)$. Then the probability that a
	$\ppt$ adversary $\adv$ breaks simulation soundness of $\marlinprot_{\fs}$ is
	upper-bounded by
	\( \epsk(\secpar) + 2\cdot\epsbind(\secpar) + \epsop(\secpar) + q_\ro^4
	\epss(\secpar)\,, \) where $q$ is the total number of queries made by the
	adversary $\adv$ to a random oracle $\ro\colon \bin^{*} \to \bin^{\secpar}$.
\end{corollary}


%%% Local Variables:
%%% mode: latex
%%% TeX-master: "main"
%%% End:
