% l\newif\ifupdate
% \updatetrue % Uncomment to compile non-updatable version.

% !TEX spellcheck = en-US
% \let\accentvec\vec              
%\documentclass[runningheads,11pt]{llncs}
%\documentclass[10pt]{llncs}
\documentclass[10pt]{llncs}
\pagestyle{plain}
\let\spvec\vec
\let\vec\accentvec

\newcommand\hmmax{0}
\newcommand\bmmax{0}

\DeclareFontFamily{U}{mathx}{\hyphenchar\font45}
\DeclareFontShape{U}{mathx}{m}{n}{<-> mathx10}{}
\DeclareSymbolFont{mathx}{U}{mathx}{m}{n}
\DeclareMathAccent{\widebar}{0}{mathx}{"73}

\let\spvec\vec
\usepackage{amssymb,amsmath}
\let\vec\spvec
%\usepackage{newtxmath,newtxtext}
\usepackage{newtxtext}
\usepackage[T1]{fontenc}
\usepackage[most]{tcolorbox}
  \def\vec#1{\mathchoice{\mbox{\boldmath$\displaystyle#1$}}
  {\mbox{\boldmath$\textstyle#1$}} {\mbox{\boldmath$\scriptstyle#1$}}
  {\mbox{\boldmath$\scriptscriptstyle#1$}}}


% lncs size (as printed in books, with small margins):
 % \usepackage[paperheight=23.5cm,paperwidth=15.5cm,text={13.2cm,20.3cm},centering]{geometry}
%\usepackage{fullpage}
\usepackage{soulutf8} \soulregister\cite7 \soulregister\ref7
\soulregister\pageref7
\usepackage{hyperref}
\usepackage[color=yellow]{todonotes} \hypersetup{final}
\usepackage{mathrsfs}
\usepackage[advantage,asymptotics,adversary,sets,keys,ff,lambda,primitives,events,operators,probability,logic,mm,complexity]{cryptocode}

\usepackage[capitalise]{cleveref}
% \crefname{appendix}{Supp.~Mat.}{Supp.~Mat.}
% \Crefname{appendix}{Supp.~Mat.}{Supp.~Mat.}
% \
\usepackage{cite} 
\usepackage{booktabs}
\usepackage{paralist}
%\usepackage[innerleftmargin=5pt,innerrightmargin=5pt]{mdframed}
\usepackage{caption}
\captionsetup{belowskip=0pt}
\usepackage{bm}
\usepackage{url}
%\usepackage{dirtytalk}
%\usepackage[margin=1in,a4paper]{geometry}
\usepackage[normalem]{ulem}
\usepackage{dashbox}
\newcommand\dboxed[1]{\dbox{\ensuremath{#1}}}
\usepackage{setspace}

\usepackage{floatrow}
\floatsetup[figure]{font=footnotesize}
 
\newcommand{\newdefs}[1] {\setlength{\fboxsep}{1pt}\colorbox{gray!20}{\(#1\)}}

\newcommand{\COMMENT}[1]  {}

%general formatting
\newcommand{\pcvarstyle}[1]{\mathsf{#1}}
\newcommand{\comment}[1]{{\color{lightgray}#1}}
\newcommand{\continue}{{\Huge{\hl{$\cdots$}}}}

% General mathematics
\newcommand{\range}[2] {[#1 \, .. \, #2]}
\newcommand{\SD}{\Delta}
\newcommand{\smallset}[1] {\{#1\}}
\newcommand{\bigset}[1] {\left\{#1\right\}}
\newcommand{\GRP} {\mathbb{G}}
\newcommand{\pair} {\hat{e}}
\newcommand{\brak}[1] {\left(#1\right)}
\newcommand{\sbrak}[1] {(#1)}
\newcommand{\alg}[1] {\pcalgostyle{#1}}
\newcommand{\image} {\operatorname{im}}
\newcommand{\myland} {\,\land\,}
\newcommand{\mylor} {\,\lor\,}
\newcommand{\vect}[1] {\operatorname{vect}(#1)}
\newcommand{\w}{\omega}
\newcommand{\const}{\pcpolynomialstyle{const}}
\newcommand{\p}[1]{\pcpolynomialstyle{#1}}
\newcommand{\ev}[1]{\tilde{\pcpolynomialstyle{#1}}}
\newcommand{\numberofconstrains}{\pcvarstyle{n}}
\newcommand{\expected}[1]{\mathbb{E}\left[#1\right]}
\newcommand{\infrac}[2]{#1 / #2}

% bilinear maps

\newcommand{\bmap}[2] {\left[#1\right]_{#2}}
\newcommand{\gone}[1] {\bmap{#1}{1}}
\newcommand{\gtwo}[1] {\bmap{#1}{2}}
\newcommand{\gi} {\iota}
\newcommand{\gtar}[1] {\bmap{#1}{T}}
\newcommand{\grpgi}[1] {\bmap{#1}{\gi}}


% zero knowledge
\newcommand{\oracleo}{\mathsf{O}}
\newcommand{\crs}{\pcvarstyle{crs}}
\newcommand{\td}{\pcvarstyle{td}}
\newcommand{\ip}[2]{\left\langle #1, #2\right\rangle}
\newcommand{\zkproof}{\pi}
\newcommand{\proofsystem}{\mathrm{\Psi}}
\newcommand{\ps}{\proofsystem}
\newcommand{\nuppt}{\pcmachinemodelstyle{NUPPT}}
\newcommand{\ro}{\mathcal{H}}
\newcommand{\rof}[2]{\mathbf{\Omega}_{#1, #2}}
\newcommand{\trans}{\pcvarstyle{trans}}
\newcommand{\tr}{\pcvarstyle{tr}}
\newcommand{\instsize}{\pcvarstyle{n}}
\newcommand{\KG} {\mathsf{K}}
\newcommand{\kcrs} {\KG_{\crs}}
\renewcommand{\dist}{\ddv}
\newcommand{\fs}{\pcalgostyle{FS}}
\newcommand{\sigmaprot}{\pcalgostyle{\Sigma}}
\newcommand{\se}{\pcvarstyle{se}}
\newcommand{\snd}{\pcvarstyle{snd}}
\newcommand{\zk}{\pcvarstyle{zk}}
\newcommand{\advse}{\adv_\se}

%rewinding---tree of transcripts
\newcommand{\pcboolstyle}[1]{\mathtt{#1}}
\newcommand{\treebuild}{\pcalgostyle{TreeBuild}}
\newcommand{\tree}{\pcvarstyle{tree}}
\newcommand{\counter}{\pcvarstyle{counter}}


%PLONK related
\newcommand{\plonkprot}{\mathbf{P}}
\newcommand{\plonkprotfs}{\mathbf{P}_\fs}
\newcommand{\selector}[1]{\pcvarstyle{q_{#1}}}
\newcommand{\selmulti}{\selector{M}}
\newcommand{\selleft}{\selector{L}}
\newcommand{\selright}{\selector{R}}
\newcommand{\seloutput}{\selector{O}}
\newcommand{\selconst}{\selector{C}}
\newcommand{\chz}{\mathfrak{z}}
\newcommand{\reduction}{\rdv}

\newcommand{\game}[1]{\pcalgostyle{G}_{#1}}

\newcommand{\lag}{\p{L}}
\newcommand{\pubinppoly}{\p{PI}}

% general complexity theory
% \newcommand{\RND}[1]{\pcalgostyle{RND}(#1)}
\newcommand{\RND}[1]{\pcvarstyle{R}(#1)}
\newcommand{\RELGEN}{\mathcal{R}}
\newcommand{\REL}{\mathbf{R}}
\newcommand{\LANG}{\mathcal{L}}
\newcommand{\inp}{\pcvarstyle{x}}
\newcommand{\wit}{\pcvarstyle{w}}
\newcommand{\class}[1]{\mathfrak{#1}}
\newcommand{\ig}{\pcalgostyle{IG}}
\newcommand{\accProb}{\event{acc}}
\newcommand{\frkProb}{\event{frk}}
\newcommand{\FS}{\pcalgostyle{FS}} % Fiat-Shamir transform
\newcommand{\aux}{\pcvarstyle{aux}} %auxiliary input

%Plonk and Sonic
\newcommand{\plonk}{\ensuremath{\textsc{Plonk}}}
\newcommand{\plonkmod}{\ensuremath{\plonk^\star}}
\newcommand{\plonkint}{\ensuremath{\plonk^\star}}
\newcommand{\polyprot}{\pcalgostyle{poly}}
\newcommand{\plonkintpoly}{\plonkint_\polyprot}
\newcommand{\sonic}{\textsc{Sonic}}
\newcommand{\maxdegree}{\pcvarstyle{N}}

\newcommand{\dlog}{\pcvarstyle{dlog}}

\newcommand{\ur}[1]{{#1\text{-}\mathsf{ur}}}

%forking
\newcommand{\forking}{\pcalgostyle{F}}
\newcommand{\genforking}{\pcalgostyle{GF}}

%colors
\definecolor{darkmagenta}{rgb}{0.5,0,0.5}
\definecolor{lightmagenta}{rgb}{1,0.85,1}
\definecolor{lightmagenta}{rgb}{0.9,0.9,0.9}
\definecolor{darkred}{rgb}{0.7,0,0}
\definecolor{blueish}{rgb}{0.1,0.1,0.5}
\definecolor{pinkish}{rgb}{0.9,0.8,0.8}
\definecolor{darkgreen}{rgb}{0,0.6,0}
\definecolor{lightgreen}{rgb}{0.85,1,0.85}
\definecolor{skyblue}{rgb}{0.3,0.9,0.99}

%comments
\DeclareRobustCommand{\markulf}[2]  {{\color{darkmagenta}\hl{\scriptsize\textsf{Markulf #1:} #2}}}
\DeclareRobustCommand{\michals}[2]  {{\color{blueish}\sethlcolor{pinkish}\hl{\scriptsize\textsf{Michal #1:} #2}}}
\newcommand{\task}[2]{\todo[author=\textbf{Task},inline]{({\textit{#1}}) #2}}
% \newcommand{\task}[2] {\xcommenti{Task}{#1}{#2}}
% \DeclareRobustCommand{\task}[2]  {{\color{black}\sethlcolor{yellow}\hl{\textsf{TASK #1:} #2}}}

%%% Local Variables:
%%% mode: latex
%%% TeX-master: "plonkext"
%%% End:


%% Save the class definition of \subparagraph
\let\llncssubparagraph\subparagraph
%% Provide a definition to \subparagraph to keep titlesec happy
\let\subparagraph\paragraph
%% Load titlesec
\usepackage[compact]{titlesec}
%% Revert \subparagraph to the llncs definition
\let\subparagraph\llncssubparagraph

\newcommand{\oursubsub}[1] {\smallskip\noindent\textbf{#1}}
\newcommand{\ourpar}[1] {\smallskip\noindent\emph{#1}}

%\title{SNARKY Signatures with Universal and Updatable~Setup}

\title{What Makes Fiat--Shamir zkSNARKs \\ (Updatable SRS) Simulation Extractable?}


%\author{(Submission to EUROCRYPT 2022)}
\institute{}
\author{}

\iffalse
\author{Chaya Ganesh\inst{1} \and Hamidreza Khoshakhlagh\inst{2} \and Markulf Kohlweiss\inst{3,4} \and \\ Anca Nitulescu\inst{5} \and Michał Zając\inst{6}} 
\institute{Indian Institute of Science \\
  \email{chaya@iisc.ac.in}\\
  \and
  Aarhus University \\
  \email{hamidreza@cs.au.dk} \\
  \and
  University of Edinburgh, Edinburgh, UK \and IOHK \\
  \email{mkohlwei@inf.ed.ac.uk}
  \and
  Protocol Labs \\ \email{anca@protocol.ai} \\
  \and
  Nethermind, London, UK \\
\email{m.p.zajac@gmail.com}}
\fi

\allowdisplaybreaks

\begin{document} \sloppy
%\titlerunning{Non-Malleability of the FS transform Revisited [\ldots]}
\maketitle


\begin{abstract}
  We show that three popular universal zero-knowledge SNARKs (Plonk, Sonic, and Marlin) are simulation extractable NIZKs and signatures of knowledge
  (SoK) with updatable SRS out-of-the-box avoiding unnecessary compilation overheads.

  Towards this we generalize results for the Fiat--Shamir (FS) transformation that turns interactive protocols into
  signature schemes, non-interactive proof systems, or SoK in the random oracle model (ROM).  The security of the transformation relies on rewinding to
  extract the secret key or the witness, even in the presence of signing queries for
  signatures and simulation queries for proof systems and SoK, respectively.  We
  build on this line of work and analyze multi-round FS for interactive arguments with a structured reference string (SRS). The combination of ROM and SRS, while redundant in theory, is the model of choice for the most efficient practical argument systems to date. We also consider the case where the SRS
  is updatable and define a strong simulation extractability notion that allows for
  simulated proofs with respect to an SRS to which the adversary can contribute
  updates.
  
  We define three properties (trapdoor-less zero-knowledge, rewinding-based knowledge soundness, and a unique response property) that are sufficient for argument systems based on multi-round FS to be also simulation extractable in this strong sense. We show that Plonk, Sonic, and Marlin satisfy these properties, and conjecture that many other proof systems such as Lunar, Basilisk, and transparent variants of Plonk also fall within the reach of our main theorem.
  
  
\end{abstract}

%\ifupdate
%
%%!TEX root=main.tex

\textbf{To do list:}\\
	-- Investigating the following differences between \textcolor{red}{Attema et al. result} and \textcolor{green}{Ours}:\\
	\begin{tabular}{{m{0.7\textwidth}m{0.1\textwidth}m{0.15\textwidth}}}
		Task & & Who \\ \hline
		\textcolor{red}{expected poly-time extractors (that extract with non-negligible probability)} vs \textcolor{green}{strict poly-time extractors (that extract with overwhelming probability)}&  & --- \\ \hline 
		\textcolor{red}{passive security} vs \textcolor{green}{adaptive security} & &--- \\ \hline
		\textcolor{red}{proofs} vs \textcolor{green}{arguments} -- game hops from arguments to
         idealised proof system then using Attema et al for proofs& & Michał \\ \hline
	\end{tabular}\vspace{0.5cm}\\
		-- Improving readability, including\\
		\begin{tabular}{{m{0.7\textwidth}m{0.1\textwidth}m{0.15\textwidth}}}
			Task & & Who \\ \hline
			 checking proofs and making sure they are readable, (and still necessary!) & &--- \\ \hline
			 checking readability changes in git commit ``Wednesday 01.19.22, 8:13pm'' & &--- \\ 
		\end{tabular}



% !TEX root = main.tex
% !TEX spellcheck = en-US
\section{Introduction}
Zero-knowledge proof systems that allow a prover to convince a verifier of a statement without revealing anything beyond the truth of the statement have broad application in cryptography and theory of computation~\cite{FOCS:GolMicWig86,STOC:Fortnow87,C:BGGHKMR88}.
When restricted to computationally sound proofs, called \emph{argument systems}, proof length can be shorter than the length of the witness~\cite{brassard1988minimum}. 
Zero-knowledge Succinct Non-interactive ARguments of Knowledge (zkSNARKs) are zero-knowledge argument systems that additionally have a succinctness property -- small proof sizes and fast verification. 
Since their introduction in~\cite{FOCS:Micali94}, zk-SNARKs have been a powerful and versatile design tool for secure cryptographic protocols. They became particularly relevant for blockchain applications that demand short proofs and fast verification, such as privacy-preserving cryptocurrencies~\cite{SP:BCGGMT14} in Zcash and scalable and private smart contracts in Ethereum\footnote{\url{https://z.cash/} and \url{https://ethereum.org} respectively}.

%The work of~\cite{EC:GGPR13} proposed a preprocessing zk-SNARK for general NP statements phrased in the language of Quadratic Span Programs (QSP) and Quadratic Arithmetic Programs (QAP) for Boolean and arithmetic circuits respectively. This built on previous works of~\cite{IKO07,AC:Groth10a,TCC:Lipmaa12} and led to several works~\cite{TCC:BCIOP13,SP:PHGR13,C:BCGTV13,AC:Lipmaa13,USENIX:BCTV14,EC:Groth16} with very short proof sizes and fast verification.

While research on zkSNARKs has seen rapid
progress~\cite{EC:GGPR13,AC:Groth10a,TCC:Lipmaa12,TCC:BCIOP13,SP:PHGR13,C:BCGTV13,AC:Lipmaa13,USENIX:BCTV14,EC:Groth16}
with many works proposing significant improvements in efficiency of different
parameters like proof size, verifier efficiency, and complexity of the public setup,
less attention has been paid to non-malleable zkSNARKs and succinct signatures of
knowledge, (SoK) also known as SNARKY signatures~\cite{C:GroMal17,EPRINT:BKSV20}. A
signature of knowledge~\cite{C:CamSta97,C:ChaLys06} uses an instance of an
NP-language as the public verification key. Instead of signing using a secret key,
which typically would be related to the public key via a discrete logarithm or some
other hard relation~\cite{AC:DHLW10}, SoK signing requires knowledge of the
NP-witness. Chase and Lysyanskaya~\cite{C:ChaLys06} require signatures of knowledge
to be simulatable to assure protection against signing key/witness extraction. Given
a trapdoor associated with the public setup, signatures can be simulated without the
witness. Furthermore, to model strong existential unforgeability of signatures, even
when given an oracle for obtaining signatures on different instances, an attacker
must not be able to produce new signatures. Chase and Lysyanskaya model this via the
notion of simulation extractability (SE) which guarantees extraction of the witness
even in the presence of simulated signatures.  Moreover, Groth and Maller
\cite{C:GroMal17} showed how to construct SoK from zkSNARK schemes that are
simulation-extractable.  Therefore, our focus can be moved from SoK to the main
building block, zkSNARK schemes, for which we have many new efficient constructions
in recent literature.
 

\paragraph{Relevance of simulation extractability.}

Most zkSNARKs are shown to only satisfy a standard knowledge soundness
property. Intuitively, this guarantees that a prover that creates a valid proof knows
a valid witness. However, deployments of zkSNARKs in real-world applications, unless
they are carefully designed to have application specific malleability protection,
e.g.~\cite{SP:BCGGMT14}, require a stronger property --
\textit{simulation-extractability} -- that as discussed above corresponds more
closely to existential unforgeability of signatures.  In practice, an adversary
against the zkSNARK has access to proofs provided by other parties using the same
zkSNARK. The definition of knowledge soundness ignores the ability of an adversary
to see other valid proofs that may occur in real-world applications. For instance,
in applications of zkSNARKs in privacy-preserving blockchains, proofs are posted on
the chain for all blockchain-participants to see.
% Therefore, it is necessary for a zero-knowledge proof system to be
% \emph{non-malleable}, that is, resilient against adversaries that additionally get
% to see proofs generated by different parties before trying to forge.  Therefore, it
% is necessary for a zero-knowledge proof system to be \emph{simulation-extractable},
% that is, resilient against adversaries that additionally get to see proofs
% generated by different parties before trying to forge.  This captures the more
% general scenario where an adversary against the zkSNARK has access to proofs
% provided by other parties as it is in applications of zkSNARKs in
% privacy-preserving blockchains, where proofs are posted on the chain for all
% participants in the network to verify.

\paragraph{Fiat-Shamir based zkSNARKs.}
Most modern zkSNARK constructions follow a modular blueprint that involves the design of an information theoretic interactive protocol, e.g. an Interactive Oracle Proof (IOP), that is then compiled via cryptographic tools to obtain an interactive argument system.  This is then turned into a zkSNARK using the hash-based Fiat-Shamir transform. By additionally hashing the message, the Fiat-Shamir transform is also a popular technique for constructing signatures. While well understood for 3-message sigma protocols and justifiable in the random oracle
model~\cite{CCS:BelRog93}, it is only a heuristic that should be used with
care since there are counterexamples that Fiat-Shamir is
unsound~\cite{FOCS:GolKal03} and there are real-world attacks when implemented incorrectly~\cite{Blog:FrozenHeart20}.

%The Fiat--Shamir (FS) transform takes a public-coin interactive protocol and makes it interactive by hashing the current protocol transcript to compute the verifier's public coins.
%
%The FS transform is a popular design tool for constructing
%zkSNARKs. In the updatable universal SRS setting, works like \sonic{}~\cite{CCS:MBKM19}
%\plonk{}~\cite{EPRINT:GabWilCio19}, and \marlin~\cite{EC:CHMMVW20} are designed
%and proven secure as multi-round interactive protocols. Security is then only
%\emph{conjectured} for their non-interactive variants by employing the FS
%transform.

In particular, several schemes such as
$\sonic$~\cite{CCS:MBKM19}, $\plonk$~\cite{EPRINT:GabWilCio19}, $\marlin$~\cite{EC:CHMMVW20} 
follow this approach where the information theoretic object is a multi-message algebraic variant of IOP, and the cryptographic primitive in the compiler is a polynomial commitment scheme (PC) that requires a trusted setup. To date, this blueprint lacks an analysis in the ROM in terms of simulation extractability.


\paragraph{Updatable setup zkSNARKs.}
One of the downsides of efficient zkSNARKs like~\cite{AC:Groth10a,TCC:Lipmaa12,EC:GGPR13,SP:PHGR13,AC:Lipmaa13,AC:DFGK14,EC:Groth16} is that they rely on a \textit{trusted setup}, where there is a structured reference string (SRS) that is assumed to be generated by a trusted party. In practice, however, this assumption is not well founded; if the party that generates the SRS is not honest, then they can produce proofs for false statements. That is, if the trusted setup assumption does not hold, knowledge soundness breaks down.
Groth et al~\cite{C:GKMMM18} propose a setting to tackle this challenge which allows parties -- provers and verifiers -- to \emph{update} the SRS, that is, take a current SRS and contribute to its randomness in a verifiable way to obtain a new SRS. The guarantee in this \textit{updatable setting} is that knowledge soundness holds as long as one of the parties who updates the SRS is honest. The SRS is also \emph{universal}, in that it does not depend on the relation to be proved but only on an upper bound on the size of the statements.
Although inefficient, as the SRS length is quadratic in the size of the statement,~\cite{C:GKMMM18} set a new
paradigm of universal updatable setting for designing zkSNARKs.

The first universal zkSNARK with updatable and linear size SRS was
$\sonic$ proposed by Maller et al.~in \cite{CCS:MBKM19}. Subsequently, Gabizon et
al.~designed $\plonk$~\cite{EPRINT:GabWilCio19} which currently is the
most efficient updatable universal zkSNARK. Independently, Chiesa et
al.~\cite{EC:CHMMVW20} proposed $\textsf{Marlin}$ with comparable efficiency to
$\plonk$.

\paragraph{The challenge of SE in the updatable setting.}

The notion of simulation-extractability for zkSNARKs which is well motivated in practice has not been studied in the updatable setting.
Consider the following scenario: We assume a rushing adversary that starts off with a sequence of malicious updates by colluding parties resulting in a subverted reference string $\srs$. By combining their trapdoor contributions and employing the simulation algorithm, these parties can easily compute a proof $(\srs,\inp,\pi)$ for a statement $\inp$ without knowing a witness. Now, assume that at a later stage a party produces a proof $(\srs',\inp,\pi')$ for the same statement with respect to an updated $\srs'$ that has an honest update contribution. We would like the guarantee that this party must know a witness corresponding to $\inp$. The ability to ``move" the proof $\pi$ from the old SRS to a proof $\pi'$ for the new SRS without knowing a witness would clearly violate security. A natural idea is to require that honestly \emph{updated} reference strings are indistinguishable from honestly \emph{generated} reference strings. However, this is not sufficient as the rushing adversary can also rush toward the end of the SRS generation ceremony to perform the last update.
%That is, an adversary who does not knows the trapdoor for the update from $\srs$ to $\srs'$ should not be able to break SE. % as long as there was at least one honest update to $\srs$.\markulf{30/09/2021}{We currently don't achieve this strong USE notion.}


A definition of SE in the updatable setting should take these additional powers of the adversary, which are not captured by existing definitions of SE, into consideration.
While generic lifting techniques/compilers~\cite{EPRINT:KZMQCP15,CCS:AbdRamSla20} can be applied to updatable SRS SNARKs to obtain SE, not only do they inevitably incur overheads and lead to efficiency loss, we contend that the standard definition of SE does not suffice in the updatable setting.

We investigate the non-malleability properties of a class of zkSNARK protocols obtained by FS-compiling multi-message protocols in the updatable SRS setting and give a modular approach to analyse the simulation-extractability of zkSNARKs.
\subsection{Our Contributions}
\begin{itemize}
\item 
\emph{Updatable simulation extractability (USE)}. 
We propose a definition of simulation extractability in the updatable SRS setting called USE, that captures the additional power the adversary gets by being able to update the SRS.% and seeing proofs with respect to different SRSs.\michals{28.09}{Now the adversary sees additional proofs wrt to the final SRS.}
    
  \item \emph{General theorem for USE of FS-compiled interactive protocols.} We
        then show that a class of interactive proofs of knowledge that when
        the Fiat--Shamir transform is applied to them are trapdoor-less zero-knowledge, have a
        unique response property in the updatable setting, and satisfy a
        property we define called rewinding-based
        knowledge soundness \emph{are USE
        out-of-the box} in the random oracle model. Informally, our notion of rewinding-based knowledge soundness is a variant of special soundness where 
        %the transcripts provided to the extractor are obtained through interaction with an honest verifier, and 
        the extraction guarantee is computational
        instead of unconditional. Our extractor only needs oracle access to the
        adversary, it does not depend on the adversary’s code, nor does it rely on
        knowledge assumptions.
    
\item
\emph{USE for concrete zkSNARKs.}
We then prove that the most efficient updatable SRS SNARKS -- Plonk/Sonic/Marlin -- satisfy the notions necessary to invoke our general theorem. We thus show that these zkSNARKs are updatable simulation extractable.
In instantiating our general theorem for these concrete zkSNARK candidates, we rely on the algebraic group model (AGM).

\item
  \emph{SNARKY signatures in the updatable setting.} Our results validate the folklore that the Fiat--Shamir transform is a natural means for constructing signatures of knowledge. This gives rise to the first SoK in the updatable setting and confirms that a much larger class of zkSNARKs, besides \cite{C:GroMal17}, can be lifted to SoK.
	
%\item
%We make several technical contributions along the way. Our generalized forking lemma is of independent interest.
\end{itemize}



\subsection{Technical Overview}
%unique response, forking special soundness. general theorem without additional assumptions. to apply the theorem to concrete schemes like plonk, we show it satisfies unique response, forking soundness, in AGM.

At a high level, the proof of our general theorem for updatable simulation
extractability is along the lines of the simulation extractability proof for
FS-compiled sigma protocol from~\cite{INDOCRYPT:FKMV12}. However, our theorem
introduces new notions that are more general to allow us to consider proof
systems that are richer than sigma protocols and support an updatable setup. We
discuss some of the technical challenges below.

\plonk{}, \sonic{}, and \marlin{} were originally presented as interactive
proofs of knowledge that are made non-interactive by the Fiat--Shamir transform.
In the following, we denote the underlying interactive protocols by $\plonkprot$
(for $\plonk$), $\sonicprot$ (for $\sonic$), and $\marlinprot$ (for \marlin) and
the resulting non-interactive proof systems by $\plonkprotfs$, $\sonicprotfs$,
$\marlinprotfs$ respectively.

\oursubsub{Rewinding-Based Knowledge Soundness.}
Following~\cite{INDOCRYPT:FKMV12}, one would have to show that for the protocols
we consider a witness can be extracted from sufficiently many valid transcripts
with a common prefix. The standard definition of special soundness for sigma
protocols requires extraction of a witness from any two transcripts with the
same first message. However, most zkSNARK protocols do not satisfy this notion.
We put forth a notion analogous to special soundness, that is more general and
applicable to a wider class of protocols. Namely, protocols can have more than three
messages and can rely on an (updatable) SRS. $\plonkprot$, $\sonicprot$, and
$\marlinprot$ have more than 3 messages and the number of transcripts required for extraction is more
than two. Concretely, $(3 \noofc + 16)$ -- where $\noofc$ is the number of
constraints in the proven circuit -- for Plonk, $(\multconstr + \linconstr + 1)$
-- where $\multconstr$ and $\linconstr$ are the numbers of multiplicative and
linear constraints -- for Sonic, and $(\multconstr + 3)$ -- where $\multconstr$
is the number of multiplicative constraints -- for Marlin. Hence, we do not have
a pair of transcripts, but a \emph{tree of transcripts}.

Furthermore, the protocols we consider are arguments and rely on a SRS that comes with a trapdoor. An adversary in
possession of the trapdoor can produce multiple valid proof transcripts without
knowing the witness and potentially for false statements. This is true even in
the updatable setting, where there still exists a trapdoor for any updated SRS---it is just harder to subvert. Recall
that the standard special soundness definition requires witness extraction from
\emph{any} tree of accepting transcripts that share a common root. This means
that there are no such trees for false statements. 

Instead, we give a rewinding-based knowledge soundness definition with an extractor that proceeds in two steps. It first uses a tree building algorithm $\tdv$ to obtain a tree of transcripts. In the second step, it uses a tree extraction algorithm $\extcss$ to compute a witness from this tree. Tree-based knowledge soundness guarantees that it is possible to extract a witness from all
(but negligibly many) trees of accepting transcripts produced by probabilistic
polynomial time (PPT) adversaries. That
is, if extraction from such a tree fails, then we break an underlying
computational assumption. Moreover, this should hold even against adversaries
that contribute to the SRS generation.

\oursubsub{Unique Response Property.}  Another property
required to show USE is the unique response property~\cite{C:Fischlin05} which says
that for $3$-message sigma protocols, all but the first messages sent by the prover are
deterministic (intuitively, the prover can only employ fresh randomness in the first
message of the protocol). We cannot use this definition since the protocols
we consider have more than one move where the prover sends randomized messages. In
Plonk, both the first and the third messages (i.e.~first two prover's messages when
we consider Plonk as an interactive protocol) are randomized. Although Sonic prover
is deterministic after it picks its first message, the protocol has more than $3$
messages. The same holds for Marlin. We propose a generalisation of the definition which
states that a protocol is $\ur{k}$ if the prover is deterministic starting from its
$(k + 1)$-th message. For our proof, it is sufficient that this property is met by Plonk
for $k = 2$. Since Sonic and Marlin provers are deterministic from the second message
on, they are $\ur{1}$.


\oursubsub{Trapdoor-Less Zero-Knowledge (TLZK).}  In order to invoke our main theorem
on (Fiat--Shamir transformed) Plonk, Sonic and Marlin to conclude USE, we also need
to show that simulators in these protocols produce proofs without relying on the
knowledge of trapdoor. More precisely, for our reduction, we need simulators that rely
only on reordering the messages and picking suitable verifier challenges, without
knowing the SRS trapdoor.  That is, any PPT party should be able to produce a
simulated proof on its own in a trapdoor-less way. Note that this property does not
necessarily break soundness of the protocol as the simulator is required only to
produce a transcript and is not involved in a real conversation with a real
verifier. We show simulators for $\plonkprotfs$, $\sonicprotfs$, and $\marlinprotfs$
that rely only on the programmability of the random oracle, where programmability is only needed
for the $k$-th challenge. This property can be understood as a generalization of
honest-verifier zero-knowledge for Fiat--Shamir transformed proof systems with an
SRS.

Technically we will make use of the $k$-UR property together with the $k$-TLZK property to bound the probability that the tree produced by $\tdv$ contains any programmed random oracle queries.

\subsection{Related Work}
There are many results on simulation extractability for
non-interactive zero-knowledge proofs (NIZKs). First, Groth \cite{AC:Groth07}
noticed that a (black-box) SE NIZK is
universally-composable (UC)~\cite{EPRINT:Canetti00}. Then Dodis et al.~\cite{AC:DHLW10} introduced a
notion of (black-box) \emph{true simulation extractability} and showed that no
NIZK can be UC-secure if it does not have this property. 

In the context of zkSNARKs, the first
SE zkSNARK was proposed by Groth and Maller~\cite{C:GroMal17} and a SE
zkSNARK for QAP was designed by Lipmaa~\cite{EPRINT:Lipmaa19a}.
Kosba's et
al.~\cite{EPRINT:KZMQCP15} give a general transformation from a NIZK to a
black-box SE NIZK. Although their transformation works for zkSNARKs as well,
succinctness of the proof system is not preserved by the transformation.
Abdolmaleki et al.~\cite{CCS:AbdRamSla20} showed another transformation that
obtains non-black-box simulation extractability but also preserves
succinctness of the argument. 
The zkSNARK of~\cite{EC:Groth16} has been shown to be SE by introducing minor modifications to the construction and making
stronger assumptions \cite{EPRINT:BowGab18,EPRINT:AtaBag19}. Recently,~\cite{EPRINT:BKSV20} showed that the
original Groth's proof system from~\cite{EC:Groth16} is weakly SE and
randomizable. None of these results are for zkSNARKs in the updatable SRS setting or for zkSNARKs obtained via the Fiat--Shamir transformation. The recent work of~\cite{cryptoeprint:GOPTT22} shows that Fiat-Shamir transformed Bulletproofs are simulation extractable. While they show a general theorem for multi-round protocols, they do not consider a setting with an SRS, and are therefore inapplicable to zkSNARKs in the updatable SRS setting.



%%% Local Variables:
%%% mode: latex
%%% TeX-master: "main"
%%% End:

% !TEX root = main.tex
% !TEX spellcheck = en-US
%\section{Preliminaries}



\paragraph{Notation.} Let $\ppt$ denote probabilistic polynomial-time and $\secpar \in \NN$ be the
security parameter. %All adversaries are stateful. 
For an algorithm $\adv$, 
%let
%$\image (\adv)$ be the image of $\adv$ (the set of valid outputs of $\adv$), 
let
$\RND{\adv}$ denote the set of random tapes of correct length for $\adv$
(assuming the given value of $\secpar$), and let $r \sample \RND{\adv}$ denote
the random choice of the randomizer $r$ from $\RND{\adv}$. We denote by $\negl$
($\poly$) an arbitrary negligible (resp.~polynomial) function.
%
%For probability ensembles $X = \smallset{X_\secpar}_\secpar$ and
%$Y = \smallset{Y_\secpar}_\secpar$, with distributions $X_\secpar, Y_\secpar$ that have
%\emph{statistical distance} $\SD(X_\secpar, Y_\secpar) = \epsilon(\secpar)$ 
%if
%$\sum_{a \in \supp{X_\secpar \cup Y_\secpar}} \abs{\prob{X_\secpar = a} -
%  \prob{Y_\secpar = a}} = \epsilon(\secpar)$. 
%we write $X \approx_\secpar Y$ if
%$\SD(X_\secpar, Y_\secpar) \leq \negl$. 
For functions $a(\secpar)$, $b(\secpar)$ and probability ensembles $X = \smallset{X_\secpar}_\secpar$,
$Y = \smallset{Y_\secpar}_\secpar$ we
write $a(\secpar) \approx_\secpar b(\secpar)$ if
$\abs{a(\secpar) - b(\secpar)} \leq \negl$ and $X \approx_\secpar Y$ if they have \emph{statistical distance} $\SD(X_\secpar, Y_\secpar) \leq \negl )$, respectively. \medskip

% For a probability space
% $(\samplespace, \eventspace, \probfunction)$ and event $\event{E} \in \eventspace$ we
% denote by $\nevent{E}$ an event that is complementary to $\event{E}$,
% i.e.~$\nevent{E} = \samplespace \setminus \event{E}$.

% \begin{lemma}[Difference lemma,~{\cite[Lemma 1]{EPRINT:Shoup04}}]
% 	\label{lem:difference_lemma}
% 	Let $\event{A}, \event{B}, \event{F}$ be events defined in some probability
% 	space, and suppose that $\event{A} \land \nevent{F} \iff \event{B}
% 		\land \nevent{F}$.  Then 
% 	$
% 		\abs{\prob{\event{A}} - \prob{\event{B}}} \leq \prob{\event{F}}\,.
% 	$
% \end{lemma}


\section{Definitions and Lemmas for Multi-message SRS-based Protocols}
\label{sec:se_definitions}
\label{sec:preliminaries}



\ourpar{Simulation-extractability for multi-message protocols.}
Most recent SNARK schemes follow the same blueprint of constructing an interactive information-theoretic proof system 
that is then compiled into a public coin computationally sound scheme using cryptographic tools such as polynomial commitments,
and finally made non-interactive via the Fiat--Shamir transformation.
Existing results on simulation extractability (for proof systems and
signatures of knowledge) for Fiat--Shamir transformed systems work for $3$-message protocols without reference string that
require two transcripts for standard model extraction, e.g.,
\cite{JC:PoiSte00,INDOCRYPT:FKMV12,C:RotSeg21}.

In this section, we define properties that are necessary for our
analysis of multi-message protocols with a universal updatable SRS.  In order to
prove simulation-extractability for such protocols, we require more than just two
transcripts for extraction. Moreover, in the updatable setting we consider protocols
that rely on an SRS where the adversary gets to contribute to the SRS. We first recall the updatable SRS setting and the Fiat-Shamir transform for $(2\nu+1)$ message protocols.
Next, we define trapdoor-less zero-knowledge and simulation-extractability
which we base on~\cite{INDOCRYPT:FKMV12} adapted to the updatable SRS setting. Then,
to support multi-message SRS-based protocols compiled using the Fiat--Shamir transform,
we generalize the unique response property, and define a notion of computational special
soundness called rewinding-based knowledge soundness.\medskip


%\subsection{Proof Systems With Updateable Setups and Random Oracles}
\noindent Let $\prover$ and $\verifier$ be $\ppt$ algorithms, the former called the \emph{prover}
and the latter the \emph{verifier} of a proof system. Both algorithms take a pre-agreed structured reference string $\srs$ as input. The structured reference strings we consider are (potentially) updatable, a notion we recall shortly.
%
We focus on proof systems made non-interactive via the multi-message Fiat--Shamir transform presented below where prover and
verifier are provided with a random oracle $\ro$. 
We denote by $\zkproof$ a proof created by $\prover$ on input
$(\srs, \inp, \wit)$. We say that proof is acceptable if $\verifier(\srs, \inp,
\zkproof)$ accepts it.



%The simulator
%$\simulator$ is not only given access to $\ro$, but it can also \emph{program}
%it. That is, it can require that for $(x, y)$ of its choice, $\ro (x) = y$.


  
\iffalse
\subsection{Zero-Knowledge Proof Systems}\label{prelim:nizk}
In a zero-knowledge proof system, a prover convinces the verifier of the veracity of a statement
without leaking any other information. The zero-knowledge property is proven by constructing a
simulator that can simulate the view of a cheating verifier without knowing the secret
information -- witness -- of the prover. A proof system has to be sound as well, i.e.~for a
malicious prover it should be infeasible to convince a verifier of a false statement. Here, we
focus on proof systems, so-called arguments, that guarantee soundness against $\ppt$ malicious provers. 
\markulf{24.04}{Nice text, but maybe too basic? Also we don't really have views, but really reconstruct the non-interactive proof.}
Typically, a stronger notion of soundness is required -- besides requiring that the
verifier rejects proofs of statements outside the language, we request from the
prover to know a witness corresponding to the proven statement. This property is
called \emph{knowledge soundness}. In this work we investigate an even stronger notion of soundness, \emph{simulation extractability} for non-interactive protocols obtained from interactive arguments via the \emph{Fiat--Shamir heuristic} via random oracles in a setting with an \emph{updatable common reference string.} 

We now introduce these concepts.
\fi


\subsection{Updatable SRS Setup Ceremonies}\label{def:upd-scheme}

%Let $\prover$ and $\verifier$ be $\ppt$ algorithms, the former called \emph{prover}
%and the latter \emph{verifier}. We allow our proof system to have a setup, i.e.~there is a
%$\kgen$ algorithm that takes as input the relation description $\REL$ and outputs a common
%reference string $\srs$.

The definition of updatable SRS ceremonies of~\cite{C:GKMMM18} requires the following algorithms.

\begin{itemize} 
	\item
	$(\srs,\rho) \gets \kgen(\REL)$ is a PPT algorithm that takes a relation $\REL$ and outputs a SRS $\srs$, and correctness proof $\rho$.
	\item
	$ (\srs',\rho') \gets \upd(\srs, \{\rho_j \}_{j=1}^n)$ is a PPT algorithm that takes a SRS $\srs$, a list of update proofs and outputs an updated SRS together with a proof of correct update. 
	\item
	$b \gets \verifyCRS(\srs, \{\rho_j \}_{j=1}^n)$ is a DPT algorithm that takes a SRS $\srs$, a list of update proofs, and outputs a bit indicating acceptance or not.\footnote{For instance \plonk{} and \marlin{} will use the $\kgen$, $\upd$ and $\verifyCRS$ algorithms in~\cref{fig:upd-scheme}.}
\end{itemize}


In the next section, we define security notions in the updatable setting by giving the adversary access to an SRS update oracle $\initU$, defined in~\cref{fig:upd}. The oracle allows the adversary to control the SRS generation. A trusted setup can be expressed by the updatable setup definition simply by restricting the adversary to only call the oracle on $\intent = \setup$ and $\intent = \final$. Note that a soundness adversary now has access to both the random oracle $\ro$ and $\initU$:  $(\inp, \zkproof) \gets \adv^{\initU,\ro}(1^\secpar)$. %\chaya{29.04}{can we move the update oracle figure to section 5? I know we talk about the oracle here, but I am wondering if there is a way to keep $\initU$ general here and give the description below specific to Plonk/Marlin later on}.\hamid{29.4}{Since we use $\initU$ in most of the definitions 1-5 in section 3, I believe it's better to keep it here.}>>>>>>> bbec4288f7bec60d201228aa23c50b277515b3f5

\newcommand*{\Scale}[2][4]{\scalebox{#1}{$#2$}}% 

\begin{figure}[t!]
	\centering
	\centerline{\fbox{
		\begin{minipage}[t]{1.20\linewidth}
			\begin{pchstack}
			\procedure{$\initU(\intent, \srs_n,\{\rho_j \}_{j=1}^n)$}{
				\pcif \srs \neq \bot: \pcreturn \bot \\
				\pcif (\intent = \setup): \\
				\t (\srs',\rho') \leftarrow \kgen(\REL)\\
				\t Q_\srs \gets Q_\srs \cup \{(\srs',\rho')\}\\
				\t \pcreturn (\srs',\rho')}
			%
			\procedure{}{
				\\
				\pcif (\intent = \update): \\
				\t b \gets \verifyCRS(\srs_n, \{\rho_j \}_{j=1}^{n})\\
				\t \pcif (b=0): \pcreturn \bot \\
				\t (\srs',\rho') \leftarrow \upd (\srs_n,\{\rho_j \}_{j=1}^n)\\
				\t Q_\srs \gets Q_\srs \cup \{(\srs',\rho')\}\\
		 	% \pccomment{$Q_\srs = (Q^{(1)}_\srs, Q^{(2)}_\srs) \text{ s.t. }   Q^{(2)}_\srs \text{ contains the update proofs in } Q_\srs$ } \\
				\t \pcreturn (\srs',\rho')}
				%
			\procedure{}{
				\\
				\pcif (\intent = \final): \\
				\t b \gets \verifyCRS(\srs_n,\{\rho_j \}_{j=1}^{n})\\
				\t \pcif (b=0) \vee Q^{(2)}_\srs \cap \{ \rho_j \}_i = \emptyset: \\
				\t \pcreturn \bot \\
				\t 
				\t \srs \gets \srs_n, \pcreturn \srs \\
				%
				\pcelse \pcreturn \bot
			}
			\end{pchstack}
		\end{minipage}
	}}
	\caption{The oracle defines the notion of updatable SRS setup.} 
		\label{fig:upd}
\end{figure}

\ourpar{Remark on universality of the SRS.} The proof systems we consider in this work are universal. This means that both the relation $\REL$ and the reference string $\srs$ allows to prove arithmetic constraints defined over a particular field up to some size bound. The public instance $\inp$ must determine the constraints. 
If $\REL$ comes with any auxiliary input, the latter is benign. 
%The
%description of $\REL$ fixes the security parameter $\secpar$ and the order of a
%group $\GRP$, if the relation is a relation on field elements (as usually the case for zkSNARKs). 
%We assume that the $\srs$ defines the relation, and for universal proof
%systems, such as Plonk and Sonic, we treat both the reference string and the relation as
%universal. 
We elide public preprocessing of constraint specific proving and verification keys. While important for performance, this modeling is not critical for security.
%\markulf{24.04}{Tighten the discussion of universality, one paragraph is enough!}


\subsection{Multi-message Fiat-Shamir Compiled Provers and Verifiers}
Given interactive prover and (public coin) verifier $\prover', \verifier'$ that exchange messages $\tzkproof = (a_1, c_1, \ldots, a_{\mu}, c_{\mu}, a_{\mu + 1})$, where $a_i$ comes from
$\prover'$ and $c_i$ comes from $\verifier'$, the $(2\mu + 1)$-message Fiat-Shamir heuristic defines non-interactive provers and verifiers $\prover, \verifier$ as follows:

\begin{compactitem}
	\item $\prover$ behaves as $\prover'$ except after sending message
	  $a_i$, $i \in \range{1}{\mu}$, the prover does not wait for
	  the message from the verifier but computes it locally setting $c_i
	  = \ro(\tzkproof[0..i])$, where $\tzkproof[0..j] = (\inp, a_1, c_1, \ldots,
	  a_{j - 1}, c_{j - 1}, a_j)$.\footnote{For Fiat--Shamir based SoK the message signed $m$ is added to $\inp$ before hashing.} 
	  
	  $\prover$ output the non-interactive proof $\pi=(a_1,\ldots, a_{\mu}, a_{\mu + 1})$, that is it omits challenges as they can be recomputed using $\ro$

	\item $\verifier$ takes $\inp$ and $\pi$ as input and behaves as $\verifier'$ would but does not provide
	  challenges to the prover. Instead it computes the
	  challenges locally as $\prover$ would, starting from $\tzkproof[0..1]=(\inp,a_1)$ which can be obtained from $\inp$ and $\pi$. Then it verifies the
	  resulting transcript $\tzkproof$ as the verifier $\verifier'$ would. We note that since the verifier can compute the challenges by querying the random oracle, they do not need to be sent by the prover.
	\end{compactitem}



\paragraph{Notation for $(2\mu + 1)$-message Fiat--Shamir compiled proof systems.}
Let $\SRScer= (\kgen,\upd, \verifyCRS)$ be the algorithm of an updatable SRS ceremony.
All our definitions and theorems are about non-interactive proof systems $\ps = (\SRScer, \prover, \verifier, \simulator)$ compiled via the $(2\mu + 1)$-message FS transform. 
%
That is $\pi = (a_1, \ldots, a_{\mu}, a_{\mu + 1})$ and $\tzkproof = (a_1, c_1, \ldots, a_{\mu}, c_{\mu}, a_{\mu + 1})$, with $c_i
= \ro(\tzkproof[0..i])$, where $\tzkproof[0..j] = (\inp, a_1, c_1, \ldots,
a_{j - 1}, c_{j - 1}, a_j)$.

\markulf{30.04}{This paragraph seems a bit of overkill, remove notation that we do not actually use}
 We use $\zkproof[i]$ (and $\tzkproof[i]$) to denote $i$-th prover's
message $a_i$. $\tzkproof[i].\ch$ denotes the challenge that is given to the prover after $a_i$, and $\tzkproof[i..j]$ denotes all messages of the proof transcript 
between $i$-th and $j$-th prover's messages, but not challenge
$\tzkproof[j].\ch$. When it is not explicitly stated, we denote the proven instance
$\inp$ by $\tzkproof[0]$. If the proof system starts with a message from the verifier,
we denote it by $\tzkproof[0].\ch$; else, we state that $\tzkproof[0].\ch$ is empty.

%%% Local Variables:
%%% mode: latex
%%% TeX-master: "main"
%%% End:

% !TEX root = main.tex
% !TEX spellcheck = en-US


\begin{figure}
	\centering
		\begin{pcvstack}[center,boxed]
		\begin{pchstack}
			\procedure{$\simOH (x)$}
			{
			\pcif H[x] = \bot \pcthen \\
			\pcind H[x] \sample \mathsf{Im}(\ro) \\
		%	Q_\ro \gets \{(x,H[x])\}\\
			\pcreturn H[x]
			  }
			%
			\pchspace
			%
			\procedure{$\simulator\oracleo.\prog(x, h)$}
			{ 
				\pcif H[x] = \bot \pcthen \\ 
				\pcind H[x] \gets h \\
				\pcind \Qprog \gets \Qprog \cup \{x\}\\
		%		Q_\ro \gets \{(x,H[x])\}\\
				\pcreturn H[x]
			}
			\pchspace
			%
			\procedure{$\boxed{\simOP(\inp, \wit)} \ \simOP'(\inp)$}
			{ 
				\boxed{\pcassert (\inp,\wit)\in \REL} \\ 
				\pi \gets \simulator^{\simOH,\simulator\oracleo.\prog}(\srs,\inp)\\
				Q \gets Q \cup \{(\inp,\pi)\}\\
				\pcreturn \pi
			}
		\end{pchstack}
	\end{pcvstack}
	\caption{Simulation oracles: $\srs$ is the finalized SRS, only $\simOP'$ allows for simulation of false statements}
	\label{fig:real_simulator_oracles}
\end{figure}

\subsection{Trapdoor-Less Zero-Knowledge (TLZK)}



%We model $\simulator$ as a stateful
%algorithm that runs in two modes. The first mode,
%$(h, st') \gets \simulator (1, st, \srs, q)$ answers random oracle calls to $\ro$ on input
%$q$. The second mode $(\zkproof, st') \gets \simulator (2, st, \srs, \inp)$ simulates the
%actual argument for instance $\inp$.  



% \begin{itemize}
% \item $\simOH (\srs,q)$ denotes an oracle that returns the first output of
%   $\simulator (1, st, \srs, q)$;
% \item $\simOP (\srs,\inp, \wit)$ denotes an oracle that returns the first output of
%   $\simulator (2, st, \srs, \inp)$ if $(\inp, \wit) \in \REL$, and returns $\bot$ otherwise;
% \end{itemize}

% We call a proof system $\proofsystem$ \emph{zero-knowledge} if for any
% $\REL \in \RELGEN(\secparam)$, and adversary $\adv$ there exists a $\ppt$ simulator
% $\simulator$ such that for any $(\inp, \wit) \in \REL$, $\eps_0 \approx \eps_1$,
% where
% \[
%   \eps_b = \condprob{\adv^{\oracleo_b} (\srs)}{\srs \gets \kgen (\REL)}\,,
% \]
% \changedm{where $\oracleo_0$ on input $(1, q)$ responds with $h$ such that
%   $(h, st') \gets \simOH (1, st, q)$, where $st$ and $st'$ are old and new
%   states of the simulator $\simulator$ and on input $(2, \inp, \wit)$ returns
%   $\zkproof \gets \simOP (2, st, \srs, \inp)$. Alternatively, $\oracleo_1$ on
%   input $(1, q)$ responds with $h \gets \ro (q)$ and on input $(2, \inp, \wit)$
%   returns $\zkproof \gets \prover (\srs, \inp, \wit)$.}
% 	%
% We call zero knowledge \emph{perfect} if the distributions are equal and
% \emph{computational} if they are indistinguishable for any $\ppt$ distinguisher.
\iffalse
We call a proof system $\proofsystem$ \emph{zero-knowledge} if for any
$\REL \in \RELGEN(\secparam)$, and adversary $\adv$ there exists a $\ppt$ simulator
$\simulator$ with oracles $(\simOH,\simOP)$ such that for any $(\inp, \wit) \in \REL$, $\eps_0 \approx \eps_1$,
where
\[
  \eps_0(\secpar) = \condprob{\adv^{\simOH,\simOP} (\srs)}{\srs \gets \kgen
    (\REL)},\,  \eps_1 (\secpar) = \condprob{\adv^{\ro,\prover} (\srs)}{\srs \gets \kgen (\REL)},
\]

We call zero knowledge \emph{perfect} if the distributions are equal and
\emph{computational} if they are indistinguishable for any $\ppt$ distinguisher.


% \end{description}
Alternatively, zero-knowledge can be defined by allowing the simulator to use the
trapdoor $\td$ that is generated along the $\srs$. In this paper we distinguish
simulators that requires a trapdoor to simulate and those that do not. We call the
former \emph{SRS-simulators}. We say that a protocol is \emph{trapdoor-less
  zero-knowledge} (TLZK) if its simulator does not require the trapdoor, cf.~\cref{def:TLZK}.

  \markulf{22.04}{We don't consider SRS-simulators in this paper. I would simplify and move the above after the Updatable SRS scheme section, or maybe the start of Section 3.}
\fi

%In this paper we distinguish between simulators that require a trapdoor to simulate and those that do not. 
We call a protocol \emph{trapdoor-less
 zero-knowledge} (TLZK) if there exists a simulator that does not require the trapdoor, and works by programming the random oracle.
 Moreover, the simulator may only be allowed to program the random oracle on point $\tzkproof[0,k]$, that is the simulator can only program the challenges that come after the $k$-th prover message. We call protocols which allow for such a simulation $k$-\emph{programmable trapdoor-less zero-knowledge}. %, and define this in~\cref{def:TLZK}.

 Our definition of zero-knowledge for non-interactive
arguments is in the programmable ROM.
%the explicitly programmable random oracle model where the simulator $\simulator$ can program the random oracle. 
We model this using the oracles from Fig.~\ref{fig:real_simulator_oracles} that provide a stateful wrapper around $\simulator$.
$\simOH (x)$ simulates $\ro$ using lazy sampling, $\simulator\oracleo.\prog(x, h)$ allows for programming the simulated $\ro$ and is available only to $\simulator$. $\simOP(\inp, \wit)$ and $\simOP'(\inp)$ call the simulator. The former is used in the zero-knowledge definition and requires the statement and witness to be in the relation, the latter is used in the simulation extraction definition and does not require a witness input.

\begin{definition}[Updatable k-Programmable Trapdoor-Less Zero-Knowledge]
  \label{def:TLZK}
  Let 
  $\psfs = (\SRScer, \prover, \verifier, \simulator)$ be a $(2\mu + 1)$-message FS-transformed NIZK proof system with an updatable SRS setup. %Let $\ro$ be the random oracle. 
  %$\simulator_{\fs}$ takes as input $\srs$ and instance $\inp$, programs $\ro$, and outputs a proof $\zkproof_\simulator$.  
  We call $\psfs$ \emph{trapdoor-less zero-knowledge} with security $\epszk$ if for any
  adversary $\adv$, $\abs{\eps_0(\secpar) - \eps_1(\secpar)} \leq \epszk(\secpar)$, where
  \begin{align*}
    \eps_0 (\secpar) = \Pr\left[ \adv^{\initU, \ro, \prover} (\secparam) \right],\,
    \eps_1 (\secpar)=  \Pr \left[\adv^{\initU, \simOH, \simOP} (\secparam) \right].
  \end{align*}
  
  If $\epszk(\secpar)$ is negligible, we say $\ps_{\fs}$ is trapdoor-less zero-knowledge. Additionally, we say that $\ps_{\fs}$ is $k$-programmable, if  $\simulator$ before returning a proof $\pi$ only calls $\simulator\oracleo.\prog$ on $(\tzkproof[0..k],h)$. That is, it only programs the $k$-th message.
  \end{definition}

  
\begin{remark}[TLZK vs HVZK]
  We note that TLZK notion is closely related to honest-verifier zero-knowledge in the
  standard model. That is, if we consider an interactive proof system $\proofsystem$
  that is HVZK in the standard model then $\proofsystem_\fs$ is TLZK. This comes as the simulator $\simulator$ in
  $\proofsystem$ produces a valid simulated proof by picking verifier's challenges
  according to a predefined distribution and $\proofsystem_\fs$'s simulator
  $\simulator_\fs$ produces its proofs similarly by picking the challenges and
  additionally programming the random oracle to return the picked
  challenges. Importantly, in both $\proofsystem$ and $\proofsystem_\fs$ success of
  the simulator does not depend on access to an SRS trapdoor.
\end{remark}

We note that $\plonk$ is $3$-programmable TLZK, and $\sonic$ and $\marlin$ are $2$-programmable TLZK. This follows directly from the proofs of
their standard model zero-knowledge property in
\cref{lem:plonk_tlzk,lem:sonic_hvzk,lem:marlin_hvzk}.

\subsection{Updatable Simulation Extractability (USE)}
We note that the zero-knowledge property is only guaranteed for statements in the
language.
%; for simulator $\simulator = (\simOH, \simOP)$, $\simOP$
%answers with simulated proofs only for true statements.  This is, however, not sufficient
For \emph{simulation extractability} where the simulator
should be able to provide simulated proofs for false statements as well, we thus use the oracle $\simOP'$
\footnote{Note,
  that simulation extractability property where the simulator is required to give
  simulated proofs for true statements only is called \emph{true simulation
    extractability.}}. 
	
%Therefore, we introduce a wrapper oracle around the simulator
%called $\simOP'$ that on input $(\srs, \inp)$ always returns the first output of
%$\simulator (2, st, \srs, \inp)$, regardless of whether $\inp$ is in the language. We
%define \emph{simulation-extractability} with respect to oracle $\simOP'$; that is,
%simulation-extractability is with respect to a simulator
%$\simulator' = (\simOH, \simOP')$.
%



\begin{definition}[Updatable Simulation Extractability]
	\label{def:updsimext}
  \label{def:simext}
	Let $\psni = (\SRScer, \prover, \verifier, \simulator)$ be a NIZK proof system with an updatable SRS setup. 
	We say that
  $\psni$ is \emph{updatable simulation-extractable} with security loss $\epsse(\secpar,\accProb, q)$ if for
  any $\ppt$ adversary $\adv$ that is given oracle access to setup oracle\hamid{10.5}{We always call this "update oracle"!}
  $\initU$ and simulation oracle $\simulator\oracleo$ and that produces an accepting
  proof for $\psni$ with probability $\accProb$, where
	\[
	\accProb = \condprob{
	\begin{matrix}
	  \verifier(\srs, \inp, \zkproof) = 1  \\
	  \wedge
	(\inp, \zkproof) \not\in Q
	\end{matrix}
}{
	\begin{matrix}
	  r \sample \RND{\advse}\\
	(\inp, \zkproof) \gets \advse^{\initU, \simOH, \simOP'
		} (1^\secpar; r)
	\end{matrix}
}
	\]
	there exists an expected PPT extractor $\extse$ such that
	\[
	 \condprob{
	\begin{matrix}
  \verifier(\srs, \inp, \zkproof) = 1, \\
   (\inp, \zkproof) \not\in Q,  \\
	   \REL(\inp, \wit) = 0
	\end{matrix}
}{
	\begin{aligned}
	& r \sample \RND{\advse},
	(\inp, \zkproof) \gets \advse^{\initU, \simOH, \simOP'
		} (1^\secpar; r) \\
	& \wit \gets \ext_\se (\srs, \advse, r,
	\Qsrs,\Qro,\Qsim ) 
	\end{aligned}
} \leq \epsse(\secpar,\accProb, q)
	\]
	%is at least 
	%\[
	%\extProb \geq \frac{1}{\poly} (\accProb - \nu)^d - \eps(\secpar)\,,
	%\]
	%for some polynomial $\poly$, constant $d$ and negligible $\eps(\secpar)$ whenever
	%$\accProb \geq \nu$. 
	Here, $\srs$ is the finalized SRS. List $\Qsrs$ contains all $(\srs, \rho)$ of update SRSs and their proofs, list $\Qro$ contains all $\advse$'s
	queries to $\simOH$ and the (simulated) random oracle's answers, $\abs{\Qro}\leq q$, and list $Q$ contains all $(\inp, \zkproof)$ pairs where 
	$\inp$ is an instance queried to $\simOP'$ by the adversary and
	$\zkproof$ is the simulator's answer .
  % \hamid{17.10}{shouldn't be "List $Q_\ro$ contains all $\advse$'s queries to $\simOH$"? Also, I think we need to remove $\ro$ from the statement of the definition!}
\end{definition}

\subsection{Unique Response (UR) Protocols}
A technical hurdle identified by Faust et al.~\cite{INDOCRYPT:FKMV12} for proving
simulation extraction via the Fiat--Shamir transformation is that the transformed
proof system satisfies a unique response property. The original formulation by Fischlin, although suitable for applications presented in
\cite{C:Fischlin05,INDOCRYPT:FKMV12}, does not suffice in our case. First, the
property assumes that the protocol has three messages, with the second being the
challenge from the verifier. That is not the case we consider here. Second, it is not
entirely clear how to generalize the property. Should one require that after the
first challenge from the verifier, the prover's responses are fixed?  That does not
work since the prover needs to answer differently on different verifier's challenges,
as otherwise the protocol could have fewer messages.  Another problem is that the
protocol could have a message, beyond the first prover's message, which is
randomized. Unique response cannot hold in this case. Finally, the protocols we
consider here are not in the standard model, but use an SRS.

We work around these obstacles by providing a generalized notion of the unique
response property. More precisely, we say that a $(2\mu + 1)$-message protocol
has \emph{unique responses from $k$}, and call it a $\ur{k}$-protocol, if it
follows the definition below:

\begin{definition}[Updatable k-Unique Response Protocol]
Let $\psfs = (\SRScer, \prover, \verifier, \simulator)$ be a $(2\mu + 1)$-message FS-transformed NIZK proof system with an updatable SRS setup. Let $\ro$ be the random oracle. 
We say that $\psfs$ has \emph{unique responses for $k$} with security $\epsur(\secpar)$ if for any $\ppt$ adversary $\advur$:
  \[
	\Pr\left[
		\left.
	\begin{aligned}
	& \zkproof \neq \zkproof', \tzkproof[0..k] = \tzkproof'[0..k],  \\
	& \verifier' (\srs, \inp, \zkproof,c) =
	\verifier' (\srs, \inp, \zkproof',c) = 1  \\
	\end{aligned}
	\,\right|\,
	\begin{aligned}
		& (\inp, \zkproof, \zkproof', c) \gets \advur^{\initU,\ro}(1^\secpar) 
	\end{aligned}
	\right] \leq \epsur(\secpar) 
	\]
	where $\srs$ is the finalized SRS and $\verifier'(\srs,\inp,\zkproof=(a_1, \dots, a_\mu,a_{\mu+1}), c)$ behaves as $\verifier (\srs, \inp, \zkproof)$ except for using c as the $k$-th challenge instead of calling $\ro(\tzkproof[0..k]) $. Thus, $\adv$ can program the $k$-th challenge. 
	We say $\psfs$  is $\ur{k}$, if $\epsur(\secpar)$ is negligible.
\end{definition}

Intuitively, a protocol is $\ur{k}$ if it is infeasible for a $\ppt$ adversary to produce a pair of accepting proofs $\zkproof \neq \zkproof'$ that are the same on the first $k$ messages of the prover.  
%We note that the definition above is also meaningful for protocols without an SRS. Intuitively in that case $\srs$ is the empty string.

The definition can be easily generalized to allow for programming the oracle on more than just a single point. We opted for this simplified presentation, since all the protocols analyzed in this paper require only single-point programming, 
  

\subsection{Rewinding-Based Knowledge Soundness (RBKS)}

Before giving the definition of rewinding-based knowledge soundness for NIZK proof systems compiled via the $(2\mu + 1)$-message FS transformation, we first recall the notion of a tree of transcripts.
\begin{definition}[Tree of accepting transcripts, cf.~{\cite{EC:BCCGP16}}]
	\label{def:tree_of_accepting_transcripts}
%	Consider a $(2\mu + 1)$-message proof system. \markulf{24.04}{Is this meaningful for interactive or also non-interactive prove systems obtained via FS?} 
	A $(n_1,
  \ldots, n_\mu)$-tree of accepting transcripts is a tree where each node on
  depth $i$, for $i \in \range{1}{\mu + 1}$, is an $i$-th prover's message in an
  accepting transcript; edges between the nodes are labeled with
  challenges, such that no two edges on the same depth have the same
  label; and each node on depth $i$ has $n_{i} - 1$ siblings and $n_{i +
    1}$ children. The tree consists of $N = \prod_{i = 1}^\mu n_i$
  branches, where $N$ is the number of accepting transcripts. We require $N = \poly$. We refer to a $(1, \ldots, n_k=n, 1, \ldots, 1)$-tree as a $(k,n)$-tree.
\end{definition}

\iffalse
		
	\begin{figure}[t]
		\centering
		\fbox{
		\procedure{$\genforking_{\zdv}^{m, m'} (y,h_1^{1}, \ldots, h_{q}^{1})$}		
		{
		\rho \sample \RND{\zdv}\\
		(i, s_1) \gets \zdv(y, h_1^{1}, \ldots, h_{q}^{1}; \rho)\\
    i_1 \gets i\\
		% \pcif i = 0\ \pcreturn (0, \bot)\\
		\pcfor j \in \range{2}{m'}\\
		\pcind h_{1}^{j}, \ldots, h_{i - 1}^{j} \gets h_{1}^{j - 1}, \ldots,
		h_{i - 1}^{j - 1}\\
		\pcind h_{i}^{j}, \ldots, h_{q}^{j} \sample H\\
		\pcind (i_j, s_j) \gets \zdv(y, h_1^{j}, \ldots, h_{i - 1}^{j}, h_{i}^{j},
		\ldots, h_{q}^{j}; \rho)\\
		%\pcind \pcif i_j = 0 \lor i_j \neq i\ \pcreturn (0, \bot)\\
   % \pcif \exists (j, j') \in \range{1}{m}^2, j \neq j' : (h_{i}^{j} = h_{i}^{j'})\
    \pcif \exists (j_1, \ldots, j_m) \in \range{1}{m'}^m, \text{ s.t. }\\
    \pcind j_k \neq j_{k'} \text{ for } k \neq k' \land \\
    \pcind i_{j_k} = i_{j_k'} \land \\
    \pcind 
		\pcelse \pcreturn (0, \bot)
	}}
\caption{Generalized forking algorithm $\genforking_{\zdv}^{m, m'}$
  \michals{5.11}{This forking lemma version is not as general as in Attema et al to
    be exact -- they allow tree of acceptable transcripts to branch at multiple
    places. however, that is not necessary in our case. Later, we should allow ``full
    generality'' but that would also require modification of the forking soundness
    definition (which now also relies on a fact that the tree branches at a single
    point)}}
	\label{fig:genforking_lemma}
\end{figure}


\fi



\iffalse
Note that the special soundness property (as usually defined) holds for
all---even computationally unbounded---adversaries. Unfortunately, since a
simulation trapdoor for $\plonkprot$, $\sonicprot$ and $\marlinprot$ exists, the protocols
cannot be special sound in that regard. This is because an unbounded adversary
can recover the trapdoor and build an arbitrary number of simulated proofs for a fake
statement. Hence, we provide a weaker, yet sufficient, definition that somehow lies between 
computational \emph{knowledge soundness} and \emph{special soundness} for unbounded adversaries. Differently from the standard definition of special soundness, we do not require from the extractor to be able to extract the witness from \emph{any} tree of acceptable transcripts. Similar to knowledge soundness, we require that the extractor fails to reconstruct the witness with some small probability only. 

More precisely, in proofs of rewinding-based knowledge soundness we show later that either the extractor extracts a witness from a tree of transcripts or the adversary who was used to produce the tree can be used to break some computationally hard problem. 
\fi
The existence of simulation trapdoor for $\plonkprot$, $\sonicprot$ and $\marlinprot$ means that they are not
special sound in the standard sense. We therefore put forth the notion of rewinding-based knowledge soundness that is a computational notion. 
Note that in the definition below, it is implicit that each transcript in the tree is accepting with respect to a ``local programming'' of the random oracle. However, the verification of the proof output by the adversary is with respect to a non-programmed random oracle.

\begin{definition}[Updatable Rewinding-Based Knowledge Soundness]
	Let $n_1, \ldots, n_\mu \in \NN$. 
	%
	Let $\psfs = (\SRScer, \prover, \verifier, \simulator)$ be a $(2\mu + 1)$-message FS-transformed NIZK proof system with an updatable SRS setup for relation $\REL$. Let $\ro$ be the random oracle.
	%
	We require existence of an expected PPT tree builder $\tdv$ that eventually outputs a $\tree$ which is either a $(n_1, \ldots, n_\mu)$-tree of accepting transcript or $\bot$ and a PPT extractor $\extcss$. Let  adversary $\advcss$ be a PPT algorithm, that outputs a valid proof with probability at least $\accProb$, 
	where
	\[
	\accProb = \condprob{
	\begin{matrix}
	  \verifier(\srs, \inp, \zkproof) = 1  \\
	  \wedge \ 
	(\inp, \zkproof) \not\in Q
	\end{matrix}
}{
	\begin{matrix}
	  r \sample \RND{\advcss}\\
	(\inp, \zkproof) \gets \advcss^{\initU, \ro
		} (1^\secpar; r)
	\end{matrix}
}.
	\]
	We say that $\psfs$ is $(n_1, \ldots, n_\mu)$-rewinding-based knowledge sound with security loss $\epscss(\secpar, \accProb, q)$ if
	\begin{align*}
	\Pr\left[
		\begin{matrix}
			\verifier(\srs, \inp, \zkproof) = 1,  \\
			\wedge \ 
			\REL(\inp, \wit) = 0
		  \end{matrix}
	\,\left|\,
	\begin{aligned}
	& 	r \sample \RND{\advcss}, \\
	& 	(\srs, \inp, \cdot) \gets \advcss^{\initU, \ro} (1^\secpar; r)\\
	&  	\tree \gets \tdv(\srs, \advcss, r, \Qsrs, \Qro),
	\wit \gets \extcss(\tree)
	\end{aligned}\right.
	\right] \leq \epscss(\secpar,\accProb, q).
	\end{align*}
	\hamid{10.5}{Should we remove $\srs$ as part of $\adv$'s output in the second probability equation? Also, why $\pi$ is not explicit as part of $\adv$'s output? (what is $\pi$ in the left side of probability in $\verifier(\srs, \inp, \zkproof) = 1$ then?)}
	Here, $\srs$ is the finalized SRS. List $\Qsrs$ contains all $(\srs, \rho)$ of updated SRSs and their proofs, and list $\Qro$ contains all of the adversaries
	queries to $\ro$ and the random oracle's answers, $\abs{\Qro}\leq q$.
\hamid{25.6}{I think we still need to define what $(k,n)$-RBKS means:} We call the protocol $(k, n)$-rewinding-based knowledge sound if $\tree$ is a $(k, n)$-tree of accepting transcripts.
%	 We call the protocol $(n_1, \ldots, n_\mu)$ rewinding-based knowledge sound when $\tdv$ builds a $(n_1, \ldots, n_\mu)$-tree of transcripts.
% \hamid{1.5}{We still need this sentence:  
%\markulf{1.5}{Cut? We call the protocol $(k, n)$-rewinding-based knowledge sound if $\tree$ is a $(k, n)$-tree of accepting transcripts.}
% }
\end{definition}
	



%%% Local Variables:
%%% mode: latex
%%% TeX-master: "main"
%%% End:

% !TEX root = main.tex
% !TEX spellcheck = en-US
\section{\COMMENT{Forking s}Simulation-extractability---the general result}
\label{sec:general}
Equipped with definitional framework of \cref{sec:se_definitions}, we now present the main result of this paper---a proof of
\COMMENT{forking }simulation extractability of Fiat-Shamir compiled multi-round protocols.

The proof proceeds by game hopping. The games are controlled by an environment $\env$
that internally runs a simulation extractability adversary $\advse$, provides it
with access to a random oracle and simulator, and when necessary, rewinds it. The
games differ by various breaking points, i.e.~points where the environment
decides to abort the game.

The first game $\game{0}$ is a simulation-extractability game, where the extractor
gets as input
\begin{inparaenum}[(1)]
\item simulated proofs $Q$,
\item random oracle queries $Q_\ro$,
\item SRS updates $Q_\srs$
\end{inparaenum}
which were made by $\advse$.  We denote by $\radv$ randomness of $\advse$ and by
$\rsim$ randomness that was used by the simulator to create responses in the list
$Q$, formally one can state that $\rsim$ could be parsed into $r_Q$ and $r_\ro$ used
to answer queries in $Q$ and $Q_\ro$. Let $r = (\radv, \rsim)$. Randomness $r$ is
used in the following games -- we pass it to a unique-response property reduction $\rdvur$
and forking special soundness adversary $\bdv$. Importantly, since we require that
$\rdvur$ and $\bdv$ are successful (with non-negligible probability) given uniformly
random randomness and $r$ is picked uniformly random we conclude that we do not
change any success probability by running $\rdvur$ and $\bdv$ on $r$. 

Denote by $\zkproof_{\advse}, \zkproof_{\simulator}$ proofs returned by the
adversary and the simulator respectively. We use $\zkproof[i]$ to denote
prover's message in the $i$-th round of the proof (counting from 1),
i.e.~$(2i - 1)$-th message exchanged in the protocol. $\zkproof[i].\ch$ denotes
the challenge that is given to the prover after $\zkproof[i]$, and
$\zkproof[i..j]$ to denote all messages of the proof including challenges
between rounds $i$ and $j$, but not challenge $\zkproof[j].\ch$. When it is not
explicitly stated, we denote the proven instance $\inp$ by $\zkproof[0]$
(however, there is no following challenge $\zkproof[0].\ch$).

Without loss of generality, we assume that whenever the accepting proof contains
a response to a challenge from a random oracle, then the adversary queried the
oracle to get it. It is straightforward to transform any adversary that violates
this condition into an adversary that makes these additional queries to the
random oracle and wins with the same probability.


\begin{theorem}[Simulation-extractable multi-message protocols]
  \label{thm:se}
  Let $\ps = (\kgen, \prover, \verifier, \simulator)$ be an interactive
  $(2 \mu + 1)$-message zero-knowledge proof system for $\RELGEN(\secparam)$. Let
  $\ro\colon \bin^{*} \to \bin^{\secpar}$ be a random oracle. If $\psfs$ has the
  $k$-\emph{unique response property} with security loss $\epsur(\secpar)$, is
  \emph{$k'$-programmable trapdoor-less simulatable}, is $(k'', n)$-\emph{forking
    special sound} with security loss $\epss(\secpar)$, and $k \geq k' \geq k''$ then
  $\psfs$ is \emph{simulation-extractable} with extraction error $\epsur(\secpar)$
  against $\ppt$ adversaries that makes up to $q$ random oracle queries and returns
  an acceptable proof with probability at least $\accProb$.  The extraction
  probability $\extProb$ is at least
  \( \extProb \geq \frac{1}{q^{n - 1}} (\accProb - \epsur(\secpar))^{n}
  -\eps(\secpar)\,, \) for some negligible $\eps(\secpar)$.
\end{theorem}
\begin{proof}		

  \ngame{0} This is the simulation-extractability game played between an adversary
  $\advse$ who is given access to an oracle $\initU$ that defines an updatable SRS
  setup, a random oracle $\ro$ and a simulation oracle $\simO$. We denote by $\rsim$
  randomness of $\simO$ which is randomly sampled. There is an extractor $\ext$ that,
  from a proof $\zkproof_{\advse}$ for instance $\inp_{\advse}$ output by the
  adversary and from transcripts of $\advse$'s operations -- which compounds of lists
  of
  \begin{inparaenum}[(1)]
  \item simulated proofs $Q$,
  \item random oracle queries $Q_\ro$,
  \item SRS updates $Q_\srs$.
  \end{inparaenum}
  -- is tasked to extract a witness $\wit_{\advse}$ such that
  $\REL(\inp_{\advse}, \wit_{\advse})$ holds. $\advse$ wins if it manages to produce
  an acceptable proof and the extractor fails to output a witness. In the following
  game hops we upper-bound the probability that this happens. Note that $\srs$ is
  with respect to the finalised SRS with respect to which $\adv$ is allowed to make
  simulation queries.

  \ngame{1} This is identical to $\game{0}$ except that now the game is aborted if
  there is $(\inp_\advse, \zkproof_{\simulator}) \in Q$ such that
  $\zkproof_\simulator[1..k] = \zkproof_{\advse}[1..k]$. That is, the adversary in
  its final proof reuses at least $k$ messages from a simulated proof, and the proof
  is accepting.  Denote this event by $\event{\errur}$.

  \ncase{Game 0 to Game 1}
  % \michals{7.10}{I think we can make this reduction work
  %   even for a standard $k$-ur, not only the updatable one.} \michals{8.10}{If we
  %   know how to move proofs between SRSs}
  We have,
  \( \prob{\game{0} \land \nevent{\errur}} = \prob{\game{1} \land \nevent{\errur}} \)
  and, from the difference lemma, cf.~\cref{lem:difference_lemma},
  $ \abs{\prob{\game{0}} - \prob{\game{1}}} \leq \prob{\event{\errur}}\,$.  Thus, to
  show that the transition from one game to another introduces only minor change in
  probability of $\adv$ winning it should be shown that $\prob{\event{\errur}}$ is
  small.
  
  We can assume that $\adv$ queried the simulator on the instance it wishes to
  output, i.e.~$\inp_{\adv}$. We show a reduction $\rdvur$ that utilises $\adv$ to
  break the $\ur{k}$ property of $\ps$.That is, for randomness $r = (r_\adv, \rsim)$,
  $\rdvur (r)$ runs $\advse(r_\adv)$ internally as a black-box:
  \begin{compactitem}
  \item The reduction answers $\adv$ update queries by asking the same query from the
    update oracle in the unique response experiment. The reduction finalises the same
    SRS $\srs$ as the one $\adv$ does.
  \item The reduction answers both queries to the simulator $\psfs.\simulator$
      and to the random oracle using $\rsim$. It also keeps lists $Q$, for the simulated proofs,
      and $Q_\ro$ for the random oracle queries.
  \item When $\adv$ makes a fake proof $\zkproof_{\adv}$ for $\inp_{\adv}$, $\rdvur$
    looks through lists $Q$ and $Q_\ro$ until it finds $\zkproof_{\simulator}[0..k]$
    such that $\zkproof_{\adv}[0..k] = \zkproof_{\simulator}[0..k]$ and a random
    oracle query $\zkproof_{\simulator}[k].\ch$ on $\zkproof_{\simulator}[0..k]$.
  \item $\rdvur$ returns two proofs for $\inp_{\adv}$:
  	\begin{align*}
      \zkproof_1 = (\zkproof_{\simulator}[1..k],
      \zkproof_{\simulator}[k].\ch, \zkproof_{\simulator}[k + 1..\mu + 1])\\
      \zkproof_2 = (\zkproof_{\simulator}[1..k],
      \zkproof_{\simulator}[k].\ch, \zkproof_{\adv}[k + 1..\mu + 1])
  	\end{align*}
  \end{compactitem}
  If $\zkproof_1 = \zkproof_2$, then $\adv$ fails to break simulation soundness, as
  $\zkproof_2 \in Q$. On the other hand, if the proofs are not equal, then $\rdvur$
  breaks $\ur{k}$-ness of $\ps$. This happens only with negligible probability
  $\epsur(\secpar)$, hence \( \prob{\event{\errur}} \leq \epsur(\secpar)\,. \)

  \COMMENT{We have,
    \( \prob{\game{0} \land \nevent{\errur}} = \prob{\game{1} \land
      \nevent{\errur}} \) and, from the difference lemma,
    cf.~\cref{lem:difference_lemma},
  \[ \abs{\prob{\game{0}} - \prob{\game{1}}} \leq \prob{\event{\errur}}\,. \]
  Thus, to show that the transition from one game to another introduces only
  minor change in probability of $\advse$ winning it should be shown that
  $\prob{\event{\errur}}$ is small.

  We can assume that $\advse$ queried the simulator on the instance it wishes to
  output---$\inp_{\advse}$. We show a reduction $\rdvur$ that utilises $\advse$,
  who outputs a valid proof for $\inp_{\advse}$, to break the $\ur{k}$ property of
  $\ps$. Let $\rdvur$ run $\advse$ internally as a black-box:
\begin{itemize}
	\item The reduction answers both queries to the simulator $\psfs.\simulator$ and to the random oracle. 
	It also keeps lists $Q$, for the simulated proofs, and $Q_\ro$ for the random oracle queries. 
\item When $\advse$ makes a fake proof $\zkproof_{\advse}$ for $\inp_{\advse}$,
  $\rdvur$ looks through lists $Q$ and $Q_\ro$ until it finds
  $\zkproof_{\simulator}[0..k]$ such that
  $\zkproof_{\advse}[0..k] = \zkproof_{\simulator}[0..k]$
  and a random oracle query $\zkproof_{\simulator}[k].\ch$ on
  $\zkproof_{\simulator}[0..k]$.
	\item $\rdvur$ returns two proofs for $\inp_{\advse}$:
	\begin{align*}
		\zkproof_1 = (\zkproof_{\simulator}[1..k],
		\zkproof_{\simulator}[k].\ch, \zkproof_{\simulator}[k + 1..\mu + 1])\\
		\zkproof_2 = (\zkproof_{\simulator}[1..k],
		\zkproof_{\simulator}[k].\ch, \zkproof_{\advse}[k + 1..\mu + 1])
	\end{align*}
	\end{itemize}  
	If $\zkproof_1 = \zkproof_2$, then $\advse$ fails to break simulation
  extractability, as $\zkproof_2 \in Q$. On the other hand, if the proofs are
  not equal, then $\rdvur$ breaks $\ur{k}$-ness of $\ps$. This happens only with
  negligible probability $\epsur(\secpar)$, hence \( \prob{\event{\errur}} \leq
  \epsur(\secpar)\,. \)
}
%
\ngame{2} Define an adversary $\bdv$ against forking special soundness such that,
given access to oracles $\initU$ and $\ro$, and randomness $r = (r_\adv, \rsim)$, it
internally runs $\advse^{\initU, \simO} (1^\secpar; r_\adv)$, where
\begin{compactenum}
\item $\bdv$ answers $\adv$ random oracle and update queries by passing the queries to the real
  oracles $\ro$ and $\initU$. When $\adv$ finalises an SRS $\srs$, $\bdv$ does the same.
\item $\bdv$ answers simulator queries by using coins $\rsim$. $\bdv$ maintains a
  list of instance-proof pairs $Q$ consisting of of all simulation queries made by
  $\adv$, and corresponding responses.
\item Eventually when $\adv$ outputs $(\inp_\advse, \zkproof_\advse)$, $\bdv$ outputs
  the same $(\inp_\advse, \zkproof_\advse)$.
\end{compactenum}
% \michals{7.10}{Need forking soundness in the updatable setting. Alternatively we
%   could show that we can move a proof from one SRS to another using partial trapdoor}

In this game the environment additionally aborts if extractor $\ext$ fails to build a
$(1, \ldots, 1, n, 1, \ldots, 1)$-tree of accepting transcripts $\tree$ by rewinding
$\bdv$.

$\ext$ proceeds as follows. First, it takes as input relation $\REL$, SRS $\srs$,
$\bdv$'s code, its randomness $r$, the output instance $\inp_{\advse}$ and proof
$\zkproof_{\advse}$, and the list of random oracle queries and responses
$Q_\ro$. Then, $\ext$ starts a forking algorithm
$\genforking^{n}_\zdv(y,h_1, \ldots, h_q)$ for
$y = (\srs, \bdv, r, \inp_{\advse}, \zkproof_{\advse})$ where we set
$h_1, \ldots, h_q$ to be the consecutive queries from list $Q_\ro$. We run $\bdv$
internally in $\zdv$.
		
To assure that in the first execution of $\zdv$ the adversary $\bdv$ produces the
same $(\inp_{\advse}, \zkproof_{\advse})$ as in the extraction game, $\zdv$ provides
$\bdv$ with the same randomness $r$ and answers queries to the random oracle with
pre-recorded responses in $Q_\ro$.
		%
Note, that since the view of the adversary when run inside $\zdv$ is the same as its
view with access to the real random oracle, it produces exactly the same
output. After the first run, $\zdv$ outputs the index $i$ of a random oracle query
that was used by $\bdv$ to compute the challenge
$\zkproof[k''].\ch = \ro(\zkproof_{\advse}[0..k''])$ it had to answer in the $(k'' + 1)$-th
round and adversary's transcript, denoted by $s_1$ in $\genforking$'s description. If
no such query took place $\zdv$ outputs $i = 0$.
		
Then, new random oracle responses are picked for queries indexed by $i, \ldots, q$
and the adversary is rewound to the point just prior to when it gets the response to
RO query $\zkproof_{\advse}[0..k'']$. The adversary gets a random oracle response from
a new set of responses $h^2_i, \ldots, h^2_q$. If the adversary requests a simulated
proof after seeing $h^2_i$, then $\zdv$ computes the simulated proof on its
own. Eventually, $\zdv$ outputs index $i'$ of a query that was used by the adversary
to compute $\ro(\zkproof_{\advse}[0..k''])$, and a new transcript $s_2$. $\zdv$ is run
$n$ times with different random oracle responses.  Eventually, if all runs of $\zdv$
were successful, then tree of acceptable transcripts $\tree$ is built.
  % If a tree
  % $\tree$ of $n$ transcripts is built, then $\ext$ internally runs the tree extractor
  % $\extt(\tree)$ and outputs what it returns.
		
We emphasize here the importance of the unique response property. If it does not hold
then in some $j$-th execution of $\zdv$ the adversary $\adv$ (run internally in
$\bdv$) could reuse a challenge that it learned from observing proofs in $Q$. In that
case, $\bdv$ would output a proof that would make $\zdv$ output $i = 0$, making the
extractor fail. Fortunately, the case that the adversary breaks the unique response
property has already been covered by the abort condition in $\game{1}$.
		
Denote by $\waccProb$ the probability that $\advse$ outputs a proof that is accepted
and does not break $\ur{k}$-ness of $\ps$. With the same probability, an accepting
proof is returned by $\bdv$. This comes since the proof system is
$k'$-programmable. That is, $\bdv$ when providing $\adv$ with simulated proofs
programs the random oracle only in rounds from $k'$ on or, to put that differently,
it does not program the random oracle for rounds $1$ to $k' - 1$.  Hence, $\adv$
outputs proof that does not contain any programmed random oracle output and $\bdv$
can output that proof as its own. This holds since (1) for rounds $1$ to $k' - 1$,
the simulator does not program the oracle, (2) $\adv$ cannot use a part of a
simulated proof in rounds $k'$ to $\mu$ because of the unique response property.
	
	\ncase{Game 1 to Game 2}
		Note that for every accepting proof $\zkproof_{\advse}$, we may
		assume that whenever $\advse$ outputs a round $k''$ message $\zkproof_{\advse}[k'']$, then the
		$(\inp_{\advse}, \zkproof_{\advse}[1..k''])$ random oracle query was made by the adversary, not
		the simulator\footnote{\cite{INDOCRYPT:FKMV12} calls these queries \emph{fresh}.}, i.e.~there
		is no simulated proof $\zkproof_\simulator$ on $\inp_\simulator$ such that
		$(\inp_{\advse}, \zkproof_{\advse} [1..k'']) = (\inp_\simulator,
		\zkproof_\simulator[1..k''])$. Otherwise, the game would be already interrupted by the error
		event in Game $\game{1}$.  As previously,
		\( \abs{\prob{\game{1}} - \prob{\game{2}}} \leq \prob{\event{\errfrk}}\,.  \)
		

    Denote by $\waccProb'$ the probability that algorithm $\zdv$, defined in the general forking lemma,
		produces an accepting proof with a fresh challenge after round $k''$. From the above argument,
		we have that $\waccProb = \waccProb'$.
		
		Next, from the generalised forking lemma, cf.~\cref{lem:generalised_forking_lemma}, we get that
		\begin{equation}
		\begin{split}
		\prob{\event{\errfrk}} \leq 1 - \waccProb \cdot \left(\infrac{\waccProb^{n -
				1}}{q^{n - 1}} + \infrac{(2^\secpar) !}{((2^\secpar - n)! \cdot
			(2^\secpar)^{n})} - 1\right).
		\end{split}
		\end{equation}
	
	\ngame{3}
    This game is identical to $\game{2}$ except that the extractor $\ext$, given tree
    of acceptable transcripts $\tree$ runs additionally $\extt (\tree)$ to learn the
    witness $\wit$. The environment additionally aborts this game if $\REL(\inp,
    \wit)$ does not hold.
	
	\ncase{Game 2 to Game 3}	
		Since $\ps$ is forking-sound the probability that $\extt(\tree)$
		fails is upper-bounded by $\epsss(\secpar)$.
		
		Since Game $\game{3}$ is aborted when it is impossible to extract a witness from
		$\tree$ and $\bdv$ only passes proofs produced by $\adv$, the adversary $\advse$ cannot
		win. Thus, by the game-hopping argument,
		\[
		\abs{\prob{\game{0}} - \prob{\game{4}}} \leq 1 -
		\left(\frac{\waccProb^{n}}{q^{n - 1}} + \waccProb \cdot \frac{(2^\secpar)
			!}{(2^\secpar - n)! \cdot (2^\secpar)^{n}} - \waccProb\right) + \epsur(\secpar) +
		%q_{\ro}^{\mu} \epss +
		\epsss(\secpar)\,.
		\]
		Thus the probability that extractor $\extss$ succeeds is at least
		\[
		\frac{\waccProb^{n}}{q^{n - 1}} + \waccProb \cdot \frac{(2^\secpar)
			!}{(2^\secpar - n)! \cdot (2^\secpar)^{n}} - \waccProb - \epsur(\secpar) 
		%- q_{\ro}^{\mu} \epss
		- \epsss(\secpar)\,.
		\]
		Since $\waccProb$ is the probability of $\advse$ producing an accepting transcript
		that does not break $\ur{k}$-ness of $\ps$, then $\waccProb \geq \accProb -
		\epsur(\secpar)$, where $\accProb$ is the probability of $\advse$ outputting an accepting
		proof as defined in \cref{def:updsimext}. Thus, 
		\begin{equation}
		\label{eq:frk}
		\extProb \geq \frac{(\accProb - \epsur(\secpar))^{n}}{q^{n - 1}} -
		\underbrace{(\accProb - \epsur(\secpar)) \cdot \left( 1 - \frac{(2^\secpar)
				!}{(2^\secpar - n)! \cdot (2^\secpar)^{n}}\right) - \epsur(\secpar) -
			% q_{\ro}^{\mu} \epss -
			\epsss(\secpar)}_{\eps(\secpar)}\,.
		\end{equation}
		Note that the part of \cref{eq:frk} denoted by $\eps(\secpar)$ is negligible as
		$\epsur(\secpar), \epsss(\secpar)$ are negligible, and
		$\frac{(2^\secpar) !}{(2^\secpar - n)! \cdot (2^\secpar)^{n}} \geq
		\left(\infrac{(2^\secpar - n)}{2^\secpar}\right)^{n}$ is overwhelming.  Therefore,
		\[
		\extProb \geq q^{-(n - 1)} (\accProb - \epsur(\secpar))^{n} -\eps(\secpar)\,.
		\] 
		and $\psfs$ is \COMMENT{forking }simulation extractable with extraction error $\epsur(\secpar)$.
	\qed
\end{proof}


We conjecture that based on the recent results on state restoration
soundness~\cite{C:GhoTes21}, which effectively allows to query the verifier multiple
times on different overlapping transcripts, the $q^{\mu}$ loss could be
avoided. However, this would reduce the class of protocols covered by our results.

%%% Local Variables:
%%% mode: latex
%%% TeX-master: "main"
%%% End:

\section{Concrete SNARKs Preliminaries}

\ourpar{Bilinear groups.}
A bilinear group generator $\pgen (\secparam)$ returns public parameters $ \pp =
(p, \GRP_1, \GRP_2, \GRP_T, \pair, \gone{1}, \gtwo{1})$, where $\GRP_1$,
$\GRP_2$, and $\GRP_T$ are additive cyclic groups of prime order $p = 2^{\Omega
  (\secpar)}$, $\gone{1}, \gtwo{1}$ are generators of $\GRP_1$, $\GRP_2$, resp.,
and $\pair: \GRP_1 \times \GRP_2 \to \GRP_T$ is a non-degenerate
$\ppt$-computable bilinear pairing. We assume the bilinear pairing to be Type-3,
i.e., that there is no efficient isomorphism from $\GRP_1$ to $\GRP_2$ or from
$\GRP_2$ to $\GRP_1$. We use the by now standard bracket notation, i.e., we
write $\bmap{a}{\gi}$ to denote $a g_{\gi}$ where $g_{\gi}$ is a fixed generator
of $\GRP_{\gi}$. We denote $\pair (\gone{a}, \gtwo{b})$ as $\gone{a} \bullet
\gtwo{b}$. Thus, $\gone{a} \bullet \gtwo{b} = \gtar{a b}$. We freely use the
bracket notation with matrices, e.g., if $\vec{A} \vec{B} = \vec{C}$ then
$\vec{A} \grpgi{\vec{B}} = \grpgi{\vec{C}}$ and $\gone{\vec{A}}\bullet
\gtwo{\vec{B}} = \gtar{\vec{C}}$. Since every algorithm $\adv$ takes as input
the public parameters we skip them when describing $\adv$'s input. Similarly, we
do not explicitly state that each protocol starts with generating these
parameters by $\pgen$.

\subsection{Algebraic Group Model}
The algebraic group model (AGM) introduced in \cite{C:FucKilLos18} lies between the
standard model and generic bilinear group model. In the AGM it is assumed that an
adversary $\adv$ can output a group element $\gnone{y} \in \GRP$ if $\gnone{y}$ has
been computed by applying group operations to group elements given to $\adv$ as
input. It is further assumed, that $\adv$ knows how to ``build'' $\gnone{y}$ from
those elements. More precisely, the AGM requires that whenever $\adv(\gnone{\vec{x}})$
outputs a group element $\gnone{y}$ then it also outputs $\vec{c}$ such that
$\gnone{y} = \vec{c}^\top \cdot \gnone{\vec{x}}$. $\plonk$, $\sonic$ and $\marlin$
have been shown secure using the AGM. An adversary that works in the AGM is called
\emph{algebraic}.

\oursubsub{Idealised Verifier and Verification Equations.} Let
$(\SRScer, \prover, \verifier, \simulator)$ be a proof system.
% or a polynomial commitment
% scheme\hamid{might be unclear as we are defining polynomial commitments as
%   $(\kgen, \com, \open, \verify)$.}.
Observe that the $\SRScer$ algorithms provide an SRS which can be interpreted as a set
of group representation of polynomials evaluated at trapdoor elements. That is, for a
trapdoor $\chi$ the SRS contains $\gone{\p{p_1}(\chi), \ldots, \p{p_k}(\chi)}$, for
some polynomials $\p{p_1}(X), \ldots, \p{p_k}(X) \in \FF_p[X]$. The verifier
$\verifier$ accepts a proof $\zkproof$ for instance $\inp$ if (a set of) verification
equation $\vereq_{\inp, \zkproof}$ (which can also be interpreted as a polynomial in
$\FF_p[X]$ whose coefficients depend on messages sent by the prover) zeroes at
$\chi$. Following \cite{EPRINT:GabWilCio19} we call verifiers who check that
$\vereq_{\inp, \zkproof}(\chi) = 0$ \emph{real verifiers} as opposed to \emph{ideal
  verifiers} who accept only when $\vereq_{\inp, \zkproof}(X) = 0$. That is, while a
real verifier accepts when a polynomial \emph{evaluates} to zero, an ideal verifier
accepts only when the polynomial \emph{is} zero.

Although ideal verifiers are impractical, they are very useful in our
proofs. More precisely, we show that
%\begin{compactenum}
%\item 
the idealised verifier accepts an incorrect proof (what ``incorrect''
  means depends on the situation) with at most negligible probability (and in many
  cases---never);
%\item 
when the real verifier accepts, but not the idealised one, then a malicious
  prover can be used to break the underlying security assumption (in our case---a
  variant of $\dlog$.)
%\end{compactenum}

Analogously, idealised verifier can be defined for polynomial commitment schemes.

\subsection{Dlog Assumptions in Standard and Updatable Setting}
\label{dlog-upd}
\label{sec:udlog_assumptions}


\label{sec:dlog_assumptions}
\begin{definition}[$(q_1, q_2)\mhyph\dlog$ assumption]\label{def:dlog}
	Let $\adv$ be a $\ppt$ adversary that gets as input
  $\gone{1, \chi, \ldots, \chi^{q_1}}, \gtwo{1, \chi, \ldots, \chi^{q_2}}$, for
  some randomly picked $\chi \in \FF_p$, the assumption requires that $\adv$ cannot compute $\chi$. That is
	\[
		\condprob{\chi = \adv(\gone{1, \chi, \ldots, \chi^{q_1}}, \gtwo{1, \chi,
        \ldots, \chi^{q_2} })}{\chi \sample \FF_p} \leq \negl.
	\]
\end{definition}

 Since all our protocols and security notions are in the updatable setting, it is natural to define the dlog assumptions also in the updatable setting. That is, instead of being given a dlog challenge the adversary $\adv$ is given access to an update oracle as defined in~\cref{fig:upd}. The honestly generated SRS is set to be a dlog challenge and the update algorithm $\upd$ re-randomizing the challenge. We define this assumptions and show a reduction between the assumptions in the updatable and standard setting. 

Note that for clarity we here refer to the SRS by $\Ch$. Further, to avoid cluttering notation, we do not make the update proofs explicit. They are generated in the same manner as the proofs in~\cref{fig:upd-scheme}.
	
%The reduction $\reduction$ proceeds as follows: given the input dlog instance, $\reduction$ answers adversary's queries for the dlog updates and sets the honest update to be the input dlog instance. Once the dlog challenge in the updatable setting is finalized, it runs the adversary and obtains the answer $\chi'$. Let $\chi_1, \ldots, \chi_\ell$ be the partial discrete logarithms of dlog instances corresponding to the adversary's dlog updates. These values can be computed by $\reduction$ by extracting from the update proofs given by the adversary. $\reduction$ now returns  $\chi = \chi' (\chi_1 \chi_2 \ldots \chi_\ell)^{-1}$. The same argument holds for the $(q_1, q_2)\mhyph\ldlog$ assumption, ~\cref{def:ldlog}.

\begin{definition}[$(q_1, q_2)\mhyph\udlog$ assumption]\label{def:udlog}
	Let $\adv$ be a $\ppt$ adversary that gets oracle access to $\initU$ with internal algorithms $(\kgen, \upd, \verifyCRS)$, where $\kgen$ and $\upd$ are defined as follows:
	\begin{itemize}
	\item $\kgen(\secpar)$ samples $\chi \sample \FF_p$ and defines 
		$\Ch:=(\gone{1, \chi, \ldots,
			\chi^{q_1}}, \gtwo{1, \chi, \ldots, \chi^{q_2}
		})$.
	\item $\upd(\Ch, \{\rho_j \}_{j=1}^n)$ 
	parses $\Ch$ as $\left( \gone{\smallset{A_i}_{i = 0}^{q_1}},
	\gtwo{\smallset{B_i}_{i = 0}^{q_2}} \right)$, samples
	$\widetilde{\chi} \sample \FF_p$, and defines
	$\widetilde{\Ch} := 
	\left( \gone{\smallset{\widetilde{\chi}^i A_i}_{i = 0}^{q_1}},
	\gtwo{\smallset{\widetilde{\chi}^i B_i}_{i = 0}^{q_2}} \right)$.
	\end{itemize}
	Then
	$
	\prob{\bar{\chi} \gets \adv^{\initU}(\secpar)} \leq \negl,
	$
	where $\left( \gone{\smallset{\bar{\chi}^i}_{i = 0}^{q_1}},
	\gtwo{\smallset{\bar{\chi}^i}_{i = 0}^{q_2}} \right)$ is the final $\Ch$.
\end{definition}



\begin{remark}[Single adversarial updates after an honest setup.]\label{rem:upd}
	As an alternative to the updatable setting defined in~\cref{fig:upd}, one can consider a slightly different model of setup, where the adversary is given an initial honestly-generated SRS and is then allowed to perform a malicious update in one-shot fashion.
	Groth
	et al.\ show in~\cite{C:GKMMM18} that the two definitions are equivalent for polynomial commitment based SNARKs. We use this simpler definition in our reductions.
\end{remark}


We show a reduction from $(q_1, q_2)\mhyph\dlog$ assumption to its variant in the updatable setting (with single adversarial update). 
%We omit showing the reduction $(q_1, q_2)\mhyph\ldlog \Rightarrow (q_1, q_2)\mhyph\uldlog$ as it can be done similarly in a straightforward manner.
\begin{lemma}
	$(q_1, q_2)\mhyph\dlog \Rightarrow (q_1, q_2)\mhyph\udlog$.
	\end{lemma}
\begin{proof}
	We construct a reduction algorithm $\reduction$ which uses an adversary $\adv$ on the $(q_1, q_2)\mhyph\udlog$ and construct an adversary on the $(q_1, q_2)\mhyph\dlog$. Specifically, $\reduction$ proceeds as follows: given a dlog instance $\Ch$ as input, it sets $\Ch$ as the initial (honestly generated) challenge. Once the dlog challenge in the updatable setting is finalized, it runs $\adv$ and obtains the answer $\chi'$. Let $\chi_\adv$ be the (possibly) partial discrete logarithm of the dlog instance corresponding to the adversary's update. This value can be computed by $\reduction$ by extracting from the update proof given by $\adv$. $\reduction$ now returns $\chi = \chi' \chi_\adv^{-1}$ as the discrete logarithm of $\Ch$.
	\end{proof}

  \oursubsub{Generalized Forking Lemma}
Although dubbed ``general'', the forking lemma of~\cite{CCS:BelNev06} is not general enough for our purpose as it is useful only for protocols where a witness can be extracted from just two transcripts. To be able to extract a witness from, say, an execution of $\plonkprot$ we need at least $(3 \numberofconstrains + 1)$ valid proofs (where $\numberofconstrains$ is the number of constrains), $(\numberofconstrains+ 1)$ for $\sonicprot$, and $\numberofconstrains + 4$ for $\marlinprot$.\hamid{2.5}{Numbers are incorrect! Maybe $(3 \numberofconstrains + 6)$ for Plonk, $(\numberofconstrains+ \linconstr+ 1)$ for Sonic and $(\numberofconstrains+ 3)$ for Marlin? } Here we use a result by Attema et
al.~\cite{EPRINT:AttFehKlo21short}\footnote{An earlier versions had its own forking lemma generalization. Attema et al.\ has a better bound.}  which lower-bounds the probability of generating a tree of accepting transcripts $\tree$.

  \begin{lemma}[Proposition 2~\cite{EPRINT:AttFehKlo21short}]\label{lem:attema}
	Let $N = n_1 \cdot \cdots \cdot n_\mu$ and $p = 2^{\Omega(\secpar)}$. Let $\epserr(\secpar) = 1 - \prod_{i=1}^{\mu}\left(1 - \frac{n_i - 1}{p}\right)$.
	Assume adversary $\adv$ that makes up to $q$ random
	oracle queries and outputs an acceptable proof with probability at least
	$\accProb$. There exists a tree building algorithm $\tdv$ for $(n_1, \ldots, n_\mu)$-trees that %utilizes the extractor $\zdv_1$ in~\cref{fig:Attema-ext} and 
	succeeds in building a
	tree of accepting transcripts in expected
	running time $N + q (N - 1)$ with probability at least
	\[
	\frac{\accProb - (q + 1) \epserr (\secpar)}{1 - \epserr (\secpar)}.
	\]
	\end{lemma}

	\iffalse
\begin{figure}[t]
	\centering
	\fbox{\parbox{\textwidth}{
			\begin{enumerate}
				\item Run $\zdv_{m + 1}$ as follows to obtain $(\vec{\zkproof}, \tzkproof_1, v)$: relay the $q + \mu$ queries to the random oracle and record all query-response pairs. Set 				$h \gets \zkproof[m].\ch $, and let $c_h$ be the response to query $i$. 
				\item If $v = 0$, abort with output $v = 0$. 
				\item Else, repeat
				\begin{itemize}
					\item sample $c'_h \in C \setminus \smallset{c_h}$ (without replacement);
					\item run $\zdv_{m + 1}$ as follows to obtain $(\vec{\zkproof'}, \tzkproof', v')$, aborting right after the initial run of $\adv$ if $\zkproof'[m].\ch  \neq \zkproof[m].\ch$: answer the query to $h$ with $c'_h$, while answering all other queries consistently if the query was performed by $\zdv_{m + 1}$ already on a previous run and with a fresh random value in $C$ otherwise; 
				\end{itemize}
				until either $n_m - 1$ additional challenges $c'_h$ with $v' = 1$ and $\zkproof'[m].\ch  = \zkproof[m]$ have been found or until all challenges $c'_h \in C$ have been tried. 
				\item In the former case, output $\vec{\zkproof}$, the $n_m$ accepting
				$(1, \ldots, 1, n_{m + 1}, \ldots, n_\mu)$-trees $\tzkproof_1, \ldots, \tzkproof_{n_m}$, and
				$v \gets 1$; in the latter case, output $v \gets 0$.
			\end{enumerate}
	}}
	\caption{Extractor $\zdv_m$ from Attema et al.~\cite{EPRINT:AttFehKlo21short}. Here $\vec{\zkproof}$ is an index vector that contains all random oracle queries made by the prover for the output proof; that is, for prover's messages $\vec{a} = a_1, \ldots, a_{\mu + 1}$, $\zkproof[1].\ch = (\inp, a_1), \ldots, \zkproof[{\mu}].\ch = (\inp, a_1, \ldots, a_\mu)$.  $\tzkproof$ is a proof transcript that contains both prover's messages and random oracle responses, that is $\tzkproof = (a_1, (\inp, a_1), a_2, (\inp, a_1, a_2), \ldots)$. Finally, $v$ is the verification bit, that is $v = \verifier (\srs, \inp, \tzkproof)$. }
	\label{fig:Attema-ext}
\end{figure} 
\fi

\oursubsub{Opening Uniqueness of Batched Polynomial Commitment Openings}
%We say that $\PCOM$ has the unique opening property if the following holds:
To show the unique response property required by our main theorem we show that the polynomial commitment schemes employed by concrete proof systems have unique openings, which, intuitively, assures that there is only one
valid opening for the committed polynomial and given evaluation point

\begin{definition}[Unique opening property]
	Let $m \in \NN$ be the number of committed polynomials, $l \in \NN$ number of evaluation points, $\vec{c} \in \GRP^m$ be the commitments, $\vec{z} \in \FF_p^l$ be the arguments the polynomials are evaluated at, $K_j$ set of indices of polynomials which are evaluated at $z_j$, $\vec{s_{i}}$ vector of evaluations of $\p{f_i}$,, and $\vec{o_j}, \vec{o'_j} \in \FF_p^{K_j}$ be the commitment openings. Then for every $\ppt$ adversary $\adv$
	  \[
		  \condprob{
			  \begin{matrix}
					\verify(\srs, \vec{c}, \vec{z}, \vec{s}, \vec{o}) = 1,  \\ 
					\verify(\srs, \vec{c}, \vec{z}, \vec{s}, \vec{o'}) = 1, \\
				   \vec{o} \neq \vec{o'}
			  \end{matrix}
		  }{
			  \begin{matrix}
  %				& \srs \gets \kgen(\secparam, \maxdeg),\\
					(\vec{c}, \vec{z}, \vec{s}, \vec{o}, \vec{o'}) \gets \adv^{\initU}(\maxdeg)
			  \end{matrix}
		  }\leq \negl\,.
	  \]
\end{definition}

We show that $\plonk$'s, $\sonic$'s, and $\marlin$'s polynomial commitment schemes satisfy this requirement in \cref{sec:uop}.

\cref{fig:pcomp}, cf.~\cref{sec:uop}, presents variants of KZG~\cite{AC:KatZavGol10} polynomial commitment schemes used in \plonk{}, \sonic{} and \marlin{}. 
To support batched verification, the inputs to $\com$, $\open$, $\verify$ are vectors, and the algorithms take an additional arbitrary auxiliary string as input. This adversarially chosen string only provides additional context for the computation of challenges. We treat auxiliary input implicitly in the definition.
 
% !TEX root = main.tex
% !TEX spellcheck = en-US


\section{Non-Malleability of $\plonkprotfs$} 
\label{sec:plonk}
In this section, we show that $\plonkprotfs$ is \COMMENT{forking }simulation-extractable. Towards this end, we first use the unique opening property to show that
$\plonkprot$ has the $\ur{2}$ property,
cf.~\cref{lem:plonkprot_ur}.
Next, we show that $\plonkprot$ is forking-special-sound. That is, given a
number of accepting transcripts whose messages match on the first $5$ messages of the
protocol, we can either extract a witness for the proven statement or use
one of the transcripts to break the $\dlog$ assumption. This result is shown in
the AGM, cf.~\cref{lem:plonkprot_ss}.

Given forking-soundness and $\ur{2}$ of $\plonkprot$, we invoke \cref{thm:se} and conclude that $\plonkprot_\fs$ is \COMMENT{forking }simulation-extractable.
%Due to page limit, we omit description of \plonk{} here and refer to
%\cref{sec:plonk_explained}.
%Unfortunately, we also have to Message some of the
% proofs to the Supplementary Materials as well, cf.~\cref{sec:plonk_supp_mat}

\newcommand{\vql}{\vec{q_{L}}}
\newcommand{\vqr}{\vec{q_{R}}}
\newcommand{\vqm}{\vec{q_{M}}}
\newcommand{\vqo}{\vec{q_{O}}}
\newcommand{\vx}{\vec{x}}
\newcommand{\vqc}{\vec{q_{C}}}

\subsection{Plonk protocol description}
\label{sec:plonk_explained}
\oursubsub{The constrain system}
Assume $\CRKT$ is a fan-in two arithmetic circuit,
which fan-out is unlimited and has $\numberofconstrains$ gates and $\noofw$ wires
($\numberofconstrains \leq \noofw \leq 2\numberofconstrains$). \plonk's constraint
system is defined as follows:
\begin{itemize}
\item Let $\vec{V} = (\va, \vb, \vc)$, where $\va, \vb, \vc
  \in \range{1}{\noofw}^\numberofconstrains$. Entries $\va_i, \vb_i, \vc_i$ represent indices of left,
  right and output wires of circuits $i$-th gate.
\item Vectors $\vec{Q} = (\vql, \vqr, \vqo, \vqm, \vqc) \in
  (\FF^\numberofconstrains)^5$ are called \emph{selector vectors}:
  \begin{itemize}
  \item If the $i$-th gate is a multiplicative gate then $\vql_i = \vqr_i = 0$,
    $\vqm_i = 1$, and $\vqo_i = -1$. 
  \item If the $i$-th gate is an addition gate then $\vql_i = \vqr_i  = 1$, $\vqm_i =
    0$, and $\vqo_i = -1$. 
  \item $\vqc_i = 0$ always. 
  \end{itemize}
\end{itemize}

We say that vector $\vx \in \FF^\noofw$ satisfies constraint system if for all $i
\in \range{1}{\numberofconstrains}$
\[
  \vql_i \cdot \vx_{\va_i} + \vqr_i \cdot \vx_{\vb_i} + \vqo \cdot \vx_{\vc_i} +
  \vqm_i \cdot (\vx_{\va_i} \vx_{\vb_i}) + \vqc_i = 0. 
\]

\oursubsub{Algorithms rolled out}
\label{sec:plonk_explained}
\plonk{} argument system is universal. That is, it allows to verify computation
of any arithmetic circuit which has no more than $\numberofconstrains$
gates using a single SRS. However, to make computation efficient, for each
circuit there is allowed a preprocessing phase which extend the SRS with
circuit-related polynomial evaluations.

For the sake of simplicity of the security reductions presented in this paper, we
include in the SRS only these elements that cannot be computed without knowing
the secret trapdoor $\chi$. The rest of the SRS---the preprocessed input---can
be computed using these SRS elements thus we leave them to be computed by the
prover, verifier, and simulator.

\ourpar{$\plonk$ SRS generating algorithm $\kgen(\REL)$:}
The SRS generating algorithm picks at random $\chi \sample \FF_p$, computes
and outputs
\[
	\srs = \left(\gone{\smallset{\chi^i}_{i = 0}^{\numberofconstrains + 2}},
	\gtwo{\chi} \right).
\]

\ourpar{Preprocessing:}
Let $H = \smallset{\omega^i}_{i = 1}^{\numberofconstrains }$ be a
(multiplicative) $\numberofconstrains$-element subgroup of a field $\FF$
compound of $\numberofconstrains$-th roots of unity in $\FF$. Let $\lag_i(X)$ be
the $i$-th element of an $\numberofconstrains$-elements Lagrange basis. During
the preprocessing phase polynomials $\p{S_{id j}}, \p{S_{\sigma j}}$, for
$\p{j} \in \range{1}{3}$, are computed:
\begin{equation*}
  \begin{aligned}
    \p{S_{id 1}}(X) & = X,\vphantom{\sum_{i = 1}^{\noofc} \sigma(i) \lag_i(X),}\\
    \p{S_{id 2}}(X) & = k_1 \cdot X,\vphantom{\sum_{i = 1}^{\noofc} \sigma(i) \lag_i(X),}\\
    \p{S_{id 3}}(X) & = k_2 \cdot X,\vphantom{\sum_{i = 1}^{\noofc} \sigma(i) \lag_i(X),}
  \end{aligned}
  \qquad
\begin{aligned}
  \p{S_{\sigma 1}}(X) & = \sum_{i = 1}^{\noofc} \sigma(i) \lag_i(X),\\
  \p{S_{\sigma 2}}(X) & = \sum_{i = 1}^{\noofc}
  \sigma(\noofc + i) \lag_i(X),\\
  \p{S_{\sigma 3}}(X) & =\sum_{i = 1}^{\noofc} \sigma(2 \noofc + i) \lag_i(X).
\end{aligned}
\end{equation*}
Coefficients $k_1$, $k_2$ are such that $H, k_1 \cdot H, k_2 \cdot H$ are
different cosets of $\FF^*$, thus they define $3 \cdot \noofc$
different elements. \cite{EPRINT:GabWilCio19} notes that it is enough to set
$k_1$ to a quadratic residue and $k_2$ to a quadratic non-residue.

Furthermore, we define polynomials $\p{q_L}, \p{q_R}, \p{q_O}, \p{q_M}, \p{q_C}$
such that
\begin{equation*}
  \begin{aligned}
  \p{q_L}(X) & = \sum_{i = 1}^{\noofc} \vql_i \lag_i(X), \\
  \p{q_R}(X) & = \sum_{i = 1}^{\noofc} \vqr_i \lag_i(X), \\
  \p{q_M}(X) & = \sum_{i = 1}^{\noofc} \vqm_i \lag_i(X),
\end{aligned}
\qquad
\begin{aligned}
  \p{q_O}(X) & = \sum_{i = 1}^{\noofc} \vqo_i \lag_i(X), \\
  \p{q_C}(X) & = \sum_{i = 1}^{\noofc} \vqc_i \lag_i(X). \\
  \vphantom{\p{q_M}(X)  = \sum_{i = 1}^{\noofc} \vqm_i \lag_i(X),}
\end{aligned}
\end{equation*}

\ourpar{Proving statements in $\plonkprotfs$} We show how prover's algorithm
$\prover(\srs, \inp, \wit = (\wit_i)_{i \in \range{1}{3 \cdot \noofc}})$ operates for
the Fiat--Shamir transformed version of Plonk.
\begin{description}
\item[Message 1] Sample $b_1, \ldots, b_9 \sample \FF_p$; compute
  $\p{a}(X), \p{b}(X), \p{c}(X)$ as
	\begin{align*}
		\p{a}(X) &= (b_1 X + b_2)\p{Z_H}(X) + \sum_{i = 1}^{\noofc} \wit_i \lag_i(X) \\
		\p{b}(X) &= (b_3 X + b_4)\p{Z_H}(X) + \sum_{i = 1}^{\noofc} \wit_{\noofc + i} \lag_i(X) \\
		\p{c}(X) &= (b_5 X + b_6)\p{Z_H}(X) + \sum_{i = 1}^{\noofc} \wit_{2 \cdot \noofc + i} \lag_i(X) 
	\end{align*}
	Output polynomial commitments $\gone{\p{a}(\chi), \p{b}(\chi), \p{c}(\chi)}$.  
	
\item[Message 2] Compute challenges $\beta, \gamma \in \FF_p$ by querying random oracle
  on partial proof, that is,
	\[
		\beta = \ro(\zkproof[0..1], 0)\,, \qquad \gamma = \ro(\zkproof[0..1], 1)\,.
	\]
  
	Compute permutation polynomial $\p{z}(X)$
	\begin{multline*}
		\p{z}(X) = (b_7 X^2 + b_8 X + b_9)\p{Z_H}(X) + \lag_1(X) + \\
    + \sum_{i = 1}^{\noofc - 1} \left(\lag_{i + 1} (X) \prod_{j = 1}^{i} \frac{
        (\wit_j +\beta \omega^{j - 1} + \gamma)(\wit_{\noofc + j} + \beta k_1
        \omega^{j - 1} + \gamma)(\wit_{2 \noofc + j} +\beta k_2 \omega^{j- 1} +
        \gamma)} {(\wit_j+\sigma(j) \beta + \gamma)(\wit_{\noofc + j} + \sigma(\noofc
        + j)\beta + \gamma)(\wit_{2 \noofc + j} + \sigma(2 \noofc + j)\beta +
        \gamma)}\right)
	\end{multline*}
	Output polynomial commitment $\gone{\p{z}(\chi)}$
		
\item[Message 3] Compute the challenge $\alpha = \ro(\zkproof[0..2])$, compute the quotient
  polynomial
	\begin{align*}
    & \p{t}(X)  = \\
    & (\p{a}(X) \p{b}(X) \selmulti(X) + \p{a}(X) \selleft(X) + 
      \p{b}(X)\selright(X) + \p{c}(X)\seloutput(X) + \pubinppoly(X) + \selconst(X)) 
      \frac{1}{\p{Z_H}(X)} +\\
    & + ((\p{a}(X) + \beta X + \gamma) (\p{b}(X) + \beta k_1 X + \gamma)(\p{c}(X) 
      + \beta k_2 X + \gamma)\p{z}(X)) \frac{\alpha}{\p{Z_H}(X)} \\
    & - (\p{a}(X) + \beta \p{S_{\sigma 1}}(X) + \gamma)(\p{b}(X) + \beta 
      \p{S_{\sigma 2}}(X) + \gamma)(\p{c}(X) + \beta \p{S_{\sigma 3}}(X) + 
      \gamma)\p{z}(X \omega))  \frac{\alpha}{\p{Z_H}(X)} \\
    & + (\p{z}(X) - 1) \lag_1(X) \frac{\alpha^2}{\p{Z_H}(X)}
	\end{align*}
	Split $\p{t}(X)$ into degree less then $\noofc$ polynomials
  $\p{t_{lo}}(X), \p{t_{mid}}(X), \p{t_{hi}}(X)$, such that
	\[
		\p{t}(X) = \p{t_{lo}}(X) + X^{\noofc} \p{t_{mid}}(X) + X^{2 \noofc}
    \p{t_{hi}}(X)\,.
	\]
	Output $\gone{\p{t_{lo}}(\chi), \p{t_{mid}}(\chi), \p{t_{hi}}(\chi)}$.
	
\item[Message 4] Get the challenge $\chz \in \FF_p$, $\chz = \ro(\zkproof[0..3])$.
  Compute opening evaluations
	\begin{align*}
    \p{a}(\chz), \p{b}(\chz), \p{c}(\chz), \p{S_{\sigma 1}}(\chz), \p{S_{\sigma 2}}(\chz), \p{t}(\chz), \p{z}(\chz \omega),
	\end{align*}
	Compute the linearisation polynomial
	\[
		\p{r}(X) =
		\begin{aligned}
      & \p{a}(\chz) \p{b}(\chz) \selmulti(X) + \p{a}(\chz) \selleft(X) + \p{b}(\chz) \selright(X) + \p{c}(\chz) \seloutput(X) + \selconst(X) \\
      & + \alpha \cdot \left( (\p{a}(\chz) + \beta \chz + \gamma) (\p{b}(\chz) + \beta k_1 \chz + \gamma)(\p{c}(\chz) + \beta k_2 \chz + \gamma) \cdot \p{z}(X)\right) \\
      & - \alpha \cdot \left( (\p{a}(\chz) + \beta \p{S_{\sigma 1}}(\chz) + \gamma) (\p{b}(\chz) + \beta \p{S_{\sigma 2}}(\chz) + \gamma)\beta \p{z}(\chz\omega) \cdot \p{S_{\sigma 3}}(X)\right) \\
      & + \alpha^2 \cdot \lag_1(\chz) \cdot \p{z}(X)
		\end{aligned}
	\]
	Output
  $\p{a}(\chz), \p{b}(\chz), \p{c}(\chz), \p{S_{\sigma 1}}(\chz), \p{S_{\sigma
      2}}(\chz), \p{t}(\chz), \p{z}(\chz \omega), \p{r}(\chz).$
	
\item[Message 5] Compute the opening challenge $v \in \FF_p$,
  $v = \ro(\zkproof[0..4])$.  Compute the openings for the polynomial commitment
  scheme
	\begin{align*}
	& \p{W_\chz}(X) = \frac{1}{X - \chz} \left(
   \begin{aligned}
     & \p{t_{lo}}(X) + \chz^\noofc \p{t_{mid}}(X) + \chz^{2 \noofc} \p{t_{hi}}(X) - \p{t}(\chz)\\
     & + v(\p{r}(X) - \p{r}(\chz)) \\
     & + v^2 (\p{a}(X) - \p{a}(\chz))\\
     & + v^3 (\p{b}(X) - \p{b}(\chz))\\
     & + v^4 (\p{c}(X) - \p{c}(\chz))\\
     & + v^5 (\p{S_{\sigma 1}}(X) - \p{S_{\sigma 1}}(\chz))\\
     & + v^6 (\p{S_{\sigma 2}}(X) - \p{S_{\sigma 2}}(\chz))
   \end{aligned}
       \right)\\
    & \p{W_{\chz \omega}}(X) = \frac{\p{z}(X) - \p{z}(\chz \omega)}{X - \chz \omega}
  \end{align*}
	Output $\gone{\p{W_{\chz}}(\chi), \p{W_{\chz \omega}}(\chi)}$.
\end{description}

\ncase{$\plonk$ verifier $\verifier(\srs, \inp, \zkproof)$}\ \newline
The \plonk{} verifier works as follows
\begin{enumerate}
	\item Validate all obtained group elements.
	\item Validate all obtained field elements.
	\item Validate the instance
      $\inp = \smallset{\wit_i}_{i = 1}^\instsize$.
	\item Compute challenges $\beta, \gamma, \alpha, \chz, v,
      u$ from the transcript.
	\item Compute zero polynomial evaluation
      $\p{Z_H} (\chz) =\chz^\noofc - 1$.
	\item Compute Lagrange polynomial evaluation
      $\lag_1 (\chz) = \frac{\chz^\noofc -1}{\noofc (\chz - 1)}$.
	\item Compute public input polynomial evaluation
      $\pubinppoly (\chz) = \sum_{i \in \range{1}{\instsize}} \wit_i
      \lag_i(\chz)$.
	\item Compute quotient polynomials evaluations
	\begin{multline*}
    \p{t} (\chz) = \frac{1}{\p{Z_H}(\chz)} \Big(
    \p{r} (\chz) + \pubinppoly(\chz) - (\p{a}(\chz) + \beta \p{S_{\sigma 1}}(\chz) + \gamma) (\p{b}(\chz) + \beta \p{S_{\sigma 2}}(\chz) + \gamma) \\
    (\p{c}(\chz) + \gamma)\p{z}(\chz \omega) \alpha - \lag_1 (\chz) \alpha^2
    \Big) \,.
	\end{multline*}
	\item Compute batched polynomial commitment
	$\gone{D} = v \gone{r} + u \gone {z}$ that is
	\begin{align*}
		\gone{D} & = v
		\left(
		\begin{aligned}
          & \p{a}(\chz)\p{b}(\chz) \cdot \gone{\selmulti} + \p{a}(\chz)  \gone{\selleft} + \p{b}  \gone{\selright} + \p{c}  \gone{\seloutput} + \\
          & + (	(\p{a}(\chz) + \beta \chz + \gamma) (\p{b}(\chz) + \beta k_1 \chz + \gamma) (\p{c} + \beta k_2 \chz + \gamma) \alpha  + \lag_1(\chz) \alpha^2)  + \\
			% &   \\
          & - (\p{a}(\chz) + \beta \p{S_{\sigma 1}}(\chz) + \gamma) (\p{b}(\chz)
          + \beta \p{S_{\sigma 2}}(\chz) + \gamma) \alpha \beta \p{z}(\chz
          \omega) \gone{\p{S_{\sigma 3}}(\chi)})
		\end{aligned}
		\right) + \\
		& + u \gone{\p{z}(\chi)}\,.
	\end{align*}
	\item Computes full batched polynomial commitment $\gone{F}$:
	\begin{align*}
      \gone{F} & = \left(\gone{\p{t_{lo}}(\chi)} + \chz^\noofc \gone{\p{t_{mid}}(\chi)} + \chz^{2 \noofc} \gone{\p{t_{hi}}(\chi)}\right) + u \gone{\p{z}(\chi)} + \\
               & + v
                 \left(
		\begin{aligned}
			& \p{a}(\chz)\p{b}(\chz) \cdot \gone{\selmulti} + \p{a}(\chz)  \gone{\selleft} + \p{b}(\chz)   \gone{\selright} + \p{c}(\chz)  \gone{\seloutput} + \\
			& + (	(\p{a}(\chz) + \beta \chz + \gamma) (\p{b}(\chz) + \beta k_1 \chz + \gamma) (\p{c}(\chz)  + \beta k_2 \chz + \gamma) \alpha  + \lag_1(\chz) \alpha^2)  + \\
			% &   \\
			& - (\p{a}(\chz) + \beta \p{S_{\sigma 1}}(\chz) + \gamma) (\p{b}(\chz) + \beta \p{S_{\sigma 2}}(\chz) + \gamma) \alpha  \beta \p{z}(\chz \omega) \gone{\p{S_{\sigma 3}}(\chi)})
		\end{aligned}
		\right) \\
		& + v^2 \gone{\p{a}(\chi)} + v^3 \gone{\p{b}(\chi)} + v^4 \gone{\p{c}(\chi)} + v^5 \gone{\p{S_{\sigma 1}(\chi)}} + v^6 \gone{\p{S_{\sigma 2}}(\chi)}\,.
	\end{align*}
	\item Compute group-encoded batch evaluation $\gone{E}$
	\begin{align*}
		\gone{E}  = \frac{1}{\p{Z_H}(\chz)} & \gone{
		\begin{aligned}
			& \p{r}(\chz) + \pubinppoly(\chz) +  \alpha^2  \lag_1 (\chz) + \\
			& - \alpha \left( (\p{a}(\chz) + \beta \p{S_{\sigma 1}} (\chz) + \gamma) (\p{b}(\chz) + \beta \p{S_{\sigma 2}} (\chz) + \gamma) (\p{c}(\chz) + \gamma) \p{z}(\chz \omega) \right)
		\end{aligned}
           }\\
      + & \gone{v \p{r}(\chz) + v^2 \p{a}(\chz) + v^3 \p{b}(\chz) + v^4 \p{c}(\chz) + v^5 \p{S_{\sigma 1}}(\chz) + v^6 \p{S_{\sigma 2}}(\chz) + u \p{z}(\chz \omega) }\,.
	\end{align*}
\item Check whether the verification
 % $\vereq_\zkproof(\chi)$
  equation holds
	\begin{multline}
		\label{eq:ver_eq}
		\left( \gone{\p{W_{\chz}}(\chi)} + u \cdot \gone{\p{W_{\chz
                \omega}}(\chi)} \right) \bullet
		\gtwo{\chi} - %\\
		\left( \chz \cdot \gone{\p{W_{\chz}}(\chi)} + u \chz \omega \cdot
          \gone{\p{W_{\chz \omega}}(\chi)} + \gone{F} - \gone{E} \right) \bullet
        \gtwo{1} = 0\,.
	\end{multline}
  The verification equation is a batched version of the verification equation
  from \cite{AC:KatZavGol10} which allows the verifier to check openings of
  multiple polynomials in two points (instead of checking an opening of a single
  polynomial at one point).
\end{enumerate}

\ncase{$\plonk$ simulator $\simulator_\chi(\srs, \td= \chi, \inp)$}\ \newline
The \plonk{} simulator proceeds as an honest prover would, except:
\begin{enumerate}
  \item In the first message, it sets $\wit = (\wit_i)_{i \in \range{1}{3 \noofc}}
    = \vec{0}$, and at random picks $b_1, \ldots, b_9$. Then it proceeds with
    that all-zero witness.
  \item In prover's $3$-rd message, it computes polynomial $\pt(X)$ honestly, however uses
    trapdoor $\chi$ to compute commitments
    $\p{t_{lo}}(\chi), \p{t_{mid}}(\chi), \p{t_{hi}}(\chi)$.
  \end{enumerate}
 
\subsection{Unique response property}
\begin{lemma}
	\label{lem:plonkprot_ur}
  Let $\PCOMp$ be commitment of knowledge with security $\epsk(\secpar)$,
  $\epsbind(\secpar)$-binding and has unique opening property with security
  $\epsop(\secpar)$, then probability that a $\ppt$ adversary $\adv$ breaks
  $\plonkprotfs$'s $\ur{2}$ property is at most
  $\epsop + 9 \cdot (\epsbind + \infrac{2}{\FF_p}) + \epss + \epsro$, where
  $\epsro$ is probability that a $\ppt$ adversary finds collision in a random
  oracle.
\end{lemma}
\begin{proof}
  Let
  $\adv$
  be an algebraic adversary tasked to break the $\ur{2}$-ness of
  $\plonkprotfs$. We show that the first 2 prover's messages determines,
  along with the verifiers challenges, the rest of it.  This is done by game
  hops. In the games, the adversary outputs two proofs $\zkproof$ and
  $\zkproof'$ for the same statement.  To distinguish polynomials and
  commitments which an honest prover sends in the proof from the polynomials and
  commitments computed by the adversary we write the latter using indices $0$
  and $1$ (two indices as we have two transcripts), e.g.~to describe the
  quotient polynomial provided by the adversary we write $\p{t}^0$ and $\p{t}^1$
  instead of $\p{t}$ as in the description of the protocol.

  \ngame{0} In this game, the adversary wins if provides two
  transcripts that match on all $5$ messages sent by the prover or finds a
  collision in the random oracle. Since such two transcripts cannot break the
  unique response property, the adversary wins this game with probability
  $\epsro$ tops.

  \ngame{1} This game is identical to Game $\game{0}$ except that now the
  adversary additionally wins if it provides two transcripts that matches on the first four
  messages of the proof.

  \ncase{Game 0 to Game 1} We show that the probability that $\adv$
  wins in one game but does not in the other is negligible.  Observe that in
  after its $4$-th message, the adversary is given a challenge $v$ and has to open
  the previously computed commitments. Since the transcripts match up to $\adv$'s 
  $4$-th message, the challenge is the same in both. Hence, to be able to give two different
  openings in its $5$-th message, $\adv$ has to break the unique opening property of the
  KZG commitment scheme which happens with probability $\epsop$ tops.
  % Since
  % there are two commitments that the adversary opens, by the union bound
  % probability that $\adv$ wins in one game but not the other is upper-bounded
  % by
  % $2 \cdot \epsop$.

  \ngame{2} This game is identical to Game $\game{1}$ except that now the
  adversary additionally wins if it provides two transcripts that matches on the
  first three messages of the proof.

  \ncase{Game 0 to Game 1} In its $4$-th message the adversary
  has to provide evaluations
  $a_\chz = \p{a}(\chz), b_\chz = \p{b}(\chz), c_\chz = \p{c}(\chz), t_\chz =
  \p{t}(\chz), S_{1, \chz} = \p{S_{\sigma 1}}(\chz), s_{2, \chz} = \p{S_{\sigma
      2}}(\chz), z_\chz = \p{z}(\chz \omega)$ of previously committed
  polynomials, and compute and evaluate a linearization polynomial $\p{r}$.

  As before, the adversary cannot provide two different evaluations for the
  committed polynomials, since that would require breaking the evaluation
  binding property, which happens (by the union bound) with probability at most
  $7 \cdot (\epsbind + \infrac{2}{\abs{\FF_p}})$. The latter terms are since
  the adversary does not provide an opening for each of the commitment
  separately, but only in a batched way. That comes with $\infrac{1}{\FF_p}$ of
  security loss. Another $\infrac{1}{\FF_p}$ security loss comes from the fact
  that the verification of commitment openings are batched as well.

  The adversary cannot also provide two different linearization polynomials
  $\p{r^0}$ and $\p{r^1}$ evaluations $r^0_\chz$ and $r^1_\chz$ as the
  linearization polynomial is determined by values known to the verifier who
  also can compute a commitment to $\p{r}(X)$ equal $\gone{\p{r}(\chi)}$ by its
  own. The evaluation of $\p{r}$ provided by the adversary is later checked, as
  $\adv$ opens the commitment in its $5$-th message. Hence, the probability that the
  adversary manages to build two convincing proofs that differ in evaluations
  $r_\chz$ and $r'_\chz$ is at most $\epsbind + \infrac{2}{\abs{\FF_p}}$.

  Hence, the probability that adversary wins in one game but does not in the
  other is upper-bounded by $8 \cdot (\epsbind + \infrac{2}{\FF_p})$

  \ngame{3} This game is identical to Game $\game{2}$ except that now the
  adversary additionally wins if it provides two transcripts that matches on the
  first two messages of the proof.

  \ncase{Game 2 to Game 3} In its $3$-rd message the adversary computes the
  quotient polynomial $\pt(X)$ and provides its commitment that compounds of
  three separate commitments
  $\gone{\p{t_{lo}}(\chi), \p{t_{mid}}(\chi), \p{t_{hi}}(\chi)}$. Let
  $\gone{\p{t^0_{lo}}(\chi), \p{t^0_{mid}}(\chi), \p{t^0_{hi}}(\chi)}$ be the
  commitments output by the adversary in one transcript, and
  $\gone{\p{t^1_{lo}}(\chi), \p{t^1_{mid}}(\chi), \p{t^1_{hi}}(\chi)}$ the commitments
  provided in the other.
%
  Since the commitment scheme is deterministic, the adversary cannot come up
  with two different valid commitments for the same polynomial.

  If the adversary picks two different polynomials: $\p{t^0}(X)$, that is committed
  as $\gone{\p{t^0_{lo}}(\chi), \p{t^0_{mid}}(\chi), \p{t^0_{hi}}(\chi)}$, and
  $\p{t^1}(X)$ that is committed as
  $\gone{\p{t^1_{lo}}(\chi), \p{t^1_{mid}}(\chi), \p{t^1_{hi}}(\chi)}$, then one of
  them has to be computed incorrectly. 

  Importantly, polynomial $\p{t}(X)$ assures that the constraints of the system
  hold. Hence, the probability that one of $\p{t^0}(X)$, $\p{t^1}(X)$ is computed
  incorrectly, the adversary gives and opens acceptably a commitment to it, and
  the proof is acceptable, is upper bounded by the soundness of the proof system
  $\epss$. Alternatively, $\adv$ may compute a commitment to an invalid
  $\p{t^0}(X)$ (or $\p{t^1}(X)$) and later open the commitment at $\chz$ to
  $\p{t}(\chz)$. That is, give an evaluation from the correct polynomial
  $\p{t}(X)$. Since the commitment scheme is evaluation binding, probability of
  such event is upper bounded by $\epsbinding + \infrac{2}{\abs{\FF_p}}$.

  \ncase{Conclusion} Taking all the games together, probability that $\adv$ wins
  in Game 3 is upper-bounded by
  \[
    2 \cdot \epsop + 9 \cdot (\epsbind + \infrac{2}{\FF_p}) + \epsro + \epss.
  \]
  \qed
\end{proof}

\subsection{Forking special soundness}
\begin{lemma}
	\label{lem:plonkprot_ss}
	Let $\plonkprot$'s idealized verifier fails with probability $\epsid (\secpar)$, and
	$(\noofc + 2, 1)$-$\dlog$ problem be $\epsdlog (\secpar)$ hard. Then $\plonkprot$ is
	$(\epsid (\secpar) + \epsdlog (\secpar) , 3, 3 \noofc + 1)$-forking special sound against algebraic
	adversary $\adv$.
\end{lemma}

\begin{proof}
	The main idea of the proof is to show that an adversary who breaks forking special
	soundness can be used to break a $\dlog$ problem instance. The proof goes by game hops. Let $\tree$
	be the tree produced by $\tdv$ by rewinding $\adv$. Note that since the tree
	branches after $\adv$'s $3$-rd message, the instance $\inp$, commitments
	$\gone{\p{a} (\chi), \p{b} (\chi), \p{c} (\chi), \p{z} (\chi), \p{t_{lo}}
		(\chi), \p{t_{mid}} (\chi), \p{t_{hi}} (\chi)}$, and challenges
	$\alpha, \beta, \gamma$ are the same. The tree branches after the third adversary's
  message where the challenge $\chz$ is presented, thus tree $\tree$ is
	build using different values of $\chz$. 
	%
	We consider the following games.
	
	\ncase{Game 0} In this game the adversary wins if
	% \begin{inparaenum}[(1)]
	% \item
	all the transcripts it produced are acceptable by the ideal verifier,
	i.e.~$\vereq_{\inp, \zkproof}(X) = 0$, cf.~\cref{eq:ver_eq}, and
	% \item
	none of commitments
	$\gone{\p{a} (\chi), \p{b} (\chi), \p{c} (\chi), \p{z} (\chi), \p{t_{lo}}
		(\chi), \p{t_{mid}} (\chi), \p{t_{hi}} (\chi)}$ use elements from a
	simulated proof, and
	% \item
	the extractor fails to extract a valid witness out of the proof.
	%\end{inparaenum}
	
	\ncase{Probability that $\adv$ wins Game 0 is negligible} Probability of $\adv$
  winning this game is $\epsid(\secpar)$ as the protocol $\plonkprot$, instantiated
  with the idealised verification equation, is perfectly knowledge sound except with
  negligible probability of the idealised verifier failure $\epsid(\secpar)$. Hence
  for a valid proof $\zkproof$ for a statement $\inp$ there exists a witness $\wit$,
  such that $\REL(\inp, \wit)$ holds. Note that since the $\tdv$ produces
  $(3 \noofc + 1)$ acceptable transcripts for different challenges $\chz$, it obtains
  the same number of different evaluations of polynomials
  $\p{a} (X), \p{b} (X), \p{c} (X), \p{z} (X), \p{t} (X)$. Since the transcripts are
  acceptable by an idealised verifier, the equality between polynomial $\p{t} (X)$
  and combination of polynomials $\p{a} (X), \p{b} (X), \p{c} (X), \p{z} (X)$ defined
  in prover's $3$-rd message description holds. Hence,
  $\p{a} (X), \p{b} (X), \p{c} (X)$ encodes the valid witness for the proven
  statement. Since $\p{a} (X), \p{b} (X), \p{c} (X)$ are of degree at most
  $(\noofc + 2)$ and there is more than $(\noofc + 2)$ their evaluations known,
  $\extt$ can recreate polynomials' coefficients by interpolation and reveal the
  witness with probability $1$. Hence, the probability that extraction fails in that
  case is upper-bounded by probability of an idealised verifier failing
  $\epsid(\secpar)$, which is negligible.
	
	\ncase{Game 1} In this game the adversary additionally wins if
	%\begin{inparaenum}
	% \item
	it produces a transcript in $\tree$ such that
	$\vereq_{\inp, \zkproof}(\chi) = 0$, but $\vereq_{\inp, \zkproof}(X) \neq 0$,
	and
	% \item
	none of commitments
	$\gone{\p{a} (\chi), \p{b} (\chi), \p{c} (\chi), \p{z} (\chi), \p{t_{lo}}
		(\chi), \p{t_{mid}} (\chi), \p{t_{hi}} (\chi)}$ use elements from a
	simulated proof.
	% \end{inparaenum}
	The first condition means that the ideal verifier does not accept the proof,
	but the real verifier does.
	
	\ncase{Game 0 to Game 1} Assume the adversary wins in Game 1, but
	does not win in Game 0. We show that such adversary may be used to break the
	$\dlog$ assumption. More precisely, let $\tdv$ be an algorithm that for
	relation $\REL$ and randomly picked $\srs \sample \kgen(\REL)$ produces a tree
	of acceptable transcripts such that the winning condition of the game
	holds. Let $\rdvdlog$ be a reduction that gets as input an
	$(\noofc + 2, 1)$-dlog instance $\gone{1, \ldots, \chi^{\noofc+2}}, \gtwo{\chi}$ and is tasked to output $\chi$.
	
	The reduction $\rdvdlog$ proceeds as follows.
	\begin{enumerate}
			\item Build $\adv$'s SRS $\srs$ in the updatable setting using the input $\dlog$ instance by answering $\adv$'s queries for SRS updates and setting the honest update of the SRS to be $\srs$. Let $\srs'$ be the finalised SRS. Start $\tdv(\adv, \srs')$;
			\item Let $(1, \tree)$ be the output returned by $\tdv$. Let $\inp$ be a
			relation proven in $\tree$.  Consider a transcript $\zkproof \in \tree$ such
			that $\vereq_{\inp, \zkproof}(X) \neq 0$, but
			$\vereq_{\inp, \zkproof}(\chi') = 0$. Since $\adv$ is algebraic, all group
			elements included in $\tree$ are extended by their representation as a
			combination of the input $\GRP_1$-elements. Hence, all coefficients of the
			verification equation polynomial $\vereq_{\inp, \zkproof}(X)$ are known.
			\item Find $\vereq_{\inp, \zkproof}(X)$ zero points and find $\chi'$ among
			them.
			\item Let $\chi_1, \ldots, \chi_\ell$ be the partial trapdoors of $\adv$'s SRS updates. These trapdoors can be extracted by the reduction from the update proofs given by $\adv$.
			\item Return  $\chi = \chi' (\chi_1 \chi_2 \ldots \chi_\ell)^{-1}$.
      \end{enumerate}

      \ncase{Conclusion} Considering all the games together, probability that adversary wins in
      Game 1 is upper-bounded by Hence, the probability that the adversary wins Game 1 is
      upper-bounded by $\epsdlog(\secpar) = \epsid (\secpar)$.
      \qed
\end{proof}

\iffalse
\begin{proof}
	Let $\srs$ be $\plonkprot$'s SRS and denote by $\srs_1$ all SRS's
  $\GRP_1$-elements; that is, $\srs_1 = \gone{1, \chi, \ldots, \chi^{\noofc +
      2}}$. Let $\tdv$ be an algebraic adversary that produces a statement $\inp$ and
  a $(1, 1, 3\noofc + 1, 1)$-tree of acceptable transcripts $\tree$.  Note that in
  all transcripts the instance $\inp$, proof elements
  $\gone{\p{a}(\chi), \p{b}(\chi), \p{c}(\chi), \p{z}(\chi), \p{t}(\chi)}$ and
  challenges $\alpha, \beta, \gamma$ are common as the transcripts share the first
  three prover's messages. The tree branches after the third prover's message of the
  protocol where the challenge $\chz$ is presented, thus tree $\tree$ is build using
  different values of $\chz$.
	
	We consider two games.
	
	\ncase{Game 0} In this game the adversary wins if all the transcripts it
	produced are acceptable by the ideal verifier,
	i.e.~$\vereq_{\inp, \zkproof}(X) = 0$, cf.~\cref{eq:ver_eq}, yet the extractor
	fails to extract a valid witness out of them.
	
	Probability of $\tdv$ winning this game is $\epsid(\secpar)$ as the protocol
	$\plonkprot$, instantiated with the idealised verification equation, is
	perfectly sound except with negligible probability of the idealised verifier
	failure $\epsid(\secpar)$. Hence for a valid proof $\zkproof$ for a statement
	$\inp$ there exists a witness $\wit$, such that $\REL(\inp, \wit)$ holds. Note
	that since the $\tdv$ produces $(3 \noofc + 1)$ acceptable transcripts for
	different challenges $\chz$, it obtains the same number of different
	evaluations of polynomials $\p{a}, \p{b}, \p{c}, \p{z}, \p{t}$. Since the
	transcripts are acceptable by an idealised verifier, the equality between
	polynomial $\p{t}$ and combination of polynomials $\p{a}, \p{b}, \p{c}, \p{z}$
	defined as in $3$-rd prover's message holds. Hence, $\p{a}, \p{b}, \p{c}$
	encodes the valid witness for the proven statement. Since
	$\p{a}, \p{b}, \p{c}$ are of degree at most $(\noofc + 2)$ and there is more
	than $(\noofc + 2)$ their evaluations known, $\extt$ can recreate their
	coefficients by interpolation and reveal the witness with probability
	$1$. Hence, the probability that extraction fails in that case is
	upper-bounded by probability of an idealised verifier failing
	$\epsid(\secpar)$, which is negligible.
	
	\ncase{Game 1} In this game the adversary additionally wins if it produces a
	transcript in $\tree$ such that $\vereq_{\inp, \zkproof}(\chi) = 0$, but
	$\vereq_{\inp, \zkproof}(X) \neq 0$. That is, the ideal verifier does not
	accept the proof, but the real verifier does.
	
	\ncase{Game 0 to Game 1} Assume the adversary wins in Game 1, but
	does not win in Game 0. We show that such adversary may be used to break the
	$\dlog$ assumption. More precisely, let $\tdv$ be an adversary that for
	relation $\REL$ and randomly picked $\srs \sample \kgen(\REL)$ produces a tree
	of acceptable transcripts such that the winning condition of the game
	holds. Let $\rdvdlog$ be a reduction that gets as input an
	$(\noofc + 2, 1)$-dlog instance $\gone{1, \ldots, \chi^{\noofc}}, \gtwo{\chi}$
	and is tasked to output $\chi$. The reduction proceeds as follows---it gives
	the input instance to the adversary as the SRS. Let $(1, \tree)$ be the output
	returned by $\adv$. Let $\inp$ be a relation proven in $\tree$.  Consider a
	transcript $\zkproof \in \tree$ such that $\vereq_{\inp, \zkproof}(X) \neq 0$,
	but $\vereq_{\inp, \zkproof}(\chi) = 0$. Since the adversary is algebraic, all
	group elements included in $\tree$ are extended by their representation as a
	combination of the input $\GRP_1$-elements. Hence all coefficients of the
	verification equation polynomial $\vereq_{\inp, \zkproof}(X)$ are known and
	$\rdvdlog$ can find its zero points. Since
	$\vereq_{\inp, \zkproof}(\chi) = 0$, the targeted discrete log value $\chi$ is
	among them.  Hence, the probability that this event happens is upper-bounded
	by $\epsdlog(\secpar)$. \qed
\end{proof}
\fi

%\subsection{Forking soundness}
%\begin{lemma}
%\label{lem:plonkprot_ss}
%Let KZG be hiding with security $\epsh (\secpar)$, $\plonkprot$'s idealized
%verifier fail with probability $\epsid (\secpar)$, and $(\noofc + 2, 1)$-$\dlog$
%problem be $\epsdlog (\secpar)$ hard. Then $\plonkprot$ is
%$(\epsid (\secpar) + \epsdlog (\secpar) + 8 \cdot S \cdot \epsh(\secpar) +
%, 3, 3 \noofc + 1)$-forking sound against algebraic adversary
%$\adv$ who makes up to $S = \poly$ simulation oracle queries.\hamid{In the new definition, $\adv$ does not have access to the simulation oracle; so this should be changed!}
%\end{lemma}
%
%\begin{proof}
%  The main idea of the proof is to show that an adversary who breaks forking
%  soundness can be used to break hiding properties of the polynomial commitment
%  scheme or a $\dlog$ problem instance. The proof goes by game hops. Let $\tree$
%  be the tree produced by $\tdv$ by rewinding $\adv$. Note that since the tree
%  branches after Round 3, the instance $\inp$, commitments
%  $\gone{\p{a} (\chi), \p{b} (\chi), \p{c} (\chi), \p{z} (\chi), \p{t_{lo}}
%    (\chi), \p{t_{mid}} (\chi), \p{t_{hi}} (\chi)}$, and challenges
%  $\alpha, \beta, \gamma$ are the same. The tree branches after the third round
%  of the protocol where the challenge $\chz$ is presented, thus tree $\tree$ is
%  built using different values of $\chz$. 
%%
%  We consider the following games.
%
%  \ncase{Game 0} In this game the adversary wins if
% % \begin{inparaenum}[(1)]
% % \item
%  all the transcripts it produced are acceptable by the ideal verifier,
%    i.e.~$\vereq_{\inp, \zkproof}(X) = 0$, cf.~\cref{eq:ver_eq}, and
%    % \item
%    none of commitments
%    $\gone{\p{a} (\chi), \p{b} (\chi), \p{c} (\chi), \p{z} (\chi), \p{t_{lo}}
%      (\chi), \p{t_{mid}} (\chi), \p{t_{hi}} (\chi)}$ use elements from a
%    simulated proof, and
%    % \item
%    the extractor fails to extract a valid witness out of the proof.
%  %\end{inparaenum}
%
%  \ncase{Probability that $\adv$ wins Game 0 is negligible} Probability of
%  $\adv$ winning this game is $\epsid(\secpar)$ as the protocol $\plonkprot$,
%  instantiated with the idealised verification equation, is perfectly knowledge
%  sound except with negligible probability of the idealised verifier failure
%  $\epsid(\secpar)$. Hence for a valid proof $\zkproof$ for a statement $\inp$
%  there exists a witness $\wit$, such that $\REL(\inp, \wit)$ holds. Note that
%  since the $\tdv$ produces $(3 \noofc + 1)$ acceptable transcripts for
%  different challenges $\chz$, it obtains the same number of different
%  evaluations of polynomials
%  $\p{a} (X), \p{b} (X), \p{c} (X), \p{z} (X), \p{t} (X)$. Since the transcripts
%  are acceptable by an idealised verifier, the equality between polynomial
%  $\p{t} (X)$ and combination of polynomials
%  $\p{a} (X), \p{b} (X), \p{c} (X), \p{z} (X)$ described in Round 3 of the
%  protocol holds. Hence, $\p{a} (X), \p{b} (X), \p{c} (X)$ encodes the valid
%  witness for the proven statement. Since $\p{a} (X), \p{b} (X), \p{c} (X)$ are
%  of degree at most $(\noofc + 2)$ and there is more than $(\noofc + 2)$ their
%  evaluations known, $\extt$ can recreate polynomials' coefficients by interpolation
%  and reveal the witness with probability $1$. Hence, the probability that
%  extraction fails in that case is upper-bounded by probability of an idealised
%  verifier failing $\epsid(\secpar)$, which is negligible.
%
%  \ncase{Game 1} In this game the adversary additionally wins if
%  %\begin{inparaenum}
%  % \item
%  it produces a transcript in $\tree$ such that
%  $\vereq_{\inp, \zkproof}(\chi) = 0$, but $\vereq_{\inp, \zkproof}(X) \neq 0$,
%  and
%  % \item
%  none of commitments
%  $\gone{\p{a} (\chi), \p{b} (\chi), \p{c} (\chi), \p{z} (\chi), \p{t_{lo}}
%    (\chi), \p{t_{mid}} (\chi), \p{t_{hi}} (\chi)}$ use elements from a
%  simulated proof.
%  % \end{inparaenum}
%  The first condition means that the ideal verifier does not accept the proof,
%  but the real verifier does.
%
%  \ncase{Game 0 to Game 1} Assume the adversary wins in Game 1, but
%  does not win in Game 0. We show that such adversary may be used to break the
%  $\dlog$ assumption. More precisely, let $\tdv$ be an algorithm that for
%  relation $\REL$ and randomly picked $\srs \sample \kgen(\REL)$ produces a tree
%  of acceptable transcripts such that the winning condition of the game
%  holds. Let $\rdvdlog$ be a reduction that gets as input an
%  $(\noofc + 2, 1)$-dlog instance $\gone{1, \ldots, \chi^{\noofc}}, \gtwo{\chi}$ \hamid{shouldn't be $\gone{1, \ldots, \chi^{\noofc+2}}, \gtwo{\chi}$}
%  and is tasked to output $\chi$.
%
%  The reduction $\rdvdlog$ proceeds as follows.
%  \hamid{\begin{enumerate}
%  \item Build a SRS $\srs$ using the input $\dlog$ instance. Answer $\adv$'s queries for SRS updates and set the honest update of the SRS to be $\srs$. Let $\srs'$ be the finalised SRS. Start $\tdv(\adv, \srs')$;
%  \item Let $(1, \tree)$ be the output returned by $\tdv$. Let $\inp$ be a
%    relation proven in $\tree$.  Consider a transcript $\zkproof \in \tree$ such
%    that $\vereq_{\inp, \zkproof}(X) \neq 0$, but
%    $\vereq_{\inp, \zkproof}(\chi') = 0$. Since $\adv$ is algebraic, all group
%    elements included in $\tree$ are extended by their representation as a
%    combination of the input $\GRP_1$-elements. Hence, all coefficients of the
%    verification equation polynomial $\vereq_{\inp, \zkproof}(X)$ are known.
%  \item Find $\vereq_{\inp, \zkproof}(X)$ zero points and find $\chi'$ among
%    them.
%  \item Let $\chi_1, \ldots, \chi_\ell$ be the partial trapdoors of $\adv$'s SRS updates. These trapdoors can be extracted by the reduction from the update proofs given by $\adv$.
%  \item Return  $\chi = \chi' (\chi_1 \chi_2 \ldots \chi_\ell)^{-1}$.
%  \end{enumerate}}
%  Hence, the probability that the adversary wins Game 1 is upper-bounded by
%  $\epsdlog(\secpar)$.
%
%  \iffalse
%  \ncase{Game 0} In this game the adversary $\adv$ wins if $\tdv$ fails to
%  output tree $\tree$ such that $\extt$ is able to extract a witness out of it
%  and none of commitments
%  $\gone{\p{a} (\chi), \p{b} (\chi), \p{c} (\chi), \p{z} (\chi), \p{t_{lo}}
%    (\chi), \p{t_{mid}} (\chi), \p{t_{hi}} (\chi)}$ use elements from a
%  simulated proof.
%
%  \ncase{Probability that $\adv$ wins Game 0 is negligible}
%  \michals{10.09}{Reduction to knowledge soundness}
%  Let $\adv$ be considered forking soundness adversary. 
%  Let $\adv$ be an adversary that breaks forking soundness.
%  Note that if
%  $\p{a}(X), \p{b} (X), \p{c} (X)$ contain witness at their coefficients, then 
%  We use $\adv$ to
%  build a reduction $\rdv$ that breaks soundness of $\plonkprotfs$. Let $\inp$
%  be the instance the adversary output and $\zkproof$ its proof. The
%  reduction proceeds as follows:
%  \begin{enumerate}
%  \item 
%  \end{enumerate}
%  \fi
%
%  \ncase{Game 2} In this game the adversary additionally wins if at least one of
%  the commitments $\p{a} (\chi), \p{b} (\chi), \p{c} (\chi), \p{z} (\chi)$
%  utilizes a commitment that comes from a simulated proof; for example, $\adv$
%  could compute its commitment to $\p{c} (X)$ as follows: it picks a polynomial
%  $\p{p} (X)$, computes $\gone{\p{p} (\chi)}$, and outputs commitment
%  $\gone{\p{c} (\chi)} = \gone{\p{p} (\chi)} + c$, where $c$ is a commitment
%  output by a simulator. In the following, w.l.o.g, we assume that $\adv$ uses
%  some simulated element to compute commitment $\gone{\p{c} (\chi)}$.
%
%  \ncase{Game 1 to Game 2} Given adversary $\adv$ that wins in Game 2, but not
%  in Game 1, we show a reduction $\rdv$ that uses $\adv$ and $\tdv$ to break
%  hiding property
%  % \michals{9.9}{Define hiding property --
%  % w.r.t.~masking, adversary can get evaluation at number of points (here --
%  % 1)}
%  of the commitment scheme. $\rdv$ proceeds as follows:
%  \begin{enumerate}
%  \item \hamid{Given polynomial commitment SRS $\srs_{\PCOM}$, produce $\plonk$'s SRS
%    $\srs$.}
%  \item Pick random polynomials $\p{p} (X), \p{p'} (X) \in \FF^{< |\HHH|} [X]$,
%    hiding parameter $k = 2$ and send them to the polynomial commitment
%    challenger $\cdv$.
%  \item From the challenger get the challenge commitment $c$.
%  \item Let $S$ be the upper bound on the number of simulator oracles queries
%    the adversary can make. %\michals{9.9}{Note -- new bound!}
%  \item Guess which simulator's response is going to be used by $\adv$ in its proof. Let $s$ be the index of this response.\hamid{should be changed!}
%  \item Guess which of the simulated polynomials in response $s$ will be
%    used. Let $i$ be the index of this polynomial.\hamid{should be changed!}
%  \item Let $\tdv'$ be an algorithm that behaves exactly as $\tdv$, except when
%    $\adv$ asks for $s$-th simulated proof, $\tdv'$'s internal procedure $\bdv'$
%    provides $\adv$ with a simulated proof such that instead of randomly picked
%    commitment $\p{c} (\chi)$ it gives $c$.
%    \michals{9.9}{Alternatively, we can parametrize $\tdv$ by $\bdv$.}
%  \item Start $\tdv'(\adv, \srs)$ and get the tree $\tree$.
%  % \item If indices $s$ or $i$ have not been guessed correctly, rewind $\adv$ to
%  %   the beginning and pick new $s$ and $i$. Since $S = \poly$ probability that
%  %   the correct $s$ will be guessed in polynomial time is overwhelming. That is,
%  %   the reduction works in expected polynomial time. Similarly, $i$ takes values
%  %   from $\range{1}{4}$, hence probability that $\rdv$ guesses $i$ in polynomial
%  %   time is overwhelming. 
%  \item Since $\tree$ contains $\noofc + 1$ evaluations of $\p{c} (X)$, the
%    polynomial can be reconstructed. 
%  \item Since $\adv$ is algebraic, $\rdv$ learns composition of $\p{c} (X)$ in
%    the $\srs$ and simulated elements. 
%  \item Hence $\rdv$ learns whether $c$ is a commitment to $\p{p} (X)$ or
%    $\p{p'} (X)$.
%  \item $\rdv$ returns its guessing bit to $\cdv$.
%  \end{enumerate}
%
%  \ncase{Game 3} In this game the adversary additionally wins if at least one of
%  the commitments $\p{t_{lo}} (\chi), \p{t_{mid}} (\chi), \p{t_{hi}} (\chi)$
%  comes from a simulated proof.
%
%  \ncase{Game 2 to Game 3} Given adversary $\adv$ that wins in Game
%  3, but not in Game 2, we show a reduction $\rdv$ that uses $\adv$ and $\tdv$
%  to break hiding property
%  % \michals{9.9}{Define hiding property --
%  %   w.r.t.~masking, adversary can get evaluation at number of points (here --
%  %   1)}
%  of the commitment scheme. $\rdv$ proceeds as follows:
%  \begin{enumerate}
%  \item Guess the simulation query index $s$ the polynomial(s) come from and
%    whether the polynomial is $\p{t_{lo}} (X), \p{t_{mid}} (X)$, or
%    $\p{t_{hi}} (X)$. Denote by $i \in \range{1}{3}$ the index of the guessed
%    polynomial. W.l.o.g.~assume $i = 1$, i.e.~it is $\p{t_{lo}} (X)$.
%  \item Produce two random polynomials $\p{p_0} (X)$ and $\p{p_1} (X)$ and
%    send them to the challenger $\cdv$. Get commitment $c$.
%  \item Let $\tdv'$ be an algorithm that behaves exactly as $\tdv$, except when
%    $\adv$ asks for $s$-th simulated proof, $\tdv'$'s internal procedure $\bdv'$
%    provides $\adv$ with a simulated proof such that:
%    \begin{enumerate}
%    \item Start making the simulated proof as a trapdoor-less simulator would.
%    \item Before polynomials $\p{t_{lo}} (X), \p{t_{mid}}, \p{t_{hi}}$ are
%      computed, pick random $\chz$ and get evaluation $p_\chz = \p{p_b} (\chz)$,
%      i.e.~evaluate the polynomial in $c$ at $\chz$.
%      % \michals{9.9}{If we give
%      %   hiding adversary possibility to evaluate polynomials, we need to use
%      %   masking}
%    \item Let $\ev{t_{lo}}$ be the evaluation of the simulated $\p{t_{lo}} (\chz)$.
%    \item Pick $r$ such that $p_\chz + r = \p{t_{lo}} (\chz)$.
%    \item For the commitment of $\p{t_{lo}}$ output $c' = c + \gone{r}$.
%    \item Compute the rest of the simulated proof as a simulator would.
%    \end{enumerate}
%  \item Let $\tdv'$ be an algorithm that behaves exactly as $\tdv$, except when
%    $\adv$ asks for $s$-th simulated proof, $\tdv'$'s internal procedure $\bdv'$
%    provides $\adv$ with a simulated proof such that instead of a simulated
%    $\p{t_{lo}} (\chi)$  it gives $c'$.
%  \item Start $\tdv'(\adv, \srs)$ and get the tree $\tree$.
%  % \item If indices $s$ or $i$ have not been guessed correctly, rewind $\adv$ to
%  %   the beginning and pick new $s$ and $i$. Since $S = \poly$ probability that
%  %   the correct $s$ will be guessed in polynomial time is overwhelming. That is,
%  %   the reduction works in expected polynomial time. Similarly, $i$ takes values
%  %   from $\range{1}{3}$, hence probability that $\rdv$ guesses $i$ in polynomial
%  %   time is overwhelming. 
%  \item Since $\tree$ contains $\noofc + 1$ evaluations of $\p{t_{lo}} (X)$, the
%    polynomial can be reconstructed. 
%  \item Since $\adv$ is algebraic, $\rdv$ learns composition of $\p{t_{lo}} (X)$ in
%    the $\srs$ and simulated elements. 
%  \item Hence $\rdv$ learns whether $c$ is a commitment to $\p{p}$ or $\p{p'}$.
%  \item $\rdv$ returns its guessing bit to $\cdv$.
%  \end{enumerate}
%  % \ncase{Conclusion}
%  % Since probability 
%\end{proof}
%
%\iffalse
%\begin{proof}
%  Let $\srs$ be $\plonkprot$'s SRS and denote by $\srs_1$ all SRS's
%  $\GRP_1$-elements; that is,
%  $\srs_1 = \gone{1, \chi, \ldots, \chi^{\noofc + 2}}$. Let $\tdv$ be an
%  algebraic adversary that produces a statement $\inp$ and a
%  $(1, 1, 3\noofc + 1, 1)$-tree of acceptable transcripts $\tree$.  Note that in
%  all transcripts the instance $\inp$, proof elements
%  $\gone{\p{a}(\chi), \p{b}(\chi), \p{c}(\chi), \p{z}(\chi), \p{t}(\chi)}$ and
%  challenges $\alpha, \beta, \gamma$ are common as the transcripts share the
%  first three rounds. The tree branches after the third round of the protocol
%  where the challenge $\chz$ is presented, thus tree $\tree$ is build using
%  different values of $\chz$.
%
%  We consider two games.
%
%  \ncase{Game 0} In this game the adversary wins if all the transcripts it
%  produced are acceptable by the ideal verifier,
%  i.e.~$\vereq_{\inp, \zkproof}(X) = 0$, cf.~\cref{eq:ver_eq}, yet the extractor
%  fails to extract a valid witness out of them.
%
%  Probability of $\tdv$ winning this game is $\epsid(\secpar)$ as the protocol
%  $\plonkprot$, instantiated with the idealised verification equation, is
%  perfectly sound except with negligible probability of the idealised verifier
%  failure $\epsid(\secpar)$. Hence for a valid proof $\zkproof$ for a statement
%  $\inp$ there exists a witness $\wit$, such that $\REL(\inp, \wit)$ holds. Note
%  that since the $\tdv$ produces $(3 \noofc + 1)$ acceptable transcripts for
%  different challenges $\chz$, it obtains the same number of different
%  evaluations of polynomials $\p{a}, \p{b}, \p{c}, \p{z}, \p{t}$. Since the
%  transcripts are acceptable by an idealised verifier, the equality between
%  polynomial $\p{t}$ and combination of polynomials $\p{a}, \p{b}, \p{c}, \p{z}$
%  described in Round 3 of the protocol holds. Hence, $\p{a}, \p{b}, \p{c}$
%  encodes the valid witness for the proven statement. Since
%  $\p{a}, \p{b}, \p{c}$ are of degree at most $(\noofc + 2)$ and there is more
%  than $(\noofc + 2)$ their evaluations known, $\extt$ can recreate their
%  coefficients by interpolation and reveal the witness with probability
%  $1$. Hence, the probability that extraction fails in that case is
%  upper-bounded by probability of an idealised verifier failing
%  $\epsid(\secpar)$, which is negligible.
%
%  \ncase{Game 1} In this game the adversary additionally wins if it produces a
%  transcript in $\tree$ such that $\vereq_{\inp, \zkproof}(\chi) = 0$, but
%  $\vereq_{\inp, \zkproof}(X) \neq 0$. That is, the ideal verifier does not
%  accept the proof, but the real verifier does.
%
%  \ncase{Game 0 to Game 1} Assume the adversary wins in Game 1, but
%  does not win in Game 0. We show that such adversary may be used to break the
%  $\dlog$ assumption. More precisely, let $\tdv$ be an adversary that for
%  relation $\REL$ and randomly picked $\srs \sample \kgen(\REL)$ produces a tree
%  of acceptable transcripts such that the winning condition of the game
%  holds. Let $\rdvdlog$ be a reduction that gets as input an
%  $(\noofc + 2, 1)$-dlog instance $\gone{1, \ldots, \chi^{\noofc}}, \gtwo{\chi}$
%  and is tasked to output $\chi$. The reduction proceeds as follows---it gives
%  the input instance to the adversary as the SRS. Let $(1, \tree)$ be the output
%  returned by $\adv$. Let $\inp$ be a relation proven in $\tree$.  Consider a
%  transcript $\zkproof \in \tree$ such that $\vereq_{\inp, \zkproof}(X) \neq 0$,
%  but $\vereq_{\inp, \zkproof}(\chi) = 0$. Since the adversary is algebraic, all
%  group elements included in $\tree$ are extended by their representation as a
%  combination of the input $\GRP_1$-elements. Hence all coefficients of the
%  verification equation polynomial $\vereq_{\inp, \zkproof}(X)$ are known and
%  $\rdvdlog$ can find its zero points. Since
%  $\vereq_{\inp, \zkproof}(\chi) = 0$, the targeted discrete log value $\chi$ is
%  among them.  Hence, the probability that this event happens is upper-bounded
%  by $\epsdlog(\secpar)$. \qed
%\end{proof}
%\fi

\subsection{Trapdoor-less simulatability of Plonk}
\begin{lemma}
  \label{lem:plonk_hvzk}
  Let $\plonkprot$ be zero knowledge with security $\epszk(\secpar)$. Let
  $(\pR, \pS, \pT, \pf, 1)$-uber assumption for $\pR, \pS, \pT, \pf$ as defined in
  \cref{eq:uber} hold with security $\epsuber(\secpar)$. Then $\plonkprot$
  2-programmable trapdoor-less simulatable zero knowledge with simulator $\simulator$
  that does not require a SRS trapdoor with security
  $\epszk(\secpar) + \epsuber(\secpar)$.\footnote{The simulator works as a simulator
    for proofs that are zero-knowledge in the standard model. However, we do not say
    that $\plonk$ is HVZK in the standard model as proof of that \emph{requires} the
    SRS simulator.}
\end{lemma}

%Due to page limit, the proof has been moved to \cref{sec:plonk-tls-proof}
\begin{proof}
  As noted in \cref{def:upd-scheme}, subvertible zero knowledge implies updatable zero
  knowledge. Hence, here we show that Plonk is TLS even against algebraic adversaries who picks
  the SRS on its own.
  
  The proof goes by game-hopping. The environment that controls the games
  provides starts the adversary who sets a SRS $\srs$, then the adversary outputs an
  instance--witness pair $(\inp, \wit)$ and, depending on the game, is provided
  with either real or simulated proof for it. In the end of the game the
  adversary outputs either $0$ if it believes that the proof it saw was provided
  by the simulator and $1$ in the other case.

  \ngame{0} In this game $\adv(\secparam)$ picks an SRS $\srs$ and instance--witness pair
  $(\inp, \wit)$ and gets a real proof $\zkproof$ for it.

  \ngame{1} In this game for $\adv(\secparam)$ picks an SRS $\srs$ and an instance--witness pair
  $(\inp, \wit)$ and gets a proof $\zkproof$ that is simulated by a simulator
  $\simulator_\chi$ which utilises for the simulation the SRS trapdoor and
  proceeds as described in \cref{sec:plonk_explained}.

  \ncase{Game 0 to Game 1} Since $\plonk$ is (subvertible) zero-knowledge,
  probability that $\adv$ outputs a different bit in both games is negligible.
  Hence
  \(
	\abs{\prob{\game{0}} - \prob{\game{1}}} \leq \epszk(\secpar).
\)

\ngame{2} In this game $\adv(\secparam)$ picks an SRS $\srs$ and instance--witness pair
$(\inp, \wit)$ and gets a proof $\zkproof$ simulated by the simulator
$\simulator$ which proceeds as follows.

For its $1$-st message the simulator  picks randomly both the randomisers $b_1, \ldots, b_6$ and
sets $\wit_i = 0$ for $i \in \range{1}{3\noofc}$. Then $\simulator$
outputs $\gone{\p{a}(\chi), \p{b}(\chi), \p{c}(\chi)}$. For the first
challenge, the simulator picks permutation argument challenges $\beta, \gamma$
randomly.

For its $2$-nd message, the simulator computes $\p{z}(X)$ from
the newly picked randomisers $b_7, b_8, b_9$ and coefficients of polynomials
$\p{a}(X), \p{b}(X), \p{c}(X)$. Then it evaluates $\p{z}(X)$ honestly and outputs
$\gone{\p{z}(\chi)}$. Challenge $\alpha$ that should be sent by the verifier
after the simulator's $2$ message is picked by the simulator at random.

In its $3$-rd message the simulator starts by picking at random a challenge $\chz$, which
in the real proof comes as a challenge from the verifier sent \emph{after} the prover
sends its $3$-rd message. Then $\simulator$ computes evaluations
\(\p{a}(\chz), \p{b}(\chz), \p{c}(\chz), \p{S_{\sigma 1}}(\chz), \p{S_{\sigma
    2}}(\chz), \pubinppoly(\chz), \lag_1(\chz), \p{Z_H}(\chz),\allowbreak
\p{z}(\chz\omega)\) and computes $\p{t}(X)$ honestly. Since for a random
$\p{a}(X), \p{b}(X), \p{c}(X), \p{z}(X)$ the constraint system is (with
overwhelming probability) not satisfied and the constraints-related polynomials
are not divisible by $\p{Z_H}(X)$, hence $\p{t}(X)$ is a rational function
rather than a polynomial. Then, the simulator evaluates $\p{t}(X)$ at $\chz$ and
picks randomly a degree-$(3 \noofc - 1)$ polynomial $\p{\tilde{t}}(X)$ such that
$\p{t}(\chz) = \p{\tilde{t}}(\chz)$ and publishes a commitment
$\gone{\p{\tilde{t}_{lo}}(\chi), \p{\tilde{t}_{mid}}(\chi),
  \p{\tilde{t}_{hi}}(\chi)}$. After that the simulator outputs $\chz$ as a
challenge.

For the next message, the simulator computes polynomial $\p{r}(X)$ as an honest
prover would, cf.~\cref{sec:plonk_explained} and evaluates $\p{r}(X)$ at $\chz$.

The rest of the evaluations are already computed, thus $\simulator$ simply outputs
\( \p{a}(\chz), \p{b}(\chz), \p{c}(\chz), \p{S_{\sigma 1}}(\chz), \p{S_{\sigma
    2}}(\chz), \p{t}(\chz), \p{z}(\chz \omega)\,.  \) After that it picks randomly
the challenge $v$, and prepares the the last message as an honest prover
would. Eventually, $\simulator$ and outputs the final challenge, $u$, by picking it
at random as well.

\ncase{Game 1 to Game 2} We now describe the reduction $\rdv$ which
relies on the $(\pR, \pS, \pT, \pF, 1)$-uber assumption, cf.~\cref{sec:uber_assumption}
where $\pR, \pS, \pT, \pF$ are polynomials over variables
$\vB = B_1, \ldots, B_9$ and are defined as follows. Let
$E = \smallset{\smallset{2}, \smallset{3, 4}, \smallset{5, 6}, \smallset{7, 8,
    9}}$ and $E' = E \setminus \smallset{2}$. Let
\begin{align}
\label{eq:uber}
\pF(\vB) & = \smallset{B_1} \cup \smallset{B_1B_i \mid i \in A,\ A \in E'} \cup
             \smallset{B_1B_iB_j \mid i \in A, j \in B,\ A, B \in E', B
             \neq A} \cup \notag\\
           & \smallset{B_1B_iB_jB_k \mid i \in A, j \in
             B, k \in C,\ A, B, C \in E', A \neq B \neq C \neq A}\notag\,,\\
  \pR(\vB) & = \smallset{B_i \mid i \in A,\ A \in E} \cup \smallset{B_i B_j \mid i \in
             A, j \in B,\ A \neq B, A, B \in E} \cup \\ 
           & \smallset{B_i B_j B_k \mid i \in A,\ j \in
             B,\ k \in C,\
             A, B, C \text{ all different and in } E} \cup \notag \\
           & \smallset{B_i B_j B_k B_l \mid i \in A,\ j \in B,\ k \in C,\ l \in D,\
             A, B, C, D \text{ all different and in } E} \notag \\
           & \setminus \pF(\vB)\,,\notag \\
  \pS(\vB) & = \emptyset, \qquad \pT(\vB) = \emptyset.
\end{align}
That is, the elements of $\pR$ are all singletons, pairs, triplets and
quadruplets of $B_i$ variables that occur in polynomial $\pt(\vB)$ except the
challenge element $\pf(\vB)$ which are all elements that depends on a variable
$B_1$. Variables $\vB$ are evaluated to randomly picked
$\vb = b_1, \ldots, b_9$.

The reduction $\rdv$ learns $\gone{\pR}$ and challenge
$\gone{\vec{w}} = \gone{w_1, \ldots, w_{12}}$ where $\vec{w}$ is either a vector
of evaluations $\pF(\vb)$ or a sequence of random values $y_1, \ldots, y_{12}$,
for the sake of concreteness we state $w_1 = b_1$ or $w_1 = y_1$ (depending on
the chosen random bit).

Then it starts $\adv$ and learns $\srs$, since $\adv$ is algebraic, the reduction also learns trapdoor $\chi$. Then $\rdv$ picks $\chz$. Elements $b_i$ are interpreted as polynomials in $X$ that are
evaluated at $\chi$, i.e. $b_i = b_i(\chi)$. Next, $\rdv$ sets for
$\xi_i, \zeta_i \sample \FF_p$
\(
  \gone{\p{\tb}_1(X)} =
(X - \chz)(X - \ochz) \gone{w_1}(X) + \xi_i (X - \chz) \gone{1} +
\zeta_i (X - \ochz) \gone{1}, % \text{ for } i \in % \range{1}{9}, u_1
\),
and
\(
  \gone{\p{\tb}_i(X)} =
(X - \chz)(X - \ochz) \gone{b_i}(X) + \xi_i (X - \chz) \gone{1} +
\zeta_i (X - \ochz) \gone{1}, % \text{ for } i \in % \range{1}{9}, u_1
\) 
for $i \in \range{2}{9}$.

Denote by $\tb_i$ evaluations of $\p{\tb}_i$ at $\chi$.  The reduction computes
all
$\gone{\tb_i \tb_j}, \gone{\tb_i \tb_j \tb_k}, \gone{\tb_i \tb_j \tb_k \tb_l}$
such that $\gone{B_i B_j, B_i B_j B_k, B_i B_j B_k B_l} \in \pR$.  This is
possible since $\rdv$ knows all singletons $\gone{w_1, b_2, \ldots, b_9}$ and pairs
$\gone{b_i b_j} \in \pR$ which can be used to compute all required pairs
$\gone{\tb_i \tb_j}$:
\begin{align*}
\gone{\tb_i \tb_j} 
& = ((\chi - \chz)(\chi - \ochz)\gone{b_i} + \xi_i (\chi - \chz)\gone{1} +
\zeta_i (\chi - \ochz) \gone{1}) 
\cdot \\
 & ((\chi - \chz)(\chi - \ochz)\gone{b_j} + \xi_j (\chi - \chz)\gone{1} +
\zeta_j (\chi - \ochz) \gone{1}) = \\
 & ((\chi - \chz)(\chi - \ochz))^2 \gone{b_i b_j} +  ((\chi - \chz)(\chi -
 \ochz)\gone{b_i} (\xi_j (\chi - \chz) \gone{1} + \zeta_j (\chi - \ochz)
 \gone{1}) + \\
 & ((\chi - \chz)(\chi -
 \ochz)\gone{b_j} (\xi_i (\chi - \chz) \gone{1} + \zeta_i (\chi - \ochz)
 \gone{1}) + \psi,
\end{align*}
where $\psi$ compounds of $\xi_i, \xi_j, \zeta_i, \zeta_j, \chz, \ochz, \chi$ which
are all known by $\rdv$ and no $b_i$ nor $b_j$.
Analogously for the triplets and quadruplets and elements dependent on~$\vec{w}$. 

Next the reduction obtains from $\adv$ an instance--witness pair $(\inp, \wit)$.  $\rdv$ now
prepares a simulated proof as follows:
\begin{compactdesc} 
\item[Message 1] $\rdv$ computes $\gone{\pa(\chi)}$ using as
randomisers $\gone{\tb_1}, \gone{\tb_2}$ and setting $\wit_i = 0$, for $i
\in \range{1}{3 \noofc}$. Similarly it computes
$\gone{\pb(\chi)}, \gone{\pc(\chi)}$.  $\rdv$ publishes the obtained values
and picks the first challenge $\beta, \gamma$ at random.  Note that regardless
$w_1 = b_1$ or a random element, $\gone{a(\chi)}$ is random. Thus $\rdv$'s
output has the same distribution as output of a real prover.  
\item[Message 2]
$\rdv$ computes $\gone{\pz(\chi)}$ using $\tb_7, \tb_8, \tb_9$ and publishes
it. Then it picks randomly the challenge $\alpha$. This message is
independent on $b_1$ thus $\rdv$'s output is indistinguishable from the prover's. 
\item[Message 3] The reduction computes
  $\p{t_{lo}}(\chi), \p{t_{mid}}(\chi), \p{t_{hi}}(\chi)$, which all depend on
  $b_1$. To that end $\gone{\tb_1}$ is used. Note that if $\vec{w}$ is a vector
  of $\pF(b_1, \ldots, b_9)$ evaluations then
  $\gone{\p{t_{lo}}(\chi), \p{t_{mid}}(\chi), \p{t_{hi}}(\chi)}$ is the same as
  the real prover's. Alternatively, if $\vec{w}$ is a vector of random values,
  then $\p{t_{lo}}(\chi), \p{t_{mid}}(\chi), \p{t_{hi}}(\chi)$ are all random
  polynomials which evaluates at $\chz$ to the same value as the polynomials
  computed by the real prover. That is, in that case
  $\p{t_{lo}}(\chi), \p{t_{mid}}(\chi), \p{t_{hi}}(\chi)$ are as the simulator
  $\simulator$ would compute. Eventually, $\rdv$ outputs $\chz$.
\item[Message 4] The reduction outputs
  $\pa(\chz), \pb(\chz), \pc(\chz), \p{S_{\sigma 1}}(\chz), \p{S_{\sigma 2} (\chz)},
  \pt(\chz), \pz(\ochz)$.  For the sake of concreteness, denote by
  $S = \smallset{\pa, \pb, \pc, \pt, \pz}$. Although for a polynomial $\p{p} \in S$,
  reduction $\rdv$ does not know $\p{p}(\chi)$ or even do not know all the
  coefficients of $\p{p}$, the polynomials in $S$ was computed such that the
  reduction always knows their evaluation at $\chz$ and $\ochz$.
\item[Message 5] $\rdv$ computes the openings of the polynomial commitments assuring
  that evaluations at $\chz$ it provided were computed honestly.
\end{compactdesc}

If the adversary $\adv$'s output distribution differ in Game $\game{1}$ and
$\game{2}$ then the reduction uses it to distinguish between
$\vec{w} = \pF(b_1, \ldots, b_9)$ and $\vec{w}$ being random, thus
\( \abs{\prob{\game{1}} - \prob{\game{2}}} \leq \epsuber(\secpar).  \)

\ncase{Conclusion} Probability that the adversary's outputs differ in Game 0 and Game
2 is upper-bounded by $\leq \epszk(\secpar) + \epsuber(\secpar)$. \qed
\end{proof}

% \begin{lemma}[Simulation for false statement]
%   Let $\simulator = (\simulator_1, \simulator_2)$ be a trapdoor-less simulator from
%   Game 2 above, define $\simulator' = (\simulator_1, \simulator'_2)$ such that
%   $\simulator'_2$ behaves as $\simulator_2$ except it produces simulated proofs for
%   statements outside the language as well. Then for every $\inp \not\in \LANG_\REL$
%   holds
%   \[
%     \Pr \left[ \verifier (\srs, \inp, \zkproof) = 1 \left|\, \srs \gets \adv
%         (\secparam), \zkproof \gets \simulator (\srs, \inp) \right.  \right] \geq
%     1 - \eps (\secpar),
%   \]
%   for some negligible $\eps (\secpar)$.\michals{15.10}{Could we have just ``$= 1$''
%     here?}
% \end{lemma}

\subsection{Simulation extractability of~$\plonkprotfs$}
Since \cref{lem:plonkprot_ur,lem:plonkprot_ss} hold, $\plonkprot$ is $\ur{2}$ and
forking special sound. We now make use of \cref{thm:se} and show that
$\plonkprot_\fs$ is simulation-extractable as defined in \cref{def:updsimext}.

\begin{corollary}[Simulation extractability of $\plonkprot_\fs$]
\label{thm:plonkprotfs_se}
Assume an idealised $\plonkprot$ verifier fails at most with probability
$\epsid(\secpar)$, the discrete logarithm advantage is bounded by $\epsdlog(\secpar)$
and the $\PCOMp$ is a commitment of knowledge with security $\epsk(\secpar)$, binding
security $\epsbind(\secpar)$ and has unique opening property with security
$\epsop(\secpar)$. Let $\ro\colon \bin^* \to \bin^\secpar$ be a random oracle. Let
$\advse$ be an adversary that can make up to $q$ random oracle queries, up to $S$
simulation oracle queries, and outputs an acceptable proof for $\plonkprotfs$ with
probability at least $\accProb$. Then $\plonkprotfs$ is \COMMENT{forking
}simulation-extractable with extraction error $\eta = \epsur(\secpar)$. The
extraction probability $\extProb$ is at least
\[
	\extProb \geq \frac{1}{q^{3 \noofc}} (\accProb - \epsk(\secpar) - 2\cdot\epsbind(\secpar) -
  \epsop(\secpar))^{3\noofc + 1} -\eps(\secpar)\,,
\]
for some negligible $\eps(\secpar)$ and $\noofc$ being the number of
constraints in the proven circuit.
\end{corollary}
 
%%% Local Variables:
%%% mode: latex
%%% TeX-master: "main"
%%% End:


%% !TEX root = main.tex
% !TEX spellcheck = en-US



\section{Polynomial Commitment Schemes}
\label{sec:pcom}
A polynomial commitment scheme $\PCOM = (\kgen, \com, \open, \verify)$ consists of four
algorithms and allows to commit to a polynomial $\p{f}$ and later open the evaluation in a
point $z$ to some value $s=\p{f}(z)$. More formally:
\begin{description}
\item[$\kgen(1^\secpar, \maxdeg)$:] The key generation algorithm takes in a security
  parameter $\secpar$ and a parameter $\maxdeg$ which determines the maximal degree of the
  committed polynomial. It outputs a structured reference string $\srs$ (the commitment
  key). In the following we will consider $\srs$ implicitly determines $\secpar$.
\item[$\com(\srs, \p{f})$:] The commitment algorithm $\com(\srs, \p{f})$ takes
  in $\srs$ and a polynomial $\p{f}$ with maximum degree $\maxdeg$, and outputs
  a commitment $c$.
\item[$\open(\srs, z, s, \p{f})$:] The opening algorithm
  takes as input $\srs$, an evaluation point $z$, a
  value $s$ and the polynomial $\p{f}$. It outputs an opening $o$.
\item[$\verify(\srs, c, z, s, o)$:] The verification algorithm takes in $\srs$,
  a commitment $c$, an evaluation point $z$, a value $s$ and an opening $o$. It
  outputs 1 if $o$ is a valid opening for $(c, z, s)$ and 0 otherwise.
\end{description} 

A secure polynomial commitment $\PCOM$ should satisfy correctness, evaluation binding,
opening uniqueness, hiding and knowledge-binding.  Note that since we are in the updatable
setting, $\srs$ in these security definitions is the SRS that $\advse$ finalises using the
update oracle $\initU$ (See~\cref{fig:upd}).

\begin{description}
\item[Evaluation binding:] A $\ppt$ adversary $\adv$ which outputs a commitment
  $\vec{c}$ and evaluation points $\vec{z}$ has at most negligible chances to open
  the commitment to two different evaluations $\vec{s}, \vec{s'}$. That is, let
  $k \in \NN$ be the number of committed polynomials, $l \in \NN$ number of
  evaluation points, $\vec{c} \in \GRP^k$ be the commitments, $\vec{z} \in
  \FF_p^l$ be the arguments the polynomials are evaluated at, $\vec{s},\vec{s}'
  \in \FF_p^k$ the evaluations, and $\vec{o},\vec{o}' \in \FF_p^l$ be the
  commitment openings. Then for every $\ppt$ adversary $\adv$
	\[
		\condprob{
			\begin{matrix}
				  \verify(\srs, \vec{c}, \vec{z}, \vec{s}, \vec{o}) = 1,  \\ 
				  \verify(\srs, \vec{c}, \vec{z}, \vec{s}', \vec{o}') = 1, \\
				  \vec{s} \neq \vec{s}'
			\end{matrix}}
			{
			\begin{matrix}
%				& \srs \gets \kgen(\secparam, \maxdeg),\\
				 (\vec{c}, \vec{z}, \vec{s}, \vec{s}', \vec{o}, \vec{o}') \gets \adv^{\initU}(\maxdeg)
			\end{matrix}
		} \leq \negl\,.
	\]

\end{description}
	
%We say that $\PCOM$ has the unique opening property if the following holds:
To show unique response property of our schemes we require that the polynomial
commitment scheme the proof system is using has unique openings defined as follows.
\begin{description}
\item[Opening uniqueness:] Intuitively, opening uniqueness assures that there is only one
  valid opening for the committed polynomial and given evaluation point. This property is
  crucial in showing \COMMENT{forking }simulation-extractability of $\plonk$, $\sonic$ and
  $\marlin$.
  Let $k \in \NN$ be the number of committed polynomials, $l \in \NN$ number of evaluation
  points, $\vec{c} \in \GRP^k$ be the commitments, $\vec{z} \in \FF_p^l$ be the arguments
  the polynomials are evaluated at, $\vec{s} \in \FF_p^k$ the evaluations, and
  $\vec{o}, \vec{o}' \in \FF_p^l$ be the commitment openings. Then for every $\ppt$ adversary $\adv$
	\[
		\condprob{
			\begin{matrix}
				  \verify(\srs, \vec{c}, \vec{z}, \vec{s}, \vec{o}) = 1,  \\ 
				  \verify(\srs, \vec{c}, \vec{z}, \vec{s}, \vec{o'}) = 1, \\
				 \vec{o} \neq \vec{o'}
			\end{matrix}
		}{
			\begin{matrix}
%				& \srs \gets \kgen(\secparam, \maxdeg),\\
				  (\vec{c}, \vec{z}, \vec{s}, \vec{o}, \vec{o'}) \gets \adv^{\initU}(\maxdeg)
			\end{matrix}
		}\leq \negl\,.
	\]
\end{description}
We show
that $\plonk$'s, $\sonic$'s, and $\marlin$'s polynomial commitment schemes satisfy this
requirement in \cref{lem:pcomp_op} and \cref{lem:pcoms_unique_op}
respectively.


\begin{description}
\item[Hiding:] We also formalize notion of $k$-hiding property of a polynomial commitment scheme. Let $\HHH$ be a set of size $\maxdeg + 1$ and $\ZERO_\HHH$ its
  vanishing polynomial. We say that a polynomial scheme is \emph{hiding} with
  security $\epsh(\secpar)$ if for every $\ppt$ adversary $\adv$, $k \in \NN$,
  probability
  \begin{align*}
    \condprob
   { b' = b}{
    (f_0, f_1, c, k, b') \gets \adv^{\initU, \oraclec}(\maxdeg, c), f_0, f_1 \in \FF^{\maxdeg}
    [X]}
\leq \frac{1}{2} + \eps(\secpar).
  \end{align*}
  Where $c = f'_b (\chi)$, for a random bit $b$ and the polynomial
      $f'_b (X) = f_b + \ZERO_\HHH (X) (a_0 + a_1 X + \ldots a_{k - 1} X^{k -
        1})$,
and the oracle $\oraclec$ on adversary's evaluation query $x$ it adds $x$ to initially empty set
      $Q_x$ and if $|Q_x| \leq k$, it provides $f'_b (x)$.
 
  \end{description}

\begin{description}
\item[Commitment of knowledge:] Intuitively, when a commitment scheme is ``of knowledge'' then if an
adversary produces a (valid) commitment $c$, which it can open correctly in an evaluation point, then it must
know the underlying polynomial $\p{f}$ which commits to that value.  For every $\ppt$ adversary $\adv$ who produces
  commitment $c$, evaluation $s$ and opening $o$ there
  exists a $\ppt$ extractor $\ext$ such that
\[
  \condprob{
    \begin{matrix}
       \deg \p{f} \leq \maxdeg,
       c = \com(\srs, \p{f}),\\
       \verify(\srs, c, z, s, o) = 1
    \end{matrix}
        }{
    \begin{matrix}
      %
     % & \srs \gets \kgen(\secparam, \maxdeg),\\
      c \gets \adv^{\initU}(\maxdeg),
      z \sample \FF_p \\
      (s, o) \gets \adv(c, z), \\
   \p{f} = \ext_\adv(\srs, c)\\
    \end{matrix}}
  \geq 1 - \epsk(\secpar).
\]
In that case we say that $\PCOM$ is $\epsk(\secpar)$-knowledge.
\end{description}


\cref{fig:pcomp,fig:pcoms} present variants of KZG~\cite{AC:KatZavGol10} polynomial
commitment schemes used in \plonk{}, \sonic{} and \marlin{}. The key generation algorithm
$\kgen$ takes as input a security parameter $\secparam$ and a parameter $\maxdeg$ which
determines the maximal degree of the committed polynomial. We assume that $\maxdeg$ can be
read from the output SRS. While the figures only describe trusted SRS setup, it is not
hard to lift the SRS generation into the updatable setting by defining the extra
algorithms $\upd$, $\verifyCRS$ (see~\cref{def:upd-scheme}) as described in~\cref{fig:upd-scheme}.  \cite{CCS:MBKM19}
shows, using AGM, that $\PCOMs$ is a commitment of knowledge.  The same reasoning could be
used to show that property for $\PCOMp$.
 



\begin{figure}[t!]
	\centering
	\fbox{
		\begin{minipage}[t]{0.76\linewidth}
			\procedure{$\kgen(\REL)$}{
				\chi \sample \FF_p \\ [\myskip]
				\srs := 
				\left( \gone{\smallset{\chi^i}_{i = 0}^{\dconst}},
				\gtwo{\chi} \right); 
				\rho =  \left(\gone{\chi, \chi}, \gtwo{\chi}\right) \\ [\myskip]
				\pcreturn (\srs, \rho) \\ [\myskip]
			}
		%
		\\
		%
		\procedure{$\verifyCRS(\srs, \{\rho_j \}_{j=1}^n)$}{
			\text{Parse }  \srs \text{ as } \left( \gone{\smallset{A_i}_{i = 0}^{\dconst}},
			\gtwo{B} \right) \text{and } \{\rho_j \}_{j=1}^n \text{ as } \left\{\left( P_j, \bP_j, \hP_j \right)\right\}_{j=1}^n \\ [\myskip]
			\text{Verify Update proofs: } \\ [\myskip]
			\t \bP_1 = P_1 \\ [\myskip]
			\t P_j \bullet \gtwo{1} = P_{j-1} \bullet \hP_j \quad \forall j \geq 2 \\ [\myskip]
			\t \bP_n \bullet \gtwo{1} = \gone{1} \bullet \hP_n \\ [\myskip]
			\text{Verify SRS structure: } \\ [\myskip]
			\t \gone{A_i} \bullet \gtwo{1} = \gone{A_{i-1}} \bullet \gtwo{B} \text{ for all } 0 < i \leq \dconst \\ [\myskip]
		}
		%
		\\
		%
		\procedure{$\upd(\srs, \{\rho_j \}_{j=1}^n)$}{
			\text{Parse } \srs \text{ as } \left( \gone{\smallset{A_i}_{i = 0}^{\dconst}},
			\gtwo{B} \right) \\ [\myskip]
			\chi' \sample \FF_p  \\ [\myskip]
			\srs' := 
			\left( \gone{\smallset{{\chi'}^i A_i}_{i = 0}^{\dconst}},
			\gtwo{\chi' B} \right); 
			\rho' =	\left( \gone{\chi' A_1, \chi'}, \gtwo{\chi'}\right) \\ [\myskip]
			\pcreturn (\srs', \rho')
		}
		\end{minipage}}
	\caption{Updatable SRS scheme for $\PCOMp$} 
	\label{fig:upd-scheme}
\end{figure}


\begin{figure}
  \small
  \hspace*{-2cm}\fbox{
\begin{minipage}{15,5cm}
\begin{pcvstack}[]
  \begin{pchstack}
			\procedure{$\kgen(\secparam, \maxdeg)$}
			{
			\chi \sample \FF_p \\ [\myskip]
			\pcreturn \srs = \gone{1, \ldots, \chi^{\maxdeg}}, \gtwo{\chi}\\ [\myskip]
      }\\
      \michals{29.04}{adjust to $\initU$}
			
			\pchspace
			
			\procedure{$\com(\srs, \vec{\p{f}}(X))$}
			{ 
				\pcreturn \gone{\vec{c}} = \gone{\vec{\p{f}}(\chi)}\\ [\myskip]
        \fbox{$\pcreturn \vec{\p{f}} (X)$}\\
			}

      \pchspace

      \procedure{$\open(\srs, \vec{z}, \vec{s}, \vec{\p{f}}(X), \aux)$}
			{
      \vec{\gamma} \gets \ro (g_0( \vec{z}, \vec{s}, \gone{\vec{c}}, \aux_0))\\[\myskip]
			\pcfor i \in \range{1}{\abs{\vec{z}}} \pcdo\\ [\myskip]
      \pcind \p{o}_j(X) \gets \sum_{i \in K_j} \gamma_{j}^{i - 1}
      \frac{\p{f}_{i}(X) - \p{f}_{i}(z_j)}{X - z_j}\\ [\myskip] 
      \pcreturn \vec{o} = \gone{\vec{\p{o}}(\chi)}\\ [\myskip]
      \fbox{$\pcreturn \vec{\p{o}} (X)$}
				% \hphantom{\hspace*{5.5cm}}	
			}

    \end{pchstack}
		 \pcvspace
    
		\begin{pchstack}
			\procedure{$\verifyb(\srs, \gone{\vec{c}}, \vec{z}, \vec{s}, \gone{\p{o}(\chi)}, \aux)$}
			{
        \vec{\gamma} \gets \ro (g_0( \vec{z}, \vec{s}, \gone{\vec{c}}, \aux_0))\\[\myskip]
				r \gets \ro (g_1(\gone{\vec{c}}, \vec{z}, \vec{s}, \gone{\p{o}(\chi)}, \aux_1))\\ [\myskip]
				% \pcfor j \in \range{1}{\abs{\vec{z}}} \pcdo \\ [\myskip]
				(*) \pcif 
          \sum_{j = 1}^{\abs{\vec{z}}} r^{j} \cdot \gone{\sum_{i \in K_j}
          \gamma_j^{i - 1} c_{i} - \sum_{i \in K_j} \gamma_j^{i - 1} s_{i_j}} \bullet \gtwo{1} + \\ [\myskip] 
          \pcind \sum_{j = 1}^{\abs{\vec{z}}} r^{j} z_j o_j
          \bullet \gtwo{1} \neq \gone{\sum_{j = 1}^{\abs{\vec{z}}} r^{j} o_j }
          \bullet \gtwo{\chi} \pcthen  \\
					\pcind \pcreturn 0\\ [\myskip]
          \fbox{
            \begin{minipage}{7cm}
            (**) $\pcif $
              $\sum_{j = 1}^{\abs{\vec{z}}} r^j \cdot (\sum_{i \in K_j}
              \gamma_j^{i - 1} \p{f}_{i} (X) - \sum_{i \in K_j} \gamma_j^{i - 1} s_{i_j}) + $\\ [\myskip] 
              $\pcind \sum_{j = 1}^{\abs{\vec{z}}} r^{j} z_j \p{o}_j (X)
               \neq \sum_{j = 1}^{\abs{\vec{z}}} r^{j} \p{o}_j (X)
              \cdot X \pcthen $ \\
              $\pcind \pcreturn 0$
            \end{minipage}
          }\\[\myskip]
					\pcreturn 1.\\
			}

      \pchspace
      
      \procedure{$\verify(\srs, \gone{\vec{c}}, \vec{z}, \vec{s}, \gone{\p{o}(\chi)})$}
			{
        \vec{\gamma} \gets \ro (g_0( \vec{z}, \vec{s}, \gone{\vec{c}}, \aux_0))\\[\myskip]
				\pcfor j \in \range{1}{\abs{\vec{z}}} \pcdo \\ [\myskip]
				\pcind \pcif 
          \gone{\sum_{i \in K_j}
          \gamma_j^{i - 1} c_{i} - \sum_{i \in K_j} \gamma_j^{i - i} s_{i, j}} \bullet
          \gtwo{1} + \\ [\myskip] \pcind  z_j
          o_j
          \bullet \gtwo{1} \neq \gone{o_j}
          \bullet \gtwo{\chi} \pcthen  \\
					\pcind \pcreturn 0\\ [\myskip]
        \fbox{
          \begin{minipage}{7cm}
          $\pcind \pcif $
          $\sum_{i \in K_j} \gamma_j^{i - 1} \p{f}_{i} (X) - \sum_{i \in K_j} \gamma_j^{i - i} s_{i, j} + $\\ [\myskip] 
          $\pcind  z_j \p{o}_j (X) \neq \p{o}_j (X) X \pcthen \pcreturn 0$
        \end{minipage}
        }\\ [\myskip]
					\pcreturn 1.
			}

    \end{pchstack}
	\end{pcvstack}
\end{minipage}
  }
	\caption{$\PCOM$ polynomial commitment scheme. Here $\abs{\vec{z}} = l$ is the number of evaluation points, the number of committed polynomials is $m$, $K_j$ is the set of polynomials that was evaluated at point $z_j$. Functions $g_0$ and $g_1$ are injective and specific to the context in which the polynomial commitment is used. (In our case, functions $g_0$ and $g_1$ are produce partial transcripts of the proof that utilizes the commitment scheme, $\aux$ contains all additional information that is needed by the functions.)
  In the boxes we describe values returned or equality computed in the ideal protocol where the verifier checks equalities on the polynomials instead of their evaluations. For algorithm $\pcalgostyle{Alg}$ we denote its ideal variant by $\pcalgostyle{Alg}'$.}
	\label{fig:pcomp}
  \end{figure}

% \begin{figure}[h!]
% \centering
% 	\begin{pcvstack}[center,boxed]
% 		\begin{pchstack}
% 			\procedure{$\kgen(\secparam, \maxdeg)$} {
% 				\alpha, \chi \sample \FF^2_p \\ [\myskip]
% 				\pcreturn \gone{\smallset{\chi^i}_{i = -\multconstr}^{\multconstr},
%           \smallset{\alpha \chi^i}_{i = -\multconstr, i \neq
%             0}^{\multconstr}},\\
%         \pcind \gtwo{\smallset{\chi^i, \alpha \chi^i}_{i =
%             -\multconstr}^{\multconstr}}, \gtar{\alpha}\\
% 				%\markulf{03.11.2020}{} \\
% 			%	\hphantom{\pcind \p{o}_i(X) \gets \sum_{j = 1}^{t_i} \gamma_i^{j - 1} \frac{\p{f}_{i,j}(X) - \p{f}_{i, j}(z_i)}{X - z_i}}
% 				\hphantom{\hspace*{5.5cm}}
% 		}
%
% 			\pchspace
%
% 			\procedure{$\com(\srs, \maxconst, \p{f}(X))$} {
% 				\p{c}(X) \gets \alpha \cdot X^{\dconst - \maxconst} \p{f}(X) \\ [\myskip]
% 				\pcreturn \gone{c} = \gone{\p{c}(\chi)}\\ [\myskip]
% 				\hphantom{\pcind \pcif \sum_{i = 1}^{\abs{\vec{z}}} r_i \cdot
%           \gone{\sum_{j = 1}^{t_j} \gamma_i^{j - 1} c_{i, j} - \sum_{j = 1}^{t_j}
%             s_{i, j}} \bullet \gtwo{1} + } }
% 		\end{pchstack}
% 		% \pcvspace
%
% 		\begin{pchstack}
% 			\procedure{$\open(\srs, z, s, f(X))$}
% 			{
% 				\p{o}(X) \gets \frac{\p{f}(X) - \p{f}(z)}{X - z}\\ [\myskip]
% 				\pcreturn \gone{\p{o}(\chi)}\\ [\myskip]
% 				\hphantom{\hspace*{5.5cm}}
% 			}
%
% 			\pchspace
%
% 			\procedure{$\verify(\srs, \maxconst, \gone{c}, z, s, \gone{\p{o}(\chi)})$}
%       {
%         \pcif \gone{\p{o}(\chi)} \bullet \gtwo{\alpha \chi} + \gone{s - z
%         \p{o}(\chi)} \bullet \gtwo{\alpha} = \\ [\myskip] \pcind \gone{c}
%         \bullet \gtwo{\chi^{- \dconst + \maxconst}} \pcthen  \pcreturn 1\\
%         [\myskip]
%         \rlap{\pcelse \pcreturn 0.} \hphantom{\pcind \pcif \sum_{i =
%             1}^{\abs{\vec{z}}} r_i \cdot \gone{\sum_{j = 1}^{t_j} \gamma_i^{j -
%               1} c_{i, j} - \sum{j = 1}^{t_j} s_{i, j}} \bullet \gtwo{1} + } }
% 		\end{pchstack}
% 	\end{pcvstack}
%
% 	\caption{$\PCOMs$ polynomial commitment scheme.}
% 	\label{fig:pcoms}
% \end{figure}

\subsection{Unique opening property of $\PCOM$}
Now, we show that the batched variant of the KZG polynomial
commitment scheme that is used in \plonk{} and $\marlin{}$, has the unique opening property.

\begin{lemma}
\label{lem:pcomp_op}
Let $\PCOM = (\kgen, \com, \open, \verifyb)$ be a batched version of a KZG polynomial commitment,
cf.~\cref{fig:pcomp}, that commits to $m$ polynomials of degree up to $\maxdeg$. Let $\vec{z} = (z_0, \ldots, z_{l - 1}) \in \FF_p^l$ be the points the polynomials are  evaluated at, $k_i \in \NN$ be the number of the committed polynomials to be evaluated at $z_i$, and $K_i$ be the set of indices of these polynomials. Let $\vec{s_{K_i}} \in \FF_p^{k_i}$ the evaluations of polynomials at $z_i$, and $\vec{o} = (o_0, \ldots, o_{l - 1}) \in \FF_p^l$ be the commitment openings. We show that the probability an algebraic adversary $\adv$, who can made up to $q$ random oracle queries, opens the same vector of commitments in two different ways is at most $\epsop(\secpar)$, for $\epsop(\secpar) \leq l \cdot  \epsudlog(\secpar) + \infrac{1}{\abs{\FF_p}}$, where $\epsudlog(\secpar)$ is security of the $(\maxdeg, 1)$-$\udlog$ assumption and $\FF_p$ is the field used in $\PCOM$.
\end{lemma}
\begin{proof}
 
  The proof goes by game hops. In the first game the adversary wins if it presents two acceptable openings of a vector of polynomials. Then, we restrict the winning condition and require that the adversary also makes the idealized batched verifier to accept the proof. In the next game, we abort if the idealized verifier rejects a proof for one of the evaluation point. 

  \ncase{Game 0} In this game the adversary wins if it provides two different openings for a vector of polynomial commitments and their evaluations that are acceptable by $\verifyb$.

  \ncase{Game 1} This game is identical to Game 0 except it is additionally aborted if the commitment opening are not acceptable by $\verifyb'$.

  \ncase{Game 0 to Game 1} %
  Any discrepancy between the idealised verifier rejection and real verifier acceptance allows one to break the (updatable) discrete logarithm problem.  The reduction $\rdvdulog$ proceeds as follows. It answers $\adv$'s queries for SRS updates according to the answers it receives from its udlog update oracle. When $\adv$ finalizes an SRS, $\rdvdulog$ finalizes the corresponding udlog challenge $(\gone{1, \ldots, {\chi'}^{\maxdeg}}, \gtwo{1})$. We consider verification equation $(**)$ as a polynomial in $X$ and the verification equation $(*)$ as it's evaluation at $\chi'$. Consider an opening such that verification equation (**), cf.~\cref{fig:pcomp}, does not hold, i.e.~(**) is not a zero polynomial, but (*) does, i.e.~(**) zeroes at $\chi'$. Since $\adv$ is algebraic, all proof elements are extended by their representation as a combination of the input $\GRP_1$-elements. Therefore, all coefficients of the verification equation polynomial related to (**) are known. Now, $\rdvdulog$ computes the roots of (**), finds $\chi'$ among them, and returns $\chi'$. Hence the probability that the adversary wins in Game 1, but does not win in Game 0 is upper-bounded by $\epsudlog (\secpar)$.

  \ncase{Game 2} This game is identical to Game 1 except it is additionally aborted if one of the opening is not acceptable by an idealised verifier $\verify'$.

  \ncase{Game 1 to Game 2}
  The ideal verifier checks whether the following equality, for $\smallset{\gamma_j}_{j = 1}^{l}, r$
  picked at random, holds:
  \begin{multline}
    \label{eq:ver_eq_poly}
    \sum_{j = 0}^{l - 1} r^{j}\left(\sum_{i \in K_j} \gamma_{j}^{i - 1} \cdot \p{f}_i(X) - \sum_{i \in K_j} \gamma_{j}^{i - 1} \cdot {s_i}_j \right) 
    \equiv \sum_{j = 0}^{l - 1} r^{j} \p{o_{j}}(X)(X - z_j).
  \end{multline}
  Since $r$ has been picked as a random oracle output, probability that
  \cref{eq:ver_eq_poly} holds while for some $j \in \range{0}{l - 1}$
  \[
    r^j \left(\sum_{i \in K_j} \gamma^{i - 1} \cdot \p{f}_i(X) - \sum_{i \in K_j}
    \gamma^{i - 1} \cdot {s_i}_j \right) \not\equiv r^j \p{o_j}(X)(X - z_j)
  \]
  is $\infrac{q}{\abs{\FF_p}}$~cf.~\cite{EPRINT:GabWilCio19}. 
  When \(
    r^j \left(\sum_{i \in K_j} \gamma^{i - 1} \cdot \p{f}_i(X) - \sum_{i \in K_j}
    \gamma^{i - 1} \cdot {s_i}_j \right) = r^j \p{o_j}(X)(X - z_j)
  \)
  holds, polynomial $\p{o_j}(X)$ is uniquely determined from the uniqueness of
  polynomial composition. 

  \ncase{Conclusion} %
  We note that the idealised verifier $\vereq(X)$ does not accept two different openings of a correct evaluation. Hence the probability that the adversary wins Game 2 is $0$ and the probability that the adversary wins in Game 0 is upper-bounded by \(\epsudlog (\secpar) + \frac{q}{\abs{\FF_p}}\).
    \qed
\end{proof}

%%% Local Variables:
%%% mode: latex
%%% TeX-master: "main"
%%% End:

%\section{Non-malleability of \plonk{}, omitted protocol descriptions}
\label{sec:plonk_supp_mat}

\newcommand{\vql}{\vec{q_{L}}}
\newcommand{\vqr}{\vec{q_{R}}}
\newcommand{\vqm}{\vec{q_{M}}}
\newcommand{\vqo}{\vec{q_{O}}}
\newcommand{\vx}{\vec{x}}
\newcommand{\vqc}{\vec{q_{C}}}
\subsection{Plonk protocol description}
\label{sec:plonk_explained}
\oursubsub{The constrain system}
Assume $\CRKT$ is a fan-in two arithmetic circuit,
which fan-out is unlimited and has $\numberofconstrains$ gates and $\noofw$ wires
($\numberofconstrains \leq \noofw \leq 2\numberofconstrains$). \plonk's constraint
system is defined as follows:
\begin{itemize}
\item Let $\vec{V} = (\va, \vb, \vc)$, where $\va, \vb, \vc
  \in \range{1}{\noofw}^\numberofconstrains$. Entries $\va_i, \vb_i, \vc_i$ represent indices of left,
  right and output wires of circuits $i$-th gate.
\item Vectors $\vec{Q} = (\vql, \vqr, \vqo, \vqm, \vqc) \in
  (\FF^\numberofconstrains)^5$ are called \emph{selector vectors}:
  \begin{itemize}
  \item If the $i$-th gate is a multiplicative gate then $\vql_i = \vqr_i = 0$,
    $\vqm_i = 1$, and $\vqo_i = -1$. 
  \item If the $i$-th gate is an addition gate then $\vql_i = \vqr_i  = 1$, $\vqm_i =
    0$, and $\vqo_i = -1$. 
  \item $\vqc_i = 0$ always. 
  \end{itemize}
\end{itemize}

We say that vector $\vx \in \FF^\noofw$ satisfies constraint system if for all $i
\in \range{1}{\numberofconstrains}$
\[
  \vql_i \cdot \vx_{\va_i} + \vqr_i \cdot \vx_{\vb_i} + \vqo \cdot \vx_{\vc_i} +
  \vqm_i \cdot (\vx_{\va_i} \vx_{\vb_i}) + \vqc_i = 0. 
\]

\oursubsub{Algorithms rolled out}
\label{sec:plonk_explained}
\plonk{} argument system is universal. That is, it allows to verify computation
of any arithmetic circuit which has no more than $\numberofconstrains$
gates using a single SRS. However, to make computation efficient, for each
circuit there is allowed a preprocessing phase which extend the SRS with
circuit-related polynomial evaluations.

For the sake of simplicity of the security reductions presented in this paper, we
include in the SRS only these elements that cannot be computed without knowing
the secret trapdoor $\chi$. The rest of the SRS---the preprocessed input---can
be computed using these SRS elements thus we leave them to be computed by the
prover, verifier, and simulator.

\ourpar{$\plonk$ SRS generating algorithm $\kgen(\REL)$:}
The SRS generating algorithm picks at random $\chi \sample \FF_p$, computes
and outputs
\[
	\srs = \left(\gone{\smallset{\chi^i}_{i = 0}^{\numberofconstrains + 2}},
	\gtwo{\chi} \right).
\]

\ourpar{Preprocessing:}
Let $H = \smallset{\omega^i}_{i = 1}^{\numberofconstrains }$ be a
(multiplicative) $\numberofconstrains$-element subgroup of a field $\FF$
compound of $\numberofconstrains$-th roots of unity in $\FF$. Let $\lag_i(X)$ be
the $i$-th element of an $\numberofconstrains$-elements Lagrange basis. During
the preprocessing phase polynomials $\p{S_{id j}}, \p{S_{\sigma j}}$, for
$\p{j} \in \range{1}{3}$, are computed:
\begin{equation*}
  \begin{aligned}
    \p{S_{id 1}}(X) & = X,\vphantom{\sum_{i = 1}^{\noofc} \sigma(i) \lag_i(X),}\\
    \p{S_{id 2}}(X) & = k_1 \cdot X,\vphantom{\sum_{i = 1}^{\noofc} \sigma(i) \lag_i(X),}\\
    \p{S_{id 3}}(X) & = k_2 \cdot X,\vphantom{\sum_{i = 1}^{\noofc} \sigma(i) \lag_i(X),}
  \end{aligned}
  \qquad
\begin{aligned}
  \p{S_{\sigma 1}}(X) & = \sum_{i = 1}^{\noofc} \sigma(i) \lag_i(X),\\
  \p{S_{\sigma 2}}(X) & = \sum_{i = 1}^{\noofc}
  \sigma(\noofc + i) \lag_i(X),\\
  \p{S_{\sigma 3}}(X) & =\sum_{i = 1}^{\noofc} \sigma(2 \noofc + i) \lag_i(X).
\end{aligned}
\end{equation*}
Coefficients $k_1$, $k_2$ are such that $H, k_1 \cdot H, k_2 \cdot H$ are
different cosets of $\FF^*$, thus they define $3 \cdot \noofc$
different elements. \cite{EPRINT:GabWilCio19} notes that it is enough to set
$k_1$ to a quadratic residue and $k_2$ to a quadratic non-residue.

Furthermore, we define polynomials $\p{q_L}, \p{q_R}, \p{q_O}, \p{q_M}, \p{q_C}$
such that
\begin{equation*}
  \begin{aligned}
  \p{q_L}(X) & = \sum_{i = 1}^{\noofc} \vql_i \lag_i(X), \\
  \p{q_R}(X) & = \sum_{i = 1}^{\noofc} \vqr_i \lag_i(X), \\
  \p{q_M}(X) & = \sum_{i = 1}^{\noofc} \vqm_i \lag_i(X),
\end{aligned}
\qquad
\begin{aligned}
  \p{q_O}(X) & = \sum_{i = 1}^{\noofc} \vqo_i \lag_i(X), \\
  \p{q_C}(X) & = \sum_{i = 1}^{\noofc} \vqc_i \lag_i(X). \\
  \vphantom{\p{q_M}(X)  = \sum_{i = 1}^{\noofc} \vqm_i \lag_i(X),}
\end{aligned}
\end{equation*}

\ourpar{$\plonk$ prover
  $\prover(\srs, \inp, \wit = (\wit_i)_{i \in \range{1}{3 \cdot
      \noofc}})$.}
\begin{description}
\item[Round 1] Sample $b_1, \ldots, b_9 \sample \FF_p$; compute
  $\p{a}(X), \p{b}(X), \p{c}(X)$ as
	\begin{align*}
		\p{a}(X) &= (b_1 X + b_2)\p{Z_H}(X) + \sum_{i = 1}^{\noofc} \wit_i \lag_i(X) \\
		\p{b}(X) &= (b_3 X + b_4)\p{Z_H}(X) + \sum_{i = 1}^{\noofc} \wit_{\noofc + i} \lag_i(X) \\
		\p{c}(X) &= (b_5 X + b_6)\p{Z_H}(X) + \sum_{i = 1}^{\noofc} \wit_{2 \cdot \noofc + i} \lag_i(X) 
	\end{align*}
	Output polynomial commitments $\gone{\p{a}(\chi), \p{b}(\chi), \p{c}(\chi)}$.
	
	\item[Round 2]
	Get challenges $\beta, \gamma \in \FF_p$
	\[
		\beta = \ro(\tzkproof[0..1], 0)\,, \qquad \gamma = \ro(\tzkproof[0..1], 1)\,.
	\]
	Compute permutation polynomial $\p{z}(X)$
	\begin{multline*}
		\p{z}(X) = (b_7 X^2 + b_8 X + b_9)\p{Z_H}(X) + \lag_1(X) + \\
			+ \sum_{i = 1}^{\noofc - 1} 
			\left(\lag_{i + 1} (X) \prod_{j = 1}^{i} 
			\frac{
			(\wit_j +\beta \omega^{j - 1} + \gamma)(\wit_{\noofc + j} + \beta k_1 \omega^{j - 1} + \gamma)(\wit_{2 \noofc + j} +\beta k_2 \omega^{j- 1} + \gamma)}
			{(\wit_j+\sigma(j) \beta + \gamma)(\wit_{\noofc + j} + \sigma(\noofc + j)\beta + \gamma)(\wit_{2 \noofc + j} + \sigma(2 \noofc + j)\beta + \gamma)}\right)
	\end{multline*}
	Output polynomial commitment $\gone{\p{z}(\chi)}$
		
	\item[Round 3]
	Get the challenge $\alpha = \ro(\tzkproof[0..2])$, compute the quotient polynomial 
	\begin{align*}
	& \p{t}(X)  = \\
	& (\p{a}(X) \p{b}(X) \selmulti(X) + \p{a}(X) \selleft(X) + 
	\p{b}(X)\selright(X) + \p{c}(X)\seloutput(X) + \pubinppoly(X) + \selconst(X)) 
	\frac{1}{\p{Z_H}(X)} +\\
	& + ((\p{a}(X) + \beta X + \gamma) (\p{b}(X) + \beta k_1 X + \gamma)(\p{c}(X) 
	+ \beta k_2 X + \gamma)\p{z}(X)) \frac{\alpha}{\p{Z_H}(X)} \\
	& - (\p{a}(X) + \beta \p{S_{\sigma 1}}(X) + \gamma)(\p{b}(X) + \beta 
	\p{S_{\sigma 2}}(X) + \gamma)(\p{c}(X) + \beta \p{S_{\sigma 3}}(X) + 
	\gamma)\p{z}(X \omega))  \frac{\alpha}{\p{Z_H}(X)} \\
	& + (\p{z}(X) - 1) \lag_1(X) \frac{\alpha^2}{\p{Z_H}(X)}
	\end{align*}
	Split $\p{t}(X)$ into degree less then $\noofc$ polynomials $\p{t_{lo}}(X), \p{t_{mid}}(X), \p{t_{hi}}(X)$, such that
	\[
		\p{t}(X) = \p{t_{lo}}(X) + X^{\noofc} \p{t_{mid}}(X) + X^{2 \noofc} \p{t_{hi}}(X)\,.
	\]
	Output $\gone{\p{t_{lo}}(\chi), \p{t_{mid}}(\chi), \p{t_{hi}}(\chi)}$.
	
	\item[Round 4]
	Get the challenge $\chz \in \FF_p$, $\chz = \ro(\tzkproof[0..3])$.
	Compute opening evaluations
	\begin{align*}
      \p{a}(\chz), \p{b}(\chz), \p{c}(\chz), \p{S_{\sigma 1}}(\chz), \p{S_{\sigma 2}}(\chz), \p{t}(\chz), \p{z}(\chz \omega),
	\end{align*}
	Compute the linearisation polynomial
	\[
		\p{r}(X) = 
		\begin{aligned}
      & \p{a}(\chz) \p{b}(\chz) \selmulti(X) + \p{a}(\chz) \selleft(X) + \p{b}(\chz) \selright(X) + \p{c}(\chz) \seloutput(X) + \selconst(X) \\
      & + \alpha \cdot \left( (\p{a}(\chz) + \beta \chz + \gamma) (\p{b}(\chz) + \beta k_1 \chz + \gamma)(\p{c}(\chz) + \beta k_2 \chz + \gamma) \cdot \p{z}(X)\right) \\
      & - \alpha \cdot \left( (\p{a}(\chz) + \beta \p{S_{\sigma 1}}(\chz) + \gamma) (\p{b}(\chz) + \beta \p{S_{\sigma 2}}(\chz) + \gamma)\beta \p{z}(\chz\omega) \cdot \p{S_{\sigma 3}}(X)\right) \\
      & + \alpha^2 \cdot \lag_1(\chz) \cdot \p{z}(X)
		\end{aligned}
	\]
	Output $\p{a}(\chz), \p{b}(\chz), \p{c}(\chz), \p{S_{\sigma 1}}(\chz), \p{S_{\sigma 2}}(\chz), \p{t}(\chz), \p{z}(\chz \omega), \p{r}(\chz).$
	
	\item[Round 5]
	Compute the opening challenge $v \in \FF_p$, $v = \ro(\tzkproof[0..4])$.
	Compute the openings for the polynomial commitment scheme 
	\begin{align*}
	& \p{W_\chz}(X) = \frac{1}{X - \chz} \left(
	\begin{aligned}
		& \p{t_{lo}}(X) + \chz^\noofc \p{t_{mid}}(X) + \chz^{2 \noofc} \p{t_{hi}}(X) - \p{t}(\chz)\\
		& + v(\p{r}(X) - \p{r}(\chz)) \\
		& + v^2 (\p{a}(X) - \p{a}(\chz))\\
		& + v^3 (\p{b}(X) - \p{b}(\chz))\\
		& + v^4 (\p{c}(X) - \p{c}(\chz))\\
		& + v^5 (\p{S_{\sigma 1}}(X) - \p{S_{\sigma 1}}(\chz))\\
		& + v^6 (\p{S_{\sigma 2}}(X) - \p{S_{\sigma 2}}(\chz))
	\end{aligned}
	\right)\\
	& \p{W_{\chz \omega}}(X) = \frac{\p{z}(X) - \p{z}(\chz \omega)}{X - \chz \omega}
\end{align*}
	Output $\gone{\p{W_{\chz}}(\chi), \p{W_{\chz \omega}}(\chi)}$.
\end{description}

\ncase{$\plonk$ verifier $\verifier(\srs, \inp, \zkproof)$}\ \newline
The \plonk{} verifier works as follows
\begin{description}
	\item[Step 1] Validate all obtained group elements.
	\item[Step 2] Validate all obtained field elements.
	\item[Step 3] Validate the instance
      $\inp = \smallset{\wit_i}_{i = 1}^\instsize$.
	\item[Step 4] Compute challenges $\beta, \gamma, \alpha, \chz, v,
      u$ from the transcript.
	\item[Step 5] Compute zero polynomial evaluation
      $\p{Z_H} (\chz) =\chz^\noofc - 1$.
	\item[Step 6] Compute Lagrange polynomial evaluation
      $\lag_1 (\chz) = \frac{\chz^\noofc -1}{\noofc (\chz - 1)}$.
	\item[Step 7] Compute public input polynomial evaluation
      $\pubinppoly (\chz) = \sum_{i \in \range{1}{\instsize}} \wit_i
      \lag_i(\chz)$.
	\item[Step 8] Compute quotient polynomials evaluations
	\begin{multline*}
    \p{t} (\chz) = \frac{1}{\p{Z_H}(\chz)} \Big(
    \p{r} (\chz) + \pubinppoly(\chz) - (\p{a}(\chz) + \beta \p{S_{\sigma 1}}(\chz) + \gamma) (\p{b}(\chz) + \beta \p{S_{\sigma 2}}(\chz) + \gamma) \\
    (\p{c}(\chz) + \gamma)\p{z}(\chz \omega) \alpha - \lag_1 (\chz) \alpha^2
    \Big) \,.
	\end{multline*}
	\item[Step 9] Compute batched polynomial commitment
	$\gone{D} = v \gone{r} + u \gone {z}$ that is
	\begin{align*}
		\gone{D} & = v
		\left(
		\begin{aligned}
          & \p{a}(\chz)\p{b}(\chz) \cdot \gone{\selmulti} + \p{a}(\chz)  \gone{\selleft} + \p{b}  \gone{\selright} + \p{c}  \gone{\seloutput} + \\
          & + (	(\p{a}(\chz) + \beta \chz + \gamma) (\p{b}(\chz) + \beta k_1 \chz + \gamma) (\p{c} + \beta k_2 \chz + \gamma) \alpha  + \lag_1(\chz) \alpha^2)  + \\
			% &   \\
          & - (\p{a}(\chz) + \beta \p{S_{\sigma 1}}(\chz) + \gamma) (\p{b}(\chz)
          + \beta \p{S_{\sigma 2}}(\chz) + \gamma) \alpha \beta \p{z}(\chz
          \omega) \gone{\p{S_{\sigma 3}}(\chi)})
		\end{aligned}
		\right) + \\
		& + u \gone{\p{z}(\chi)}\,.
	\end{align*}
	\item[Step 10] Computes full batched polynomial commitment $\gone{F}$:
	\begin{align*}
      \gone{F} & = \left(\gone{\p{t_{lo}}(\chi)} + \chz^\noofc \gone{\p{t_{mid}}(\chi)} + \chz^{2 \noofc} \gone{\p{t_{hi}}(\chi)}\right) + u \gone{\p{z}(\chi)} + \\
               & + v
                 \left(
		\begin{aligned}
			& \p{a}(\chz)\p{b}(\chz) \cdot \gone{\selmulti} + \p{a}(\chz)  \gone{\selleft} + \p{b}(\chz)   \gone{\selright} + \p{c}(\chz)  \gone{\seloutput} + \\
			& + (	(\p{a}(\chz) + \beta \chz + \gamma) (\p{b}(\chz) + \beta k_1 \chz + \gamma) (\p{c}(\chz)  + \beta k_2 \chz + \gamma) \alpha  + \lag_1(\chz) \alpha^2)  + \\
			% &   \\
			& - (\p{a}(\chz) + \beta \p{S_{\sigma 1}}(\chz) + \gamma) (\p{b}(\chz) + \beta \p{S_{\sigma 2}}(\chz) + \gamma) \alpha  \beta \p{z}(\chz \omega) \gone{\p{S_{\sigma 3}}(\chi)})
		\end{aligned}
		\right) \\
		& + v^2 \gone{\p{a}(\chi)} + v^3 \gone{\p{b}(\chi)} + v^4 \gone{\p{c}(\chi)} + v^5 \gone{\p{S_{\sigma 1}(\chi)}} + v^6 \gone{\p{S_{\sigma 2}}(\chi)}\,.
	\end{align*}
	\item[Step 11] Compute group-encoded batch evaluation $\gone{E}$
	\begin{align*}
		\gone{E}  = \frac{1}{\p{Z_H}(\chz)} & \gone{
		\begin{aligned}
			& \p{r}(\chz) + \pubinppoly(\chz) +  \alpha^2  \lag_1 (\chz) + \\
			& - \alpha \left( (\p{a}(\chz) + \beta \p{S_{\sigma 1}} (\chz) + \gamma) (\p{b}(\chz) + \beta \p{S_{\sigma 2}} (\chz) + \gamma) (\p{c}(\chz) + \gamma) \p{z}(\chz \omega) \right)
		\end{aligned}
           }\\
      + & \gone{v \p{r}(\chz) + v^2 \p{a}(\chz) + v^3 \p{b}(\chz) + v^4 \p{c}(\chz) + v^5 \p{S_{\sigma 1}}(\chz) + v^6 \p{S_{\sigma 2}}(\chz) + u \p{z}(\chz \omega) }\,.
	\end{align*}
\item[Step 12] Check whether the verification
 % $\vereq_\zkproof(\chi)$
  equation holds
	\begin{multline}
		\label{eq:ver_eq}
		\left( \gone{\p{W_{\chz}}(\chi)} + u \cdot \gone{\p{W_{\chz
                \omega}}(\chi)} \right) \bullet
		\gtwo{\chi} - %\\
		\left( \chz \cdot \gone{\p{W_{\chz}}(\chi)} + u \chz \omega \cdot
          \gone{\p{W_{\chz \omega}}(\chi)} + \gone{F} - \gone{E} \right) \bullet
        \gtwo{1} = 0\,.
	\end{multline}
  The verification equation is a batched version of the verification equation
  from \cite{AC:KatZavGol10} which allows the verifier to check openings of
  multiple polynomials in two points (instead of checking an opening of a single
  polynomial at one point).
\end{description}

\ncase{$\plonk$ simulator $\simulator_\chi(\srs, \td= \chi, \inp)$}\ \newline
The \plonk{} simulator proceeds as an honest prover would, except:
\begin{enumerate}
  \item In the first round, it sets $\wit = (\wit_i)_{i \in \range{1}{3 \noofc}}
    = \vec{0}$, and at random picks $b_1, \ldots, b_9$. Then it proceeds with
    that all-zero witness.
  \item In Round 3, it computes polynomial $\pt(X)$ honestly, however uses
    trapdoor $\chi$ to compute commitments
    $\p{t_{lo}}(\chi), \p{t_{mid}}(\chi), \p{t_{hi}}(\chi)$.
  \end{enumerate}

%  \subsection{Trapdoor-less simulatability of Plonk}
%\label{sec:plonk-TLZK-proof}


%%% Local Variables:
%%% mode: latex
%%% TeX-master: "main"
%%% End:

%% !TEX root = main.tex
% !TEX spellcheck = en-US

\section{Non-malleability of $\sonicprotfs$}
\label{sec:sonic}
\subsection{\sonic{} protocol rolled out}
In this section we present $\sonic$'s constraint system and algorithms. Reader
familiar with them may jump directly to the next section.

%\hamid{}{we should put the following figure \ref{fig:pcoms} somewhere in this section.}
 \begin{figure}[h!]
 \centering
 	\begin{pcvstack}[center,boxed]
 		\begin{pchstack}
 			\procedure{$\kgen(\secparam, \maxdeg)$} {
 				\alpha, \chi \sample \FF^2_p \\ [\myskip]
 				\pcreturn \gone{\smallset{\chi^i}_{i = -\multconstr}^{\multconstr},
           \smallset{\alpha \chi^i}_{i = -\multconstr, i \neq
             0}^{\multconstr}},\\
         \pcind \gtwo{\smallset{\chi^i, \alpha \chi^i}_{i =
             -\multconstr}^{\multconstr}}, \gtar{\alpha}\\
 				%\markulf{03.11.2020}{} \\
 			%	\hphantom{\pcind \p{o}_i(X) \gets \sum_{j = 1}^{t_i} \gamma_i^{j - 1} \frac{\p{f}_{i,j}(X) - \p{f}_{i, j}(z_i)}{X - z_i}}
 				\hphantom{\hspace*{5.5cm}}
 		}

 			\pchspace

 			\procedure{$\com(\srs, \maxconst, \p{f}(X))$} {
 				\p{c}(X) \gets \alpha \cdot X^{\dconst - \maxconst} \p{f}(X) \\ [\myskip]
 				\pcreturn \gone{c} = \gone{\p{c}(\chi)}\\ [\myskip]
 				\hphantom{\pcind \pcif \sum_{i = 1}^{\abs{\vec{z}}} r_i \cdot
           \gone{\sum_{j = 1}^{t_j} \gamma_i^{j - 1} c_{i, j} - \sum_{j = 1}^{t_j}
             s_{i, j}} \bullet \gtwo{1} + } }
 		\end{pchstack}
 		% \pcvspace

 		\begin{pchstack}
 			\procedure{$\open(\srs, z, s, f(X))$}
 			{
 				\p{o}(X) \gets \frac{\p{f}(X) - \p{f}(z)}{X - z}\\ [\myskip]
 				\pcreturn \gone{\p{o}(\chi)}\\ [\myskip]
 				\hphantom{\hspace*{5.5cm}}
 			}

 			\pchspace

 			\procedure{$\verify(\srs, \maxconst, \gone{c}, z, s, \gone{\p{o}(\chi)})$}
       {
         \pcif \gone{\p{o}(\chi)} \bullet \gtwo{\alpha \chi} + \gone{s - z
         \p{o}(\chi)} \bullet \gtwo{\alpha} = \\ [\myskip] \pcind \gone{c}
         \bullet \gtwo{\chi^{- \dconst + \maxconst}} \pcthen  \pcreturn 1\\
         [\myskip]
         \rlap{\pcelse \pcreturn 0.} \hphantom{\pcind \pcif \sum_{i =
             1}^{\abs{\vec{z}}} r_i \cdot \gone{\sum_{j = 1}^{t_j} \gamma_i^{j -
               1} c_{i, j} - \sum{j = 1}^{t_j} s_{i, j}} \bullet \gtwo{1} + } }
 		\end{pchstack}
 	\end{pcvstack}

 	\caption{$\PCOMs$ polynomial commitment scheme.}
 	\label{fig:pcoms}
 \end{figure}



\oursubsub{The constraint system}
\label{sec:sonic_constraint_system}
\cref{fig:pcoms} presents a variant of KZG~\cite{AC:KatZavGol10} polynomial commitment schemes
used in \sonic{}. \sonic's system of constraints composes of three $\multconstr$-long vectors
$\va, \vb, \vc$ which corresponds to left and right inputs to multiplication
gates and their outputs. It hence holds $\va \cdot \vb = \vc$.

There is also $\linconstr$ linear constraints of the form
\[
  \va \vec{u_q} + \vb \vec{v_q} + \vc \vec{w_q} = k_q,
\]
where $\vec{u_q}, \vec{v_q}, \vec{w_q}$ are vectors for the $q$-th linear
constraint with instance value $k_q \in \FF_p$. Furthermore define polynomials
\begin{equation}
  \begin{split}
    \p{u_i}(Y) & = \sum_{q = 1}^\linconstr Y^{q + \multconstr} u_{q, i}\,,\\
    \p{v_i}(Y) & = \sum_{q = 1}^\linconstr Y^{q + \multconstr} v_{q, i}\,,\\
  \end{split}
  \qquad
  \begin{split}
    \p{w_i}(Y) & = -Y^i - Y^{-i} + \sum_{q = 1}^\linconstr Y^{q +
      \multconstr} w_{q, i}\,,\\
    \p{k}(Y) & = \sum_{q = 1}^\linconstr Y^{q + \multconstr} k_{q}.
  \end{split}
\end{equation}

$\sonic$ constraint system requires that
\begin{align}
  \label{eq:sonic_constraint}
  \vec{a}^\top \cdot \vec{\p{u}} (Y) + \vec{b}^\top \cdot \vec{\p{v}} (Y) +
  \vec{c}^\top \cdot \vec{\p{w}} (Y) + \sum_{i = 1}^{\multconstr} a_i b_i (Y^i +
  Y^{-i}) - \p{k} (Y) = 0.
\end{align}

In \sonic{} we will use commitments to the following polynomials.
\begin{align*}
  \pr(X, Y) & = \sum_{i = 1}^{\multconstr} \left(a_i X^i Y^i + b_i X^{-i} Y^{-i}
              + c_i X^{-i - \multconstr} Y^{-i - \multconstr}\right) \\
  \p{s}(X, Y) & = \sum_{i = 1}^{\multconstr} \left( u_i (Y) X^{-i} +
                v_i(Y) X^i + w_i(Y) X^{i + \multconstr}\right)\\
  \pt(X, Y) & = \pr(X, 1) (\pr(X, Y) + \p{s}(X, Y)) - \p{k}(Y)\,.
\end{align*}

Polynomials $\p{r} (X, Y), \p{s} (X, Y), \p{t} (X, Y)$ are designed such that
$\p{t} (0, Y) = \vec{a}^\top \cdot \vec{\p{u}} (Y) + \vec{b}^\top \cdot
\vec{\p{v}} (Y) + \vec{c}^\top \cdot \vec{\p{w}} (Y) + \sum_{i =
  1}^{\multconstr} a_i b_i (Y^i + Y^{-i}) - \p{k} (Y) $. That is, the prover is
asked to show that $\p{t} (0, Y) = 0$, cf.~\cref{eq:sonic_constraint}.

Furthermore, the commitment system in $\sonic$ is designed such that it is
infeasible for a $\ppt$ algorithm to commit to a polynomial with non-zero
constant term.

\oursubsub{Algorithms rolled out}
\ourpar{$\sonic$ SRS generation $\kgen(\REL)$.} The SRS generating algorithm picks
randomly $\alpha, \chi \sample \FF_p$ and outputs
	\[
      \srs = \left( \gone{\smallset{\chi^i}_{i = -\dconst}^{\dconst},
          \smallset{\alpha \chi^i}_{i = -\dconst, i \neq 0}^{\dconst}},
        \gtwo{\smallset{\chi^i, \alpha \chi^i}_{i = - \dconst}^{\dconst}},
        \gtar{\alpha} \right)
	\]
\ourpar{$\sonic$ prover $\prover(\srs, \inp, \wit=\va, \vb, \vc)$.}
\begin{description}
\item[Message 1] The prover picks randomly randomisers
  $c_{\multconstr + 1}, c_{\multconstr + 2}, c_{\multconstr + 3}, c_{\multconstr
    + 4} \sample \FF_p$. Sets
  $\pr(X, Y) \gets \pr(X, Y) + \sum_{i = 1}^4 c_{\multconstr + i} X^{- 2
    \multconstr - i}$. Commits to $\pr(X, 1)$ and outputs
  $\gone{r} \gets \com(\srs, \multconstr, \pr(X, 1))$.  Then it computes challenge $y = \ro(\zkproof[0..1])$.
\item[Message 2] $\prover$ commits to $\pt(X, y)$ and outputs
  $\gone{t} \gets \com(\srs, \dconst, \pt(X, y))$. Then it gets a challenge $z = \ro(\zkproof[0..2])$.
\item[Message 3] The prover computes commitment openings. That is, it outputs
  \begin{align*}
    \gone{o_a} & = \open(\srs, z, \pr(z, 1), \pr(X, 1)) \\
    \gone{o_b} & = \open(\srs, yz, \pr(yz, 1), \pr(X, 1)) \\
    \gone{o_t} & = \open(\srs, z, \pt(z, y), \pt(X, y)) 
  \end{align*}
  along with evaluations $a' = \pr(z, 1), b' = \pr(y, z), t' = \pt(z, y)$.  Then it
  engages in the signature of correct computation playing the role of the
  helper, i.e.~it commits to $\p{s}(X, y)$ and sends the commitment $\gone{s}$, commitment opening
  \begin{align*}
    \gone{o_s} & = \open(\srs, z, \p{s}(z, y), \p{s}(X, y)), \\
  \end{align*} and $s'=\p{s}(z, y)$. 
%
  Then
  it obtains a challenge $u = \ro(\zkproof[0..3])$.
\item[Message 4] For the next message the prover computes
  $\gone{c} \gets \com(\srs, \dconst, \p{s}(u, Y))$ and
  computes commitments' openings
  \begin{align*}
    \gone{w} & = \open(\srs, u, \p{s}(u, y), \p{s}(X, y)), \\
    \gone{q_y} & = \open(\srs, y,\p{s}(u, y), \p{s}(u, Y)),
  \end{align*}
  and returns $\gone{w}, \gone{q_y}, s = \p{s}(u, y)$. Eventually the prover gets the last challenge
  $z' = \ro(\zkproof[0..4])$.
\item[Message 5] For the final message, $\prover$ computes opening
  $\gone{q_{z'}} = \open(\srs, z', \p{s}(u, z'), \p{s}(u, X))$ and outputs $\gone{q_{z'}}$.
\end{description}

\ourpar{$\sonic$ verifier $\verifier(\srs, \inp, \zkproof)$.} The verifier
in \sonic{} runs as subroutines the verifier for the polynomial commitment. That
is it sets $t' = a'(b' + s') - \p{k}(y)$ and checks the following:
\begin{equation*}
  \begin{split}
    &\PCOMs.\verifier(\srs, \multconstr, \gone{r}, z, a', \gone{o_a}), \\
    &\PCOMs.\verifier(\srs, \multconstr, \gone{r}, yz, b', \gone{o_b}),\\
    &\PCOMs.\verifier(\srs, \dconst, \gone{t}, z, t', \gone{o_t}),\\
    &\PCOMs.\verifier(\srs, \dconst, \gone{s}, z, s', \gone{o_s}),\\
  \end{split}
  \qquad
  \begin{split}
    &\PCOMs.\verifier(\srs, \dconst, \gone{s}, u, s, \gone{w}),\\
    &\PCOMs.\verifier(\srs, \dconst, \gone{c}, y, s, \gone{q_y}),\\
    &\PCOMs.\verifier(\srs, \dconst, \gone{c}, z', \p{s}(u, z'), \gone{q_{z'}}),
  \end{split}
\end{equation*}
and accepts the proof iff all the checks holds. Note that the value
$\p{s}(u, z')$ that is recomputed by the verifier uses separate challenges $u$
and $z'$. This enables the batching of many proof and outsourcing of this
part of the proof to an untrusted helper.

\subsection{Unique opening property of $\PCOMs$}
\begin{lemma}
\label{lem:pcoms_unique_op}
$\PCOMs$ has the unique opening property in the AGM. 
\end{lemma}
\begin{proof}
Let 
$z \in \FF_p$ be the attribute the polynomial is evaluated at,
$\gone{c} \in \GRP$ be the commitment,  
$s \in \FF_p$ the evaluation value, and 
$o \in \GRP$ be the commitment opening. 
We need to show that for every $\ppt$ adversary $\adv$ probability
\[
  \Pr \left[
    \begin{aligned}
      & \verify(\srs, \gone{c}, z, s, \gone{o}) = 1, \\
      & \verify(\srs, \gone{c}, z, \tilde{s}, \gone{\tilde{o}}) = 1
    \end{aligned}
    \,\left|\, \vphantom{\begin{aligned}
          & \verify(\srs, \gone{c}, z, s, \gone{o}),\\
          & \verify(\srs, \gone{c}, z, s, \gone{\tilde{o}}) \\
          &o \neq \tilde{o})
		\end{aligned}}
      \begin{aligned}
        %& \srs \gets \kgen(\secparam, \maxdeg), \\
        & (\gone{c}, z, s, \gone{o}, \gone{\tilde{o}}) \gets \adv^{\initU}(1^\secpar, \maxdeg)
      \end{aligned}
    \right.\right]
  % \leq \negl.
\]
is at most negligible.

As noted in \cite[Lemma 2.2]{EPRINT:GabWilCio19} it is enough to upper bound the
probability of the adversary succeeding using the idealised verification
equation---which considers equality between polynomials---instead of the real
verification equation---which considers equality of the polynomials' evaluations.

For a polynomial $f$, its degree upper bound $\maxconst$, evaluation point $z$,
evaluation result $s$, and opening $\gone{o(X)}$ the idealised check verifies that
\begin{equation}
  \alpha (X^{\dconst - \maxconst}f(X) \cdot X^{-\dconst + \maxconst} -  s) \equiv \alpha \cdot o(X) (X - z)\,,
\end{equation}
what is equivalent to 
\begin{equation}
	f(X) -  s \equiv o(X) (X - z)\,.
	\label{eq:pcoms_idealised_check}
\end{equation}
Since $o(X)(X - z) \in \FF_p[X]$ then from the uniqueness of polynomial
composition, there is only one $o(X)$ that fulfils the equation above.
\qed
\end{proof}


\subsection{Unique response property}
The unique response property of $\sonicprot$ follows from the unique opening
property of the used polynomial commitment scheme $\PCOMs$.
\begin{lemma}
\label{lem:sonicprot_ur}
If a polynomial commitment scheme $\PCOMs$ is evaluation binding with parameter
$\epsbind (\secpar)$ and has unique openings property with parameter $\epsop(\secpar)$,
$\sonicprotfs$ is $\epss (\secpar)$-sound and $(\dconst, \dconst)$-$\ldlog$ problem is
$\epsldlog (\secpar)$-hard, then $\sonicprot$ is $\ur{1}$ against algebraic adversaries with security loss
  \[
    3 \cdot \epsop (\secpar) + 2 \cdot (\epsbind (\secpar) + \epss (\secpar) +
    \epsdlog (\secpar)).
  \]
\end{lemma}

\changedm{
\begin{proof}
  Let $\adv$ be an algebraic adversary tasked to break the $\ur{1}$-ness of
  $\sonicprotfs$. We show that the first prover's message determines, along with
  the verifiers challenges, the rest of it.  This is done by game hops. In the games,
  the adversary outputs two proofs $\zkproof^0$ and $\zkproof^1$ for the same statement.
  To distinguish polynomials and commitments which an honest prover sends in the
  proof from the polynomials and commitments computed by the adversary we write the
  latter using indices $0$ and $1$ (two indices as we have two transcripts), e.g.~to
  describe the quotient polynomial provided by the adversary we write $\p{t}^0$ and
  $\p{t}^1$ instead of $\p{t}$ as in the description of the protocol.

  \ngame{0} In this game, the adversary additionally wins if it provides two transcripts that
  match on all $5$ messages sent by the prover.
  In this game the adversary cannot win.

  \ngame{1} This game is identical to Game $\game{0}$ except that now the
  adversary additionally wins if it provides two transcripts that matches on the first four
  messages of the proof.

  \ncase{Game 0 to Game 1} We show that the probability that $\adv$
  wins in one game but does not in the other is negligible.  Observe that in
  after its $4$-th message, the adversary is given a challenge $z'$ and has to open
  commitment to $\p{s} (u, z')$. Hence, to be able to give two different
  openings in its $5$-th message, $\adv$ has to break the unique opening property of the
  KZG commitment scheme which happens with probability $\epsop (\secpar)$ tops.

  \ngame{2} This game is identical to Game $\game{1}$ except that now the
  adversary additionally wins if it provides two transcripts that matches on the
  first three messages of the proof.

  \ncase{Game 2 to Game 3} In its $4$-th message the adversary computes evaluation
  $s = \p{s} (u, y)$ and the corresponding openings $\gone{w}, \gone{q_y}$. The adversary
  cannot provide two different evaluations for the committed polynomials, since that would
  require breaking the evaluation binding property, which happens (by the union bound)
  with probability at most $2 \cdot \epsbind (\secpar)$. Also, the adversary cannot provide two
  different yet valid opening except probability $2 \cdot \epsop (\secpar)$

  Hence, the probability that adversary wins in one game but does not in the
  other is upper-bounded by $2 \cdot (\epsbind (\secpar) + \epsop (\secpar))$

  \ngame{4} This game is identical to Game $\game{3}$ except that now the
  adversary additionally wins if it provides two transcripts that matches on the
  first two messages of the proof.

  \ncase{Game 3 to Game 4} In its $3$-rd message the adversary computes $4$ polynomial
  evaluations and their openings. It also sends commitment to $\p{s} (X, y)$ which is a
  signature of correct computation.

  Probability that the adversary provides different evaluations or polynomial openings is
  upper bounded by $4 \cdot (\epsop (\secpar) + \epsbind (\secpar))$. Since the polynomial
  commitment scheme is deterministic, if commitment to $\p{s^0} (X, y)$ does not equal
  commitment to $\p{s^1} (X, y)$, then at least one of these values has been computed
  incorrectly. Probability that the adversary outputs an acceptable proof, where signature
  of correct computation has been computed incorrectly is upper-bounded by $\epss
  (\secpar) + \epsdlog (\secpar)$, cf.~\cref{lem:plonkprot_ur}.

  \ncase{Game 5}  This game is identical to Game $\game{4}$ except that now the
  adversary additionally wins if it provides two transcripts that matches on the
  first messages of the proof.

  \ncase{Game 4 to Game 5} In its second message the adversary commits to polynomial
  $\p{t} (X, y)$. Since the commitment scheme is deterministic, probability that adversary
  outputs acceptable proofs where commitment to $\p{t^0} (X, y)$ does not equal commitment
  to $\p{t^1} (X)$ is upper-bounded by $\epss (\secpar) + \epsdlog (\secpar)$, cf.~\cref{lem:plonkprot_ur}.

  \ncase{Conclusion} Taking all the games together, probability that $\adv$ wins
  in Game 5 is upper-bounded by
  \[
    3 \cdot \epsop (\secpar) + 2 \cdot (\epsbind (\secpar) + \epss (\secpar) +
    \epsdlog (\secpar)).
  \]
  \qed
\end{proof}
}

\COMMENT{
  \michals{1.11}{Old proof below}
\begin{proof}
  Let $\adv$ be an adversary that breaks $\ur{1}$-ness of $\sonicprot$.  We
  consider two cases, depending on which message $\adv$ is able to provide at
  least two different outputs such that the resulting transcripts are
  acceptable.  For the first case we show that $\adv$ can be used to break the
  evaluation binding property of $\PCOMs$, while for the second case we show
  that it can be used to break the unique opening property of $\PCOMs$.

  The proof goes similarly to the proof of \cref{lem:plonkprot_ur} thus we
  provide only draft of it here.  For $i$-th message ($i > 1$) the prover
  either commits to some well-defined polynomials (deterministically), evaluates
  these on randomly picked points, or shows that the evaluations were performed
  correctly.  Obviously, for a committed polynomial $\p{p}$ evaluated at point
  $x$ only one value $y = \p{p}(x)$ is correct. If the adversary was able to
  provide two different values $y$ and $\tilde{y}$ that would be accepted as an
  evaluation of $\p{p}$ at $x$ then the $\PCOMs$'s evaluation binding would be
  broken.  Alternatively, if $\adv$ was able to provide two openings $\p{W}$ and
  $\p{\tilde{W}}$ for $y = \p{p}(x)$ then the unique opening property would be
  broken.
%
Hence the probability that $\adv$ breaks $\ur{1}$-property of $\PCOMs$ is
upper-bounded by $\epsbind(\secpar) + \epsop(\secpar)$. 
\qed

\end{proof}
}

\subsection{rewinding-based knowledge soundness}
\begin{lemma}
	\label{lem:sonicprot_ss}
	$\sonicprot$ is $(2, \multconstr + \linconstr + 1)$-computational special sound with security loss $(\epst (\accProb, \secpar), \epss(\secpar))$ against
	algebraic adversaries, where
  \[
    \epst(\accProb, \secpar) \leq \frac{\accProb - (q + 1) \epsid (\secpar)}{1 - \epsid (\secpar)}\,,
  \]
  and
	\[
	  \epss(\secpar) \leq \epsid(\secpar) + \epsldlog(\secpar) \,.
	\]
	Here $\accProb$ is a probability that the adversary outputs an acceptable proof, $q$ is the upper bound for a number of random oracle queries the adversary makes, $\epsid(\secpar)$ is a soundness error of the idealized verifier, and $\epsldlog(\secpar)$ is security of $(\dconst, \dconst)$-$\ldlog$ assumption.
\end{lemma}
\begin{proof}
	Similarly as in the case of $\plonk$, the main idea of the proof is to show that an
  adversary who breaks rewinding-based knowledge soundness can be used to break a $\dlog$
  problem instance. The proof goes by game hops. Let $\tree$ be the tree produced by
  $\tdv$ by rewinding $\adv$. Note that since the tree branches after prover's $2$-nd
  message, the instance $\inp$, commitments
  $\gone{\p{r} (\chi, 1), \p{r} (\chi, y), \p{s} (\chi, y), \p{t} (\chi, y)}$, and
  challenge $y$ are the same. The tree branches after the $2$-nd message of the
  prover when the challenge $z$ is presented, thus tree $\tree$ is build using
  different values of $z$.
	%
	We consider the following games.
	
	\ncase{Game 0} In this game the adversary wins if all the transcripts it
	produced are acceptable by the ideal verifier,
	i.e.~$\vereq_{\inp, \zkproof}(X) = 0$, cf.~\cref{eq:ver_eq}, and none of
	commitments
	$\gone{\p{r} (\chi, 1), \p{r} (\chi, y), \p{s} (\chi, y), \p{t} (\chi, y)}$ use
	elements from a simulated proof, and the extractor fails to extract a valid
	witness out of the proof.
	
	\ncase{Probability that $\adv$ wins Game 0 is negligible} Probability of
	$\adv$ winning this game is $\epsid(\secpar)$ as the protocol $\sonicprot$,
	instantiated with the idealised verification equation, is perfectly
	knowledge sound except with negligible probability of the idealised verifier
	failure $\epsid(\secpar)$. Hence for a valid proof $\zkproof$ for a
	statement $\inp$ there exists a witness $\wit$, such that $\REL(\inp, \wit)$
	holds. Note that since the $\tdv$ produces $(\multconstr + \linconstr + 1)$
	acceptable transcripts for different challenges $z$. As noted in
	\cite{CCS:MBKM19} this assures that the correct witness is encoded in
	$\p{r} (X, Y)$. Hence $\extt$ can recreate polynomials' coefficients by
	interpolation and reveal the witness with probability $1$. Moreover, the
	probability that extraction fails in that case is upper-bounded by
	probability of an idealised verifier failing $\epsid(\secpar)$, which is
	negligible.
	
	\ncase{Game 1} In this game the adversary additionally wins if it produces a
	transcript in $\tree$ such that $\vereq_{\inp, \zkproof}(\chi) = 0$, but
	$\vereq_{\inp, \zkproof}(X) \neq 0$, and none of commitments
	$\gone{\p{r} (\chi, 1), \p{r} (\chi, y), \p{s} (\chi, y), \p{t} (\chi, y)}$
	use elements from a simulated proof.  The first condition means that the
	ideal verifier does not accept the proof, but the real verifier does.
	
	\ncase{Game 0 to Game 1} Assume the adversary wins in Game 1, but does not
	win in Game 0. We show that such adversary may be used to break an
	instance of a $\ldlog$ assumption. More precisely, let $\tdv$ be an
	algorithm that for relation $\REL$ and randomly picked
	$\srs \sample \kgen(\REL)$ produces a tree of acceptable transcripts such
	that the winning condition of the game holds. Let $\rdvdlog$ be a
	reduction that gets as input an
	$(\dconst, \dconst)$-$\ldlog$ instance
	$\gone{\chi^{-\dconst}, \ldots, \chi^{\dconst}}, \gtwo{\chi^{-\dconst},
		\ldots, \chi^{\dconst}}$ and is tasked to output $\chi$.
	
	The reduction $\rdvdlog$ proceeds as follows.
	\begin{enumerate}
  \item Pick a random $\alpha$ and compute
    $\gone{\alpha \chi^{- \dconst}, \ldots, \alpha \chi^{-1}, \alpha \chi,
      \ldots, \alpha \chi^{\dconst}}$,
    $\gtwo{\alpha \chi^{- \dconst}, \ldots, \alpha \chi^{-1}, \alpha \chi,
      \ldots, \alpha \chi^{\dconst}}$. Set a SRS $\srs$ to be the
    $(\dconst, \dconst)$-$\ldlog$ instance and its multiplication with $\alpha$ as
    computed above.
    % \hamid{Is this clear?}
  \item Build $\sonicprot$'s SRS in the updatable setting by answering $\adv$'s
    queries for SRS updates and setting the honest update of the SRS to be
    $\srs$. Let $\srs'$ be the finalised SRS.
		\item Let $(1, \tree)$ be the output returned by $\tdv$. Let $\inp$ be a
		relation proven in $\tree$.  Consider a transcript $\zkproof \in \tree$ such
		that $\vereq_{\inp, \zkproof}(X) \neq 0$, but
		$\vereq_{\inp, \zkproof}(\chi') = 0$. Since $\adv$ is algebraic, all group
		elements included in $\tree$ are extended by their representation as a
		combination of the input $\GRP_1$-elements. Hence, all coefficients of the
		verification equation polynomial $\vereq_{\inp, \zkproof}(X)$ are known.
		\item Find $\vereq_{\inp, \zkproof}(X)$ zero points and find $\chi'$ among
		them.
  \item Let $\chi_1, \ldots, \chi_\ell$ be the partial trapdoors of $\adv$'s SRS
    updates, extracted by the reduction from the update proofs given by $\adv$.
		\item Return  $\chi = \chi' (\chi_1 \chi_2 \ldots \chi_\ell)^{-1}$.
	\end{enumerate}
	Hence, the probability that the adversary wins Game 1 is upper-bounded by
	$\epsldlog(\secpar)$.
\end{proof}

\subsection{Trapdoor-less simulatability of Sonic}
\begin{lemma}
\label{lem:sonic_hvzk}
$\sonic$ is 2-programmable trapdoor-less simulatable.
\end{lemma}
\begin{proof}
  The simulator proceeds as follows.
  \begin{enumerate}
  \item Pick randomly vectors $\vec{a}$, $\vec{b}$ and set
    \begin{equation}
      \label{eq:ab_eq_c}
      \vec{c} = \vec{a} \cdot \vec{b}. 
    \end{equation}
  \item Pick randomisers $c_{\multconstr + 1}, \ldots, c_{\multconstr + 4}$,
    honestly compute polynomials $\p{r}(X, Y), \p{r'}(X, Y), \p{s}(X, Y)$ and
    pick randomly challenges $y$, $z$.
  \item Output commitment $\gone{r} \gets \com(\srs, \multconstr, \p{r} (X,
    1))$ and challenge $y$. 
  \item Compute
    \begin{align*}
      & a' = \p{r}(z, 1),\\
      & b' = \p{r}(z, y),\\
      & s' = \p{s}(z, y).
    \end{align*} 
  \item Pick polynomial $\p{t}(X, Y)$ such that
    \begin{align*}
      & \p{t} (X, y) = \p{r} (X, 1) (\p{r}(X, y) + \p{s} (X, y)) - \p{k} (Y)\\
      & \p{t} (0, y) = 0
    \end{align*}
  \item Output commitment $\gone{t} = \com (\srs, \dconst, \p{t} (X, y))$ and
    challenge $z$.
  \item Continue following the protocol.
  \end{enumerate}

  We note that the simulation is perfect. This comes since, except polynomial
  $\p{t} (X, Y)$ all polynomials are computed following the protocol. For
  polynomial $\p{t} (X, Y)$ we observe that in a case of both real and simulated
  proof the verifier only learns commitment $\gone{t} = \p{t} (\chi, y)$ and
  evaluation $t' = \p{t} (z, y)$. Since the simulator picks $\p{t} (X, Y)$ such
  that 
  \begin{align*}
      \p{t} (X, y) = \p{r} (X, 1) (\p{r}(X, y) + \p{s} (X, y)) - \p{k} (Y)
  \end{align*}
  Values of $\gone{t}$ are equal in both proofs.
  Furthermore, the simulator picks its polynomial such that $\p{t}(0, y) = 0$,
  hence it does not need the trapdoor to commit to it. (Note that the proof
  system's SRS does not allow to commit to polynomials which have non-zero
  constant term). \qed
\end{proof}
\begin{remark} 
  As noted in \cite{CCS:MBKM19}, $\sonic$ is statistically subversion-zero
  knowledge (Sub-ZK). As noted in \cite{AC:ABLZ17}, one way to achieve
  subversion zero knowledge is to utilise an extractor that extracts a SRS
  trapdoor from a SRS-generator. Unfortunately, a NIZK made subversion
  zero-knowledge by this approach cannot achieve perfect Sub-ZK as one has to
  count in the probability of extraction failure. However, with the simulation
  presented in \cref{lem:sonic_hvzk}, the trapdoor is not required for the
  simulator as it is able to simulate the execution of the protocol just by
  picking appropriate (honest) verifier's challenges. This result transfers to
  $\sonicprotfs$, where the simulator can program the random oracle to provide
  challenges that fits it.
\end{remark}

\subsection{From rewinding-based knowledge soundness and unique response property to \COMMENT{forking
  }simulation extractability of $\sonicprotfs$}
Since \cref{lem:sonicprot_ur,lem:sonicprot_ss,lem:sonic_hvzk} hold, $\sonicprot$ is $\ur{1}$, computational special sound and trapdoor-less simulatable. We now make use
of \cref{thm:se} and show that $\sonicprotfs$ is \COMMENT{ forking }simulation-extractable as defined in \cref{def:simext}.

\begin{corollary}[\COMMENT{Forking s}Simulation extractability of $\sonicprotfs$]
  \label{thm:sonicprotfs_se}
  Assume that $\sonicprot$ is $\ur{1}$ with security
  $\epsur(\secpar) = \epsbind(\secpar) + \epsop(\secpar)$ -- where
  $\epsbind (\secpar)$ is polynomial commitment's binding security, $\epsop$ is
  polynomial commitment unique opening security -- and computational special sound with
  security $\epss(\secpar)$. Let $\ro\colon \bin^* \to \bin^\secpar$ be a
  random oracle. Let $\advse$ be an adversary that can make up to $q$
  random oracle queries, and outputs an
  acceptable proof for $\sonicprotfs$ with probability at least $\accProb$. Then
  $\sonicprotfs$ is \COMMENT{forking }simulation-extractable with extraction error
  $\eta = \epsur(\secpar)$. The extraction probability $\extProb$ is at least
\[
		\extProb  \geq \frac{1}{q^{\multconstr + \linconstr}} (\accProb - \epsur(\secpar))^{\multconstr +
		\linconstr + 1} - \eps(\secpar).
	\]
	for some negligible $\eps(\secpar)$, $\multconstr$ and $\linconstr$ being,
  respectively, the number of multiplicative and linear constraints of the system.
\end{corollary}

%%% Local Variables:
%%% mode: latex
%%% TeX-master: "main"
%%% End:

%% !TEX root = main.tex
% !TEX spellcheck = en-US

\section{Non-malleability of Marlin}
We show that $\marlin$ is \COMMENT{forking }simulation-extractable. To that end, we show
that $\marlin$ has all the required properties: has unique response property, is
rewinding-based knowledge sound, and its simulator can provide indistinguishable proofs
without a trapdoor, just by programming the random oracle.

\subsection{$\marlin$ Protocol Rolled-out}
$\marlin$ uses R1CS as arithmetization method. That is, the prover given
instance $\inp$ and witness $\wit$ and $|\HHH| \times |\HHH|$ matrices $\vec{A},
\vec{B}, \vec{C}$ shows that $\vec{A} (\inp^\top, \wit^\top)^\top \circ \vec{B}
(\inp^\top, \wit^\top)^\top = \vec{C} (\inp^\top, \wit^\top)^\top$. (Here
$\circ$ is a entry-wise product.)

We assume that the matrices have at most $|\KKK|$ non-zero entries. Obviously,
$|\KKK| \leq |\HHH|^2$. Let $b = 3$, the upper-bound of polynomial evaluations
the prover has to provide for each of the sent polynomials.  Denote by $\dconst$
an upper-bound for $\smallset{|\HHH| + 2b -1, 2 |\HHH| + b - 1, 6 |\KKK| - 6}$.

The idea of showing that the constraint system is fulfilled is as
follows. Denote by $\vec{z} = (\inp, \wit)$. The prover computes polynomials
$\p{z_A} (X), \p{z_B} (X), \p{z_C} (X)$ which encode vectors
$\vec{A} \vec{z}, \vec{B} \vec{z}, \vec{C} \vec{z}$ and have degree $<
|\HHH|$. Importantly, when constraints are fulfilled,
$ \p{z_A} (X) \p{z_B} (X) - \p{z_C} (X) = \p{h_0} (X) \ZERO_\HHH (X)$, for some
$\p{h_0} (X)$ and vanishing polynomial $\ZERO_\HHH (X)$. The prover sends
commitments to these polynomials and shows that they have been computed
correctly. More precisely, it shows that
\begin{equation}
  \label{eq:marlin_eq_2}
\forall \vec{M} \in \smallset{\vec{A}, \vec{B}, \vec{C}},  \forall \kappa \in \HHH,
\p{z_M} (\kappa) = \sum_{\iota \in \HHH} \vec{M}[\kappa, \iota] \p{z}(\iota).
\end{equation}

The ideal verifier checks the following equalities
\begin{equation}
  \label{eq:marlin_ver_eq}
  \begin{aligned}
    \p{h}_3 (\beta_3) \ZERO_\KKK (\beta_3) & = \p{a} (\beta_3) - \p{b} (\beta_3)
    (\beta_3 \p{g_3} (\beta_3) + \sigma_3 / |\KKK|)\\
    \p{r}(\alpha, \beta_2) \sigma_3 & = \p{h_2} (\beta_2) \ZERO_\HHH (\beta_2) +
    \beta_2 \p{g2} (\beta_2) + \sigma_2/|\HHH|\\
    \p{s}(\beta_1) + \p{r}(\alpha, \beta_1) (\sum_M \eta_M \p{z_M} (\beta_1)) -
    \sigma_2 \p{z} (\beta_1) & = \p{h_1} (\beta_1) \ZERO_\HHH (\beta_1) +
    \beta_1
    \p{g_1} (\beta_1) + \sigma_1/|\HHH| \\
    \p{z_A} (\beta_1) \p{z_B} (\beta_1) - \p{z_C} (\beta_1) & = \p{h_0}
    (\beta_1) \ZERO_\HHH (\beta_1)
  \end{aligned}
\end{equation}
where $\p{g_i} (X), \p{h_i} (X)$, $i \in \range{1}{3}$,
$\p{a} (X), \p{b} (X), \sigma_1, \sigma_2, \sigma_3$ are polynomials and
variables required by the sumcheck protocol which allows verifier to efficiently
verify that \cref{eq:marlin_eq_2} holds.
                         

\subsection{Unique Response Property}
\begin{lemma}\label{lem:marlinprot_ur}
  Let $\PCOM$ be a commitment of knowledge that is evaluation binding with security loss 
  $\epsbind(\secpar)$ and has unique opening property with security loss
  $\epsop(\secpar)$. Then
  $\marlinprotfs$ is $\ur{2}$ against algebraic adversaries with security loss $2 \cdot \epsbinding (\secpar) + \epsop (\secpar)$.
\end{lemma}

\begin{proof}
	The proof is similar to the proof of \cref{lem:plonkprot_ur} and \cref{lem:sonicprot_ur}.
	An adversary who can break the $2$-unique response property of $\marlinprotfs$ can be either used to break the commitment scheme's evaluation binding or unique opening property. The former happens with the probability upper-bounded by $2 \cdot \epsbinding (\secpar)$, the latter with probability at most $\epsop (\secpar)$.
	By the union bound, the adversary is able to break the unique response property with probability upper bounded by $2 \cdot \epsbinding (\secpar) + \epsop (\secpar)$.
	\qed
	\end{proof}
%  \changedm{$11 \cdot (\epsop (\secpar)) + 6 \cdot \epsbind (\secpar) +
%    \infrac{3}{\abs{\FF_p}}.$} \mxout{$6 \cdot (\epsbind + \epsop + \epsk) + \epss$}


\COMMENT{\changedm{
\begin{proof}
  Let $\adv$ be an algebraic adversary tasked to break the $\ur{2}$-ness of
  $\marlinprotfs$. We show that the first prover's message determines, along with
  the verifiers challenges, the rest of it.  This is done by game hops. In the games,
  the adversary outputs two proofs $\zkproof^0$ and $\zkproof^1$ for the same statement.
  To distinguish polynomials and commitments which an honest prover sends in the
  proof from the polynomials and commitments computed by the adversary we write the
  latter using indices $0$ and $1$ (two indices as we have two transcripts), e.g.~to
  describe a polynomial provided by the adversary we write $\p{f}^0$ and
  $\p{f}^1$ instead of $\p{f}$ as in the description of the protocol.

  \ngame{0} In this game, the adversary additionally wins if it provides two transcripts that
  match on all $5$ messages sent by the prover.
  Obviously, in this game the adversary cannot win.

  \ngame{1} This game is identical to Game $\game{0}$ except that now the
  adversary additionally wins if it provides two transcripts that matches on the first four
  messages of the proof.

  \ncase{Game 0 to Game 1} We show that the probability that $\adv$ wins in one game but
  does not in the other is negligible.  Observe that in its $4$-th message, the
  adversary is given a challenge $\beta_3$ and has to open the previously computed
  commitments for polynomials $\p{g_3} (X), \p{h_3} (X)$. Since the transcripts match up
  to $\adv$'s $4$-th message, the challenge is the same in both.  Also, the adversary
  evaluates and opens evaluations of polynomials $\p{g_2} (X), \p{h_2} (X)$ evaluated at
  $\beta_2$ and
  $\p{s} (X), \p{z} (X), \p{z_A} (X), \p{z_B} (X), \p{z_C} (X), \p{h_1} (X), \p{g_1} (X)$
  evaluated at $\beta_1$. Hence, to be able to give two different openings in its $5$-th
  message, $\adv$ has to break the unique opening property of the KZG commitment scheme
  which happens with probability $11 \cdot \epsop (\secpar)$ tops (as $11$ polynomials are
  evaluated).
  
  \ngame{2} This game is identical to Game $\game{1}$ except that now the
  adversary additionally wins if it provides two transcripts that matches on the
  first three messages of the proof.

  \ncase{Game 1 to Game 2} In its $4$-th message the adversary
  provides $\sigma_3$ and commits to polynomials $\p{g_3} (X), \p{h_3} (X)$ which are
  uniquely determined. Probability that the adversary commits to wrong polynomials that it
  can later evaluate to correct values (i.e.~to values of evaluated correct polynomials)
  is upper bounded by $2 \cdot \epsbind (\secpar) + 1 / \abs{\FF_p}$. This is since the
  adversary either has to break binding property in at least one of two commitments or
  guess the evaluation point.

  \ngame{3} This game is identical to Game $\game{2}$ except that now the
  adversary additionally wins if it provides two transcripts that matches on the
  first two messages of the proof.

  \ncase{Game 2 to Game 3} In its $3$-rd message the adversary
  provides $\sigma_2$ and commits to polynomials $\p{g_2} (X), \p{h_2} (X)$ which are
  uniquely determined. Probability that the adversary commits to wrong polynomials that it
  can later evaluate to correct values (i.e.~to values of evaluated correct polynomials)
  is upper bounded by $2 \cdot \epsbind (\secpar) + 1 / \abs{\FF_p}$ as explained in the
  previous point.

  \ngame{4}  This game is identical to Game $\game{3}$ except that now the
  adversary additionally wins if it provides two transcripts that matches on the
  first message of the proof.

  \ncase{Game 3 to Game 4} In its $2$-nd message the adversary
  provides $\sigma_1$ and commits to polynomials $\p{g_1} (X), \p{h_1} (X)$ which are
  uniquely determined. Probability that the adversary commits to wrong polynomials that it
  can later evaluate to correct values (i.e.~to values of evaluated correct polynomials)
  is upper bounded by $2 \cdot \epsbind (\secpar) + 1 / \abs{\FF_p}$ as explained before.
  
  \ncase{Conclusion} Taking all the games together, probability that $\adv$ wins
  in Game 4 is upper-bounded by
  \[
    11 \cdot (\epsop (\secpar)) + 6 \cdot \epsbind (\secpar) + \frac{3}{\abs{\FF_p}}.
  \]
  \qed
\end{proof}
}}

\COMMENT{
  \michals{2.11}{Old proof}
\begin{proof}
  As in previous proofs, we show the property by game hops. Let
  $N = \p{g_1}, \p{h_1}, \p{g_2}, \p{h_2}, \p{g_3}, \p{h_3}$. That is, $N$ is a
  set of all polynomials which commitments prover sends after it sends its first message.

  \ncase{Game 0} In this game the adversary wins if it breaks evaluation
  binding, unique opening property, or knowledge soundness of one of commitments
  for polynomials in $N$.

  Probability that a $\ppt$ adversary wins in Game 0, is upper bounded by $6
  \cdot (\epsbind + \epsop + \epsk)$.

  \ncase{Game 1} In this game the adversary additionally wins if it breaks the
  $\ur{1}$ property of the protocol

  \ncase{Game 0 to Game 1} Probability that the adversary wins in Game 1 but not in Game 0
  is $\epss$. This is since in the honest proof the polynomials in $N$ are uniquely
  determined. W.l.o.g.~we analyse probability that adversary is able to produce two
  (different) pairs of polynomials $(\p{h_2}, \p{g_2})$ and $(\p{h'_2}, \p{g'_2})$ such
  that
  \begin{align*}
    \p{h_2} (X) \ZERO_{\HHH} (X) + X \p{g_2} (X) & = \p{h_2} (X) \ZERO_{\HHH} (X) +
                                                   X \p{g_2} (X)\\
    (\p{h_2} (X) - \p{h'_2} (X)) \ZERO_{\HHH} (X) & = X (\p{g'_2} (X) - \p{g'_2}
    (X)).
  \end{align*}
  Since $\p{h_2}, \p{g_2} \in \FF^{< |\HHH| - 1} [X]$ and
  $\ZERO \in \FF^{|\HHH|} [X]$, LHS has different degree than RHS unless both
  sides have degree $0$. This happens when $\p{h_2} (X) = \p{h'_2} (X)$ and
  $\p{g_2} (X) - \p{g'_2} (X)$.
  Thus, for the adversary to be successful in this game it has to provide acceptable
  proofs where $(\p{h_2}, \p{g_2})$ and $(\p{h'_2}, \p{g'_2})$ differ. One of such pair
  has to be incorrect and will be accepted with probability at most $\epss$.
\end{proof}}

\subsection{Rewinding-Based Knowledge Soundness}
\begin{lemma}
	\label{lem:marlinprot_ss}
	$\marlinprotfs$ is $(2, \multconstr + 3)$-rewinding-based knowledge sound against algebraic adversaries who make up to $q$ random oracle queries with security loss 
	\[
	\epscss(\secpar,\accProb, q) \leq \left(1 - \frac{\accProb - (q + 1) \left(1 - \frac{\multconstr + 3}{p}\right)}{\frac{\multconstr + 3}{p}}\right) + (\multconstr + 3)\cdot\epsudlog (\secpar) + (\multconstr + 3) \cdot \epsid (\secpar)\,,
	\]
	Here $\accProb$ is a probability that the adversary outputs an acceptable proof, $\epsid(\secpar)$ is a soundness error of the ideal verifier for $\marlinprotfs$, and $\epsudlog(\secpar)$ is the security of $(\multconstr + 2, 1)$-$\udlog$ \hamid{2.5}{Not sure about $(\multconstr + 2, 1)$!}assumption.
\end{lemma}
\begin{proof}
The proof is similar to the proof of \cref{lem:plonkprot_ss} and \cref{lem:sonicprot_ss}. 
We use Attema et al.~\cite[Proposition 2]{EPRINT:AttFehKlo21} to bound the probability that the tree-building algorithm $\tdv$ does not obtain a tree of acceptable transcript in an expected number of runs. This happens with probability at most
\[
1 - \frac{\accProb - (q + 1) \left(1 - \frac{ \noofc + 3}{p} \right)}{\frac{\noofc + 3}{p}}
\]
Let $\tree$ be the tree outputted by $\tdv$. If one of the proofs in $\tree$ is not acceptable by the ideal verifier, one can break an instance of an updatable dlog assumption which happens with probability at most $(\noofc + 3)  \cdot \epsudlog (\secpar)$. In the case that all the transcripts are acceptable by the ideal verifier, but $\extss$ fails to extract a valid witness from $\tree$, one can break the soundness of the ideal verifier in one of the transcripts. That happens with probability at most $(\noofc + 3) \cdot \epsid (\secpar)$. Taking a union bound completes the proof.
\qed
\end{proof}


\COMMENT{\begin{lemma}\label{lem:marlinprot_ss}
	Assume that an idealised $\marlinprot$ verifier fails with probability at most
	$\epsid(\secpar)$ and probability that a $\ppt$ adversary breaks $\udlog$ is
	bounded by $\epsudlog(\secpar)$. Then $\marlinprotfs$ is
	$(2, d + 1)$-rewinding-based knowledge
	sound with security loss $\epsid (\secpar) + \epsudlog (\secpar)$.
\end{lemma}
\begin{proof}
	% \michals{8.9}{Need to check the degrees}
	The proof goes similarly to the respective proofs for $\plonk$ and
	$\sonic$. That is, let $\srs$ be $\marlinprot$'s finalized SRS and denote by $\srs_1$
	all SRS's $\GRP_1$-elements. Let $\tdv$ be an algebraic adversary that
	produces a statement $\inp$ and a $(1, \dconst + 1, 1, 1)$-tree of
	accepting transcripts $\tree$. Note that in all transcripts the instance
	$\inp$, proof elements
	$\sigma_1, \gone{\p{w}(\chi), \p{z_A}(\chi), \p{z_B}(\chi), \p{z_C}(\chi),
		\p{h_0}(\chi), \p{s}(\chi)}, \gone{\p{g_1}(\chi), \p{h_1}(\chi)}$
	and challenges $\alpha, \eta_1, \eta_2, \eta_3$ are common as the transcripts
	share the first $3$ messages. The tree branches after the third message of the
	protocol where the challenge $\beta_1$ is presented, thus tree $\tree$ is
	build using different values of $\beta_1$.
	
	We consider the following games.
	
	\ncase{Game 0} In this game the adversary wins if all the transcripts it
	produced are acceptable by the ideal verifier,
	i.e.~$\vereq_{\inp, \zkproof}(X) = 0$, cf.~\cref{eq:marlin_ver_eq}, yet the extractor
	fails to extract a valid witness out of them.
	
	Probability of $\tdv$ winning this game is $\epsid(\secpar)$ as the protocol
	$\marlinprot$, instantiated with the idealised verification equation, is
	perfectly sound except with negligible probability of the idealised verifier
	failure $\epsid(\secpar)$. Hence for a valid proof $\zkproof$ for a statement
	$\inp$ there exists a witness $\wit$, such that $\REL(\inp, \wit)$ holds. Note
	that since the $\tdv$ produces $(\dconst + 1)$ accepting transcripts for
	different challenges $\beta_1$, it obtains the same number of different
	evaluations of polynomials $\p{z_A}, \p{z_B}, \p{z_C}$.
	
	Since the transcripts are acceptable by an idealised verifier, the equality
	$\p{z_A} (X) \p{z_B} (X) - \p{z_C} (X) = \p{h_0} (X) \ZERO_\HHH (X)$ holds and
	each of $\p{z}_M$, $M \in \smallset{A, B, C}$, has been computed
	correctly. Hence, $\p{z_A}, \p{z_B}, \p{z_C}$ encodes the valid witness for
	the proven statement. Since $\p{z_A}, \p{z_B}, \p{z_C}$ are of degree at most
	$\dconst$ and there is more than $(\dconst + 1)$ their evaluations
	known, $\extt$ can recreate their coefficients by interpolation and reveal the
	witness with probability $1$. Hence, the probability that extraction fails in
	that case is upper-bounded by probability of an idealised verifier failing
	$\epsid(\secpar)$, which is negligible.
	
	\ncase{Game 1} In this game the adversary additionally wins if it produces a
	transcript in $\tree$ such that $\vereq_{\inp, \zkproof}(\chi) = 0$, but
	$\vereq_{\inp, \zkproof}(X) \neq 0$. That is, the ideal verifier does not
	accept the proof, but the real verifier does.
	
	\ncase{Game 0 to Game 1} Assume the adversary wins in Game 1, but
	does not win in Game 0. We show that such adversary may be used to break the
	$\udlog$ assumption. More precisely, let $\tdv$ be an adversary that for
	relation $\REL$ and randomly picked $\srs \sample \kgen(\REL)$ produces a tree
	of accepting transcripts such that the winning condition of the game
	holds. Let $\rdvdulog$ be a reduction that gets as input an
	$(\dconst, 1)$-$\udlog$ instance $\gone{1, \ldots, \chi^\dconst}, \gtwo{1, \chi}$ and
	is tasked to output $\chi$. The reduction proceeds as follows---it builds $\adv$'s SRS $\srs$ in the updatable setting using the input $\udlog$ instance. Namely it answers $\adv$'s queries for SRS updates and sets the honest update of the SRS to be $\srs$. Let $\srs'$ be the finalized SRS and $(1, \tree)$ be the output
	returned by $\adv$. Let $\inp$ be a relation proven in $\tree$.  Consider a
	transcript $\zkproof \in \tree$ such that $\vereq_{\inp, \zkproof}(X) \neq 0$,
	but $\vereq_{\inp, \zkproof}(\chi) = 0$. Since the adversary is algebraic, all
	group elements included in $\tree$ are extended by their representation as a
	combination of the input $\GRP_1$-elements. Hence all coefficients of the
	verification equation polynomial $\vereq_{\inp, \zkproof}(X)$ are known and
	$\rdvudlog$ can find its zero points. Since
	$\vereq_{\inp, \zkproof}(\chi) = 0$, the targeted discrete log value $\chi'$ is
	among them.  Let $\chi_1, \ldots, \chi_\ell$ be the partial trapdoors of $\adv$'s SRS updates,  extracted by the reduction from the update proofs given by $\adv$. Now $\rdvudlog$ returns $\chi = \chi' (\chi_1 \chi_2 \ldots \chi_\ell)^{-1}$ and breaks the $\udlog$ assumption. Hence, the probability that this event happens is upper-bounded
	by $\epsudlog(\secpar)$.
	
\end{proof}}

\subsection{Trapdoor-Less Zero-Knowledge of Marlin}
\begin{lemma}
  \label{lem:marlin_hvzk}
  $\marlinprotfs$ is 2-programmable trapdoor-less zero-knowledge.
\end{lemma}
\begin{proof}
The simulator follows the protocol except it picks the challenges $\alpha,
\eta_A, \eta_B, \eta_C, \beta_1, \beta_2, \beta_3$ before it picks polynomials
it sends.

First, it picks $\p{\tilde{z}}_A (X)$, $\p{\tilde{z}}_B (X)$ at random and
$\p{\tilde{z}}_C (X)$ such that
$\p{\tilde{z}}_A (\beta_1) \p{\tilde{z}}_B (\beta_1) = \p{\tilde{z}}_C
(\beta_1)$.  Given the challenges and polynomials $\p{\tilde{z}}_A (X)$,
$\p{\tilde{z}}_B (X)$, $\p{\tilde{z}}_C (X)$ the simulator computes
$\sigma_1 \gets \sum_{\kappa \in \HHH} \p{s}(\kappa) + \p{r}(\alpha, X) (\sum_{M
  \in \smallset{A, B, C}}\eta_M \p{\tilde{z}}_M(X)) - \sum_{M \in \smallset{A,
    B, C}} \eta_M \p{r}_M (\alpha, X) \p{\tilde{z}} (X)$.

Then the simulator starts the protocol and follows it, except it programs the
random oracle that on partial transcripts it returns the challenges picked by
$\simulator$.
\end{proof}

%\subsection{From Rewinding-Based Knowledge Soundness and Unique Response Property to\COMMENT{ forking}
%  Simulation Extractability of $\marlinprotfs$}
%\begin{corollary}
%  Assume that $\marlinprotfs$ is $\ur{2}$ with security
%  $\epsur(\secpar) = 6 \cdot (\epsbind + \epsop + \epsk)$, and rewinding-based knowledge sound
%  with security $\epss(\secpar)$. Let $\ro\colon \bin^* \to \bin^\secpar$ be a
%  random oracle. Let $\advse$ be an adversary that can make up to $q$
%  random oracle queries, and outputs an
%  acceptable proof for $\marlinprotfs$ with probability at least
%  $\accProb$. Then $\marlinprotfs$ is \COMMENT{forking }simulation-extractable with
%  extraction error $\eta = \epsur(\secpar)$. The extraction probability
%  $\extProb$ is at least
%  \[
%    \extProb \geq q^{-\dconst} (\accProb - 6 \cdot (\epsbind + \epsop +
%    \epsk))^{\dconst + 1} -\eps(\secpar)\,.
%\]
%	for some negligible $\eps(\secpar)$, $\dconst$ being, the upper bound of
%  constraints of the system.
%\end{corollary}

\subsection{Simulation Extractability of $\marlinprotfs$}
Since \cref{lem:marlinprot_ur,lem:marlinprot_ss,lem:marlin_hvzk} hold, $\marlinprotfs$ is $\ur{2}$, rewinding-based knowledge sound and trapdoor-less zero-knowledge. By making use
of \cref{thm:se}, we conclude that $\marlinprotfs$ is simulation-extractable as defined in \cref{def:simext}.

\begin{corollary}[Simulation extractability of $\marlinprotfs$]
	\label{thm:marlinprotfs_se}
	$\marlinprotfs$ is \emph{updatable simulation-extractable} against any $\ppt$ adversary $\advse$ who makes up to $q$ random oracle queries and returns an acceptable proof with probability at least $\accProb$ with extraction failure probability 
	\[
	\epsse(\secpar, \accProb, q) \leq \left(1 - \frac{\accProb - \epsur (\secpar) - (q + 1) \epserr (\secpar)} {1 - \epserr (\secpar)}\right) + (\multconstr  + 3) \cdot \epsudlog (\secpar) + (\multconstr + 3) \cdot \epsid (\secpar),
	\]
	where $\epserr (\secpar) = \left(1 - \frac{\multconstr + 3}{p} \right)$, $p$ is the size of the field, and $\noofc$ is the number of constrains in the circuit. 
\end{corollary}


%\section{Further work}
%We identify a number of problems which we left as further work. First of all,
%the generalised version of the forking lemma presented in this paper can be
%generalised even further to include protocols where forking soundness holds for
%protocols where $\extt$ extracts a witness from a $(n_1, \ldots, n_\mu)$-tree of
%acceptable transcripts, where more than one $n_j > 1$. I.e.~to include
%protocols that for witness extraction require transcripts that branch at more
%than one point.
%
%Although we picked $\plonk$ and $\sonic$ as examples for our framework, it is
%not limited to SRS-based NIZKs. Thus, it would be interesting to apply it to
%known so-called transparent zkSNARKs like Bulletproofs \cite{SP:BBBPWM18},
%Aurora \cite{EC:BCRSVW19} or AuroraLight \cite{EPRINT:Gabizon19a}.
%
%Since the rewinding technique and the forking lemma used to show simulation
%extractability of $\plonkprotfs$ and $\sonicprotfs$ come with security loss,
%it would be interesting to show SE of these protocols directly in the
%algebraic group model.
%
%Although we focused here only on zkSNARKs, it is worth to
%investigating other protocols that may benefit from our framework, like
%e.g.~identification schemes.
%
%Last, but not least, this paper would benefit greatly if a more tight version
%of the generalised forking lemma was provided. However, we have to note here
%that some of the inequalities used in the proof are already tight, i.e.~for
%specific adversaries, some of the inequalities are already equalities.

%%% Local Variables:
%%% mode: latex
%%% TeX-master: "main"
%%% End:

%% !TEX root = main.tex
% !TEX spellcheck = en-US

\section{Simulation Soundness---definitions and the general result}
\noindent \textbf{Simulation sound NIZKs in the updatable setting.}
Another notion for non-malleable NIZKs is \emph{simulation soundness}. It allows the adversary to see simulated proof, however, in contrast to simulation
extractability it does not require an extractor to provide a witness for the
proven statement. Instead, it is only necessary, that an adversary who sees
simulated proofs cannot make the verifier accept a proof of an incorrect
statement. More precisely,


\begin{definition}[Simulation soundness in the updatable setting]
	\label{def:simsnd}
	Let $\ps = (\kgen, \prover, \verifier, \simulator)$ be a NIZK proof system. We say that
  $\ps$ is \emph{updatable simulation-sound} if for any $\ppt$ adversary $\adv$ that is
  given oracle access to an updatable SRS setup $\initU$, cf.~\cref{fig:upd}, a simulation oracle $\simulator
  = (\simOH, \simOP')$, and a random oracle $\ro$, probability
	\[
	\ssndProb = \condprob{
		\begin{matrix}
		\verifier(\srs, \inp_{\advse}, \zkproof_{\advse}) = 1 \\
		\wedge  ~(\inp_{\advse}, \zkproof_{\advse}) \not\in Q   \\
		\wedge \neg \exists \wit_{\adv}: \REL(\inp_{\adv}, \wit_{\adv}) = 1
		\end{matrix}
	}{
		\begin{aligned}
		& r \sample \RND{\advse},
		(\inp_{\advse}, \zkproof_{\advse}) \gets \advse^{\initU, \simOH, \simOP'} (1^\secpar; r) \\
		\end{aligned}
	}
	\]
	is at most negligible.  
	Here, $\srs$ is the finalized SRS, list $Q$ contains all $(\inp, \zkproof)$ pairs where 
	$\inp$ is an instance provided to the simulator by the adversary and
	$\zkproof$ is the simulator's answer. List $Q_\ro$ contains all $\advse$'s
	queries to $\ro$ and $\ro$'s answers.  
\end{definition}

\label{rem:simext_to_simsnd}
We note that the probability $\ssndProb$ in~\cref{def:simsnd} can be expressed in
terms of simulation-extractability. More precisely, the
condition $\neg \exists \wit: \REL(\inp_\adv, \wit_\adv) = 1$ can be substituted with
$\REL(\inp_\adv, \wit_\adv) = 0$, where $\wit_\adv$, returned by a possibly unbounded
extractor, is either a witness to $\inp_\adv$ (if there exists any) or $\bot$ (if
there is none). More precisely,
\[
\ssndProb = \condprob{
	\begin{matrix}
	\verifier(\srs, \inp_{\advse}, \zkproof_{\advse}) = 1 \\
	\wedge  ~(\inp_{\advse}, \zkproof_{\advse}) \not\in Q   \\
	\wedge  ~\REL(\inp_{\advse}, \wit_{\advse}) = 0
	\end{matrix}
}{
	\begin{aligned}
	& r \sample \RND{\advse},
	(\inp_{\advse}, \zkproof_{\advse}) \gets \advse^{\initU, \simOH, \simOP'} (1^\secpar; r) \\
	& \wit_{\advse} \gets \ext(\srs, \advse, r, \inp_{\advse}, \zkproof_{\advse},
	Q, Q_\ro, Q_\srs) 
	\end{aligned}
}
\]
The only necessary input to the unbounded extractor $\ext$ is the instance
$\inp_\adv$ (the rest is given for the consistency with the simulation extractability
definition). 
%
With the probabilities in \cref{def:simext} holding regardless of whether the extractor
is unbounded or not, we obtain the following equality
$ \ssndProb = \accProb - \extProb$.

\subsection{Simulation soundness---the general result}
\label{sec:general}
Equipped with definitional framework of \cref{sec:se_definitions}, we can also show the proof of simulation soundness of Fiat-Shamir NIZKs based on multi-round protocols.

%\begin{theorem}[Forking simulation-extractable multi-message protocols]
%	\label{thm:se}
%	Let $\ps = (\kgen, \prover, \verifier, \simulator)$ be an interactive $(2 \mu + 1)$-message
%	zero-knowledge proof system for $\RELGEN(\secparam)$, which is trapdoor-less simulatable, has
%	$\ur{k}$ property with security $\epsur(\secpar)$, and is $(\epss(\secpar), k, n)$-forking
%	sound.  Let $\ro\colon \bin^{*} \to \bin^{\secpar}$ be a random oracle.  Then $\psfs$ is
%	forking simulation-extractable with extraction error $\epsur(\secpar)$ against $\ppt$
%	algebraic adversaries that makes up to $q$ random oracle queries and returns an acceptable
%	proof with probability at least $\accProb$.  The extraction probability $\extProb$ is at
%	least
%	\( \extProb \geq \frac{1}{q^{n - 1}} (\accProb - \epsur(\secpar))^{n} -\eps(\secpar)\,, \)
%	for some negligible $\eps(\secpar)$.
%\end{theorem}

\begin{theorem}[Simulation soundness]
	\label{thm:simsnd}
	Let $\ps = (\kgen, \prover, \verifier, \simulator)$ be an interactive $(2 \mu + 1)$-message
	zero-knowledge proof system for $\RELGEN(\secparam)$, which is trapdoor-less simulatable, has
	$\ur{k}$ property with security $\epsur(\secpar)$. Let $\ro\colon \bin^{*} \to \bin^{\secpar}$ be a random oracle. Then, the
	probability that a $\ppt$ adversary $\adv$ breaks simulation soundness of
	$\ps$ is upper-bounded by
	\(
	\epsur(\secpar) + q_\ro^\mu  \epss(\secpar)\,,
	\)
	where $q$ is the total number of queries made by the adversary $\adv$ to $\ro$.
\end{theorem}

\begin{proof}
	\ngame{0} This is a simulation soundness game played between an adversary
	$\advse$ who is given access to an oracle $\initU$ that defines an updatable SRS setup, a random oracle $\ro$ and a simulation oracle
	$\simulator$. $\adv$ wins if it manages to produce an accepting proof
	for a false statement. In the following game hops, we upper-bound the
	probability that this happens.
	
	\ngame{1} This is identical to $\game{0}$ except that the game is aborted if
	there is a simulated proof $\zkproof_\simulator$ for $\inp_{\adv}$ such that
	$(\inp_{\adv}, \zkproof_\simulator[1..k]) = (\inp_{\adv},
	\zkproof_{\adv}[1..k])$. That is, the adversary in its final proof reuses at
	least $k$ messages from a simulated proof it saw before and the proof is
	accepting.  Denote this event by $\event{\errur}$.
	
	\ncase{Game 0 to Game 1} We have, \( \prob{\game{0} \land
		\nevent{\errur}} = \prob{\game{1} \land \nevent{\errur}} \) and, from the
	difference lemma, cf.~\cref{lem:difference_lemma},
	$ \abs{\prob{\game{0}} - \prob{\game{1}}} \leq \prob{\event{\errur}}\,$.
	Thus, to show that the transition from one game to another introduces only
	minor change in probability of $\adv$ winning it should be shown that
	$\prob{\event{\errur}}$ is small.
	
	We can assume that $\adv$ queried the simulator on the instance it wishes to
	output, i.e.~$\inp_{\adv}$. We show a reduction $\rdvur$ that utilises $\adv$
	to break the $\ur{k}$ property of $\ps$. Let $\rdvur$ run $\advse$ internally
	as a black-box:
	\begin{compactitem}
		\item The reduction answers $\adv$ update queries by asking the same query from the update oracle in the unique response experiment. The reduction finalises the same SRS $\srs$ as the one $\adv$ does.
		\item The reduction answers both queries to the simulator $\psfs.\simulator$
		and to the random oracle.  It also keeps lists $Q$, for the simulated
		proofs, and $Q_\ro$ for the random oracle queries.
		\item When $\adv$ makes a fake proof $\zkproof_{\adv}$ for $\inp_{\adv}$,
		$\rdvur$ looks through lists $Q$ and $Q_\ro$ until it finds
		$\zkproof_{\simulator}[0..k]$ such that
		$\zkproof_{\adv}[0..k] = \zkproof_{\simulator}[0..k]$ and a random oracle
		query $\zkproof_{\simulator}[k].\ch$ on $\zkproof_{\simulator}[0..k]$.
		\item $\rdvur$ returns two proofs for $\inp_{\adv}$:
		\begin{align*}
		\zkproof_1 = (\zkproof_{\simulator}[1..k],
		\zkproof_{\simulator}[k].\ch, \zkproof_{\simulator}[k + 1..\mu + 1])\\
		\zkproof_2 = (\zkproof_{\simulator}[1..k],
		\zkproof_{\simulator}[k].\ch, \zkproof_{\adv}[k + 1..\mu + 1])
		\end{align*}
	\end{compactitem}  
	If $\zkproof_1 = \zkproof_2$, then $\adv$ fails to break simulation soundness,
	as $\zkproof_2 \in Q$. On the other hand, if the proofs are not equal, then
	$\rdvur$ breaks $\ur{k}$-ness of $\ps$. This happens only with negligible
	probability $\epsur(\secpar)$, hence
	\( \prob{\event{\errur}} \leq \epsur(\secpar)\,. \)
	
	\ngame{2} This is identical to $\game{1}$ except that now the environment
	aborts if the instance the adversary proves is not in the language.
	
	\ncase{Game 1 to Game 2} 
	% REDUCTION TO INTERACTIVE SOUNDNESS:
	We show that
	$\abs{\prob{\game{1}} - \prob{\game{2}}} \leq q^{\mu} \cdot \epss(\secpar)$,
	where $\epss(\secpar)$ is the probability of breaking soundness of the underlying
	\emph{interactive} protocol $\ps$. Note that
	$\abs{\prob{\game{1}} - \prob{\game{2}}}$ is the probability that $\adv$
	outputs an acceptable proof for a false statement which does not break the
	unique response property (such proofs have been excluded by
	$\game{1}$). Consider a soundness adversary $\adv'$ who initiates a proof with
	$\ps$'s verifier $\ps.\verifier$, internally runs $\adv$ and proceeds as
	follows:
	\begin{compactitem}
		\item It guesses indices $i_1, \ldots, i_\mu$ such that random oracle queries
		$h_{i_1}, \ldots, h_{i_\mu}$ are the queries used in the $\zkproof_\adv$
		proof eventually output by $\adv$. This is done with probability at least
		$1/q^\mu$ (since there are $\mu$ challenges from the verifier in
		$\ps$).
		\item On input $h$ for the $i$-th,
		$i \not\in \smallset{{i_1}, \ldots, {i_\mu}}$, random oracle query, $\adv'$
		returns randomly picked $y$, sets $\ro(h) = y $ and stores $(h, y)$ in
		$Q_\ro$ if $h$ is sent to $\ro$ the first time. If that is not the case,
		$\adv$ finds $h$ in $Q_\ro$ and returns the corresponding $y$.
		\item On input $h_{i_j}$ for the $i_j$-th,
		$i_j \in \smallset{{i_1}, \ldots, {i_\mu}}$, random oracle query, $\adv'$
		parses $h_{i_j}$ as a partial proof transcript $\zkproof_\adv[1..j]$ and
		runs $\ps$ using $\zkproof_\adv[j]$ as a $\ps.\prover$'s $j$-th message to
		$\ps.\verifier$. The verifier responds with a challenge
		$\zkproof_\adv[j].\ch$. $\adv'$ sets $\ro(h_{i_j}) =
		\zkproof_\adv[j].\ch$. If we guessed the indices correctly we have that
		$h_{i_{j'}}$, for $j' \leq j$, parsed as $\zkproof_\adv[1..j']$ is a prefix
		of $\zkproof_\adv[1..j]$.
		\item On query $\inp_\simulator$ to $\simulator$, $\adv'$ runs the simulator
		$\ps.\simulator$ internally. Note that we require a simulator that only
		programs the random oracle for $j \geq k$.  If the simulator makes a
		previously unanswered random oracle query with input
		$\zkproof_\simulator[1..j]$, $1 \leq j < k$, and this is the $i_j$-th query,
		it generates $\zkproof_\simulator[j].\ch$ by invoking $\ps.\verifier$ on
		$\zkproof_\simulator[j]$ and programs
		$\ro(h_{i_j}) = \zkproof_\simulator[j].\ch$.  It returns
		$\zkproof_\simulator$.
		\item Answers $\ps.\verifier$'s final challenge $\zkproof_\adv[\mu].\ch$ using the
		answer given by $\adv$, i.e.~$\zkproof_\adv[\mu]$.
	\end{compactitem}
	That is, $\adv'$ manages to break soundness of $\ps$ if $\adv$ manages to
	break simulation soundness without breaking the unique response property and
	$\adv'$ correctly guesses the indices of $\adv$ random oracle queries. This
	happens with probability upper-bounded by $\abs{\prob{\game{1}} -
		\prob{\game{2}}} \cdot \infrac{1}{q^{\mu}}$. Hence $\abs{\prob{\game{1}} -
		\prob{\game{2}}} \leq q^{\mu} \cdot \epss(\secpar)$.
	
	Note that in $\game{2}$ the adversary cannot win. Thus the probability
	that $\advss$ is successful is upper-bounded by
	$\epsur(\secpar) + q^{\mu} \cdot \epss(\secpar)$.  \qed
\end{proof}


We conjecture that based on the recent results on state restoration soundness~\cite{cryptoeprint:2020:1351}, which effectively allows to query the verifier multiple times on different overlapping transcripts, the $q^{\mu}$ loss could be avoided. However, this would reduce the class of protocols covered by our results. 


\subsection{Simulation soundness of~$\plonkprotfs$}
Since \cref{lem:plonkprot_ur,lem:plonkprot_ss,lem:plonk_hvzk} hold, $\plonkprot$ is $\ur{2}$, computational special sound and trapdoor-less simulatable. We now make use of \cref{thm:simsnd} and show that
$\plonkprot_\fs$ is simulation sound as defined in
\cref{def:simsnd}.

 \begin{corollary}[Simulation soundness of $\plonkprot_\fs$]
   \label{cor:simsnd-P}
   Assume that $\plonkprot$ is $2$-programmable HVZK in the standard model, that
   is computational special sound with security $\epss(\secpar)$, and the $\PCOMp$ is a commitment of knowledge with
   security $\epsk(\secpar)$, binding security $\epsbind(\secpar)$ and has unique
   opening property with security $\epsop(\secpar)$. Then the probability that a
   $\ppt$ adversary $\adv$ breaks simulation soundness of $\plonkprot_{\fs}$ is
   upper-bounded by
   \( \epsk(\secpar) + 2\cdot\epsbind(\secpar) + \epsop(\secpar) + q_\ro^4
   \epss(\secpar)\,, \) where $q$ is the total number of queries made by the
   adversary $\adv$ to a random oracle $\ro\colon \bin^{*} \to \bin^{\secpar}$.
 \end{corollary}

\subsection{Simulation soundness of~$\sonicprotfs$}
The following corollary shows the simulation soundness of $\sonicprotfs$ based on~\cref{lem:sonicprot_ur,lem:sonicprot_ss,lem:sonic_hvzk} and~\cref{thm:simsnd}.
\begin{corollary}[Simulation soundness of $\sonicprot_\fs$]
	\label{cor:simsnd-S}
	Assume that $\sonicprot$ is $1$-programmable HVZK in the standard model, that
	is computational special sound with security $\epss(\secpar)$, and the $\PCOMs$ is a commitment of knowledge with
	security $\epsk(\secpar)$, binding security $\epsbind(\secpar)$ and has unique
	opening property with security $\epsop(\secpar)$. Then the probability that a
	$\ppt$ adversary $\adv$ breaks simulation soundness of $\sonicprot_{\fs}$ is
	upper-bounded by
	\( \epsk(\secpar) + 2\cdot\epsbind(\secpar) + \epsop(\secpar) + q_\ro^4
	\epss(\secpar)\,, \) where $q$ is the total number of queries made by the
	adversary $\adv$ to a random oracle $\ro\colon \bin^{*} \to \bin^{\secpar}$.
\end{corollary}

\subsection{Simulation soundness of~$\marlinprotfs$}
	The simulation soundness of $\marlinprot_\fs$ follows from \cref{thm:simsnd}, and \cref{lem:marlinprot_ur,lem:marlinprot_ss,lem:marlin_hvzk}.
\begin{corollary}[Simulation soundness of $\marlinprot_\fs$]
	\label{cor:simsnd-M}
	Assume that $\marlinprot$ is $1$-programmable HVZK in the standard model, that
	is computational special sound with security $\epss(\secpar)$, and the $\PCOM$ is a commitment of knowledge with
	security $\epsk(\secpar)$, binding security $\epsbind(\secpar)$ and has unique
	opening property with security $\epsop(\secpar)$. Then the probability that a
	$\ppt$ adversary $\adv$ breaks simulation soundness of $\marlinprot_{\fs}$ is
	upper-bounded by
	\( \epsk(\secpar) + 2\cdot\epsbind(\secpar) + \epsop(\secpar) + q_\ro^4
	\epss(\secpar)\,, \) where $q$ is the total number of queries made by the
	adversary $\adv$ to a random oracle $\ro\colon \bin^{*} \to \bin^{\secpar}$.
\end{corollary}


%%% Local Variables:
%%% mode: latex
%%% TeX-master: "main"
%%% End:


\section{Conclusion and Future Work}

Since the SRS is being continually updated, one could argue that there might be
simulated proofs with respect to \textit{different} SRSs available for the adversary
to see before attempting to forge a proof with respect to a current SRS.  That is,
each SRS in the update chain spawns a simulation oracle. Intuitively, the
updatability of the SRS allows an adversarial prover to contribute to updating, and
see proofs with respect to different updated SRSs before attempting to provide a
proof for a false statement (potentially output a proof wrt a SRS that is different
from the SRSs corresponding to all the simulated proofs seen).  This is a stronger
definition of SE than the one we consider in this paper, and we leave exploring this
to future work.

\bibliographystyle{abbrv}
\bibliography{../cryptobib/abbrev3,../cryptobib/crypto,additional_bib}

% \clearpage
%{\Huge{Supplementary Materials}}
\appendix

\section{Additional preliminaries}

\subsection{Security of uber assumption}
\label{sec:uber-assumption-security-proof}.
We show security of our version of the uber assumption using the generic group model as
introduced by Shoup \cite{EC:Shoup97} where all group elements are represented by random binary
strings of length $\secpar$. That is, there are random encodings $\xi_1, \xi_2, \xi_T$ which
are injective functions from $\ZZ_p^+$ to $\bin^{\secpar}$. We write
$\GRP_i = \smallset{\xi_i(x) \mid x \in \ZZ_p^+}$, for $i \in \smallset{1, 2, T}$. For the sake
of clarity we denote by $\xi_{i, j}$ the $j$-th encoding in group $\GRP_i$.

Let
$\p{P}_i = \smallset{p_1, \ldots, p_{\tau_i}} \subset \FF_p[X_1, \ldots, X_n]$,
for $i \in \smallset{1, 2, T}, \tau_i, n \in \NN$, be sets of multivariate
polynomials. Denote by $\p{P}_i(x_1, \ldots, x_n)$ a set of evaluations of
polynomials in $\p{P_i}$ at $(x_1, \ldots, x_n)$. Denote by
$L_i = \smallset{(p_j, \xi_{i, j}) \mid j \leq \tau_i}$.

Let $\adv$ be an algorithm that is given encodings $\xi_{i, j_i}$ of polynomials
in $\p{P}_i$ for $i \in \smallset{1, 2, T}, j_i = \tau_i$. There is an oracle $\oracleo$
that allows to perform $\adv$ the following queries:
\begin{description}
\item[Group operations in $\GRP_1, \GRP_2, \GRP_T$:] On input
  $(\xi_{i, j}, \xi_{i, j'}, i, op)$, $j, j' \leq \tau_i$,
  $op \in \smallset{\msg{add}, \msg{sub}}$, $\oracleo$ sets $\tau'_i \gets \tau_i + 1$,
  computes
  $p_{i, \tau'_i} = p_{i, j}(x_1, \ldots, x_n) \pm p_{i, j'}(x_1, \ldots, x_n)$
  respectively to $op$. If there is an element  $p_{i, k} \in L_i$ such 
  that $p_{i, k} = p_{\tau'_i}$, then the oracle returns encoding of $p_{i,
    k}$. Otherwise it sets the encoding $\xi_{i, \tau'_i}$ to a new unused
  random string, adds $(p_{i, \tau'_i}, \xi_{i, \tau'_i})$ to $L_i$, and returns
  $\xi_{i, \tau'_i}$.
\item[Bilinear pairing:] On input $(\xi_{1, j}, \xi_{2, j'})$ the oracle sets
  $\tau' \gets \tau_T + 1$ and computes
  $r_{\tau'} \gets p_{i, j}(x_1, \ldots, x_n) \cdot p_{i, j'}(x_1, \ldots,
  x_n)$. If $r_{\tau'} \in L_T$ then return encoding found in the list $L_T$,
  else pick a new unused random string and set $\xi_{T, \tau'}$ to it. Return
  the encoding to the algorithm.
\end{description}

Given that, we are ready to show security of our variant of the Boneh et
al.~uber assumption. The proof goes similarly to the original proof given in
\cite{EC:BonBoyGoh05} with minor differences.

\begin{theorem}[Security of the uber assumption]
  \label{thm:uber_assumption}
  Let $\p{P}_i \in \FF_p[X_1, \ldots, X_n]^{m_i}$, for
  $i \in \smallset{1, 2, T}$ be $\tau_i$ tuples of $n$-variate polynomials over
  $\FF_p$ and let $\p{F} \in \FF_p[X_1, \ldots, X_n]^m$. Let
  $\xi_0, \xi_1, \xi_T$, $\GRP_1, \GRP_2, \GRP_T$ be as defined above. If
  polynomials $f \in \p{F}$ are pair-wise independent and are independent of
  $\p{P}_1, \p{P}_2, \p{P}_T$, then for any $\adv$ that makes up to $q$ queries to the
  GGM oracle holds:
  \begin{equation*}
    \begin{split}
     \left|\,
    \Pr\left[
    \adv\left(
      \begin{aligned}
        \xi_1(\p{P}_1(x_1, \ldots, x_n)), \\
        \xi_2(\p{P}_2(x_1, \ldots, x_n)), \\
        \xi_T(\p{P}_T(x_1, \ldots, x_n)), \\
        \xi_{1}(\p{F}_0), \xi_{1}(\p{F}_1)
      \end{aligned}
    \right) = b
    \, \left|\,
      \begin{aligned}
        x_1, \ldots, x_n, y_1, \ldots, y_m \sample \FF_p,\\
        b \sample \bin, \\
        \p{F}_b \gets \p{F}(x_1, \ldots, x_n),\\
        \p{F}_{1 - b} \gets (y_1, \ldots, y_m)
      \end{aligned}
    \right.  \right] - \frac{1}{2} \, \right| \\
     \leq \frac{d(q + m_1 + m_2 + m_T +
      m)^2 }{2p}
    \end{split}
  \end{equation*}
\end{theorem}
\begin{proof}
  Let $\cdv$ be a challenger that plays with $\adv$ in the following
  game. $\cdv$ maintains three lists
  \[
    L_i = \smallset{(p_j, \xi_{i, j}) \mid j \in \range{1}{\tau_i}},
  \]
  for $i \in \smallset{1, 2, T}$. Invariant $\tau$ states that
  $\tau_1 + \tau_2 + \tau_T = \tau + m_1 + m_2 + m$.

  Challenger $\cdv$ answers $\adv$'s oracle queries. However, it does it a bit
  differently that the oracle $\oracleo$ would:
  \begin{description}
  \item[Group operations in $\GRP_1, \GRP_2, \GRP_T$:] On input
    $(\xi_{i, j}, \xi_{i, j'}, i, op)$, $j, j' \leq \tau_i$,
    $op \in \smallset{\msg{add}, \msg{sub}}$, $\cdv$ sets
    $\tau' \gets \tau_i + 1$, computes
    $p_{i, \tau'}(X_1, \ldots, X_n) = p_{i, j}(X_1, \ldots, X_n) \pm p_{i,
      j'}(X_1, \ldots, X_n)$ respectively to $op$. If there is a polynomial
    $p_{i, k}(X_1, \ldots, X_n) \in L_i$ such that
    $p_{i, k}(X_1, \ldots, X_n) = p_{\tau'}(X_1, \ldots, X_n)$, then the
    challenger returns encoding of $p_{i, k}$. Otherwise it sets the encoding
    $\xi_{i, \tau'}$ to a new unused random string, adds
    $(p_{i, \tau'}, \xi_{i, \tau'})$ to $L_i$, and returns $\xi_{i, \tau'}$.
  \item[Bilinear pairing:] On input $(\xi_{1, j}, \xi_{2, j'})$ the challenger
    sets $\tau' \gets \tau_T + 1$ and computes
    $r_{\tau'}(X_1, \ldots, X_n) \gets p_{i, j}(X_1, \ldots, X_n) \cdot p_{i,
      j'}(X_1, \ldots, X_n)$. If $r_{\tau'}(X_1, \ldots, X_n) \in L_T$, $\cdv$
    returns encoding found in the list $L_T$. Else it picks a new unused random
    string and set $\xi_{T, \tau'}$ to it. Finally it returns the encoding to
    the algorithm.
\end{description}
  
After at most $q$ queries to the oracle, the adversary returns a bit $b'$. At
that point the challenger $\cdv$ chooses randomly $x_1, \ldots, x_n, y_1 \ldots, y_m$,
random bit $b$, and sets $X_i = x_i$, for $i \in \range{1}{n}$, and $Y_i = y_i$,
for $i \in \range{1}{m}$; furthermore, $\p{F}_b \gets \p{F}(x_1, \ldots, x_n)$
and $\p{F}_{1 - b} \gets (y_1, \ldots, y_m)$. Note that $\cdv$ simulates
perfectly unless the chosen values $x_1, \ldots, x_n, y_1, \ldots, y_m$ result
in equalities between polynomial evaluations that are not equalities between the
polynomials. That is, the simulation is perfect unless for some $i, j, j'$ holds
\[
  p_{i, j}(x_1, \ldots, x_n) - p_{i, j'}(x_1, \ldots, x_n) = 0,
  \]
  for $p_{i, j}(X_1, \ldots, X_n) \neq p_{i, j'}(X_1, \ldots, X_n)$.  Denote by
  $\bad$ an event that at least one of the three conditions holds. When $\bad$
  happens, the answer $\cdv$ gives to $\adv$ differs from an answer that a real
  oracle would give. We bound the probability that $\bad$ occurs in two steps.

  First we set $\p{F}_b = \p{F}(X_1, \ldots, X_n)$. Note that symbolic
  substitutions do not introduce any new equalities in $\GRP_1$. That is, if for
  all $j, j'$ holds $p_{1, j} \neq p_{1, j'}$, then $p_{1, j} \neq p_{1, j'}$
  even after setting $\p{F}_b = \p{F}(X_1, \ldots, X_n)$. This follows since all
  polynomials in $\p{F}$ are pairwise independent and $\p{F}$ independent on
  $\p{P}_1, \p{P}_2, \p{P}_T$. Indeed, $p_{1, j} - p_{1, j'}$ is a polynomial of
  the form
  \[
    \sum_{j = 1}^{m_1}a_j p_{1, j} + \sum_{j = 1}^{m} b_j f_j (X_1, \ldots, X_n),
  \]
  for some constants $a_j, b_j$. If the polynomial is non-zero, but setting
  $\p{F}_b = \p{F}(X_1, \ldots, X_n)$ makes this polynomial vanish, then some
  $f_k$ must be dependent on some $\p{P}_1, \p{F} \setminus \smallset{f_k}$.

  Now we set $X_1 \ldots, X_n, \p{F}_{1 - b}$ and bound probability that for
  some $i$ and $j, j'$ holds
  $(p_{i, j}(x_1, \ldots, x_n) - p_{i, j'}(x_1, \ldots, x_n) = 0$ for
  $p_{i, j} \neq p_{i, j'}$. By the construction, the maximum total degree of
  these polynomials is
  $d = \max(d_{\p{P}_1}+ d_{\p{P}_2}, d_{\p{P}_T}, d_{\p{F}})$, where $d_f$ is
  the total degree of some polynomial $f$ and for a set of polynomials
  $F = \smallset{f_1, \ldots, f_k}$, we write
  $d_F = \smallset{d_{f_1}, \ldots, d_{f_k}}$. Thus, for a given $j, j'$ probability that a random assignment to
  $X_1, \ldots, X_n, Y_1, \ldots, Y_n$ is a root of $p_{i, j} - p_{i, j'}$ is,
  by the Schwartz-Zippel lemma, bounded by $\infrac{d}{p}$, which is
  negligible. There is at most $2 \cdot {q + m_0 + m_1 + m  \choose 2}$ such
  pairs $p_{i, j}, p_{i, j'}$ we have that
  \[
    \prob{\bad} \leq  {q + m_0 + m_1 + m  \choose 2} \cdot \frac{2d}{p} \leq (q
    + m_0 + m_1 + m)^2 \frac{d}{p}.
  \]

  As noted, if $\bad$ does not occur then the simulation is perfect. Also the
  bit $b$ has been chosen independently on the $\adv$'s view, thus $\condprob{b
    = b'}{\neg \bad} = \infrac{1}{2}$. Hence,
  \[
    \begin{aligned}
      \prob{b = b'} & \leq \condprob{b = b'}{\neg \bad}(1 - \prob{\bad}) + \prob{\bad} =
      \frac{1}{2} + \frac{\prob{\bad}}{2} \\
      \prob{b = b'} & \geq \condprob{b = b'}{\neq \bad}(1 - \prob{\bad}) =
      \frac{1}{2} - \frac{\prob{\bad}}{2}.
    \end{aligned}
  \]
  Finally,
  \[
    \abs{\Pr[b = b'] - \frac{1}{2}} \leq \prob{\bad}/2 \leq (q
    + m_0 + m_1 + m)^2 \frac{d}{2p}
  \]
  as required.
\end{proof}


\section{SNARKY Signatures Construction}\label{sec:SoKconstruction}

We will recall the construction of 
a signature of knowledge scheme for messages in $\{0, 1\}^*$
from an SE-SNARK as shown in \cite{C:GroMal17}. 
 
Consider a target collision-resistant hash function also known as universal oneway hash function $H:\{0, 1\}^{\ell_k} \times \{0, 1\}^* \to \{0, 1\}^{\ell_h}$ with $\ell_K, \ell_h$ polynomials in the security parameter $\secpar$. 

The target collision-resistance of $H$ requires that for any $\ppt$ adversary $\adv$ : 
 
	\[  \condprob{
	m_0 \gets \adv(1^\secpar); K \gets \{0, 1\}^{\ell_K}; m_1 \gets \adv(K)
}{
	m_0 \neq m_1 ~\land  ~H_K(m_0)=H_K(m_1)
} \leq \negl.
	\]
	
For any given relation $\REL'$, we consider the  following relation $\REL$:
 	\[\REL= \{((K,h,\inp),\wit) : K\in  \{0, 1\}^{\ell_K} ~\land ~h \in  \{0, 1\}^{\ell_h} ~\land ~(\inp, \wit) \in \REL'\}.
 	\]
 	
Let $\RELGEN(\secparam)$ be the relation generator that runs $\REL' \gets \RELGEN'(\secparam)$ and returns $\REL$ defined as above. 
Let $H$ be a target collision-resistant hash function and let $\ps = (\kgen, \prover, \verifier, \simulator)$ be a SE-NIZK argument for $\RELGEN$. 
We build a signature of knowledge  $\SoK$ for $\RELGEN'$ that works as follows: 

\begin{description}
    \item[$\signsetup(1^\secpar, \REL') \rightarrow  \param$:]
 Run $(\srs,\rho) \leftarrow \kgen(\REL)$. 
 Return $ \param = \srs$.

    \item[$\sign(\mesage, \inp, \wit)  \rightarrow \signature$:]
Sample $K \gets \{0, 1\}^{\ell_K}$.  Compute
$\zkproof \gets  \prover(\srs, (K, H_K(\mesage), \inp), \wit) $.
Return $\signature = (K, \zkproof)$.

    \item[$\verify(\mesage, \inp, \signature) \rightarrow b$:]
Parse $\signature = (K, \zkproof)$.
Return $b \gets \verifier(\srs, (K, H_K(\mesage), \inp), \zkproof)$.
	
    \item[$\simsetup(\REL') \rightarrow (\param, \td)$:]
  Run   $(\srs,\rho) \leftarrow \kgen(\REL)$. 
 Return $( \param = \srs,  \td = \rho)$.

    	
   \item[$\simsign(\td, \mesage, \inp) \rightarrow \signature'$:]
   Sample $K \gets \{0, 1\}^{\ell_K}$.  Compute
$\zkproof \gets \simulator(\srs, \rho, (K, H_K(\mesage), \inp)) $.
Return $\signature' = (K, \zkproof)$.

\end{description} 


\begin{theorem}[Signature of Knowledge] 
  \label{thm:SoK}
If $H:\{0, 1\}^{\ell_k} \times \{0, 1\}^* \to \{0, 1\}^{\ell_h}$ is a target collision-resistant hash function 
and $\ps = (\kgen, \prover, \verifier, \simulator)$ be is a SE-NIZK for $\RELGEN(\secparam)$, then the scheme
 $\SoK = (\signsetup,  \sign, \allowbreak \verify,  \simsetup, \simsign)$ above is a signature of knowledge for $\RELGEN'(\secparam)$ for  the message space $\{0, 1\}^*$.
\end{theorem}
 The proof can be found in 
 \cite{C:GroMal17}.
	

%%% Local Variables:
%%% mode: latex
%%% TeX-master: "main"
%%% End:

% !TEX root = main.tex
% !TEX spellcheck = en-US



\section{Polynomial Commitment Schemes}
\label{sec:pcom}
A polynomial commitment scheme $\PCOM = (\kgen, \com, \open, \verify)$ consists of four
algorithms and allows to commit to a polynomial $\p{f}$ and later open the evaluation in a
point $z$ to some value $s=\p{f}(z)$. More formally:
\begin{description}
\item[$\kgen(1^\secpar, \maxdeg)$:] The key generation algorithm takes in a security
  parameter $\secpar$ and a parameter $\maxdeg$ which determines the maximal degree of the
  committed polynomial. It outputs a structured reference string $\srs$ (the commitment
  key). In the following we will consider $\srs$ implicitly determines $\secpar$.
\item[$\com(\srs, \p{f})$:] The commitment algorithm $\com(\srs, \p{f})$ takes
  in $\srs$ and a polynomial $\p{f}$ with maximum degree $\maxdeg$, and outputs
  a commitment $c$.
\item[$\open(\srs, z, s, \p{f})$:] The opening algorithm
  takes as input $\srs$, an evaluation point $z$, a
  value $s$ and the polynomial $\p{f}$. It outputs an opening $o$.
\item[$\verify(\srs, c, z, s, o)$:] The verification algorithm takes in $\srs$,
  a commitment $c$, an evaluation point $z$, a value $s$ and an opening $o$. It
  outputs 1 if $o$ is a valid opening for $(c, z, s)$ and 0 otherwise.
\end{description} 

A secure polynomial commitment $\PCOM$ should satisfy correctness, evaluation binding,
opening uniqueness, hiding and knowledge-binding.  Note that since we are in the updatable
setting, $\srs$ in these security definitions is the SRS that $\advse$ finalises using the
update oracle $\initU$ (See~\cref{fig:upd}).

\begin{description}
\item[Evaluation binding:] A $\ppt$ adversary $\adv$ which outputs a commitment
  $\vec{c}$ and evaluation points $\vec{z}$ has at most negligible chances to open
  the commitment to two different evaluations $\vec{s}, \vec{s'}$. That is, let
  $k \in \NN$ be the number of committed polynomials, $l \in \NN$ number of
  evaluation points, $\vec{c} \in \GRP^k$ be the commitments, $\vec{z} \in
  \FF_p^l$ be the arguments the polynomials are evaluated at, $\vec{s},\vec{s}'
  \in \FF_p^k$ the evaluations, and $\vec{o},\vec{o}' \in \FF_p^l$ be the
  commitment openings. Then for every $\ppt$ adversary $\adv$
	\[
		\condprob{
			\begin{matrix}
				  \verify(\srs, \vec{c}, \vec{z}, \vec{s}, \vec{o}) = 1,  \\ 
				  \verify(\srs, \vec{c}, \vec{z}, \vec{s}', \vec{o}') = 1, \\
				  \vec{s} \neq \vec{s}'
			\end{matrix}}
			{
			\begin{matrix}
%				& \srs \gets \kgen(\secparam, \maxdeg),\\
				 (\vec{c}, \vec{z}, \vec{s}, \vec{s}', \vec{o}, \vec{o}') \gets \adv^{\initU}(\maxdeg)
			\end{matrix}
		} \leq \negl\,.
	\]

\end{description}
	
%We say that $\PCOM$ has the unique opening property if the following holds:
To show unique response property of our schemes we require that the polynomial
commitment scheme the proof system is using has unique openings defined as follows.
\begin{description}
\item[Opening uniqueness:] Intuitively, opening uniqueness assures that there is only one
  valid opening for the committed polynomial and given evaluation point. This property is
  crucial in showing \COMMENT{forking }simulation-extractability of $\plonk$, $\sonic$ and
  $\marlin$.
  Let $k \in \NN$ be the number of committed polynomials, $l \in \NN$ number of evaluation
  points, $\vec{c} \in \GRP^k$ be the commitments, $\vec{z} \in \FF_p^l$ be the arguments
  the polynomials are evaluated at, $\vec{s} \in \FF_p^k$ the evaluations, and
  $\vec{o}, \vec{o}' \in \FF_p^l$ be the commitment openings. Then for every $\ppt$ adversary $\adv$
	\[
		\condprob{
			\begin{matrix}
				  \verify(\srs, \vec{c}, \vec{z}, \vec{s}, \vec{o}) = 1,  \\ 
				  \verify(\srs, \vec{c}, \vec{z}, \vec{s}, \vec{o'}) = 1, \\
				 \vec{o} \neq \vec{o'}
			\end{matrix}
		}{
			\begin{matrix}
%				& \srs \gets \kgen(\secparam, \maxdeg),\\
				  (\vec{c}, \vec{z}, \vec{s}, \vec{o}, \vec{o'}) \gets \adv^{\initU}(\maxdeg)
			\end{matrix}
		}\leq \negl\,.
	\]
\end{description}
We show
that $\plonk$'s, $\sonic$'s, and $\marlin$'s polynomial commitment schemes satisfy this
requirement in \cref{lem:pcomp_op} and \cref{lem:pcoms_unique_op}
respectively.


\begin{description}
\item[Hiding:] We also formalize notion of $k$-hiding property of a polynomial commitment scheme. Let $\HHH$ be a set of size $\maxdeg + 1$ and $\ZERO_\HHH$ its
  vanishing polynomial. We say that a polynomial scheme is \emph{hiding} with
  security $\epsh(\secpar)$ if for every $\ppt$ adversary $\adv$, $k \in \NN$,
  probability
  \begin{align*}
    \condprob
   { b' = b}{
    (f_0, f_1, c, k, b') \gets \adv^{\initU, \oraclec}(\maxdeg, c), f_0, f_1 \in \FF^{\maxdeg}
    [X]}
\leq \frac{1}{2} + \eps(\secpar).
  \end{align*}
  Where $c = f'_b (\chi)$, for a random bit $b$ and the polynomial
      $f'_b (X) = f_b + \ZERO_\HHH (X) (a_0 + a_1 X + \ldots a_{k - 1} X^{k -
        1})$,
and the oracle $\oraclec$ on adversary's evaluation query $x$ it adds $x$ to initially empty set
      $Q_x$ and if $|Q_x| \leq k$, it provides $f'_b (x)$.
 
  \end{description}

\begin{description}
\item[Commitment of knowledge:] Intuitively, when a commitment scheme is ``of knowledge'' then if an
adversary produces a (valid) commitment $c$, which it can open correctly in an evaluation point, then it must
know the underlying polynomial $\p{f}$ which commits to that value.  For every $\ppt$ adversary $\adv$ who produces
  commitment $c$, evaluation $s$ and opening $o$ there
  exists a $\ppt$ extractor $\ext$ such that
\[
  \condprob{
    \begin{matrix}
       \deg \p{f} \leq \maxdeg,
       c = \com(\srs, \p{f}),\\
       \verify(\srs, c, z, s, o) = 1
    \end{matrix}
        }{
    \begin{matrix}
      %
     % & \srs \gets \kgen(\secparam, \maxdeg),\\
      c \gets \adv^{\initU}(\maxdeg),
      z \sample \FF_p \\
      (s, o) \gets \adv(c, z), \\
   \p{f} = \ext_\adv(\srs, c)\\
    \end{matrix}}
  \geq 1 - \epsk(\secpar).
\]
In that case we say that $\PCOM$ is $\epsk(\secpar)$-knowledge.
\end{description}


\cref{fig:pcomp,fig:pcoms} present variants of KZG~\cite{AC:KatZavGol10} polynomial
commitment schemes used in \plonk{}, \sonic{} and \marlin{}. The key generation algorithm
$\kgen$ takes as input a security parameter $\secparam$ and a parameter $\maxdeg$ which
determines the maximal degree of the committed polynomial. We assume that $\maxdeg$ can be
read from the output SRS. While the figures only describe trusted SRS setup, it is not
hard to lift the SRS generation into the updatable setting by defining the extra
algorithms $\upd$, $\verifyCRS$ (see~\cref{def:upd-scheme}) as described in~\cref{fig:upd-scheme}.  \cite{CCS:MBKM19}
shows, using AGM, that $\PCOMs$ is a commitment of knowledge.  The same reasoning could be
used to show that property for $\PCOMp$.
 



\begin{figure}[t!]
	\centering
	\fbox{
		\begin{minipage}[t]{0.76\linewidth}
			\procedure{$\kgen(\REL)$}{
				\chi \sample \FF_p \\ [\myskip]
				\srs := 
				\left( \gone{\smallset{\chi^i}_{i = 0}^{\dconst}},
				\gtwo{\chi} \right); 
				\rho =  \left(\gone{\chi, \chi}, \gtwo{\chi}\right) \\ [\myskip]
				\pcreturn (\srs, \rho) \\ [\myskip]
			}
		%
		\\
		%
		\procedure{$\verifyCRS(\srs, \{\rho_j \}_{j=1}^n)$}{
			\text{Parse }  \srs \text{ as } \left( \gone{\smallset{A_i}_{i = 0}^{\dconst}},
			\gtwo{B} \right) \text{and } \{\rho_j \}_{j=1}^n \text{ as } \left\{\left( P_j, \bP_j, \hP_j \right)\right\}_{j=1}^n \\ [\myskip]
			\text{Verify Update proofs: } \\ [\myskip]
			\t \bP_1 = P_1 \\ [\myskip]
			\t P_j \bullet \gtwo{1} = P_{j-1} \bullet \hP_j \quad \forall j \geq 2 \\ [\myskip]
			\t \bP_n \bullet \gtwo{1} = \gone{1} \bullet \hP_n \\ [\myskip]
			\text{Verify SRS structure: } \\ [\myskip]
			\t \gone{A_i} \bullet \gtwo{1} = \gone{A_{i-1}} \bullet \gtwo{B} \text{ for all } 0 < i \leq \dconst \\ [\myskip]
		}
		%
		\\
		%
		\procedure{$\upd(\srs, \{\rho_j \}_{j=1}^n)$}{
			\text{Parse } \srs \text{ as } \left( \gone{\smallset{A_i}_{i = 0}^{\dconst}},
			\gtwo{B} \right) \\ [\myskip]
			\chi' \sample \FF_p  \\ [\myskip]
			\srs' := 
			\left( \gone{\smallset{{\chi'}^i A_i}_{i = 0}^{\dconst}},
			\gtwo{\chi' B} \right); 
			\rho' =	\left( \gone{\chi' A_1, \chi'}, \gtwo{\chi'}\right) \\ [\myskip]
			\pcreturn (\srs', \rho')
		}
		\end{minipage}}
	\caption{Updatable SRS scheme for $\PCOMp$} 
	\label{fig:upd-scheme}
\end{figure}


\begin{figure}
  \small
  \hspace*{-2cm}\fbox{
\begin{minipage}{15,5cm}
\begin{pcvstack}[]
  \begin{pchstack}
			\procedure{$\kgen(\secparam, \maxdeg)$}
			{
			\chi \sample \FF_p \\ [\myskip]
			\pcreturn \srs = \gone{1, \ldots, \chi^{\maxdeg}}, \gtwo{\chi}\\ [\myskip]
      }\\
      \michals{29.04}{adjust to $\initU$}
			
			\pchspace
			
			\procedure{$\com(\srs, \vec{\p{f}}(X))$}
			{ 
				\pcreturn \gone{\vec{c}} = \gone{\vec{\p{f}}(\chi)}\\ [\myskip]
        \fbox{$\pcreturn \vec{\p{f}} (X)$}\\
			}

      \pchspace

      \procedure{$\open(\srs, \vec{z}, \vec{s}, \vec{\p{f}}(X), \aux)$}
			{
      \vec{\gamma} \gets \ro (g_0( \vec{z}, \vec{s}, \gone{\vec{c}}, \aux_0))\\[\myskip]
			\pcfor i \in \range{1}{\abs{\vec{z}}} \pcdo\\ [\myskip]
      \pcind \p{o}_j(X) \gets \sum_{i \in K_j} \gamma_{j}^{i - 1}
      \frac{\p{f}_{i}(X) - \p{f}_{i}(z_j)}{X - z_j}\\ [\myskip] 
      \pcreturn \vec{o} = \gone{\vec{\p{o}}(\chi)}\\ [\myskip]
      \fbox{$\pcreturn \vec{\p{o}} (X)$}
				% \hphantom{\hspace*{5.5cm}}	
			}

    \end{pchstack}
		 \pcvspace
    
		\begin{pchstack}
			\procedure{$\verifyb(\srs, \gone{\vec{c}}, \vec{z}, \vec{s}, \gone{\p{o}(\chi)}, \aux)$}
			{
        \vec{\gamma} \gets \ro (g_0( \vec{z}, \vec{s}, \gone{\vec{c}}, \aux_0))\\[\myskip]
				r \gets \ro (g_1(\gone{\vec{c}}, \vec{z}, \vec{s}, \gone{\p{o}(\chi)}, \aux_1))\\ [\myskip]
				% \pcfor j \in \range{1}{\abs{\vec{z}}} \pcdo \\ [\myskip]
				(*) \pcif 
          \sum_{j = 1}^{\abs{\vec{z}}} r^{j} \cdot \gone{\sum_{i \in K_j}
          \gamma_j^{i - 1} c_{i} - \sum_{i \in K_j} \gamma_j^{i - 1} s_{i_j}} \bullet \gtwo{1} + \\ [\myskip] 
          \pcind \sum_{j = 1}^{\abs{\vec{z}}} r^{j} z_j o_j
          \bullet \gtwo{1} \neq \gone{\sum_{j = 1}^{\abs{\vec{z}}} r^{j} o_j }
          \bullet \gtwo{\chi} \pcthen  \\
					\pcind \pcreturn 0\\ [\myskip]
          \fbox{
            \begin{minipage}{7cm}
            (**) $\pcif $
              $\sum_{j = 1}^{\abs{\vec{z}}} r^j \cdot (\sum_{i \in K_j}
              \gamma_j^{i - 1} \p{f}_{i} (X) - \sum_{i \in K_j} \gamma_j^{i - 1} s_{i_j}) + $\\ [\myskip] 
              $\pcind \sum_{j = 1}^{\abs{\vec{z}}} r^{j} z_j \p{o}_j (X)
               \neq \sum_{j = 1}^{\abs{\vec{z}}} r^{j} \p{o}_j (X)
              \cdot X \pcthen $ \\
              $\pcind \pcreturn 0$
            \end{minipage}
          }\\[\myskip]
					\pcreturn 1.\\
			}

      \pchspace
      
      \procedure{$\verify(\srs, \gone{\vec{c}}, \vec{z}, \vec{s}, \gone{\p{o}(\chi)})$}
			{
        \vec{\gamma} \gets \ro (g_0( \vec{z}, \vec{s}, \gone{\vec{c}}, \aux_0))\\[\myskip]
				\pcfor j \in \range{1}{\abs{\vec{z}}} \pcdo \\ [\myskip]
				\pcind \pcif 
          \gone{\sum_{i \in K_j}
          \gamma_j^{i - 1} c_{i} - \sum_{i \in K_j} \gamma_j^{i - i} s_{i, j}} \bullet
          \gtwo{1} + \\ [\myskip] \pcind  z_j
          o_j
          \bullet \gtwo{1} \neq \gone{o_j}
          \bullet \gtwo{\chi} \pcthen  \\
					\pcind \pcreturn 0\\ [\myskip]
        \fbox{
          \begin{minipage}{7cm}
          $\pcind \pcif $
          $\sum_{i \in K_j} \gamma_j^{i - 1} \p{f}_{i} (X) - \sum_{i \in K_j} \gamma_j^{i - i} s_{i, j} + $\\ [\myskip] 
          $\pcind  z_j \p{o}_j (X) \neq \p{o}_j (X) X \pcthen \pcreturn 0$
        \end{minipage}
        }\\ [\myskip]
					\pcreturn 1.
			}

    \end{pchstack}
	\end{pcvstack}
\end{minipage}
  }
	\caption{$\PCOM$ polynomial commitment scheme. Here $\abs{\vec{z}} = l$ is the number of evaluation points, the number of committed polynomials is $m$, $K_j$ is the set of polynomials that was evaluated at point $z_j$. Functions $g_0$ and $g_1$ are injective and specific to the context in which the polynomial commitment is used. (In our case, functions $g_0$ and $g_1$ are produce partial transcripts of the proof that utilizes the commitment scheme, $\aux$ contains all additional information that is needed by the functions.)
  In the boxes we describe values returned or equality computed in the ideal protocol where the verifier checks equalities on the polynomials instead of their evaluations. For algorithm $\pcalgostyle{Alg}$ we denote its ideal variant by $\pcalgostyle{Alg}'$.}
	\label{fig:pcomp}
  \end{figure}

% \begin{figure}[h!]
% \centering
% 	\begin{pcvstack}[center,boxed]
% 		\begin{pchstack}
% 			\procedure{$\kgen(\secparam, \maxdeg)$} {
% 				\alpha, \chi \sample \FF^2_p \\ [\myskip]
% 				\pcreturn \gone{\smallset{\chi^i}_{i = -\multconstr}^{\multconstr},
%           \smallset{\alpha \chi^i}_{i = -\multconstr, i \neq
%             0}^{\multconstr}},\\
%         \pcind \gtwo{\smallset{\chi^i, \alpha \chi^i}_{i =
%             -\multconstr}^{\multconstr}}, \gtar{\alpha}\\
% 				%\markulf{03.11.2020}{} \\
% 			%	\hphantom{\pcind \p{o}_i(X) \gets \sum_{j = 1}^{t_i} \gamma_i^{j - 1} \frac{\p{f}_{i,j}(X) - \p{f}_{i, j}(z_i)}{X - z_i}}
% 				\hphantom{\hspace*{5.5cm}}
% 		}
%
% 			\pchspace
%
% 			\procedure{$\com(\srs, \maxconst, \p{f}(X))$} {
% 				\p{c}(X) \gets \alpha \cdot X^{\dconst - \maxconst} \p{f}(X) \\ [\myskip]
% 				\pcreturn \gone{c} = \gone{\p{c}(\chi)}\\ [\myskip]
% 				\hphantom{\pcind \pcif \sum_{i = 1}^{\abs{\vec{z}}} r_i \cdot
%           \gone{\sum_{j = 1}^{t_j} \gamma_i^{j - 1} c_{i, j} - \sum_{j = 1}^{t_j}
%             s_{i, j}} \bullet \gtwo{1} + } }
% 		\end{pchstack}
% 		% \pcvspace
%
% 		\begin{pchstack}
% 			\procedure{$\open(\srs, z, s, f(X))$}
% 			{
% 				\p{o}(X) \gets \frac{\p{f}(X) - \p{f}(z)}{X - z}\\ [\myskip]
% 				\pcreturn \gone{\p{o}(\chi)}\\ [\myskip]
% 				\hphantom{\hspace*{5.5cm}}
% 			}
%
% 			\pchspace
%
% 			\procedure{$\verify(\srs, \maxconst, \gone{c}, z, s, \gone{\p{o}(\chi)})$}
%       {
%         \pcif \gone{\p{o}(\chi)} \bullet \gtwo{\alpha \chi} + \gone{s - z
%         \p{o}(\chi)} \bullet \gtwo{\alpha} = \\ [\myskip] \pcind \gone{c}
%         \bullet \gtwo{\chi^{- \dconst + \maxconst}} \pcthen  \pcreturn 1\\
%         [\myskip]
%         \rlap{\pcelse \pcreturn 0.} \hphantom{\pcind \pcif \sum_{i =
%             1}^{\abs{\vec{z}}} r_i \cdot \gone{\sum_{j = 1}^{t_j} \gamma_i^{j -
%               1} c_{i, j} - \sum{j = 1}^{t_j} s_{i, j}} \bullet \gtwo{1} + } }
% 		\end{pchstack}
% 	\end{pcvstack}
%
% 	\caption{$\PCOMs$ polynomial commitment scheme.}
% 	\label{fig:pcoms}
% \end{figure}

\subsection{Unique opening property of $\PCOM$}
Now, we show that the batched variant of the KZG polynomial
commitment scheme that is used in \plonk{} and $\marlin{}$, has the unique opening property.

\begin{lemma}
\label{lem:pcomp_op}
Let $\PCOM = (\kgen, \com, \open, \verifyb)$ be a batched version of a KZG polynomial commitment,
cf.~\cref{fig:pcomp}, that commits to $m$ polynomials of degree up to $\maxdeg$. Let $\vec{z} = (z_0, \ldots, z_{l - 1}) \in \FF_p^l$ be the points the polynomials are  evaluated at, $k_i \in \NN$ be the number of the committed polynomials to be evaluated at $z_i$, and $K_i$ be the set of indices of these polynomials. Let $\vec{s_{K_i}} \in \FF_p^{k_i}$ the evaluations of polynomials at $z_i$, and $\vec{o} = (o_0, \ldots, o_{l - 1}) \in \FF_p^l$ be the commitment openings. We show that the probability an algebraic adversary $\adv$, who can made up to $q$ random oracle queries, opens the same vector of commitments in two different ways is at most $\epsop(\secpar)$, for $\epsop(\secpar) \leq l \cdot  \epsudlog(\secpar) + \infrac{1}{\abs{\FF_p}}$, where $\epsudlog(\secpar)$ is security of the $(\maxdeg, 1)$-$\udlog$ assumption and $\FF_p$ is the field used in $\PCOM$.
\end{lemma}
\begin{proof}
 
  The proof goes by game hops. In the first game the adversary wins if it presents two acceptable openings of a vector of polynomials. Then, we restrict the winning condition and require that the adversary also makes the idealized batched verifier to accept the proof. In the next game, we abort if the idealized verifier rejects a proof for one of the evaluation point. 

  \ncase{Game 0} In this game the adversary wins if it provides two different openings for a vector of polynomial commitments and their evaluations that are acceptable by $\verifyb$.

  \ncase{Game 1} This game is identical to Game 0 except it is additionally aborted if the commitment opening are not acceptable by $\verifyb'$.

  \ncase{Game 0 to Game 1} %
  Any discrepancy between the idealised verifier rejection and real verifier acceptance allows one to break the (updatable) discrete logarithm problem.  The reduction $\rdvdulog$ proceeds as follows. It answers $\adv$'s queries for SRS updates according to the answers it receives from its udlog update oracle. When $\adv$ finalizes an SRS, $\rdvdulog$ finalizes the corresponding udlog challenge $(\gone{1, \ldots, {\chi'}^{\maxdeg}}, \gtwo{1})$. We consider verification equation $(**)$ as a polynomial in $X$ and the verification equation $(*)$ as it's evaluation at $\chi'$. Consider an opening such that verification equation (**), cf.~\cref{fig:pcomp}, does not hold, i.e.~(**) is not a zero polynomial, but (*) does, i.e.~(**) zeroes at $\chi'$. Since $\adv$ is algebraic, all proof elements are extended by their representation as a combination of the input $\GRP_1$-elements. Therefore, all coefficients of the verification equation polynomial related to (**) are known. Now, $\rdvdulog$ computes the roots of (**), finds $\chi'$ among them, and returns $\chi'$. Hence the probability that the adversary wins in Game 1, but does not win in Game 0 is upper-bounded by $\epsudlog (\secpar)$.

  \ncase{Game 2} This game is identical to Game 1 except it is additionally aborted if one of the opening is not acceptable by an idealised verifier $\verify'$.

  \ncase{Game 1 to Game 2}
  The ideal verifier checks whether the following equality, for $\smallset{\gamma_j}_{j = 1}^{l}, r$
  picked at random, holds:
  \begin{multline}
    \label{eq:ver_eq_poly}
    \sum_{j = 0}^{l - 1} r^{j}\left(\sum_{i \in K_j} \gamma_{j}^{i - 1} \cdot \p{f}_i(X) - \sum_{i \in K_j} \gamma_{j}^{i - 1} \cdot {s_i}_j \right) 
    \equiv \sum_{j = 0}^{l - 1} r^{j} \p{o_{j}}(X)(X - z_j).
  \end{multline}
  Since $r$ has been picked as a random oracle output, probability that
  \cref{eq:ver_eq_poly} holds while for some $j \in \range{0}{l - 1}$
  \[
    r^j \left(\sum_{i \in K_j} \gamma^{i - 1} \cdot \p{f}_i(X) - \sum_{i \in K_j}
    \gamma^{i - 1} \cdot {s_i}_j \right) \not\equiv r^j \p{o_j}(X)(X - z_j)
  \]
  is $\infrac{q}{\abs{\FF_p}}$~cf.~\cite{EPRINT:GabWilCio19}. 
  When \(
    r^j \left(\sum_{i \in K_j} \gamma^{i - 1} \cdot \p{f}_i(X) - \sum_{i \in K_j}
    \gamma^{i - 1} \cdot {s_i}_j \right) = r^j \p{o_j}(X)(X - z_j)
  \)
  holds, polynomial $\p{o_j}(X)$ is uniquely determined from the uniqueness of
  polynomial composition. 

  \ncase{Conclusion} %
  We note that the idealised verifier $\vereq(X)$ does not accept two different openings of a correct evaluation. Hence the probability that the adversary wins Game 2 is $0$ and the probability that the adversary wins in Game 0 is upper-bounded by \(\epsudlog (\secpar) + \frac{q}{\abs{\FF_p}}\).
    \qed
\end{proof}

%%% Local Variables:
%%% mode: latex
%%% TeX-master: "main"
%%% End:

\begin{figure}[t!]
	\centering
	% \small
	\centerline{\fbox{
		\begin{minipage}[t]{0.75\linewidth}
		\begin{pcvstack}
		\begin{pchstack}
			\procedure{$\kgen(\secparam, \maxdeg)$}{
				\chi \sample \FF_p \\ [\myskip]
				\srs := 
				\left( \gone{\smallset{\chi^i}_{i = 0}^{\maxdeg}},
				\gtwo{\chi} \right); \\
				\rho =  \left(\gone{\chi, \chi}, \gtwo{\chi}\right) \\ [\myskip]
				\pcreturn (\srs, \rho) \\ [\myskip]
			}
		%
		\pchspace
		% 
		\procedure{$\upd(\srs, \{\rho_j \}_{j=1}^n)$}{
			\text{Parse } \srs \text{ as } \left( \gone{\smallset{A_i}_{i = 0}^{\maxdeg}},
			\gtwo{B} \right) \\ [\myskip]
			\chi' \sample \FF_p  \\ [\myskip]
			\srs' := 
			\left( \gone{\smallset{{\chi'}^i A_i}_{i = 0}^{\maxdeg}},
			\gtwo{\chi' B} \right); \\
			\rho' =	\left( \gone{\chi' A_1, \chi'}, \gtwo{\chi'}\right) \\ [\myskip]
			\pcreturn (\srs', \rho')
		}
		\end{pchstack}
		\pcvspace
		\begin{pchstack}
		%
		\procedure{$\verifyCRS(\srs, \{\rho_j \}_{j=1}^n)$}{
			\text{Parse }  \srs \text{ as } \left( \gone{\smallset{A_i}_{i = 0}^{\maxdeg}},
			\gtwo{B} \right) \\ \text{and } \{\rho_j \}_{j=1}^n \text{ as } \left\{\left( P_j, \bP_j, \hP_j \right)\right\}_{j=1}^n \\ [\myskip]
			\text{Verify Update proofs: } \\ [\myskip]
			\t \bP_1 = P_1 \\ [\myskip]
			\t P_j \bullet \gtwo{1} = P_{j-1} \bullet \hP_j \quad \forall j \geq 2 \\ [\myskip]
			\t \bP_n \bullet \gtwo{1} = \gone{1} \bullet \hP_n \\ [\myskip]
			\text{Verify SRS structure: } \\ [\myskip]
			\t \gone{A_i} \bullet \gtwo{1} = \gone{A_{i-1}} \bullet \gtwo{B} \text{ for all } 0 < i \leq \maxdeg %\\ [\myskip]
		}
		%
		\end{pchstack}
		\end{pcvstack}
		%
		\end{minipage}}}
	\caption{Updatable SRS scheme $\SRScer$ for $\PCOMp$} 
	\label{fig:upd-scheme}
\end{figure}


\begin{figure}
  \small
  \centerline{\fbox{
\begin{minipage}{14.3cm}
\begin{pcvstack}[]
  \begin{pchstack}
			\procedure{$\SRScer(\secparam, \maxdeg)$}
			{
			\text{cf.~\cref{fig:upd-scheme}}
      }\\
      
			\hspace*{1.7cm}
			\pchspace
			
			\procedure{$\com(\srs, \vec{\p{f}}(X))$}
			{ 
				\pcreturn \gone{\vec{c}} = \gone{\vec{\p{f}}(\chi)}\\ [\myskip]
        \fbox{$\pcreturn \vec{\p{f}} (X)$}\\
			}

      \pchspace
	  \hspace*{1.7cm}

      \procedure{$\open(\srs, \vec{z}, \vec{s}, \vec{\p{f}}(X), \aux_0)$}
			{
      \vec{\gamma} \gets \ro (g_0( \vec{z}, \vec{s}, \gone{\vec{c}}, \aux_0))\\[\myskip]
			\pcfor i \in \range{1}{\abs{\vec{z}}} \pcdo\\ [\myskip]
      \pcind \p{o}_j(X) \gets \sum_{i \in K_j} \gamma_{j}^{i - 1}
      \frac{\p{f}_{i}(X) - \p{f}_{i}(z_j)}{X - z_j}\\ [\myskip] 
      \pcreturn \vec{o} = \gone{\vec{\p{o}}(\chi)}\\ [\myskip]
      \fbox{$\pcreturn \vec{\p{o}} (X)$}
				% \hphantom{\hspace*{5.5cm}}	
			}

    \end{pchstack}
		 \pcvspace
    
		\begin{pchstack}
			\procedure{$\verifyb(\srs, \gone{\vec{c}}, \vec{z}, \vec{s}, \gone{\p{o}(\chi)}, (\aux_0,\aux_1))$}
			{
        \vec{\gamma} \gets \ro (g_0( \vec{z}, \vec{s}, \gone{\vec{c}}, \aux_0))\\[\myskip]
				r \gets \ro (g_1(\gone{\vec{c}}, \vec{z}, \vec{s}, \gone{\p{o}(\chi)}, \aux_1))\\ [\myskip]
				% \pcfor j \in \range{1}{\abs{\vec{z}}} \pcdo \\ [\myskip]
				(*) \pcif 
          \sum_{j = 1}^{\abs{\vec{z}}} r^{j} \cdot \gone{\sum_{i \in K_j}
          \gamma_j^{i - 1} c_{i} - \sum_{i \in K_j} \gamma_j^{i - 1} s_{i_j}} \bullet \gtwo{1} + \\ [\myskip] 
          \pcind \sum_{j = 1}^{\abs{\vec{z}}} r^{j} z_j o_j
          \bullet \gtwo{1} \neq \gone{\sum_{j = 1}^{\abs{\vec{z}}} r^{j} o_j }
          \bullet \gtwo{\chi} \pcthen  \\
					\pcind \pcreturn 0\\ [\myskip]
          \fbox{
            \begin{minipage}{7cm}
            (**) $\pcif $
              $\sum_{j = 1}^{\abs{\vec{z}}} r^j \cdot (\sum_{i \in K_j}
              \gamma_j^{i - 1} \p{f}_{i} (X) - \sum_{i \in K_j} \gamma_j^{i - 1} s_{i_j}) + $\\ [\myskip] 
              $\pcind \sum_{j = 1}^{\abs{\vec{z}}} r^{j} z_j \p{o}_j (X)
               \neq \sum_{j = 1}^{\abs{\vec{z}}} r^{j} \p{o}_j (X)
              \cdot X \pcthen $ \\
              $\pcind \pcreturn 0$
            \end{minipage}
          }\\[\myskip]
					\pcreturn 1.\\
			}

      \pchspace
      
      \procedure{$\verify(\srs, \gone{\vec{c}}, \vec{z}, \vec{s}, \gone{\p{o}(\chi)},\aux_0)$}
			{
        \vec{\gamma} \gets \ro (g_0( \vec{z}, \vec{s}, \gone{\vec{c}}, \aux_0))\\[\myskip]
				\pcfor j \in \range{1}{\abs{\vec{z}}} \pcdo \\ [\myskip]
				\pcind \pcif 
          \gone{\sum_{i \in K_j}
          \gamma_j^{i - 1} c_{i} - \sum_{i \in K_j} \gamma_j^{i - i} s_{i, j}} \bullet
          \gtwo{1} + \\ [\myskip] \pcind  z_j
          o_j
          \bullet \gtwo{1} \neq \gone{o_j}
          \bullet \gtwo{\chi} \pcthen  \\
					\pcind \pcreturn 0\\ [\myskip]
        \fbox{
          \begin{minipage}{6cm}
          $\pcind \pcif $
          $\sum_{i \in K_j} \gamma_j^{i - 1} \p{f}_{i} (X) - \sum_{i \in K_j} \gamma_j^{i - i} s_{i, j} + $\\ [\myskip] 
          $\pcind  z_j \p{o}_j (X) \neq \p{o}_j (X) X \pcthen \pcreturn 0$
        \end{minipage}
        }\\ [\myskip]
					\pcreturn 1.
			}
    \end{pchstack}
	\end{pcvstack}
\end{minipage}
  }}
	\caption{$\PCOMp$ polynomial commitment scheme. Here $\abs{\vec{z}} = l$ is the number of evaluation points, the number of committed polynomials is $m$, $K_j$ is the set of polynomials that was evaluated at point $z_j$. Functions $g_0$ and $g_1$ are injective and specific to the context in which the polynomial commitment is used. (In our case, functions $g_0$ and $g_1$ are produce partial transcripts of the proof that utilizes the commitment scheme, $\aux$ contains all additional information that is needed by the functions.)
  In the boxes we describe values returned or equality computed in the ideal protocol where the verifier checks equalities on the polynomials instead of their evaluations. For algorithm $\pcalgostyle{Alg}$ we denote its ideal variant by $\pcalgostyle{Alg}'$.}
	\label{fig:pcomp}
  \end{figure}

% \begin{figure}[h!]
% \centering
% 	\begin{pcvstack}[center,boxed]
% 		\begin{pchstack}
% 			\procedure{$\kgen(\secparam, \maxdeg)$} {
% 				\alpha, \chi \sample \FF^2_p \\ [\myskip]
% 				\pcreturn \gone{\smallset{\chi^i}_{i = -\multconstr}^{\multconstr},
%           \smallset{\alpha \chi^i}_{i = -\multconstr, i \neq
%             0}^{\multconstr}},\\
%         \pcind \gtwo{\smallset{\chi^i, \alpha \chi^i}_{i =
%             -\multconstr}^{\multconstr}}, \gtar{\alpha}\\
% 				%\markulf{03.11.2020}{} \\
% 			%	\hphantom{\pcind \p{o}_i(X) \gets \sum_{j = 1}^{t_i} \gamma_i^{j - 1} \frac{\p{f}_{i,j}(X) - \p{f}_{i, j}(z_i)}{X - z_i}}
% 				\hphantom{\hspace*{5.5cm}}
% 		}
%
% 			\pchspace
%
% 			\procedure{$\com(\srs, \maxconst, \p{f}(X))$} {
% 				\p{c}(X) \gets \alpha \cdot X^{\dconst - \maxconst} \p{f}(X) \\ [\myskip]
% 				\pcreturn \gone{c} = \gone{\p{c}(\chi)}\\ [\myskip]
% 				\hphantom{\pcind \pcif \sum_{i = 1}^{\abs{\vec{z}}} r_i \cdot
%           \gone{\sum_{j = 1}^{t_j} \gamma_i^{j - 1} c_{i, j} - \sum_{j = 1}^{t_j}
%             s_{i, j}} \bullet \gtwo{1} + } }
% 		\end{pchstack}
% 		% \pcvspace
%
% 		\begin{pchstack}
% 			\procedure{$\open(\srs, z, s, f(X))$}
% 			{
% 				\p{o}(X) \gets \frac{\p{f}(X) - \p{f}(z)}{X - z}\\ [\myskip]
% 				\pcreturn \gone{\p{o}(\chi)}\\ [\myskip]
% 				\hphantom{\hspace*{5.5cm}}
% 			}
%
% 			\pchspace
%
% 			\procedure{$\verify(\srs, \maxconst, \gone{c}, z, s, \gone{\p{o}(\chi)})$}
%       {
%         \pcif \gone{\p{o}(\chi)} \bullet \gtwo{\alpha \chi} + \gone{s - z
%         \p{o}(\chi)} \bullet \gtwo{\alpha} = \\ [\myskip] \pcind \gone{c}
%         \bullet \gtwo{\chi^{- \dconst + \maxconst}} \pcthen  \pcreturn 1\\
%         [\myskip]
%         \rlap{\pcelse \pcreturn 0.} \hphantom{\pcind \pcif \sum_{i =
%             1}^{\abs{\vec{z}}} r_i \cdot \gone{\sum_{j = 1}^{t_j} \gamma_i^{j -
%               1} c_{i, j} - \sum{j = 1}^{t_j} s_{i, j}} \bullet \gtwo{1} + } }
% 		\end{pchstack}
% 	\end{pcvstack}
%
% 	\caption{$\PCOMs$ polynomial commitment scheme.}
% 	\label{fig:pcoms}
% \end{figure}

\subsection{Unique Opening Property of $\PCOMp$}
\label{sec:uop}
Now, we show that the batched variant of the KZG polynomial
commitment scheme that is used in \plonk{} and $\marlin{}$, has the unique opening property.
%\markulf{30.4}{Changed this to $\PCOMp$, hope that's correct. Could simplfy this, given that the body will be only about Plonk?}

\begin{lemma}
\label{lem:pcomp_op}
Let $\PCOMp = (\kgen, \com, \open, \verifyb)$ be a batched version of a KZG polynomial commitment,
cf.~\cref{fig:pcomp}, that commits to $m$ polynomials of degree up to $\maxdeg$. Let $\vec{z} = (z_0, \ldots, z_{l - 1}) \in \FF_p^l$ be the points the polynomials are  evaluated at, $k_i \in \NN$ be the number of the committed polynomials to be evaluated at $z_i$, and $K_i$ be the set of indices of these polynomials. Let $\vec{s_{K_i}} \in \FF_p^{k_i}$ be the evaluations of polynomials at $z_i$, and $\vec{o} = (o_0, \ldots, o_{l - 1}) \in \FF_p^l$ be the commitment openings. We show that the probability an algebraic adversary $\adv$, who can made up to $q$ random oracle queries, opens the same vector of commitments in two different ways is at most $\epsop(\secpar)$, for $\epsop(\secpar) \leq l \cdot  \epsudlog(\secpar) + \infrac{q}{p}$, where $\epsudlog(\secpar)$ is security of the $(\maxdeg, 1)$-$\udlog$ assumption and $p$ is the field size used in $\PCOMp$.
\end{lemma}
\begin{proof}
 
  The proof goes by game hops. In the first game the adversary wins if it presents two accepting openings of a vector of polynomials. Then, we restrict the winning condition and require that the adversary also makes the idealized batched verifier to accept the proof. In the next game, we abort if the idealized verifier rejects a proof for one of the evaluation point. 

  \ncase{Game 0} In this game the adversary wins if it provides two different openings for a vector of polynomial commitments and their evaluations that are accepting by $\verifyb$.

  \ncase{Game 1} This game is identical to Game 0 except it is additionally aborted if the commitment opening are not accepting by $\verifyb'$.

  \ncase{Game 0 to Game 1} %
  Any discrepancy between the idealized verifier rejection and real verifier acceptance allows one to break the (updatable) discrete logarithm problem.  The reduction $\rdvdulog$ proceeds as follows. It answers $\adv$'s queries for SRS updates according to the answers it receives from its udlog update oracle. When $\adv$ finalizes an SRS, $\rdvdulog$ finalizes the corresponding udlog challenge $(\gone{1, \ldots, {\chi'}^{\maxdeg}}, \gtwo{1})$. We consider verification equation $(**)$ as a polynomial in $X$ and the verification equation $(*)$ as it's evaluation at $\chi'$. Consider an opening such that verification equation (**), cf.~\cref{fig:pcomp}, does not hold, i.e.~(**) is not a zero polynomial, but (*) does, i.e.~(**) zeroes at $\chi'$. Since $\adv$ is algebraic, all proof elements are extended by their representation as a combination of the input $\GRP_1$-elements. Therefore, all coefficients of the verification equation polynomial related to (**) are known. Now, $\rdvdulog$ computes the roots of (**), finds $\chi'$ among them, and returns $\chi'$. Hence the probability that the adversary wins in Game 1, but does not win in Game 0 is upper-bounded by $\epsudlog (\secpar)$.

  \ncase{Game 2} This game is identical to Game 1 except it is additionally aborted if one of the opening is not accepting by an idealized verifier $\verify'$.

  \ncase{Game 1 to Game 2}
  The ideal verifier checks whether the following equality, for $\smallset{\gamma_j}_{j = 1}^{l}, r$
  picked at random, holds:
  \begin{multline}
    \label{eq:ver_eq_poly}
    \sum_{j = 0}^{l - 1} r^{j}\left(\sum_{i \in K_j} \gamma_{j}^{i - 1} \cdot \p{f}_i(X) - \sum_{i \in K_j} \gamma_{j}^{i - 1} \cdot {s_i}_j \right) 
    \equiv \sum_{j = 0}^{l - 1} r^{j} \p{o_{j}}(X)(X - z_j).
  \end{multline}
  Since $r$ has been picked as a random oracle output, probability that
  \cref{eq:ver_eq_poly} holds while for some $j \in \range{0}{l - 1}$
  \[
    r^j \left(\sum_{i \in K_j} \gamma^{i - 1} \cdot \p{f}_i(X) - \sum_{i \in K_j}
    \gamma^{i - 1} \cdot {s_i}_j \right) \not\equiv r^j \p{o_j}(X)(X - z_j)
  \]
  is $\infrac{q}{p}$~cf.~\cite{EPRINT:GabWilCio19}. 
  When \(
    r^j \left(\sum_{i \in K_j} \gamma^{i - 1} \cdot \p{f}_i(X) - \sum_{i \in K_j}
    \gamma^{i - 1} \cdot {s_i}_j \right) = r^j \p{o_j}(X)(X - z_j)
  \)
  holds, polynomial $\p{o_j}(X)$ is uniquely determined from the uniqueness of
  polynomial composition. 

  \ncase{Conclusion} %
  We note that the idealized verifier $\vereq(X)$ does not accept two different openings of a correct evaluation. Hence the probability that the adversary wins Game 2 is $0$ and the probability that the adversary wins in Game 0 is upper-bounded by \(\epsudlog (\secpar) + \frac{q}{p}\).
    \qed
\end{proof}
%\section{Non-malleability of \plonk{}, omitted protocol descriptions}
\label{sec:plonk_supp_mat}

\newcommand{\vql}{\vec{q_{L}}}
\newcommand{\vqr}{\vec{q_{R}}}
\newcommand{\vqm}{\vec{q_{M}}}
\newcommand{\vqo}{\vec{q_{O}}}
\newcommand{\vx}{\vec{x}}
\newcommand{\vqc}{\vec{q_{C}}}
\subsection{Plonk protocol description}
\label{sec:plonk_explained}
\oursubsub{The constrain system}
Assume $\CRKT$ is a fan-in two arithmetic circuit,
which fan-out is unlimited and has $\numberofconstrains$ gates and $\noofw$ wires
($\numberofconstrains \leq \noofw \leq 2\numberofconstrains$). \plonk's constraint
system is defined as follows:
\begin{itemize}
\item Let $\vec{V} = (\va, \vb, \vc)$, where $\va, \vb, \vc
  \in \range{1}{\noofw}^\numberofconstrains$. Entries $\va_i, \vb_i, \vc_i$ represent indices of left,
  right and output wires of circuits $i$-th gate.
\item Vectors $\vec{Q} = (\vql, \vqr, \vqo, \vqm, \vqc) \in
  (\FF^\numberofconstrains)^5$ are called \emph{selector vectors}:
  \begin{itemize}
  \item If the $i$-th gate is a multiplicative gate then $\vql_i = \vqr_i = 0$,
    $\vqm_i = 1$, and $\vqo_i = -1$. 
  \item If the $i$-th gate is an addition gate then $\vql_i = \vqr_i  = 1$, $\vqm_i =
    0$, and $\vqo_i = -1$. 
  \item $\vqc_i = 0$ always. 
  \end{itemize}
\end{itemize}

We say that vector $\vx \in \FF^\noofw$ satisfies constraint system if for all $i
\in \range{1}{\numberofconstrains}$
\[
  \vql_i \cdot \vx_{\va_i} + \vqr_i \cdot \vx_{\vb_i} + \vqo \cdot \vx_{\vc_i} +
  \vqm_i \cdot (\vx_{\va_i} \vx_{\vb_i}) + \vqc_i = 0. 
\]

\oursubsub{Algorithms rolled out}
\label{sec:plonk_explained}
\plonk{} argument system is universal. That is, it allows to verify computation
of any arithmetic circuit which has no more than $\numberofconstrains$
gates using a single SRS. However, to make computation efficient, for each
circuit there is allowed a preprocessing phase which extend the SRS with
circuit-related polynomial evaluations.

For the sake of simplicity of the security reductions presented in this paper, we
include in the SRS only these elements that cannot be computed without knowing
the secret trapdoor $\chi$. The rest of the SRS---the preprocessed input---can
be computed using these SRS elements thus we leave them to be computed by the
prover, verifier, and simulator.

\ourpar{$\plonk$ SRS generating algorithm $\kgen(\REL)$:}
The SRS generating algorithm picks at random $\chi \sample \FF_p$, computes
and outputs
\[
	\srs = \left(\gone{\smallset{\chi^i}_{i = 0}^{\numberofconstrains + 2}},
	\gtwo{\chi} \right).
\]

\ourpar{Preprocessing:}
Let $H = \smallset{\omega^i}_{i = 1}^{\numberofconstrains }$ be a
(multiplicative) $\numberofconstrains$-element subgroup of a field $\FF$
compound of $\numberofconstrains$-th roots of unity in $\FF$. Let $\lag_i(X)$ be
the $i$-th element of an $\numberofconstrains$-elements Lagrange basis. During
the preprocessing phase polynomials $\p{S_{id j}}, \p{S_{\sigma j}}$, for
$\p{j} \in \range{1}{3}$, are computed:
\begin{equation*}
  \begin{aligned}
    \p{S_{id 1}}(X) & = X,\vphantom{\sum_{i = 1}^{\noofc} \sigma(i) \lag_i(X),}\\
    \p{S_{id 2}}(X) & = k_1 \cdot X,\vphantom{\sum_{i = 1}^{\noofc} \sigma(i) \lag_i(X),}\\
    \p{S_{id 3}}(X) & = k_2 \cdot X,\vphantom{\sum_{i = 1}^{\noofc} \sigma(i) \lag_i(X),}
  \end{aligned}
  \qquad
\begin{aligned}
  \p{S_{\sigma 1}}(X) & = \sum_{i = 1}^{\noofc} \sigma(i) \lag_i(X),\\
  \p{S_{\sigma 2}}(X) & = \sum_{i = 1}^{\noofc}
  \sigma(\noofc + i) \lag_i(X),\\
  \p{S_{\sigma 3}}(X) & =\sum_{i = 1}^{\noofc} \sigma(2 \noofc + i) \lag_i(X).
\end{aligned}
\end{equation*}
Coefficients $k_1$, $k_2$ are such that $H, k_1 \cdot H, k_2 \cdot H$ are
different cosets of $\FF^*$, thus they define $3 \cdot \noofc$
different elements. \cite{EPRINT:GabWilCio19} notes that it is enough to set
$k_1$ to a quadratic residue and $k_2$ to a quadratic non-residue.

Furthermore, we define polynomials $\p{q_L}, \p{q_R}, \p{q_O}, \p{q_M}, \p{q_C}$
such that
\begin{equation*}
  \begin{aligned}
  \p{q_L}(X) & = \sum_{i = 1}^{\noofc} \vql_i \lag_i(X), \\
  \p{q_R}(X) & = \sum_{i = 1}^{\noofc} \vqr_i \lag_i(X), \\
  \p{q_M}(X) & = \sum_{i = 1}^{\noofc} \vqm_i \lag_i(X),
\end{aligned}
\qquad
\begin{aligned}
  \p{q_O}(X) & = \sum_{i = 1}^{\noofc} \vqo_i \lag_i(X), \\
  \p{q_C}(X) & = \sum_{i = 1}^{\noofc} \vqc_i \lag_i(X). \\
  \vphantom{\p{q_M}(X)  = \sum_{i = 1}^{\noofc} \vqm_i \lag_i(X),}
\end{aligned}
\end{equation*}

\ourpar{$\plonk$ prover
  $\prover(\srs, \inp, \wit = (\wit_i)_{i \in \range{1}{3 \cdot
      \noofc}})$.}
\begin{description}
\item[Round 1] Sample $b_1, \ldots, b_9 \sample \FF_p$; compute
  $\p{a}(X), \p{b}(X), \p{c}(X)$ as
	\begin{align*}
		\p{a}(X) &= (b_1 X + b_2)\p{Z_H}(X) + \sum_{i = 1}^{\noofc} \wit_i \lag_i(X) \\
		\p{b}(X) &= (b_3 X + b_4)\p{Z_H}(X) + \sum_{i = 1}^{\noofc} \wit_{\noofc + i} \lag_i(X) \\
		\p{c}(X) &= (b_5 X + b_6)\p{Z_H}(X) + \sum_{i = 1}^{\noofc} \wit_{2 \cdot \noofc + i} \lag_i(X) 
	\end{align*}
	Output polynomial commitments $\gone{\p{a}(\chi), \p{b}(\chi), \p{c}(\chi)}$.
	
	\item[Round 2]
	Get challenges $\beta, \gamma \in \FF_p$
	\[
		\beta = \ro(\tzkproof[0..1], 0)\,, \qquad \gamma = \ro(\tzkproof[0..1], 1)\,.
	\]
	Compute permutation polynomial $\p{z}(X)$
	\begin{multline*}
		\p{z}(X) = (b_7 X^2 + b_8 X + b_9)\p{Z_H}(X) + \lag_1(X) + \\
			+ \sum_{i = 1}^{\noofc - 1} 
			\left(\lag_{i + 1} (X) \prod_{j = 1}^{i} 
			\frac{
			(\wit_j +\beta \omega^{j - 1} + \gamma)(\wit_{\noofc + j} + \beta k_1 \omega^{j - 1} + \gamma)(\wit_{2 \noofc + j} +\beta k_2 \omega^{j- 1} + \gamma)}
			{(\wit_j+\sigma(j) \beta + \gamma)(\wit_{\noofc + j} + \sigma(\noofc + j)\beta + \gamma)(\wit_{2 \noofc + j} + \sigma(2 \noofc + j)\beta + \gamma)}\right)
	\end{multline*}
	Output polynomial commitment $\gone{\p{z}(\chi)}$
		
	\item[Round 3]
	Get the challenge $\alpha = \ro(\tzkproof[0..2])$, compute the quotient polynomial 
	\begin{align*}
	& \p{t}(X)  = \\
	& (\p{a}(X) \p{b}(X) \selmulti(X) + \p{a}(X) \selleft(X) + 
	\p{b}(X)\selright(X) + \p{c}(X)\seloutput(X) + \pubinppoly(X) + \selconst(X)) 
	\frac{1}{\p{Z_H}(X)} +\\
	& + ((\p{a}(X) + \beta X + \gamma) (\p{b}(X) + \beta k_1 X + \gamma)(\p{c}(X) 
	+ \beta k_2 X + \gamma)\p{z}(X)) \frac{\alpha}{\p{Z_H}(X)} \\
	& - (\p{a}(X) + \beta \p{S_{\sigma 1}}(X) + \gamma)(\p{b}(X) + \beta 
	\p{S_{\sigma 2}}(X) + \gamma)(\p{c}(X) + \beta \p{S_{\sigma 3}}(X) + 
	\gamma)\p{z}(X \omega))  \frac{\alpha}{\p{Z_H}(X)} \\
	& + (\p{z}(X) - 1) \lag_1(X) \frac{\alpha^2}{\p{Z_H}(X)}
	\end{align*}
	Split $\p{t}(X)$ into degree less then $\noofc$ polynomials $\p{t_{lo}}(X), \p{t_{mid}}(X), \p{t_{hi}}(X)$, such that
	\[
		\p{t}(X) = \p{t_{lo}}(X) + X^{\noofc} \p{t_{mid}}(X) + X^{2 \noofc} \p{t_{hi}}(X)\,.
	\]
	Output $\gone{\p{t_{lo}}(\chi), \p{t_{mid}}(\chi), \p{t_{hi}}(\chi)}$.
	
	\item[Round 4]
	Get the challenge $\chz \in \FF_p$, $\chz = \ro(\tzkproof[0..3])$.
	Compute opening evaluations
	\begin{align*}
      \p{a}(\chz), \p{b}(\chz), \p{c}(\chz), \p{S_{\sigma 1}}(\chz), \p{S_{\sigma 2}}(\chz), \p{t}(\chz), \p{z}(\chz \omega),
	\end{align*}
	Compute the linearisation polynomial
	\[
		\p{r}(X) = 
		\begin{aligned}
      & \p{a}(\chz) \p{b}(\chz) \selmulti(X) + \p{a}(\chz) \selleft(X) + \p{b}(\chz) \selright(X) + \p{c}(\chz) \seloutput(X) + \selconst(X) \\
      & + \alpha \cdot \left( (\p{a}(\chz) + \beta \chz + \gamma) (\p{b}(\chz) + \beta k_1 \chz + \gamma)(\p{c}(\chz) + \beta k_2 \chz + \gamma) \cdot \p{z}(X)\right) \\
      & - \alpha \cdot \left( (\p{a}(\chz) + \beta \p{S_{\sigma 1}}(\chz) + \gamma) (\p{b}(\chz) + \beta \p{S_{\sigma 2}}(\chz) + \gamma)\beta \p{z}(\chz\omega) \cdot \p{S_{\sigma 3}}(X)\right) \\
      & + \alpha^2 \cdot \lag_1(\chz) \cdot \p{z}(X)
		\end{aligned}
	\]
	Output $\p{a}(\chz), \p{b}(\chz), \p{c}(\chz), \p{S_{\sigma 1}}(\chz), \p{S_{\sigma 2}}(\chz), \p{t}(\chz), \p{z}(\chz \omega), \p{r}(\chz).$
	
	\item[Round 5]
	Compute the opening challenge $v \in \FF_p$, $v = \ro(\tzkproof[0..4])$.
	Compute the openings for the polynomial commitment scheme 
	\begin{align*}
	& \p{W_\chz}(X) = \frac{1}{X - \chz} \left(
	\begin{aligned}
		& \p{t_{lo}}(X) + \chz^\noofc \p{t_{mid}}(X) + \chz^{2 \noofc} \p{t_{hi}}(X) - \p{t}(\chz)\\
		& + v(\p{r}(X) - \p{r}(\chz)) \\
		& + v^2 (\p{a}(X) - \p{a}(\chz))\\
		& + v^3 (\p{b}(X) - \p{b}(\chz))\\
		& + v^4 (\p{c}(X) - \p{c}(\chz))\\
		& + v^5 (\p{S_{\sigma 1}}(X) - \p{S_{\sigma 1}}(\chz))\\
		& + v^6 (\p{S_{\sigma 2}}(X) - \p{S_{\sigma 2}}(\chz))
	\end{aligned}
	\right)\\
	& \p{W_{\chz \omega}}(X) = \frac{\p{z}(X) - \p{z}(\chz \omega)}{X - \chz \omega}
\end{align*}
	Output $\gone{\p{W_{\chz}}(\chi), \p{W_{\chz \omega}}(\chi)}$.
\end{description}

\ncase{$\plonk$ verifier $\verifier(\srs, \inp, \zkproof)$}\ \newline
The \plonk{} verifier works as follows
\begin{description}
	\item[Step 1] Validate all obtained group elements.
	\item[Step 2] Validate all obtained field elements.
	\item[Step 3] Validate the instance
      $\inp = \smallset{\wit_i}_{i = 1}^\instsize$.
	\item[Step 4] Compute challenges $\beta, \gamma, \alpha, \chz, v,
      u$ from the transcript.
	\item[Step 5] Compute zero polynomial evaluation
      $\p{Z_H} (\chz) =\chz^\noofc - 1$.
	\item[Step 6] Compute Lagrange polynomial evaluation
      $\lag_1 (\chz) = \frac{\chz^\noofc -1}{\noofc (\chz - 1)}$.
	\item[Step 7] Compute public input polynomial evaluation
      $\pubinppoly (\chz) = \sum_{i \in \range{1}{\instsize}} \wit_i
      \lag_i(\chz)$.
	\item[Step 8] Compute quotient polynomials evaluations
	\begin{multline*}
    \p{t} (\chz) = \frac{1}{\p{Z_H}(\chz)} \Big(
    \p{r} (\chz) + \pubinppoly(\chz) - (\p{a}(\chz) + \beta \p{S_{\sigma 1}}(\chz) + \gamma) (\p{b}(\chz) + \beta \p{S_{\sigma 2}}(\chz) + \gamma) \\
    (\p{c}(\chz) + \gamma)\p{z}(\chz \omega) \alpha - \lag_1 (\chz) \alpha^2
    \Big) \,.
	\end{multline*}
	\item[Step 9] Compute batched polynomial commitment
	$\gone{D} = v \gone{r} + u \gone {z}$ that is
	\begin{align*}
		\gone{D} & = v
		\left(
		\begin{aligned}
          & \p{a}(\chz)\p{b}(\chz) \cdot \gone{\selmulti} + \p{a}(\chz)  \gone{\selleft} + \p{b}  \gone{\selright} + \p{c}  \gone{\seloutput} + \\
          & + (	(\p{a}(\chz) + \beta \chz + \gamma) (\p{b}(\chz) + \beta k_1 \chz + \gamma) (\p{c} + \beta k_2 \chz + \gamma) \alpha  + \lag_1(\chz) \alpha^2)  + \\
			% &   \\
          & - (\p{a}(\chz) + \beta \p{S_{\sigma 1}}(\chz) + \gamma) (\p{b}(\chz)
          + \beta \p{S_{\sigma 2}}(\chz) + \gamma) \alpha \beta \p{z}(\chz
          \omega) \gone{\p{S_{\sigma 3}}(\chi)})
		\end{aligned}
		\right) + \\
		& + u \gone{\p{z}(\chi)}\,.
	\end{align*}
	\item[Step 10] Computes full batched polynomial commitment $\gone{F}$:
	\begin{align*}
      \gone{F} & = \left(\gone{\p{t_{lo}}(\chi)} + \chz^\noofc \gone{\p{t_{mid}}(\chi)} + \chz^{2 \noofc} \gone{\p{t_{hi}}(\chi)}\right) + u \gone{\p{z}(\chi)} + \\
               & + v
                 \left(
		\begin{aligned}
			& \p{a}(\chz)\p{b}(\chz) \cdot \gone{\selmulti} + \p{a}(\chz)  \gone{\selleft} + \p{b}(\chz)   \gone{\selright} + \p{c}(\chz)  \gone{\seloutput} + \\
			& + (	(\p{a}(\chz) + \beta \chz + \gamma) (\p{b}(\chz) + \beta k_1 \chz + \gamma) (\p{c}(\chz)  + \beta k_2 \chz + \gamma) \alpha  + \lag_1(\chz) \alpha^2)  + \\
			% &   \\
			& - (\p{a}(\chz) + \beta \p{S_{\sigma 1}}(\chz) + \gamma) (\p{b}(\chz) + \beta \p{S_{\sigma 2}}(\chz) + \gamma) \alpha  \beta \p{z}(\chz \omega) \gone{\p{S_{\sigma 3}}(\chi)})
		\end{aligned}
		\right) \\
		& + v^2 \gone{\p{a}(\chi)} + v^3 \gone{\p{b}(\chi)} + v^4 \gone{\p{c}(\chi)} + v^5 \gone{\p{S_{\sigma 1}(\chi)}} + v^6 \gone{\p{S_{\sigma 2}}(\chi)}\,.
	\end{align*}
	\item[Step 11] Compute group-encoded batch evaluation $\gone{E}$
	\begin{align*}
		\gone{E}  = \frac{1}{\p{Z_H}(\chz)} & \gone{
		\begin{aligned}
			& \p{r}(\chz) + \pubinppoly(\chz) +  \alpha^2  \lag_1 (\chz) + \\
			& - \alpha \left( (\p{a}(\chz) + \beta \p{S_{\sigma 1}} (\chz) + \gamma) (\p{b}(\chz) + \beta \p{S_{\sigma 2}} (\chz) + \gamma) (\p{c}(\chz) + \gamma) \p{z}(\chz \omega) \right)
		\end{aligned}
           }\\
      + & \gone{v \p{r}(\chz) + v^2 \p{a}(\chz) + v^3 \p{b}(\chz) + v^4 \p{c}(\chz) + v^5 \p{S_{\sigma 1}}(\chz) + v^6 \p{S_{\sigma 2}}(\chz) + u \p{z}(\chz \omega) }\,.
	\end{align*}
\item[Step 12] Check whether the verification
 % $\vereq_\zkproof(\chi)$
  equation holds
	\begin{multline}
		\label{eq:ver_eq}
		\left( \gone{\p{W_{\chz}}(\chi)} + u \cdot \gone{\p{W_{\chz
                \omega}}(\chi)} \right) \bullet
		\gtwo{\chi} - %\\
		\left( \chz \cdot \gone{\p{W_{\chz}}(\chi)} + u \chz \omega \cdot
          \gone{\p{W_{\chz \omega}}(\chi)} + \gone{F} - \gone{E} \right) \bullet
        \gtwo{1} = 0\,.
	\end{multline}
  The verification equation is a batched version of the verification equation
  from \cite{AC:KatZavGol10} which allows the verifier to check openings of
  multiple polynomials in two points (instead of checking an opening of a single
  polynomial at one point).
\end{description}

\ncase{$\plonk$ simulator $\simulator_\chi(\srs, \td= \chi, \inp)$}\ \newline
The \plonk{} simulator proceeds as an honest prover would, except:
\begin{enumerate}
  \item In the first round, it sets $\wit = (\wit_i)_{i \in \range{1}{3 \noofc}}
    = \vec{0}$, and at random picks $b_1, \ldots, b_9$. Then it proceeds with
    that all-zero witness.
  \item In Round 3, it computes polynomial $\pt(X)$ honestly, however uses
    trapdoor $\chi$ to compute commitments
    $\p{t_{lo}}(\chi), \p{t_{mid}}(\chi), \p{t_{hi}}(\chi)$.
  \end{enumerate}

%  \subsection{Trapdoor-less simulatability of Plonk}
%\label{sec:plonk-TLZK-proof}


%%% Local Variables:
%%% mode: latex
%%% TeX-master: "main"
%%% End:

% !TEX root = main.tex
% !TEX spellcheck = en-US

\section{Non-malleability of $\sonicprotfs$}
\label{sec:sonic}
\subsection{\sonic{} protocol rolled out}
In this section we present $\sonic$'s constraint system and algorithms. Reader
familiar with them may jump directly to the next section.

%\hamid{}{we should put the following figure \ref{fig:pcoms} somewhere in this section.}
 \begin{figure}[h!]
 \centering
 	\begin{pcvstack}[center,boxed]
 		\begin{pchstack}
 			\procedure{$\kgen(\secparam, \maxdeg)$} {
 				\alpha, \chi \sample \FF^2_p \\ [\myskip]
 				\pcreturn \gone{\smallset{\chi^i}_{i = -\multconstr}^{\multconstr},
           \smallset{\alpha \chi^i}_{i = -\multconstr, i \neq
             0}^{\multconstr}},\\
         \pcind \gtwo{\smallset{\chi^i, \alpha \chi^i}_{i =
             -\multconstr}^{\multconstr}}, \gtar{\alpha}\\
 				%\markulf{03.11.2020}{} \\
 			%	\hphantom{\pcind \p{o}_i(X) \gets \sum_{j = 1}^{t_i} \gamma_i^{j - 1} \frac{\p{f}_{i,j}(X) - \p{f}_{i, j}(z_i)}{X - z_i}}
 				\hphantom{\hspace*{5.5cm}}
 		}

 			\pchspace

 			\procedure{$\com(\srs, \maxconst, \p{f}(X))$} {
 				\p{c}(X) \gets \alpha \cdot X^{\dconst - \maxconst} \p{f}(X) \\ [\myskip]
 				\pcreturn \gone{c} = \gone{\p{c}(\chi)}\\ [\myskip]
 				\hphantom{\pcind \pcif \sum_{i = 1}^{\abs{\vec{z}}} r_i \cdot
           \gone{\sum_{j = 1}^{t_j} \gamma_i^{j - 1} c_{i, j} - \sum_{j = 1}^{t_j}
             s_{i, j}} \bullet \gtwo{1} + } }
 		\end{pchstack}
 		% \pcvspace

 		\begin{pchstack}
 			\procedure{$\open(\srs, z, s, f(X))$}
 			{
 				\p{o}(X) \gets \frac{\p{f}(X) - \p{f}(z)}{X - z}\\ [\myskip]
 				\pcreturn \gone{\p{o}(\chi)}\\ [\myskip]
 				\hphantom{\hspace*{5.5cm}}
 			}

 			\pchspace

 			\procedure{$\verify(\srs, \maxconst, \gone{c}, z, s, \gone{\p{o}(\chi)})$}
       {
         \pcif \gone{\p{o}(\chi)} \bullet \gtwo{\alpha \chi} + \gone{s - z
         \p{o}(\chi)} \bullet \gtwo{\alpha} = \\ [\myskip] \pcind \gone{c}
         \bullet \gtwo{\chi^{- \dconst + \maxconst}} \pcthen  \pcreturn 1\\
         [\myskip]
         \rlap{\pcelse \pcreturn 0.} \hphantom{\pcind \pcif \sum_{i =
             1}^{\abs{\vec{z}}} r_i \cdot \gone{\sum_{j = 1}^{t_j} \gamma_i^{j -
               1} c_{i, j} - \sum{j = 1}^{t_j} s_{i, j}} \bullet \gtwo{1} + } }
 		\end{pchstack}
 	\end{pcvstack}

 	\caption{$\PCOMs$ polynomial commitment scheme.}
 	\label{fig:pcoms}
 \end{figure}



\oursubsub{The constraint system}
\label{sec:sonic_constraint_system}
\cref{fig:pcoms} presents a variant of KZG~\cite{AC:KatZavGol10} polynomial commitment schemes
used in \sonic{}. \sonic's system of constraints composes of three $\multconstr$-long vectors
$\va, \vb, \vc$ which corresponds to left and right inputs to multiplication
gates and their outputs. It hence holds $\va \cdot \vb = \vc$.

There is also $\linconstr$ linear constraints of the form
\[
  \va \vec{u_q} + \vb \vec{v_q} + \vc \vec{w_q} = k_q,
\]
where $\vec{u_q}, \vec{v_q}, \vec{w_q}$ are vectors for the $q$-th linear
constraint with instance value $k_q \in \FF_p$. Furthermore define polynomials
\begin{equation}
  \begin{split}
    \p{u_i}(Y) & = \sum_{q = 1}^\linconstr Y^{q + \multconstr} u_{q, i}\,,\\
    \p{v_i}(Y) & = \sum_{q = 1}^\linconstr Y^{q + \multconstr} v_{q, i}\,,\\
  \end{split}
  \qquad
  \begin{split}
    \p{w_i}(Y) & = -Y^i - Y^{-i} + \sum_{q = 1}^\linconstr Y^{q +
      \multconstr} w_{q, i}\,,\\
    \p{k}(Y) & = \sum_{q = 1}^\linconstr Y^{q + \multconstr} k_{q}.
  \end{split}
\end{equation}

$\sonic$ constraint system requires that
\begin{align}
  \label{eq:sonic_constraint}
  \vec{a}^\top \cdot \vec{\p{u}} (Y) + \vec{b}^\top \cdot \vec{\p{v}} (Y) +
  \vec{c}^\top \cdot \vec{\p{w}} (Y) + \sum_{i = 1}^{\multconstr} a_i b_i (Y^i +
  Y^{-i}) - \p{k} (Y) = 0.
\end{align}

In \sonic{} we will use commitments to the following polynomials.
\begin{align*}
  \pr(X, Y) & = \sum_{i = 1}^{\multconstr} \left(a_i X^i Y^i + b_i X^{-i} Y^{-i}
              + c_i X^{-i - \multconstr} Y^{-i - \multconstr}\right) \\
  \p{s}(X, Y) & = \sum_{i = 1}^{\multconstr} \left( u_i (Y) X^{-i} +
                v_i(Y) X^i + w_i(Y) X^{i + \multconstr}\right)\\
  \pt(X, Y) & = \pr(X, 1) (\pr(X, Y) + \p{s}(X, Y)) - \p{k}(Y)\,.
\end{align*}

Polynomials $\p{r} (X, Y), \p{s} (X, Y), \p{t} (X, Y)$ are designed such that
$\p{t} (0, Y) = \vec{a}^\top \cdot \vec{\p{u}} (Y) + \vec{b}^\top \cdot
\vec{\p{v}} (Y) + \vec{c}^\top \cdot \vec{\p{w}} (Y) + \sum_{i =
  1}^{\multconstr} a_i b_i (Y^i + Y^{-i}) - \p{k} (Y) $. That is, the prover is
asked to show that $\p{t} (0, Y) = 0$, cf.~\cref{eq:sonic_constraint}.

Furthermore, the commitment system in $\sonic$ is designed such that it is
infeasible for a $\ppt$ algorithm to commit to a polynomial with non-zero
constant term.

\oursubsub{Algorithms rolled out}
\ourpar{$\sonic$ SRS generation $\kgen(\REL)$.} The SRS generating algorithm picks
randomly $\alpha, \chi \sample \FF_p$ and outputs
	\[
      \srs = \left( \gone{\smallset{\chi^i}_{i = -\dconst}^{\dconst},
          \smallset{\alpha \chi^i}_{i = -\dconst, i \neq 0}^{\dconst}},
        \gtwo{\smallset{\chi^i, \alpha \chi^i}_{i = - \dconst}^{\dconst}},
        \gtar{\alpha} \right)
	\]
\ourpar{$\sonic$ prover $\prover(\srs, \inp, \wit=\va, \vb, \vc)$.}
\begin{description}
\item[Message 1] The prover picks randomly randomisers
  $c_{\multconstr + 1}, c_{\multconstr + 2}, c_{\multconstr + 3}, c_{\multconstr
    + 4} \sample \FF_p$. Sets
  $\pr(X, Y) \gets \pr(X, Y) + \sum_{i = 1}^4 c_{\multconstr + i} X^{- 2
    \multconstr - i}$. Commits to $\pr(X, 1)$ and outputs
  $\gone{r} \gets \com(\srs, \multconstr, \pr(X, 1))$.  Then it computes challenge $y = \ro(\zkproof[0..1])$.
\item[Message 2] $\prover$ commits to $\pt(X, y)$ and outputs
  $\gone{t} \gets \com(\srs, \dconst, \pt(X, y))$. Then it gets a challenge $z = \ro(\zkproof[0..2])$.
\item[Message 3] The prover computes commitment openings. That is, it outputs
  \begin{align*}
    \gone{o_a} & = \open(\srs, z, \pr(z, 1), \pr(X, 1)) \\
    \gone{o_b} & = \open(\srs, yz, \pr(yz, 1), \pr(X, 1)) \\
    \gone{o_t} & = \open(\srs, z, \pt(z, y), \pt(X, y)) 
  \end{align*}
  along with evaluations $a' = \pr(z, 1), b' = \pr(y, z), t' = \pt(z, y)$.  Then it
  engages in the signature of correct computation playing the role of the
  helper, i.e.~it commits to $\p{s}(X, y)$ and sends the commitment $\gone{s}$, commitment opening
  \begin{align*}
    \gone{o_s} & = \open(\srs, z, \p{s}(z, y), \p{s}(X, y)), \\
  \end{align*} and $s'=\p{s}(z, y)$. 
%
  Then
  it obtains a challenge $u = \ro(\zkproof[0..3])$.
\item[Message 4] For the next message the prover computes
  $\gone{c} \gets \com(\srs, \dconst, \p{s}(u, Y))$ and
  computes commitments' openings
  \begin{align*}
    \gone{w} & = \open(\srs, u, \p{s}(u, y), \p{s}(X, y)), \\
    \gone{q_y} & = \open(\srs, y,\p{s}(u, y), \p{s}(u, Y)),
  \end{align*}
  and returns $\gone{w}, \gone{q_y}, s = \p{s}(u, y)$. Eventually the prover gets the last challenge
  $z' = \ro(\zkproof[0..4])$.
\item[Message 5] For the final message, $\prover$ computes opening
  $\gone{q_{z'}} = \open(\srs, z', \p{s}(u, z'), \p{s}(u, X))$ and outputs $\gone{q_{z'}}$.
\end{description}

\ourpar{$\sonic$ verifier $\verifier(\srs, \inp, \zkproof)$.} The verifier
in \sonic{} runs as subroutines the verifier for the polynomial commitment. That
is it sets $t' = a'(b' + s') - \p{k}(y)$ and checks the following:
\begin{equation*}
  \begin{split}
    &\PCOMs.\verifier(\srs, \multconstr, \gone{r}, z, a', \gone{o_a}), \\
    &\PCOMs.\verifier(\srs, \multconstr, \gone{r}, yz, b', \gone{o_b}),\\
    &\PCOMs.\verifier(\srs, \dconst, \gone{t}, z, t', \gone{o_t}),\\
    &\PCOMs.\verifier(\srs, \dconst, \gone{s}, z, s', \gone{o_s}),\\
  \end{split}
  \qquad
  \begin{split}
    &\PCOMs.\verifier(\srs, \dconst, \gone{s}, u, s, \gone{w}),\\
    &\PCOMs.\verifier(\srs, \dconst, \gone{c}, y, s, \gone{q_y}),\\
    &\PCOMs.\verifier(\srs, \dconst, \gone{c}, z', \p{s}(u, z'), \gone{q_{z'}}),
  \end{split}
\end{equation*}
and accepts the proof iff all the checks holds. Note that the value
$\p{s}(u, z')$ that is recomputed by the verifier uses separate challenges $u$
and $z'$. This enables the batching of many proof and outsourcing of this
part of the proof to an untrusted helper.

\subsection{Unique opening property of $\PCOMs$}
\begin{lemma}
\label{lem:pcoms_unique_op}
$\PCOMs$ has the unique opening property in the AGM. 
\end{lemma}
\begin{proof}
Let 
$z \in \FF_p$ be the attribute the polynomial is evaluated at,
$\gone{c} \in \GRP$ be the commitment,  
$s \in \FF_p$ the evaluation value, and 
$o \in \GRP$ be the commitment opening. 
We need to show that for every $\ppt$ adversary $\adv$ probability
\[
  \Pr \left[
    \begin{aligned}
      & \verify(\srs, \gone{c}, z, s, \gone{o}) = 1, \\
      & \verify(\srs, \gone{c}, z, \tilde{s}, \gone{\tilde{o}}) = 1
    \end{aligned}
    \,\left|\, \vphantom{\begin{aligned}
          & \verify(\srs, \gone{c}, z, s, \gone{o}),\\
          & \verify(\srs, \gone{c}, z, s, \gone{\tilde{o}}) \\
          &o \neq \tilde{o})
		\end{aligned}}
      \begin{aligned}
        %& \srs \gets \kgen(\secparam, \maxdeg), \\
        & (\gone{c}, z, s, \gone{o}, \gone{\tilde{o}}) \gets \adv^{\initU}(1^\secpar, \maxdeg)
      \end{aligned}
    \right.\right]
  % \leq \negl.
\]
is at most negligible.

As noted in \cite[Lemma 2.2]{EPRINT:GabWilCio19} it is enough to upper bound the
probability of the adversary succeeding using the idealised verification
equation---which considers equality between polynomials---instead of the real
verification equation---which considers equality of the polynomials' evaluations.

For a polynomial $f$, its degree upper bound $\maxconst$, evaluation point $z$,
evaluation result $s$, and opening $\gone{o(X)}$ the idealised check verifies that
\begin{equation}
  \alpha (X^{\dconst - \maxconst}f(X) \cdot X^{-\dconst + \maxconst} -  s) \equiv \alpha \cdot o(X) (X - z)\,,
\end{equation}
what is equivalent to 
\begin{equation}
	f(X) -  s \equiv o(X) (X - z)\,.
	\label{eq:pcoms_idealised_check}
\end{equation}
Since $o(X)(X - z) \in \FF_p[X]$ then from the uniqueness of polynomial
composition, there is only one $o(X)$ that fulfils the equation above.
\qed
\end{proof}


\subsection{Unique response property}
The unique response property of $\sonicprot$ follows from the unique opening
property of the used polynomial commitment scheme $\PCOMs$.
\begin{lemma}
\label{lem:sonicprot_ur}
If a polynomial commitment scheme $\PCOMs$ is evaluation binding with parameter
$\epsbind (\secpar)$ and has unique openings property with parameter $\epsop(\secpar)$,
$\sonicprotfs$ is $\epss (\secpar)$-sound and $(\dconst, \dconst)$-$\ldlog$ problem is
$\epsldlog (\secpar)$-hard, then $\sonicprot$ is $\ur{1}$ against algebraic adversaries with security loss
  \[
    3 \cdot \epsop (\secpar) + 2 \cdot (\epsbind (\secpar) + \epss (\secpar) +
    \epsdlog (\secpar)).
  \]
\end{lemma}

\changedm{
\begin{proof}
  Let $\adv$ be an algebraic adversary tasked to break the $\ur{1}$-ness of
  $\sonicprotfs$. We show that the first prover's message determines, along with
  the verifiers challenges, the rest of it.  This is done by game hops. In the games,
  the adversary outputs two proofs $\zkproof^0$ and $\zkproof^1$ for the same statement.
  To distinguish polynomials and commitments which an honest prover sends in the
  proof from the polynomials and commitments computed by the adversary we write the
  latter using indices $0$ and $1$ (two indices as we have two transcripts), e.g.~to
  describe the quotient polynomial provided by the adversary we write $\p{t}^0$ and
  $\p{t}^1$ instead of $\p{t}$ as in the description of the protocol.

  \ngame{0} In this game, the adversary additionally wins if it provides two transcripts that
  match on all $5$ messages sent by the prover.
  In this game the adversary cannot win.

  \ngame{1} This game is identical to Game $\game{0}$ except that now the
  adversary additionally wins if it provides two transcripts that matches on the first four
  messages of the proof.

  \ncase{Game 0 to Game 1} We show that the probability that $\adv$
  wins in one game but does not in the other is negligible.  Observe that in
  after its $4$-th message, the adversary is given a challenge $z'$ and has to open
  commitment to $\p{s} (u, z')$. Hence, to be able to give two different
  openings in its $5$-th message, $\adv$ has to break the unique opening property of the
  KZG commitment scheme which happens with probability $\epsop (\secpar)$ tops.

  \ngame{2} This game is identical to Game $\game{1}$ except that now the
  adversary additionally wins if it provides two transcripts that matches on the
  first three messages of the proof.

  \ncase{Game 2 to Game 3} In its $4$-th message the adversary computes evaluation
  $s = \p{s} (u, y)$ and the corresponding openings $\gone{w}, \gone{q_y}$. The adversary
  cannot provide two different evaluations for the committed polynomials, since that would
  require breaking the evaluation binding property, which happens (by the union bound)
  with probability at most $2 \cdot \epsbind (\secpar)$. Also, the adversary cannot provide two
  different yet valid opening except probability $2 \cdot \epsop (\secpar)$

  Hence, the probability that adversary wins in one game but does not in the
  other is upper-bounded by $2 \cdot (\epsbind (\secpar) + \epsop (\secpar))$

  \ngame{4} This game is identical to Game $\game{3}$ except that now the
  adversary additionally wins if it provides two transcripts that matches on the
  first two messages of the proof.

  \ncase{Game 3 to Game 4} In its $3$-rd message the adversary computes $4$ polynomial
  evaluations and their openings. It also sends commitment to $\p{s} (X, y)$ which is a
  signature of correct computation.

  Probability that the adversary provides different evaluations or polynomial openings is
  upper bounded by $4 \cdot (\epsop (\secpar) + \epsbind (\secpar))$. Since the polynomial
  commitment scheme is deterministic, if commitment to $\p{s^0} (X, y)$ does not equal
  commitment to $\p{s^1} (X, y)$, then at least one of these values has been computed
  incorrectly. Probability that the adversary outputs an acceptable proof, where signature
  of correct computation has been computed incorrectly is upper-bounded by $\epss
  (\secpar) + \epsdlog (\secpar)$, cf.~\cref{lem:plonkprot_ur}.

  \ncase{Game 5}  This game is identical to Game $\game{4}$ except that now the
  adversary additionally wins if it provides two transcripts that matches on the
  first messages of the proof.

  \ncase{Game 4 to Game 5} In its second message the adversary commits to polynomial
  $\p{t} (X, y)$. Since the commitment scheme is deterministic, probability that adversary
  outputs acceptable proofs where commitment to $\p{t^0} (X, y)$ does not equal commitment
  to $\p{t^1} (X)$ is upper-bounded by $\epss (\secpar) + \epsdlog (\secpar)$, cf.~\cref{lem:plonkprot_ur}.

  \ncase{Conclusion} Taking all the games together, probability that $\adv$ wins
  in Game 5 is upper-bounded by
  \[
    3 \cdot \epsop (\secpar) + 2 \cdot (\epsbind (\secpar) + \epss (\secpar) +
    \epsdlog (\secpar)).
  \]
  \qed
\end{proof}
}

\COMMENT{
  \michals{1.11}{Old proof below}
\begin{proof}
  Let $\adv$ be an adversary that breaks $\ur{1}$-ness of $\sonicprot$.  We
  consider two cases, depending on which message $\adv$ is able to provide at
  least two different outputs such that the resulting transcripts are
  acceptable.  For the first case we show that $\adv$ can be used to break the
  evaluation binding property of $\PCOMs$, while for the second case we show
  that it can be used to break the unique opening property of $\PCOMs$.

  The proof goes similarly to the proof of \cref{lem:plonkprot_ur} thus we
  provide only draft of it here.  For $i$-th message ($i > 1$) the prover
  either commits to some well-defined polynomials (deterministically), evaluates
  these on randomly picked points, or shows that the evaluations were performed
  correctly.  Obviously, for a committed polynomial $\p{p}$ evaluated at point
  $x$ only one value $y = \p{p}(x)$ is correct. If the adversary was able to
  provide two different values $y$ and $\tilde{y}$ that would be accepted as an
  evaluation of $\p{p}$ at $x$ then the $\PCOMs$'s evaluation binding would be
  broken.  Alternatively, if $\adv$ was able to provide two openings $\p{W}$ and
  $\p{\tilde{W}}$ for $y = \p{p}(x)$ then the unique opening property would be
  broken.
%
Hence the probability that $\adv$ breaks $\ur{1}$-property of $\PCOMs$ is
upper-bounded by $\epsbind(\secpar) + \epsop(\secpar)$. 
\qed

\end{proof}
}

\subsection{rewinding-based knowledge soundness}
\begin{lemma}
	\label{lem:sonicprot_ss}
	$\sonicprot$ is $(2, \multconstr + \linconstr + 1)$-computational special sound with security loss $(\epst (\accProb, \secpar), \epss(\secpar))$ against
	algebraic adversaries, where
  \[
    \epst(\accProb, \secpar) \leq \frac{\accProb - (q + 1) \epsid (\secpar)}{1 - \epsid (\secpar)}\,,
  \]
  and
	\[
	  \epss(\secpar) \leq \epsid(\secpar) + \epsldlog(\secpar) \,.
	\]
	Here $\accProb$ is a probability that the adversary outputs an acceptable proof, $q$ is the upper bound for a number of random oracle queries the adversary makes, $\epsid(\secpar)$ is a soundness error of the idealized verifier, and $\epsldlog(\secpar)$ is security of $(\dconst, \dconst)$-$\ldlog$ assumption.
\end{lemma}
\begin{proof}
	Similarly as in the case of $\plonk$, the main idea of the proof is to show that an
  adversary who breaks rewinding-based knowledge soundness can be used to break a $\dlog$
  problem instance. The proof goes by game hops. Let $\tree$ be the tree produced by
  $\tdv$ by rewinding $\adv$. Note that since the tree branches after prover's $2$-nd
  message, the instance $\inp$, commitments
  $\gone{\p{r} (\chi, 1), \p{r} (\chi, y), \p{s} (\chi, y), \p{t} (\chi, y)}$, and
  challenge $y$ are the same. The tree branches after the $2$-nd message of the
  prover when the challenge $z$ is presented, thus tree $\tree$ is build using
  different values of $z$.
	%
	We consider the following games.
	
	\ncase{Game 0} In this game the adversary wins if all the transcripts it
	produced are acceptable by the ideal verifier,
	i.e.~$\vereq_{\inp, \zkproof}(X) = 0$, cf.~\cref{eq:ver_eq}, and none of
	commitments
	$\gone{\p{r} (\chi, 1), \p{r} (\chi, y), \p{s} (\chi, y), \p{t} (\chi, y)}$ use
	elements from a simulated proof, and the extractor fails to extract a valid
	witness out of the proof.
	
	\ncase{Probability that $\adv$ wins Game 0 is negligible} Probability of
	$\adv$ winning this game is $\epsid(\secpar)$ as the protocol $\sonicprot$,
	instantiated with the idealised verification equation, is perfectly
	knowledge sound except with negligible probability of the idealised verifier
	failure $\epsid(\secpar)$. Hence for a valid proof $\zkproof$ for a
	statement $\inp$ there exists a witness $\wit$, such that $\REL(\inp, \wit)$
	holds. Note that since the $\tdv$ produces $(\multconstr + \linconstr + 1)$
	acceptable transcripts for different challenges $z$. As noted in
	\cite{CCS:MBKM19} this assures that the correct witness is encoded in
	$\p{r} (X, Y)$. Hence $\extt$ can recreate polynomials' coefficients by
	interpolation and reveal the witness with probability $1$. Moreover, the
	probability that extraction fails in that case is upper-bounded by
	probability of an idealised verifier failing $\epsid(\secpar)$, which is
	negligible.
	
	\ncase{Game 1} In this game the adversary additionally wins if it produces a
	transcript in $\tree$ such that $\vereq_{\inp, \zkproof}(\chi) = 0$, but
	$\vereq_{\inp, \zkproof}(X) \neq 0$, and none of commitments
	$\gone{\p{r} (\chi, 1), \p{r} (\chi, y), \p{s} (\chi, y), \p{t} (\chi, y)}$
	use elements from a simulated proof.  The first condition means that the
	ideal verifier does not accept the proof, but the real verifier does.
	
	\ncase{Game 0 to Game 1} Assume the adversary wins in Game 1, but does not
	win in Game 0. We show that such adversary may be used to break an
	instance of a $\ldlog$ assumption. More precisely, let $\tdv$ be an
	algorithm that for relation $\REL$ and randomly picked
	$\srs \sample \kgen(\REL)$ produces a tree of acceptable transcripts such
	that the winning condition of the game holds. Let $\rdvdlog$ be a
	reduction that gets as input an
	$(\dconst, \dconst)$-$\ldlog$ instance
	$\gone{\chi^{-\dconst}, \ldots, \chi^{\dconst}}, \gtwo{\chi^{-\dconst},
		\ldots, \chi^{\dconst}}$ and is tasked to output $\chi$.
	
	The reduction $\rdvdlog$ proceeds as follows.
	\begin{enumerate}
  \item Pick a random $\alpha$ and compute
    $\gone{\alpha \chi^{- \dconst}, \ldots, \alpha \chi^{-1}, \alpha \chi,
      \ldots, \alpha \chi^{\dconst}}$,
    $\gtwo{\alpha \chi^{- \dconst}, \ldots, \alpha \chi^{-1}, \alpha \chi,
      \ldots, \alpha \chi^{\dconst}}$. Set a SRS $\srs$ to be the
    $(\dconst, \dconst)$-$\ldlog$ instance and its multiplication with $\alpha$ as
    computed above.
    % \hamid{Is this clear?}
  \item Build $\sonicprot$'s SRS in the updatable setting by answering $\adv$'s
    queries for SRS updates and setting the honest update of the SRS to be
    $\srs$. Let $\srs'$ be the finalised SRS.
		\item Let $(1, \tree)$ be the output returned by $\tdv$. Let $\inp$ be a
		relation proven in $\tree$.  Consider a transcript $\zkproof \in \tree$ such
		that $\vereq_{\inp, \zkproof}(X) \neq 0$, but
		$\vereq_{\inp, \zkproof}(\chi') = 0$. Since $\adv$ is algebraic, all group
		elements included in $\tree$ are extended by their representation as a
		combination of the input $\GRP_1$-elements. Hence, all coefficients of the
		verification equation polynomial $\vereq_{\inp, \zkproof}(X)$ are known.
		\item Find $\vereq_{\inp, \zkproof}(X)$ zero points and find $\chi'$ among
		them.
  \item Let $\chi_1, \ldots, \chi_\ell$ be the partial trapdoors of $\adv$'s SRS
    updates, extracted by the reduction from the update proofs given by $\adv$.
		\item Return  $\chi = \chi' (\chi_1 \chi_2 \ldots \chi_\ell)^{-1}$.
	\end{enumerate}
	Hence, the probability that the adversary wins Game 1 is upper-bounded by
	$\epsldlog(\secpar)$.
\end{proof}

\subsection{Trapdoor-less simulatability of Sonic}
\begin{lemma}
\label{lem:sonic_hvzk}
$\sonic$ is 2-programmable trapdoor-less simulatable.
\end{lemma}
\begin{proof}
  The simulator proceeds as follows.
  \begin{enumerate}
  \item Pick randomly vectors $\vec{a}$, $\vec{b}$ and set
    \begin{equation}
      \label{eq:ab_eq_c}
      \vec{c} = \vec{a} \cdot \vec{b}. 
    \end{equation}
  \item Pick randomisers $c_{\multconstr + 1}, \ldots, c_{\multconstr + 4}$,
    honestly compute polynomials $\p{r}(X, Y), \p{r'}(X, Y), \p{s}(X, Y)$ and
    pick randomly challenges $y$, $z$.
  \item Output commitment $\gone{r} \gets \com(\srs, \multconstr, \p{r} (X,
    1))$ and challenge $y$. 
  \item Compute
    \begin{align*}
      & a' = \p{r}(z, 1),\\
      & b' = \p{r}(z, y),\\
      & s' = \p{s}(z, y).
    \end{align*} 
  \item Pick polynomial $\p{t}(X, Y)$ such that
    \begin{align*}
      & \p{t} (X, y) = \p{r} (X, 1) (\p{r}(X, y) + \p{s} (X, y)) - \p{k} (Y)\\
      & \p{t} (0, y) = 0
    \end{align*}
  \item Output commitment $\gone{t} = \com (\srs, \dconst, \p{t} (X, y))$ and
    challenge $z$.
  \item Continue following the protocol.
  \end{enumerate}

  We note that the simulation is perfect. This comes since, except polynomial
  $\p{t} (X, Y)$ all polynomials are computed following the protocol. For
  polynomial $\p{t} (X, Y)$ we observe that in a case of both real and simulated
  proof the verifier only learns commitment $\gone{t} = \p{t} (\chi, y)$ and
  evaluation $t' = \p{t} (z, y)$. Since the simulator picks $\p{t} (X, Y)$ such
  that 
  \begin{align*}
      \p{t} (X, y) = \p{r} (X, 1) (\p{r}(X, y) + \p{s} (X, y)) - \p{k} (Y)
  \end{align*}
  Values of $\gone{t}$ are equal in both proofs.
  Furthermore, the simulator picks its polynomial such that $\p{t}(0, y) = 0$,
  hence it does not need the trapdoor to commit to it. (Note that the proof
  system's SRS does not allow to commit to polynomials which have non-zero
  constant term). \qed
\end{proof}
\begin{remark} 
  As noted in \cite{CCS:MBKM19}, $\sonic$ is statistically subversion-zero
  knowledge (Sub-ZK). As noted in \cite{AC:ABLZ17}, one way to achieve
  subversion zero knowledge is to utilise an extractor that extracts a SRS
  trapdoor from a SRS-generator. Unfortunately, a NIZK made subversion
  zero-knowledge by this approach cannot achieve perfect Sub-ZK as one has to
  count in the probability of extraction failure. However, with the simulation
  presented in \cref{lem:sonic_hvzk}, the trapdoor is not required for the
  simulator as it is able to simulate the execution of the protocol just by
  picking appropriate (honest) verifier's challenges. This result transfers to
  $\sonicprotfs$, where the simulator can program the random oracle to provide
  challenges that fits it.
\end{remark}

\subsection{From rewinding-based knowledge soundness and unique response property to \COMMENT{forking
  }simulation extractability of $\sonicprotfs$}
Since \cref{lem:sonicprot_ur,lem:sonicprot_ss,lem:sonic_hvzk} hold, $\sonicprot$ is $\ur{1}$, computational special sound and trapdoor-less simulatable. We now make use
of \cref{thm:se} and show that $\sonicprotfs$ is \COMMENT{ forking }simulation-extractable as defined in \cref{def:simext}.

\begin{corollary}[\COMMENT{Forking s}Simulation extractability of $\sonicprotfs$]
  \label{thm:sonicprotfs_se}
  Assume that $\sonicprot$ is $\ur{1}$ with security
  $\epsur(\secpar) = \epsbind(\secpar) + \epsop(\secpar)$ -- where
  $\epsbind (\secpar)$ is polynomial commitment's binding security, $\epsop$ is
  polynomial commitment unique opening security -- and computational special sound with
  security $\epss(\secpar)$. Let $\ro\colon \bin^* \to \bin^\secpar$ be a
  random oracle. Let $\advse$ be an adversary that can make up to $q$
  random oracle queries, and outputs an
  acceptable proof for $\sonicprotfs$ with probability at least $\accProb$. Then
  $\sonicprotfs$ is \COMMENT{forking }simulation-extractable with extraction error
  $\eta = \epsur(\secpar)$. The extraction probability $\extProb$ is at least
\[
		\extProb  \geq \frac{1}{q^{\multconstr + \linconstr}} (\accProb - \epsur(\secpar))^{\multconstr +
		\linconstr + 1} - \eps(\secpar).
	\]
	for some negligible $\eps(\secpar)$, $\multconstr$ and $\linconstr$ being,
  respectively, the number of multiplicative and linear constraints of the system.
\end{corollary}

%%% Local Variables:
%%% mode: latex
%%% TeX-master: "main"
%%% End:

% !TEX root = main.tex
% !TEX spellcheck = en-US

\section{Non-malleability of Marlin}
We show that $\marlin$ is \COMMENT{forking }simulation-extractable. To that end, we show
that $\marlin$ has all the required properties: has unique response property, is
rewinding-based knowledge sound, and its simulator can provide indistinguishable proofs
without a trapdoor, just by programming the random oracle.

\subsection{$\marlin$ Protocol Rolled-out}
$\marlin$ uses R1CS as arithmetization method. That is, the prover given
instance $\inp$ and witness $\wit$ and $|\HHH| \times |\HHH|$ matrices $\vec{A},
\vec{B}, \vec{C}$ shows that $\vec{A} (\inp^\top, \wit^\top)^\top \circ \vec{B}
(\inp^\top, \wit^\top)^\top = \vec{C} (\inp^\top, \wit^\top)^\top$. (Here
$\circ$ is a entry-wise product.)

We assume that the matrices have at most $|\KKK|$ non-zero entries. Obviously,
$|\KKK| \leq |\HHH|^2$. Let $b = 3$, the upper-bound of polynomial evaluations
the prover has to provide for each of the sent polynomials.  Denote by $\dconst$
an upper-bound for $\smallset{|\HHH| + 2b -1, 2 |\HHH| + b - 1, 6 |\KKK| - 6}$.

The idea of showing that the constraint system is fulfilled is as
follows. Denote by $\vec{z} = (\inp, \wit)$. The prover computes polynomials
$\p{z_A} (X), \p{z_B} (X), \p{z_C} (X)$ which encode vectors
$\vec{A} \vec{z}, \vec{B} \vec{z}, \vec{C} \vec{z}$ and have degree $<
|\HHH|$. Importantly, when constraints are fulfilled,
$ \p{z_A} (X) \p{z_B} (X) - \p{z_C} (X) = \p{h_0} (X) \ZERO_\HHH (X)$, for some
$\p{h_0} (X)$ and vanishing polynomial $\ZERO_\HHH (X)$. The prover sends
commitments to these polynomials and shows that they have been computed
correctly. More precisely, it shows that
\begin{equation}
  \label{eq:marlin_eq_2}
\forall \vec{M} \in \smallset{\vec{A}, \vec{B}, \vec{C}},  \forall \kappa \in \HHH,
\p{z_M} (\kappa) = \sum_{\iota \in \HHH} \vec{M}[\kappa, \iota] \p{z}(\iota).
\end{equation}

The ideal verifier checks the following equalities
\begin{equation}
  \label{eq:marlin_ver_eq}
  \begin{aligned}
    \p{h}_3 (\beta_3) \ZERO_\KKK (\beta_3) & = \p{a} (\beta_3) - \p{b} (\beta_3)
    (\beta_3 \p{g_3} (\beta_3) + \sigma_3 / |\KKK|)\\
    \p{r}(\alpha, \beta_2) \sigma_3 & = \p{h_2} (\beta_2) \ZERO_\HHH (\beta_2) +
    \beta_2 \p{g2} (\beta_2) + \sigma_2/|\HHH|\\
    \p{s}(\beta_1) + \p{r}(\alpha, \beta_1) (\sum_M \eta_M \p{z_M} (\beta_1)) -
    \sigma_2 \p{z} (\beta_1) & = \p{h_1} (\beta_1) \ZERO_\HHH (\beta_1) +
    \beta_1
    \p{g_1} (\beta_1) + \sigma_1/|\HHH| \\
    \p{z_A} (\beta_1) \p{z_B} (\beta_1) - \p{z_C} (\beta_1) & = \p{h_0}
    (\beta_1) \ZERO_\HHH (\beta_1)
  \end{aligned}
\end{equation}
where $\p{g_i} (X), \p{h_i} (X)$, $i \in \range{1}{3}$,
$\p{a} (X), \p{b} (X), \sigma_1, \sigma_2, \sigma_3$ are polynomials and
variables required by the sumcheck protocol which allows verifier to efficiently
verify that \cref{eq:marlin_eq_2} holds.
                         

\subsection{Unique Response Property}
\begin{lemma}\label{lem:marlinprot_ur}
  Let $\PCOM$ be a commitment of knowledge that is evaluation binding with security loss 
  $\epsbind(\secpar)$ and has unique opening property with security loss
  $\epsop(\secpar)$. Then
  $\marlinprotfs$ is $\ur{2}$ against algebraic adversaries with security loss $2 \cdot \epsbinding (\secpar) + \epsop (\secpar)$.
\end{lemma}

\begin{proof}
	The proof is similar to the proof of \cref{lem:plonkprot_ur} and \cref{lem:sonicprot_ur}.
	An adversary who can break the $2$-unique response property of $\marlinprotfs$ can be either used to break the commitment scheme's evaluation binding or unique opening property. The former happens with the probability upper-bounded by $2 \cdot \epsbinding (\secpar)$, the latter with probability at most $\epsop (\secpar)$.
	By the union bound, the adversary is able to break the unique response property with probability upper bounded by $2 \cdot \epsbinding (\secpar) + \epsop (\secpar)$.
	\qed
	\end{proof}
%  \changedm{$11 \cdot (\epsop (\secpar)) + 6 \cdot \epsbind (\secpar) +
%    \infrac{3}{\abs{\FF_p}}.$} \mxout{$6 \cdot (\epsbind + \epsop + \epsk) + \epss$}


\COMMENT{\changedm{
\begin{proof}
  Let $\adv$ be an algebraic adversary tasked to break the $\ur{2}$-ness of
  $\marlinprotfs$. We show that the first prover's message determines, along with
  the verifiers challenges, the rest of it.  This is done by game hops. In the games,
  the adversary outputs two proofs $\zkproof^0$ and $\zkproof^1$ for the same statement.
  To distinguish polynomials and commitments which an honest prover sends in the
  proof from the polynomials and commitments computed by the adversary we write the
  latter using indices $0$ and $1$ (two indices as we have two transcripts), e.g.~to
  describe a polynomial provided by the adversary we write $\p{f}^0$ and
  $\p{f}^1$ instead of $\p{f}$ as in the description of the protocol.

  \ngame{0} In this game, the adversary additionally wins if it provides two transcripts that
  match on all $5$ messages sent by the prover.
  Obviously, in this game the adversary cannot win.

  \ngame{1} This game is identical to Game $\game{0}$ except that now the
  adversary additionally wins if it provides two transcripts that matches on the first four
  messages of the proof.

  \ncase{Game 0 to Game 1} We show that the probability that $\adv$ wins in one game but
  does not in the other is negligible.  Observe that in its $4$-th message, the
  adversary is given a challenge $\beta_3$ and has to open the previously computed
  commitments for polynomials $\p{g_3} (X), \p{h_3} (X)$. Since the transcripts match up
  to $\adv$'s $4$-th message, the challenge is the same in both.  Also, the adversary
  evaluates and opens evaluations of polynomials $\p{g_2} (X), \p{h_2} (X)$ evaluated at
  $\beta_2$ and
  $\p{s} (X), \p{z} (X), \p{z_A} (X), \p{z_B} (X), \p{z_C} (X), \p{h_1} (X), \p{g_1} (X)$
  evaluated at $\beta_1$. Hence, to be able to give two different openings in its $5$-th
  message, $\adv$ has to break the unique opening property of the KZG commitment scheme
  which happens with probability $11 \cdot \epsop (\secpar)$ tops (as $11$ polynomials are
  evaluated).
  
  \ngame{2} This game is identical to Game $\game{1}$ except that now the
  adversary additionally wins if it provides two transcripts that matches on the
  first three messages of the proof.

  \ncase{Game 1 to Game 2} In its $4$-th message the adversary
  provides $\sigma_3$ and commits to polynomials $\p{g_3} (X), \p{h_3} (X)$ which are
  uniquely determined. Probability that the adversary commits to wrong polynomials that it
  can later evaluate to correct values (i.e.~to values of evaluated correct polynomials)
  is upper bounded by $2 \cdot \epsbind (\secpar) + 1 / \abs{\FF_p}$. This is since the
  adversary either has to break binding property in at least one of two commitments or
  guess the evaluation point.

  \ngame{3} This game is identical to Game $\game{2}$ except that now the
  adversary additionally wins if it provides two transcripts that matches on the
  first two messages of the proof.

  \ncase{Game 2 to Game 3} In its $3$-rd message the adversary
  provides $\sigma_2$ and commits to polynomials $\p{g_2} (X), \p{h_2} (X)$ which are
  uniquely determined. Probability that the adversary commits to wrong polynomials that it
  can later evaluate to correct values (i.e.~to values of evaluated correct polynomials)
  is upper bounded by $2 \cdot \epsbind (\secpar) + 1 / \abs{\FF_p}$ as explained in the
  previous point.

  \ngame{4}  This game is identical to Game $\game{3}$ except that now the
  adversary additionally wins if it provides two transcripts that matches on the
  first message of the proof.

  \ncase{Game 3 to Game 4} In its $2$-nd message the adversary
  provides $\sigma_1$ and commits to polynomials $\p{g_1} (X), \p{h_1} (X)$ which are
  uniquely determined. Probability that the adversary commits to wrong polynomials that it
  can later evaluate to correct values (i.e.~to values of evaluated correct polynomials)
  is upper bounded by $2 \cdot \epsbind (\secpar) + 1 / \abs{\FF_p}$ as explained before.
  
  \ncase{Conclusion} Taking all the games together, probability that $\adv$ wins
  in Game 4 is upper-bounded by
  \[
    11 \cdot (\epsop (\secpar)) + 6 \cdot \epsbind (\secpar) + \frac{3}{\abs{\FF_p}}.
  \]
  \qed
\end{proof}
}}

\COMMENT{
  \michals{2.11}{Old proof}
\begin{proof}
  As in previous proofs, we show the property by game hops. Let
  $N = \p{g_1}, \p{h_1}, \p{g_2}, \p{h_2}, \p{g_3}, \p{h_3}$. That is, $N$ is a
  set of all polynomials which commitments prover sends after it sends its first message.

  \ncase{Game 0} In this game the adversary wins if it breaks evaluation
  binding, unique opening property, or knowledge soundness of one of commitments
  for polynomials in $N$.

  Probability that a $\ppt$ adversary wins in Game 0, is upper bounded by $6
  \cdot (\epsbind + \epsop + \epsk)$.

  \ncase{Game 1} In this game the adversary additionally wins if it breaks the
  $\ur{1}$ property of the protocol

  \ncase{Game 0 to Game 1} Probability that the adversary wins in Game 1 but not in Game 0
  is $\epss$. This is since in the honest proof the polynomials in $N$ are uniquely
  determined. W.l.o.g.~we analyse probability that adversary is able to produce two
  (different) pairs of polynomials $(\p{h_2}, \p{g_2})$ and $(\p{h'_2}, \p{g'_2})$ such
  that
  \begin{align*}
    \p{h_2} (X) \ZERO_{\HHH} (X) + X \p{g_2} (X) & = \p{h_2} (X) \ZERO_{\HHH} (X) +
                                                   X \p{g_2} (X)\\
    (\p{h_2} (X) - \p{h'_2} (X)) \ZERO_{\HHH} (X) & = X (\p{g'_2} (X) - \p{g'_2}
    (X)).
  \end{align*}
  Since $\p{h_2}, \p{g_2} \in \FF^{< |\HHH| - 1} [X]$ and
  $\ZERO \in \FF^{|\HHH|} [X]$, LHS has different degree than RHS unless both
  sides have degree $0$. This happens when $\p{h_2} (X) = \p{h'_2} (X)$ and
  $\p{g_2} (X) - \p{g'_2} (X)$.
  Thus, for the adversary to be successful in this game it has to provide acceptable
  proofs where $(\p{h_2}, \p{g_2})$ and $(\p{h'_2}, \p{g'_2})$ differ. One of such pair
  has to be incorrect and will be accepted with probability at most $\epss$.
\end{proof}}

\subsection{Rewinding-Based Knowledge Soundness}
\begin{lemma}
	\label{lem:marlinprot_ss}
	$\marlinprotfs$ is $(2, \multconstr + 3)$-rewinding-based knowledge sound against algebraic adversaries who make up to $q$ random oracle queries with security loss 
	\[
	\epscss(\secpar,\accProb, q) \leq \left(1 - \frac{\accProb - (q + 1) \left(1 - \frac{\multconstr + 3}{p}\right)}{\frac{\multconstr + 3}{p}}\right) + (\multconstr + 3)\cdot\epsudlog (\secpar) + (\multconstr + 3) \cdot \epsid (\secpar)\,,
	\]
	Here $\accProb$ is a probability that the adversary outputs an acceptable proof, $\epsid(\secpar)$ is a soundness error of the ideal verifier for $\marlinprotfs$, and $\epsudlog(\secpar)$ is the security of $(\multconstr + 2, 1)$-$\udlog$ \hamid{2.5}{Not sure about $(\multconstr + 2, 1)$!}assumption.
\end{lemma}
\begin{proof}
The proof is similar to the proof of \cref{lem:plonkprot_ss} and \cref{lem:sonicprot_ss}. 
We use Attema et al.~\cite[Proposition 2]{EPRINT:AttFehKlo21} to bound the probability that the tree-building algorithm $\tdv$ does not obtain a tree of acceptable transcript in an expected number of runs. This happens with probability at most
\[
1 - \frac{\accProb - (q + 1) \left(1 - \frac{ \noofc + 3}{p} \right)}{\frac{\noofc + 3}{p}}
\]
Let $\tree$ be the tree outputted by $\tdv$. If one of the proofs in $\tree$ is not acceptable by the ideal verifier, one can break an instance of an updatable dlog assumption which happens with probability at most $(\noofc + 3)  \cdot \epsudlog (\secpar)$. In the case that all the transcripts are acceptable by the ideal verifier, but $\extss$ fails to extract a valid witness from $\tree$, one can break the soundness of the ideal verifier in one of the transcripts. That happens with probability at most $(\noofc + 3) \cdot \epsid (\secpar)$. Taking a union bound completes the proof.
\qed
\end{proof}


\COMMENT{\begin{lemma}\label{lem:marlinprot_ss}
	Assume that an idealised $\marlinprot$ verifier fails with probability at most
	$\epsid(\secpar)$ and probability that a $\ppt$ adversary breaks $\udlog$ is
	bounded by $\epsudlog(\secpar)$. Then $\marlinprotfs$ is
	$(2, d + 1)$-rewinding-based knowledge
	sound with security loss $\epsid (\secpar) + \epsudlog (\secpar)$.
\end{lemma}
\begin{proof}
	% \michals{8.9}{Need to check the degrees}
	The proof goes similarly to the respective proofs for $\plonk$ and
	$\sonic$. That is, let $\srs$ be $\marlinprot$'s finalized SRS and denote by $\srs_1$
	all SRS's $\GRP_1$-elements. Let $\tdv$ be an algebraic adversary that
	produces a statement $\inp$ and a $(1, \dconst + 1, 1, 1)$-tree of
	accepting transcripts $\tree$. Note that in all transcripts the instance
	$\inp$, proof elements
	$\sigma_1, \gone{\p{w}(\chi), \p{z_A}(\chi), \p{z_B}(\chi), \p{z_C}(\chi),
		\p{h_0}(\chi), \p{s}(\chi)}, \gone{\p{g_1}(\chi), \p{h_1}(\chi)}$
	and challenges $\alpha, \eta_1, \eta_2, \eta_3$ are common as the transcripts
	share the first $3$ messages. The tree branches after the third message of the
	protocol where the challenge $\beta_1$ is presented, thus tree $\tree$ is
	build using different values of $\beta_1$.
	
	We consider the following games.
	
	\ncase{Game 0} In this game the adversary wins if all the transcripts it
	produced are acceptable by the ideal verifier,
	i.e.~$\vereq_{\inp, \zkproof}(X) = 0$, cf.~\cref{eq:marlin_ver_eq}, yet the extractor
	fails to extract a valid witness out of them.
	
	Probability of $\tdv$ winning this game is $\epsid(\secpar)$ as the protocol
	$\marlinprot$, instantiated with the idealised verification equation, is
	perfectly sound except with negligible probability of the idealised verifier
	failure $\epsid(\secpar)$. Hence for a valid proof $\zkproof$ for a statement
	$\inp$ there exists a witness $\wit$, such that $\REL(\inp, \wit)$ holds. Note
	that since the $\tdv$ produces $(\dconst + 1)$ accepting transcripts for
	different challenges $\beta_1$, it obtains the same number of different
	evaluations of polynomials $\p{z_A}, \p{z_B}, \p{z_C}$.
	
	Since the transcripts are acceptable by an idealised verifier, the equality
	$\p{z_A} (X) \p{z_B} (X) - \p{z_C} (X) = \p{h_0} (X) \ZERO_\HHH (X)$ holds and
	each of $\p{z}_M$, $M \in \smallset{A, B, C}$, has been computed
	correctly. Hence, $\p{z_A}, \p{z_B}, \p{z_C}$ encodes the valid witness for
	the proven statement. Since $\p{z_A}, \p{z_B}, \p{z_C}$ are of degree at most
	$\dconst$ and there is more than $(\dconst + 1)$ their evaluations
	known, $\extt$ can recreate their coefficients by interpolation and reveal the
	witness with probability $1$. Hence, the probability that extraction fails in
	that case is upper-bounded by probability of an idealised verifier failing
	$\epsid(\secpar)$, which is negligible.
	
	\ncase{Game 1} In this game the adversary additionally wins if it produces a
	transcript in $\tree$ such that $\vereq_{\inp, \zkproof}(\chi) = 0$, but
	$\vereq_{\inp, \zkproof}(X) \neq 0$. That is, the ideal verifier does not
	accept the proof, but the real verifier does.
	
	\ncase{Game 0 to Game 1} Assume the adversary wins in Game 1, but
	does not win in Game 0. We show that such adversary may be used to break the
	$\udlog$ assumption. More precisely, let $\tdv$ be an adversary that for
	relation $\REL$ and randomly picked $\srs \sample \kgen(\REL)$ produces a tree
	of accepting transcripts such that the winning condition of the game
	holds. Let $\rdvdulog$ be a reduction that gets as input an
	$(\dconst, 1)$-$\udlog$ instance $\gone{1, \ldots, \chi^\dconst}, \gtwo{1, \chi}$ and
	is tasked to output $\chi$. The reduction proceeds as follows---it builds $\adv$'s SRS $\srs$ in the updatable setting using the input $\udlog$ instance. Namely it answers $\adv$'s queries for SRS updates and sets the honest update of the SRS to be $\srs$. Let $\srs'$ be the finalized SRS and $(1, \tree)$ be the output
	returned by $\adv$. Let $\inp$ be a relation proven in $\tree$.  Consider a
	transcript $\zkproof \in \tree$ such that $\vereq_{\inp, \zkproof}(X) \neq 0$,
	but $\vereq_{\inp, \zkproof}(\chi) = 0$. Since the adversary is algebraic, all
	group elements included in $\tree$ are extended by their representation as a
	combination of the input $\GRP_1$-elements. Hence all coefficients of the
	verification equation polynomial $\vereq_{\inp, \zkproof}(X)$ are known and
	$\rdvudlog$ can find its zero points. Since
	$\vereq_{\inp, \zkproof}(\chi) = 0$, the targeted discrete log value $\chi'$ is
	among them.  Let $\chi_1, \ldots, \chi_\ell$ be the partial trapdoors of $\adv$'s SRS updates,  extracted by the reduction from the update proofs given by $\adv$. Now $\rdvudlog$ returns $\chi = \chi' (\chi_1 \chi_2 \ldots \chi_\ell)^{-1}$ and breaks the $\udlog$ assumption. Hence, the probability that this event happens is upper-bounded
	by $\epsudlog(\secpar)$.
	
\end{proof}}

\subsection{Trapdoor-Less Zero-Knowledge of Marlin}
\begin{lemma}
  \label{lem:marlin_hvzk}
  $\marlinprotfs$ is 2-programmable trapdoor-less zero-knowledge.
\end{lemma}
\begin{proof}
The simulator follows the protocol except it picks the challenges $\alpha,
\eta_A, \eta_B, \eta_C, \beta_1, \beta_2, \beta_3$ before it picks polynomials
it sends.

First, it picks $\p{\tilde{z}}_A (X)$, $\p{\tilde{z}}_B (X)$ at random and
$\p{\tilde{z}}_C (X)$ such that
$\p{\tilde{z}}_A (\beta_1) \p{\tilde{z}}_B (\beta_1) = \p{\tilde{z}}_C
(\beta_1)$.  Given the challenges and polynomials $\p{\tilde{z}}_A (X)$,
$\p{\tilde{z}}_B (X)$, $\p{\tilde{z}}_C (X)$ the simulator computes
$\sigma_1 \gets \sum_{\kappa \in \HHH} \p{s}(\kappa) + \p{r}(\alpha, X) (\sum_{M
  \in \smallset{A, B, C}}\eta_M \p{\tilde{z}}_M(X)) - \sum_{M \in \smallset{A,
    B, C}} \eta_M \p{r}_M (\alpha, X) \p{\tilde{z}} (X)$.

Then the simulator starts the protocol and follows it, except it programs the
random oracle that on partial transcripts it returns the challenges picked by
$\simulator$.
\end{proof}

%\subsection{From Rewinding-Based Knowledge Soundness and Unique Response Property to\COMMENT{ forking}
%  Simulation Extractability of $\marlinprotfs$}
%\begin{corollary}
%  Assume that $\marlinprotfs$ is $\ur{2}$ with security
%  $\epsur(\secpar) = 6 \cdot (\epsbind + \epsop + \epsk)$, and rewinding-based knowledge sound
%  with security $\epss(\secpar)$. Let $\ro\colon \bin^* \to \bin^\secpar$ be a
%  random oracle. Let $\advse$ be an adversary that can make up to $q$
%  random oracle queries, and outputs an
%  acceptable proof for $\marlinprotfs$ with probability at least
%  $\accProb$. Then $\marlinprotfs$ is \COMMENT{forking }simulation-extractable with
%  extraction error $\eta = \epsur(\secpar)$. The extraction probability
%  $\extProb$ is at least
%  \[
%    \extProb \geq q^{-\dconst} (\accProb - 6 \cdot (\epsbind + \epsop +
%    \epsk))^{\dconst + 1} -\eps(\secpar)\,.
%\]
%	for some negligible $\eps(\secpar)$, $\dconst$ being, the upper bound of
%  constraints of the system.
%\end{corollary}

\subsection{Simulation Extractability of $\marlinprotfs$}
Since \cref{lem:marlinprot_ur,lem:marlinprot_ss,lem:marlin_hvzk} hold, $\marlinprotfs$ is $\ur{2}$, rewinding-based knowledge sound and trapdoor-less zero-knowledge. By making use
of \cref{thm:se}, we conclude that $\marlinprotfs$ is simulation-extractable as defined in \cref{def:simext}.

\begin{corollary}[Simulation extractability of $\marlinprotfs$]
	\label{thm:marlinprotfs_se}
	$\marlinprotfs$ is \emph{updatable simulation-extractable} against any $\ppt$ adversary $\advse$ who makes up to $q$ random oracle queries and returns an acceptable proof with probability at least $\accProb$ with extraction failure probability 
	\[
	\epsse(\secpar, \accProb, q) \leq \left(1 - \frac{\accProb - \epsur (\secpar) - (q + 1) \epserr (\secpar)} {1 - \epserr (\secpar)}\right) + (\multconstr  + 3) \cdot \epsudlog (\secpar) + (\multconstr + 3) \cdot \epsid (\secpar),
	\]
	where $\epserr (\secpar) = \left(1 - \frac{\multconstr + 3}{p} \right)$, $p$ is the size of the field, and $\noofc$ is the number of constrains in the circuit. 
\end{corollary}


%\section{Further work}
%We identify a number of problems which we left as further work. First of all,
%the generalised version of the forking lemma presented in this paper can be
%generalised even further to include protocols where forking soundness holds for
%protocols where $\extt$ extracts a witness from a $(n_1, \ldots, n_\mu)$-tree of
%acceptable transcripts, where more than one $n_j > 1$. I.e.~to include
%protocols that for witness extraction require transcripts that branch at more
%than one point.
%
%Although we picked $\plonk$ and $\sonic$ as examples for our framework, it is
%not limited to SRS-based NIZKs. Thus, it would be interesting to apply it to
%known so-called transparent zkSNARKs like Bulletproofs \cite{SP:BBBPWM18},
%Aurora \cite{EC:BCRSVW19} or AuroraLight \cite{EPRINT:Gabizon19a}.
%
%Since the rewinding technique and the forking lemma used to show simulation
%extractability of $\plonkprotfs$ and $\sonicprotfs$ come with security loss,
%it would be interesting to show SE of these protocols directly in the
%algebraic group model.
%
%Although we focused here only on zkSNARKs, it is worth to
%investigating other protocols that may benefit from our framework, like
%e.g.~identification schemes.
%
%Last, but not least, this paper would benefit greatly if a more tight version
%of the generalised forking lemma was provided. However, we have to note here
%that some of the inequalities used in the proof are already tight, i.e.~for
%specific adversaries, some of the inequalities are already equalities.

%%% Local Variables:
%%% mode: latex
%%% TeX-master: "main"
%%% End:

%% !TEX root = main.tex
% !TEX spellcheck = en-US

\section{Simulation Soundness---definitions and the general result}
\noindent \textbf{Simulation sound NIZKs in the updatable setting.}
Another notion for non-malleable NIZKs is \emph{simulation soundness}. It allows the adversary to see simulated proof, however, in contrast to simulation
extractability it does not require an extractor to provide a witness for the
proven statement. Instead, it is only necessary, that an adversary who sees
simulated proofs cannot make the verifier accept a proof of an incorrect
statement. More precisely,


\begin{definition}[Simulation soundness in the updatable setting]
	\label{def:simsnd}
	Let $\ps = (\kgen, \prover, \verifier, \simulator)$ be a NIZK proof system. We say that
  $\ps$ is \emph{updatable simulation-sound} if for any $\ppt$ adversary $\adv$ that is
  given oracle access to an updatable SRS setup $\initU$, cf.~\cref{fig:upd}, a simulation oracle $\simulator
  = (\simOH, \simOP')$, and a random oracle $\ro$, probability
	\[
	\ssndProb = \condprob{
		\begin{matrix}
		\verifier(\srs, \inp_{\advse}, \zkproof_{\advse}) = 1 \\
		\wedge  ~(\inp_{\advse}, \zkproof_{\advse}) \not\in Q   \\
		\wedge \neg \exists \wit_{\adv}: \REL(\inp_{\adv}, \wit_{\adv}) = 1
		\end{matrix}
	}{
		\begin{aligned}
		& r \sample \RND{\advse},
		(\inp_{\advse}, \zkproof_{\advse}) \gets \advse^{\initU, \simOH, \simOP'} (1^\secpar; r) \\
		\end{aligned}
	}
	\]
	is at most negligible.  
	Here, $\srs$ is the finalized SRS, list $Q$ contains all $(\inp, \zkproof)$ pairs where 
	$\inp$ is an instance provided to the simulator by the adversary and
	$\zkproof$ is the simulator's answer. List $Q_\ro$ contains all $\advse$'s
	queries to $\ro$ and $\ro$'s answers.  
\end{definition}

\label{rem:simext_to_simsnd}
We note that the probability $\ssndProb$ in~\cref{def:simsnd} can be expressed in
terms of simulation-extractability. More precisely, the
condition $\neg \exists \wit: \REL(\inp_\adv, \wit_\adv) = 1$ can be substituted with
$\REL(\inp_\adv, \wit_\adv) = 0$, where $\wit_\adv$, returned by a possibly unbounded
extractor, is either a witness to $\inp_\adv$ (if there exists any) or $\bot$ (if
there is none). More precisely,
\[
\ssndProb = \condprob{
	\begin{matrix}
	\verifier(\srs, \inp_{\advse}, \zkproof_{\advse}) = 1 \\
	\wedge  ~(\inp_{\advse}, \zkproof_{\advse}) \not\in Q   \\
	\wedge  ~\REL(\inp_{\advse}, \wit_{\advse}) = 0
	\end{matrix}
}{
	\begin{aligned}
	& r \sample \RND{\advse},
	(\inp_{\advse}, \zkproof_{\advse}) \gets \advse^{\initU, \simOH, \simOP'} (1^\secpar; r) \\
	& \wit_{\advse} \gets \ext(\srs, \advse, r, \inp_{\advse}, \zkproof_{\advse},
	Q, Q_\ro, Q_\srs) 
	\end{aligned}
}
\]
The only necessary input to the unbounded extractor $\ext$ is the instance
$\inp_\adv$ (the rest is given for the consistency with the simulation extractability
definition). 
%
With the probabilities in \cref{def:simext} holding regardless of whether the extractor
is unbounded or not, we obtain the following equality
$ \ssndProb = \accProb - \extProb$.

\subsection{Simulation soundness---the general result}
\label{sec:general}
Equipped with definitional framework of \cref{sec:se_definitions}, we can also show the proof of simulation soundness of Fiat-Shamir NIZKs based on multi-round protocols.

%\begin{theorem}[Forking simulation-extractable multi-message protocols]
%	\label{thm:se}
%	Let $\ps = (\kgen, \prover, \verifier, \simulator)$ be an interactive $(2 \mu + 1)$-message
%	zero-knowledge proof system for $\RELGEN(\secparam)$, which is trapdoor-less simulatable, has
%	$\ur{k}$ property with security $\epsur(\secpar)$, and is $(\epss(\secpar), k, n)$-forking
%	sound.  Let $\ro\colon \bin^{*} \to \bin^{\secpar}$ be a random oracle.  Then $\psfs$ is
%	forking simulation-extractable with extraction error $\epsur(\secpar)$ against $\ppt$
%	algebraic adversaries that makes up to $q$ random oracle queries and returns an acceptable
%	proof with probability at least $\accProb$.  The extraction probability $\extProb$ is at
%	least
%	\( \extProb \geq \frac{1}{q^{n - 1}} (\accProb - \epsur(\secpar))^{n} -\eps(\secpar)\,, \)
%	for some negligible $\eps(\secpar)$.
%\end{theorem}

\begin{theorem}[Simulation soundness]
	\label{thm:simsnd}
	Let $\ps = (\kgen, \prover, \verifier, \simulator)$ be an interactive $(2 \mu + 1)$-message
	zero-knowledge proof system for $\RELGEN(\secparam)$, which is trapdoor-less simulatable, has
	$\ur{k}$ property with security $\epsur(\secpar)$. Let $\ro\colon \bin^{*} \to \bin^{\secpar}$ be a random oracle. Then, the
	probability that a $\ppt$ adversary $\adv$ breaks simulation soundness of
	$\ps$ is upper-bounded by
	\(
	\epsur(\secpar) + q_\ro^\mu  \epss(\secpar)\,,
	\)
	where $q$ is the total number of queries made by the adversary $\adv$ to $\ro$.
\end{theorem}

\begin{proof}
	\ngame{0} This is a simulation soundness game played between an adversary
	$\advse$ who is given access to an oracle $\initU$ that defines an updatable SRS setup, a random oracle $\ro$ and a simulation oracle
	$\simulator$. $\adv$ wins if it manages to produce an accepting proof
	for a false statement. In the following game hops, we upper-bound the
	probability that this happens.
	
	\ngame{1} This is identical to $\game{0}$ except that the game is aborted if
	there is a simulated proof $\zkproof_\simulator$ for $\inp_{\adv}$ such that
	$(\inp_{\adv}, \zkproof_\simulator[1..k]) = (\inp_{\adv},
	\zkproof_{\adv}[1..k])$. That is, the adversary in its final proof reuses at
	least $k$ messages from a simulated proof it saw before and the proof is
	accepting.  Denote this event by $\event{\errur}$.
	
	\ncase{Game 0 to Game 1} We have, \( \prob{\game{0} \land
		\nevent{\errur}} = \prob{\game{1} \land \nevent{\errur}} \) and, from the
	difference lemma, cf.~\cref{lem:difference_lemma},
	$ \abs{\prob{\game{0}} - \prob{\game{1}}} \leq \prob{\event{\errur}}\,$.
	Thus, to show that the transition from one game to another introduces only
	minor change in probability of $\adv$ winning it should be shown that
	$\prob{\event{\errur}}$ is small.
	
	We can assume that $\adv$ queried the simulator on the instance it wishes to
	output, i.e.~$\inp_{\adv}$. We show a reduction $\rdvur$ that utilises $\adv$
	to break the $\ur{k}$ property of $\ps$. Let $\rdvur$ run $\advse$ internally
	as a black-box:
	\begin{compactitem}
		\item The reduction answers $\adv$ update queries by asking the same query from the update oracle in the unique response experiment. The reduction finalises the same SRS $\srs$ as the one $\adv$ does.
		\item The reduction answers both queries to the simulator $\psfs.\simulator$
		and to the random oracle.  It also keeps lists $Q$, for the simulated
		proofs, and $Q_\ro$ for the random oracle queries.
		\item When $\adv$ makes a fake proof $\zkproof_{\adv}$ for $\inp_{\adv}$,
		$\rdvur$ looks through lists $Q$ and $Q_\ro$ until it finds
		$\zkproof_{\simulator}[0..k]$ such that
		$\zkproof_{\adv}[0..k] = \zkproof_{\simulator}[0..k]$ and a random oracle
		query $\zkproof_{\simulator}[k].\ch$ on $\zkproof_{\simulator}[0..k]$.
		\item $\rdvur$ returns two proofs for $\inp_{\adv}$:
		\begin{align*}
		\zkproof_1 = (\zkproof_{\simulator}[1..k],
		\zkproof_{\simulator}[k].\ch, \zkproof_{\simulator}[k + 1..\mu + 1])\\
		\zkproof_2 = (\zkproof_{\simulator}[1..k],
		\zkproof_{\simulator}[k].\ch, \zkproof_{\adv}[k + 1..\mu + 1])
		\end{align*}
	\end{compactitem}  
	If $\zkproof_1 = \zkproof_2$, then $\adv$ fails to break simulation soundness,
	as $\zkproof_2 \in Q$. On the other hand, if the proofs are not equal, then
	$\rdvur$ breaks $\ur{k}$-ness of $\ps$. This happens only with negligible
	probability $\epsur(\secpar)$, hence
	\( \prob{\event{\errur}} \leq \epsur(\secpar)\,. \)
	
	\ngame{2} This is identical to $\game{1}$ except that now the environment
	aborts if the instance the adversary proves is not in the language.
	
	\ncase{Game 1 to Game 2} 
	% REDUCTION TO INTERACTIVE SOUNDNESS:
	We show that
	$\abs{\prob{\game{1}} - \prob{\game{2}}} \leq q^{\mu} \cdot \epss(\secpar)$,
	where $\epss(\secpar)$ is the probability of breaking soundness of the underlying
	\emph{interactive} protocol $\ps$. Note that
	$\abs{\prob{\game{1}} - \prob{\game{2}}}$ is the probability that $\adv$
	outputs an acceptable proof for a false statement which does not break the
	unique response property (such proofs have been excluded by
	$\game{1}$). Consider a soundness adversary $\adv'$ who initiates a proof with
	$\ps$'s verifier $\ps.\verifier$, internally runs $\adv$ and proceeds as
	follows:
	\begin{compactitem}
		\item It guesses indices $i_1, \ldots, i_\mu$ such that random oracle queries
		$h_{i_1}, \ldots, h_{i_\mu}$ are the queries used in the $\zkproof_\adv$
		proof eventually output by $\adv$. This is done with probability at least
		$1/q^\mu$ (since there are $\mu$ challenges from the verifier in
		$\ps$).
		\item On input $h$ for the $i$-th,
		$i \not\in \smallset{{i_1}, \ldots, {i_\mu}}$, random oracle query, $\adv'$
		returns randomly picked $y$, sets $\ro(h) = y $ and stores $(h, y)$ in
		$Q_\ro$ if $h$ is sent to $\ro$ the first time. If that is not the case,
		$\adv$ finds $h$ in $Q_\ro$ and returns the corresponding $y$.
		\item On input $h_{i_j}$ for the $i_j$-th,
		$i_j \in \smallset{{i_1}, \ldots, {i_\mu}}$, random oracle query, $\adv'$
		parses $h_{i_j}$ as a partial proof transcript $\zkproof_\adv[1..j]$ and
		runs $\ps$ using $\zkproof_\adv[j]$ as a $\ps.\prover$'s $j$-th message to
		$\ps.\verifier$. The verifier responds with a challenge
		$\zkproof_\adv[j].\ch$. $\adv'$ sets $\ro(h_{i_j}) =
		\zkproof_\adv[j].\ch$. If we guessed the indices correctly we have that
		$h_{i_{j'}}$, for $j' \leq j$, parsed as $\zkproof_\adv[1..j']$ is a prefix
		of $\zkproof_\adv[1..j]$.
		\item On query $\inp_\simulator$ to $\simulator$, $\adv'$ runs the simulator
		$\ps.\simulator$ internally. Note that we require a simulator that only
		programs the random oracle for $j \geq k$.  If the simulator makes a
		previously unanswered random oracle query with input
		$\zkproof_\simulator[1..j]$, $1 \leq j < k$, and this is the $i_j$-th query,
		it generates $\zkproof_\simulator[j].\ch$ by invoking $\ps.\verifier$ on
		$\zkproof_\simulator[j]$ and programs
		$\ro(h_{i_j}) = \zkproof_\simulator[j].\ch$.  It returns
		$\zkproof_\simulator$.
		\item Answers $\ps.\verifier$'s final challenge $\zkproof_\adv[\mu].\ch$ using the
		answer given by $\adv$, i.e.~$\zkproof_\adv[\mu]$.
	\end{compactitem}
	That is, $\adv'$ manages to break soundness of $\ps$ if $\adv$ manages to
	break simulation soundness without breaking the unique response property and
	$\adv'$ correctly guesses the indices of $\adv$ random oracle queries. This
	happens with probability upper-bounded by $\abs{\prob{\game{1}} -
		\prob{\game{2}}} \cdot \infrac{1}{q^{\mu}}$. Hence $\abs{\prob{\game{1}} -
		\prob{\game{2}}} \leq q^{\mu} \cdot \epss(\secpar)$.
	
	Note that in $\game{2}$ the adversary cannot win. Thus the probability
	that $\advss$ is successful is upper-bounded by
	$\epsur(\secpar) + q^{\mu} \cdot \epss(\secpar)$.  \qed
\end{proof}


We conjecture that based on the recent results on state restoration soundness~\cite{cryptoeprint:2020:1351}, which effectively allows to query the verifier multiple times on different overlapping transcripts, the $q^{\mu}$ loss could be avoided. However, this would reduce the class of protocols covered by our results. 


\subsection{Simulation soundness of~$\plonkprotfs$}
Since \cref{lem:plonkprot_ur,lem:plonkprot_ss,lem:plonk_hvzk} hold, $\plonkprot$ is $\ur{2}$, computational special sound and trapdoor-less simulatable. We now make use of \cref{thm:simsnd} and show that
$\plonkprot_\fs$ is simulation sound as defined in
\cref{def:simsnd}.

 \begin{corollary}[Simulation soundness of $\plonkprot_\fs$]
   \label{cor:simsnd-P}
   Assume that $\plonkprot$ is $2$-programmable HVZK in the standard model, that
   is computational special sound with security $\epss(\secpar)$, and the $\PCOMp$ is a commitment of knowledge with
   security $\epsk(\secpar)$, binding security $\epsbind(\secpar)$ and has unique
   opening property with security $\epsop(\secpar)$. Then the probability that a
   $\ppt$ adversary $\adv$ breaks simulation soundness of $\plonkprot_{\fs}$ is
   upper-bounded by
   \( \epsk(\secpar) + 2\cdot\epsbind(\secpar) + \epsop(\secpar) + q_\ro^4
   \epss(\secpar)\,, \) where $q$ is the total number of queries made by the
   adversary $\adv$ to a random oracle $\ro\colon \bin^{*} \to \bin^{\secpar}$.
 \end{corollary}

\subsection{Simulation soundness of~$\sonicprotfs$}
The following corollary shows the simulation soundness of $\sonicprotfs$ based on~\cref{lem:sonicprot_ur,lem:sonicprot_ss,lem:sonic_hvzk} and~\cref{thm:simsnd}.
\begin{corollary}[Simulation soundness of $\sonicprot_\fs$]
	\label{cor:simsnd-S}
	Assume that $\sonicprot$ is $1$-programmable HVZK in the standard model, that
	is computational special sound with security $\epss(\secpar)$, and the $\PCOMs$ is a commitment of knowledge with
	security $\epsk(\secpar)$, binding security $\epsbind(\secpar)$ and has unique
	opening property with security $\epsop(\secpar)$. Then the probability that a
	$\ppt$ adversary $\adv$ breaks simulation soundness of $\sonicprot_{\fs}$ is
	upper-bounded by
	\( \epsk(\secpar) + 2\cdot\epsbind(\secpar) + \epsop(\secpar) + q_\ro^4
	\epss(\secpar)\,, \) where $q$ is the total number of queries made by the
	adversary $\adv$ to a random oracle $\ro\colon \bin^{*} \to \bin^{\secpar}$.
\end{corollary}

\subsection{Simulation soundness of~$\marlinprotfs$}
	The simulation soundness of $\marlinprot_\fs$ follows from \cref{thm:simsnd}, and \cref{lem:marlinprot_ur,lem:marlinprot_ss,lem:marlin_hvzk}.
\begin{corollary}[Simulation soundness of $\marlinprot_\fs$]
	\label{cor:simsnd-M}
	Assume that $\marlinprot$ is $1$-programmable HVZK in the standard model, that
	is computational special sound with security $\epss(\secpar)$, and the $\PCOM$ is a commitment of knowledge with
	security $\epsk(\secpar)$, binding security $\epsbind(\secpar)$ and has unique
	opening property with security $\epsop(\secpar)$. Then the probability that a
	$\ppt$ adversary $\adv$ breaks simulation soundness of $\marlinprot_{\fs}$ is
	upper-bounded by
	\( \epsk(\secpar) + 2\cdot\epsbind(\secpar) + \epsop(\secpar) + q_\ro^4
	\epss(\secpar)\,, \) where $q$ is the total number of queries made by the
	adversary $\adv$ to a random oracle $\ro\colon \bin^{*} \to \bin^{\secpar}$.
\end{corollary}


%%% Local Variables:
%%% mode: latex
%%% TeX-master: "main"
%%% End:

%% !TEX root = main.tex
% !TEX spellcheck = en-US

\section{From arguments to proofs}
\newcommand{\epsa}{\eps_{\mathsf{attema}}}
\michals{25.01}{Here I describe how to make Attema et al result ``for proofs'' usable in
  our ``arguments''}
%
\michals{26.1}{Should we give adversary access to the simulator oracle? Probably not, as
  that would force us to write the whole SE proof here. Let's focus on a simpler case and
  introduce the simulator later on
}
%
\michals{26.1}{Should we show this result for every proof system separately?}
%
Let $\proofsystem'$ be a statistically sound idealised proof system and $\proofsystem$ a
computationally sound proof system (i.e.~an argument) obtained from $\proofsystem'$. We
show how to build a knowledge extractor for $\proofsystem$ using extractor for $\proofsystem'$.

\begin{lemma}
  Let $\proofsystem'$ be a knowledge sound proof system with knowledge error
  $\epsid$. Let $\proofsystem$ be a computationally sound proof system \changedm{compiled
    from $\proofsystem'$ with security loss $\eps$}. Let $\adv'$ be a $\proofsystem'$
  adversary that outputs an acceptable proof with probability $\accProb'$, then \hl{...}  
\end{lemma}


\ncase{Game 1} %
The $\proofsystem$ adversary $\adv$ is a PPT machine that gets as input an SRS $\srs$ --
that encodes relation $\REL$ --, oracle access to a random oracle $\ro$, and outputs an
instance--proof pair $(\inp, \zkproof)$. We denote by $\accProb$ probability that
$\verifier (\srs, \inp, \zkproof)$ accepts the proof. We build an extractor $\ext_\adv$
that given $\adv$'s code, and random oracle queries $Q_{ro}$
outputs $\wit$ such that $\REL(\inp, \wit) = 1$ with \changedm{non-negligible}
probability.  Adversary $\adv$ wins whenever $\ext_\adv$ fails to output the witness.

\ncase{Game 2} %
This game is identical to Game 1, except now the environment aborts if the proof
submitted by the adversary $\adv$ in not acceptable by the idealized verifier
$\verifier'$.

\ncase{Game 1 to Game 2} %
\michals{26.1}{Here we are using the reduction known from, e.g.~plonk -- if the adversary
manages to provide a proof acceptable by $\verifier$, but not $\verifier'$ then we can
break the dlog assumption. This happens with probability at most $\epsdlog$.
}

\ncase{Game 3}
\michals{26.1} {(this game was previously a part of Game 2, maybe that was a good idea.)
  We now introduce $\proofsystem'$ adversary $\adv'$ that internally runs $\adv$ (provides
  it with the SRS, passes $\verifier'$ challenges as $\verifier$ challenges) and whenever
  $\adv$ outputs a polynomial commitment it extracts the underlying polynomial and sends
  it to $\prover'$.

  Probability $\accProb'$ that $\adv'$ makes an acceptable proof is thus at least
  $\accProb - \epsdlog - k \cdot \epsk$. The last term, $k \cdot \epsk$ comes as there is
  $k$ polynomial commitments in $\proofsystem$ and probability that $\adv'$ fails to
  extract a polynomial from a polynomial commitment is upper-bounded by $\epsk$.
}

\ncase{Game 2 to Game 3}

\ncase{Game 4} \michals{26.1} { Here we run the Attema et al extractor on $\proofsystem'$
  adversary $\adv'$. Probability that this extractor fails is upper-bounded by
  $\epsa + N \cdot (\epsdlog + k \cdot \epsk)$, where $N$ is the expected number of
  $\ext'_{\adv}$ runs.

  Note that $\epsa$ already includes
  probability that $\adv'$ breaks knowledge soundness of $\proofsystem'$, in our paper
  denoted by $\epsid$.  }

\ncase{Conclusion}
\michals{26.1}{Probability that $\adv$ outputs a valid proof that is acceptable by
  $\verifier$ but $\ext_\adv$ fails is upper-bounded by
  \[
    N \cdot (\epsdlog + k \cdot \epsk) + \epsa
  \]
  }


%%% Local Variables:
%%% mode: latex
%%% TeX-master: "main"
%%% End:


\iffalse
Let $\prover$ and $\verifier$ be $\ppt$ algorithms, the former called \emph{prover}
and the latter \emph{verifier}. 

We denote by $\zkproof$ a proof created by $\prover$ with input
$(\srs, \inp, \wit)$. We say that proof is acceptable if $\verifier (\srs, \inp,
\zkproof)$ accepts it. We focus on proof systems where $\prover$ and
$\verifier$ are given oracle access to a random oracle $\ro$. The simulator
$\simulator$ is not only given access to $\ro$, but it can also \emph{program}
it. That is, it can require that for $(x, y)$ of its choice, $\ro (x) = y$.

A non-interactive  proof system $\proofsystem = (\kgen, \prover, \verifier, \simulator)$ for $\RELGEN$ is
required to have three properties: completeness, soundness and zero knowledge, which are
defined as follows:

\ourpar{Completeness.}

  A non-interactive proof system $\proofsystem$ is
  \emph{complete} if an honest prover always convinces an honest verifier, that
  is for all $\REL \in \RELGEN(\secparam)$ and $(\inp, \wit) \in \REL$
	\[
		\condprob{\verifier^\ro (\srs, \inp, \zkproof) = 1} {\srs \gets \kgen(\REL),
      \zkproof \gets \prover^\ro (\srs, \inp, \wit)}\,.
	\]

\ourpar{Soundness.}
    We say that $\proofsystem$ for $\RELGEN$ is \emph{sound} if no
  $\ppt$ prover $\adv$ can convince an honest verifier $\verifier$ to accept a
  proof for a false statement $\inp \not\in\LANG$. More precisely, for
  all $\REL \in \RELGEN(\secparam)$
	\[
    \condprob{ \verifier^\ro (\srs, \inp, \zkproof) = 1 \land \inp \not\in
      \LANG_\REL}{\srs \gets \kgen(\REL), (\inp, \zkproof) \gets \adv^\ro(\srs)} \leq
    \negl.
	\]

\fi
 
\end{document}
%%% Local Variables:
%%% mode: latex
%%% TeX-master: t
%%% End:
