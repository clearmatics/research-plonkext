\section{Simulation extractability of $\plonk$, omitted proofs}
\label{sec:plonkse_proofs}

\oursubsub{Proof of \cref{lem:plonkprot_ur}}

\begin{proof}
    Let $\adv$ be an algebraic adversary tasked to break the $\ur{3}$-ness of
      $\plonkprotfs$. We show that the first three prover's messages determine, along with 	the verifiers challenges, the rest of it. We denote by $\zkproof^0$ and $\zkproof^1$ the two proofs that the adversary outputs. To distinguish polynomials and commitments which an honest prover would send in the proof from the polynomials and commitments computed by the adversary we write the latter using indices $0$ and $1$ (two indices as we have two transcripts), e.g.~to describe the quotient polynomial provided by the adversary we write $\p{t}^0$ and $\p{t}^1$ instead of $\p{t}$ as in the description of the protocol.
  
    We note that since the unique response property requires from $\zkproof^{0}$ and $\zkproof^{1}$ that the first place they possibly differ is the $4$-th prover's message, then the challenge $\chz$, that is picked by the adversary after the $3$-rd message is the same in both transcripts. This challenge determines the evaluation point of polynomials $\p{a}(X), \p{b}(X), \p{c}(X), \p{t}(X), \p{z}(X)$ which commitments are already sent.
  
    In its fourth message, the prover provides evaluations of the aforementioned polynomials, along with evaluations of publicly known polynomials $
    \p{S_{\sigma 1}} (\chz), \p{S_{\sigma 2}} (\chz)$, and evaluation of a linearization polynomial $\p{r}(\chz)$.
    Note that the adversary can output two acceptable proofs that differ on their fourth message only if it either manages to break evaluation binding of one of the opening, or provides an unacceptable opening which is accepted due to a batching error. Since the commitment scheme is evaluation binding with security loss $\epsbinding (\secpar)$, and the batched verification equation accepts an incorrect opening with probability at most $\infrac{q}{\abs{\FF_p}}$, cf.~\cite{EPRINT:GabWilCio19}, the adversary can make $\zkproof^{0}$ and $\zkproof^{1}$ differ on the fourth message with probability at most $8  \cdot \epsbinding (\secpar) + \infrac{q}{\abs{\FF_p}}$. 
  
    Next, assume that the transcripts are the same up to the fourth message, but differ at the fifth. In that message, the adversary provides openings of the evaluations. Since the unique opening property, the adversary can open the valid evaluation of a polynomial to two different values with probability at most $\epsop (\secpar)$, which is upper-bounded by $\epsudlog (\secpar) + \infrac{q}{\abs{\FF_p}}$, cf.~\cref{lem:pcomp_op}.
  
    By the union bound, the adversary is able to break the unique response property with probability upper bounded by $8  \cdot \epsbinding (\secpar) + \epsudlog (\secpar) + 2 \infrac{q}{\abs{\FF_p}}$.
    \end{proof}

\oursubsub{Proof of \cref{lem:plonkprot_ss}}
\begin{proof}
	Let $\adv^{\ro, \initU}(\secparam; r)$ be the adversary who outputs $(\inp, \zkproof)$ such that $\plonkprotfs.\verifier$ accepts the proof. Let $\tdv$ be a tree-building algorithm of \cref{lem:attema} that outputs a tree $\tree$, and let $\extcss$ be an extractor that given the tree output by $\tdv$ reveals the witness for $\inp$. The main idea of the proof is to show that an adversary who breaks rewinding-based knowledge soundness can be used to break a $\udlog$-problem instance. The proof goes by game hops. Note that since the tree branches after $\adv$'s $3$-rd message, the instance $\inp$, commitments $\gone{\p{a} (\chi), \p{b} (\chi), \p{c} (\chi), \p{z} (\chi), \p{t_{lo}} (\chi), \p{t_{mid}} (\chi), \p{t_{hi}} (\chi)}$, and challenges $\alpha, \beta, \gamma$ are the same in all the transcripts. Also, the tree branches after the third adversary's message where the challenge $\chz$ is presented, thus tree $\tree$ is built using different values of $\chz$.	We consider the following games.

  \ncase{Game 0} %
  In this game the adversary wins if it outputs a valid instance--proof pair $(\inp, \zkproof)$, and the extractor $\extcss$ does not manage to output a witness $\wit$ such that $\REL (\inp, \wit)$ holds.

  \ncase{Game 1} %
  In this game the environment aborts the game if the tree building algorithm $\tdv$ fails in building a tree of accepting transcripts $\tree$. 

  \ncase{Game 0 to Game 1} %
  By \cref{lem:attema} probability that Game 1 is aborted, while Game 0 is not, is at most 
  \[
    \frac{\accProb - (q + 1) \left(1 - \frac{3 \noofc + 15}{\abs{\FF_p}} \right)} {\frac{3 \noofc + 15}{\abs{\FF_p}}} \,.
  \]

  \ncase{Game 2} %
  In this game the environment additionally aborts if at least one of its proofs in $\tree$ is not acceptable by an ideal verifier.

  \ncase{Game 1 to Game 2} % 
  As usual, we show a reduction that breaks an instance of a $\udlog$ assumption when Game 2 is aborted, while Game 1 is not.

  Let $\rdvudlog$ be a reduction that gets as input an $(\noofc + 5, 1)$-$\udlog$ instance $\gone{1, \ldots, \chi^{\noofc + 5}}, \gtwo{\chi}$. Then it can update the instance to another one $\gone{1, \ldots, {\chi'}^{\noofc + 5}}, \gtwo{\chi'}$. Eventually, the reduction outputs $\chi'$.
	%
	The reduction $\rdvudlog$ proceeds as follows.
	First, it builds $\adv$'s SRS $\srs$ using the input $\udlog$ instance. Then it processes the adversary's update query by adding it to the list $\Qsrs$ and passing it to its own update oracle getting instance $\gone{1, \ldots, {\chi'}^{\noofc + 5}}, \gtwo{\chi'}$. The updated SRS $\srs'$ is then computed and given to $\adv$. $\rdvdulog$ also takes care of the random oracle queries made by $\adv$. It picks their answers honestly and write them in $\Qro$. The reduction then starts $\tdv(\srs, \adv, r, \Qro, \Qsrs)$.
	
  Let $(1, \tree)$ be the output returned by $\tdv$. Let $\inp$ be a relation proven in $\tree$.  Consider a transcript $\zkproof \in \tree$ such that $\vereq_{\inp, \zkproof}(X) \neq 0$, but $\vereq_{\inp, \zkproof}(\chi') = 0$. Since $\adv$ is algebraic, all group elements included in $\tree$ are extended by their representation as a combination of the input $\GRP_1$-elements. Hence, all coefficients of the verification equation polynomial $\vereq_{\inp, \zkproof}(X)$ are known. 
  Eventually, the reduction finds $\vereq_{\inp, \zkproof}(X)$ zero points and returns $\chi'$ which is one of them.
    
  Hence, the probability that the adversary wins in Game 2 but does not win in Game 1 is upper-bounded by $(3 \noofc + 6) \cdot \epsudlog (\secpar)$.

  \ncase{Conclusion}
  Note that the adversary can win in Game 2 only if $\tdv$ manages to produce a tree of accepting transcript $\tree$, such that each of the transcripts in $\tree$ is acceptable by an ideal verifier. Note that since $\tdv$ produces $(3 \noofc + 6)$ accepting transcripts for different challenges $\chz$, it obtains the same number of different evaluations of polynomials $\p{a} (X), \p{b} (X), \p{c} (X), \p{z} (X), \p{t} (X)$. With probability $(1 - (3 \noofc + 6) \epsid (\secpar)$ all the transcripts are acceptable by an idealised verifier, the equality between polynomial $\p{t} (X)$ and combination of polynomials $\p{a} (X), \p{b} (X), \p{c} (X), \p{z} (X)$ defined in prover's $3$-rd message description holds. Hence, $\p{a} (X), \p{b} (X), \p{c} (X)$ encodes the valid witness for the proven statement. $\extcss$ can recreate polynomials' coefficients by interpolation and reveal the witness given $(3 \noofc + 6)$ evaluations. Thus, the probability that extraction fails in that case is upper-bounded by $(3 \noofc + 6) \cdot \epsid(\secpar)$.

  Hence, the probability that the adversary wins in Game 0 is upper-bounded by 
  \[
    \epscss(\secpar,\accProb, q) \leq \left(1 - \frac{\accProb - (q + 1) \left(1 - \frac{3 \noofc + 15}{\abs{\FF_p}} \right)}{\frac{3 \noofc + 15}{\abs{\FF_p}}}\right) + (3 \noofc + 6) \cdot \epsudlog (\secpar) + (3 \noofc + 6) \cdot \epsid (\secpar)\,. 
  \]
 \end{proof}

\oursubsub{Proof of \cref{lem:plonk_tlzk}}
\begin{proof}
    As noted in \cref{def:upd-scheme}, subvertible zero-knowledge implies updatable zero-knowledge. Hence, here we show that Plonk is TLZK even against adversaries who picks
    the SRS on its own.
  
  The adversary $\adv(\secparam)$ picks an SRS $\srs$ and instance--witness pair
  $(\inp, \wit)$ and gets a proof $\zkproof$ simulated by the simulator
  $\simulator$ which proceeds as follows.
  
  For its $1$-st message the simulator  picks randomly both the randomizers $b_1, \ldots, b_6$ and
  sets $\wit_i = 0$ for $i \in \range{1}{3\noofc}$. Then $\simulator$
  outputs $\gone{\p{a}(\chi), \p{b}(\chi), \p{c}(\chi)}$. For the first
  challenge, the simulator picks permutation argument challenges $\beta, \gamma$
  randomly.
  
  For its $2$-nd message, the simulator computes $\p{z}(X)$ from
  the newly picked randomizers $b_7, b_8, b_9$ and coefficients of polynomials
  $\p{a}(X), \p{b}(X), \p{c}(X)$. Then it evaluates $\p{z}(X)$ honestly and outputs
  $\gone{\p{z}(\chi)}$. Challenge $\alpha$ that should be sent by the verifier
  after the simulator's $2$ message is picked by the simulator at random.
  
  In its $3$-rd message the simulator starts by picking at random a challenge $\chz$, which
  in the real proof comes as a challenge from the verifier sent \emph{after} the prover
  sends its $3$-rd message. Then $\simulator$ computes evaluations
  \(\p{a}(\chz), \p{b}(\chz), \p{c}(\chz), \p{S_{\sigma 1}}(\chz), \p{S_{\sigma
      2}}(\chz), \pubinppoly(\chz), \lag_1(\chz), \p{Z_H}(\chz),\allowbreak
  \p{z}(\chz\omega)\) and computes $\p{t}(X)$ honestly. Since for a random
  $\p{a}(X), \p{b}(X), \p{c}(X), \p{z}(X)$ the constraint system is (with
  overwhelming probability) not satisfied and the constraints-related polynomials
  are not divisible by $\p{Z_H}(X)$, hence $\p{t}(X)$ is a rational function
  rather than a polynomial. Then, the simulator evaluates $\p{t}(X)$ at $\chz$ and
  picks randomly a degree-$(3 \noofc + 15)$ polynomial $\p{\tilde{t}}(X)$ such that
  $\p{t}(\chz) = \p{\tilde{t}}(\chz)$ and publishes a commitment
  $\gone{\p{\tilde{t}_{lo}}(\chi), \p{\tilde{t}_{mid}}(\chi),
    \p{\tilde{t}_{hi}}(\chi)}$. After that the simulator outputs $\chz$ as a
  challenge.
  
  For the next message, the simulator computes polynomial $\p{r}(X)$ as an honest
  prover would, cf.~\cref{sec:plonk_explained} and evaluates $\p{r}(X)$ at $\chz$.
  
  The rest of the evaluations are already computed, thus $\simulator$ simply outputs
  \( \p{a}(\chz), \p{b}(\chz), \p{c}(\chz), \p{S_{\sigma 1}}(\chz), \p{S_{\sigma
      2}}(\chz), \p{t}(\chz), \p{z}(\chz \omega)\,.  \) After that it picks randomly
  the challenge $v$, and prepares the the last message as an honest prover
  would. Eventually, $\simulator$ and outputs the final challenge, $u$, by picking it
  at random as well.
  
  We argue about zero-knowledge as usual. The property holds since the polynomials that has witness elements at their coefficients are randomized by at least two randomizers and are evaluated at at most two points; and the simulator computes all polynomials as an honest prover would.
  \end{proof}
  