% !TEX root = main.tex
% !TEX spellcheck = en-US

\section{Non-malleability of Marlin}
We show that $\marlin$ is \COMMENT{forking }simulation-extractable. To that end, we show
that $\marlin$ has all the required properties: has unique response property, is
rewinding-based knowledge sound, and its simulator can provide indistinguishable proofs
without a trapdoor, just by programming the random oracle.

\subsection{$\marlin$ Protocol Rolled-out}
$\marlin$ uses R1CS as arithmetization method. That is, the prover given
instance $\inp$ and witness $\wit$ and $|\HHH| \times |\HHH|$ matrices $\vec{A},
\vec{B}, \vec{C}$ shows that $\vec{A} (\inp^\top, \wit^\top)^\top \circ \vec{B}
(\inp^\top, \wit^\top)^\top = \vec{C} (\inp^\top, \wit^\top)^\top$. (Here
$\circ$ is a entry-wise product.)

We assume that the matrices have at most $|\KKK|$ non-zero entries. Obviously,
$|\KKK| \leq |\HHH|^2$. Let $b = 3$, the upper-bound of polynomial evaluations
the prover has to provide for each of the sent polynomials.  Denote by $\dconst$
an upper-bound for $\smallset{|\HHH| + 2b -1, 2 |\HHH| + b - 1, 6 |\KKK| - 6}$.

The idea of showing that the constraint system is fulfilled is as
follows. Denote by $\vec{z} = (\inp, \wit)$. The prover computes polynomials
$\p{z_A} (X), \p{z_B} (X), \p{z_C} (X)$ which encode vectors
$\vec{A} \vec{z}, \vec{B} \vec{z}, \vec{C} \vec{z}$ and have degree $<
|\HHH|$. Importantly, when constraints are fulfilled,
$ \p{z_A} (X) \p{z_B} (X) - \p{z_C} (X) = \p{h_0} (X) \ZERO_\HHH (X)$, for some
$\p{h_0} (X)$ and vanishing polynomial $\ZERO_\HHH (X)$. The prover sends
commitments to these polynomials and shows that they have been computed
correctly. More precisely, it shows that
\begin{equation}
  \label{eq:marlin_eq_2}
\forall \vec{M} \in \smallset{\vec{A}, \vec{B}, \vec{C}},  \forall \kappa \in \HHH,
\p{z_M} (\kappa) = \sum_{\iota \in \HHH} \vec{M}[\kappa, \iota] \p{z}(\iota).
\end{equation}

The ideal verifier checks the following equalities
\begin{equation}
  \label{eq:marlin_ver_eq}
  \begin{aligned}
    \p{h}_3 (\beta_3) \ZERO_\KKK (\beta_3) & = \p{a} (\beta_3) - \p{b} (\beta_3)
    (\beta_3 \p{g_3} (\beta_3) + \sigma_3 / |\KKK|)\\
    \p{r}(\alpha, \beta_2) \sigma_3 & = \p{h_2} (\beta_2) \ZERO_\HHH (\beta_2) +
    \beta_2 \p{g2} (\beta_2) + \sigma_2/|\HHH|\\
    \p{s}(\beta_1) + \p{r}(\alpha, \beta_1) (\sum_M \eta_M \p{z_M} (\beta_1)) -
    \sigma_2 \p{z} (\beta_1) & = \p{h_1} (\beta_1) \ZERO_\HHH (\beta_1) +
    \beta_1
    \p{g_1} (\beta_1) + \sigma_1/|\HHH| \\
    \p{z_A} (\beta_1) \p{z_B} (\beta_1) - \p{z_C} (\beta_1) & = \p{h_0}
    (\beta_1) \ZERO_\HHH (\beta_1)
  \end{aligned}
\end{equation}
where $\p{g_i} (X), \p{h_i} (X)$, $i \in \range{1}{3}$,
$\p{a} (X), \p{b} (X), \sigma_1, \sigma_2, \sigma_3$ are polynomials and
variables required by the sumcheck protocol which allows verifier to efficiently
verify that \cref{eq:marlin_eq_2} holds.
                         

\subsection{Unique Response Property}
\begin{lemma}\label{lem:marlinprot_ur}
  Let $\PCOM$ be a commitment of knowledge that is evaluation binding with security loss 
  $\epsbind(\secpar)$ and has unique opening property with security loss
  $\epsop(\secpar)$. Then
  $\marlinprotfs$ is $\ur{2}$ against algebraic adversaries with security loss $2 \cdot \epsbinding (\secpar) + \epsop (\secpar)$.
\end{lemma}

\begin{proof}
	The proof is similar to the proof of \cref{lem:plonkprot_ur} and \cref{lem:sonicprot_ur}.
	An adversary who can break the $2$-unique response property of $\marlinprotfs$ can be either used to break the commitment scheme's evaluation binding or unique opening property. The former happens with the probability upper-bounded by $2 \cdot \epsbinding (\secpar)$, the latter with probability at most $\epsop (\secpar)$.
	By the union bound, the adversary is able to break the unique response property with probability upper bounded by $2 \cdot \epsbinding (\secpar) + \epsop (\secpar)$.
	\qed
	\end{proof}
%  \changedm{$11 \cdot (\epsop (\secpar)) + 6 \cdot \epsbind (\secpar) +
%    \infrac{3}{\abs{\FF_p}}.$} \mxout{$6 \cdot (\epsbind + \epsop + \epsk) + \epss$}


\COMMENT{\changedm{
\begin{proof}
  Let $\adv$ be an algebraic adversary tasked to break the $\ur{2}$-ness of
  $\marlinprotfs$. We show that the first prover's message determines, along with
  the verifiers challenges, the rest of it.  This is done by game hops. In the games,
  the adversary outputs two proofs $\zkproof^0$ and $\zkproof^1$ for the same statement.
  To distinguish polynomials and commitments which an honest prover sends in the
  proof from the polynomials and commitments computed by the adversary we write the
  latter using indices $0$ and $1$ (two indices as we have two transcripts), e.g.~to
  describe a polynomial provided by the adversary we write $\p{f}^0$ and
  $\p{f}^1$ instead of $\p{f}$ as in the description of the protocol.

  \ngame{0} In this game, the adversary additionally wins if it provides two transcripts that
  match on all $5$ messages sent by the prover.
  Obviously, in this game the adversary cannot win.

  \ngame{1} This game is identical to Game $\game{0}$ except that now the
  adversary additionally wins if it provides two transcripts that matches on the first four
  messages of the proof.

  \ncase{Game 0 to Game 1} We show that the probability that $\adv$ wins in one game but
  does not in the other is negligible.  Observe that in its $4$-th message, the
  adversary is given a challenge $\beta_3$ and has to open the previously computed
  commitments for polynomials $\p{g_3} (X), \p{h_3} (X)$. Since the transcripts match up
  to $\adv$'s $4$-th message, the challenge is the same in both.  Also, the adversary
  evaluates and opens evaluations of polynomials $\p{g_2} (X), \p{h_2} (X)$ evaluated at
  $\beta_2$ and
  $\p{s} (X), \p{z} (X), \p{z_A} (X), \p{z_B} (X), \p{z_C} (X), \p{h_1} (X), \p{g_1} (X)$
  evaluated at $\beta_1$. Hence, to be able to give two different openings in its $5$-th
  message, $\adv$ has to break the unique opening property of the KZG commitment scheme
  which happens with probability $11 \cdot \epsop (\secpar)$ tops (as $11$ polynomials are
  evaluated).
  
  \ngame{2} This game is identical to Game $\game{1}$ except that now the
  adversary additionally wins if it provides two transcripts that matches on the
  first three messages of the proof.

  \ncase{Game 1 to Game 2} In its $4$-th message the adversary
  provides $\sigma_3$ and commits to polynomials $\p{g_3} (X), \p{h_3} (X)$ which are
  uniquely determined. Probability that the adversary commits to wrong polynomials that it
  can later evaluate to correct values (i.e.~to values of evaluated correct polynomials)
  is upper bounded by $2 \cdot \epsbind (\secpar) + 1 / \abs{\FF_p}$. This is since the
  adversary either has to break binding property in at least one of two commitments or
  guess the evaluation point.

  \ngame{3} This game is identical to Game $\game{2}$ except that now the
  adversary additionally wins if it provides two transcripts that matches on the
  first two messages of the proof.

  \ncase{Game 2 to Game 3} In its $3$-rd message the adversary
  provides $\sigma_2$ and commits to polynomials $\p{g_2} (X), \p{h_2} (X)$ which are
  uniquely determined. Probability that the adversary commits to wrong polynomials that it
  can later evaluate to correct values (i.e.~to values of evaluated correct polynomials)
  is upper bounded by $2 \cdot \epsbind (\secpar) + 1 / \abs{\FF_p}$ as explained in the
  previous point.

  \ngame{4}  This game is identical to Game $\game{3}$ except that now the
  adversary additionally wins if it provides two transcripts that matches on the
  first message of the proof.

  \ncase{Game 3 to Game 4} In its $2$-nd message the adversary
  provides $\sigma_1$ and commits to polynomials $\p{g_1} (X), \p{h_1} (X)$ which are
  uniquely determined. Probability that the adversary commits to wrong polynomials that it
  can later evaluate to correct values (i.e.~to values of evaluated correct polynomials)
  is upper bounded by $2 \cdot \epsbind (\secpar) + 1 / \abs{\FF_p}$ as explained before.
  
  \ncase{Conclusion} Taking all the games together, probability that $\adv$ wins
  in Game 4 is upper-bounded by
  \[
    11 \cdot (\epsop (\secpar)) + 6 \cdot \epsbind (\secpar) + \frac{3}{\abs{\FF_p}}.
  \]
  \qed
\end{proof}
}}

\COMMENT{
  \michals{2.11}{Old proof}
\begin{proof}
  As in previous proofs, we show the property by game hops. Let
  $N = \p{g_1}, \p{h_1}, \p{g_2}, \p{h_2}, \p{g_3}, \p{h_3}$. That is, $N$ is a
  set of all polynomials which commitments prover sends after it sends its first message.

  \ncase{Game 0} In this game the adversary wins if it breaks evaluation
  binding, unique opening property, or knowledge soundness of one of commitments
  for polynomials in $N$.

  Probability that a $\ppt$ adversary wins in Game 0, is upper bounded by $6
  \cdot (\epsbind + \epsop + \epsk)$.

  \ncase{Game 1} In this game the adversary additionally wins if it breaks the
  $\ur{1}$ property of the protocol

  \ncase{Game 0 to Game 1} Probability that the adversary wins in Game 1 but not in Game 0
  is $\epss$. This is since in the honest proof the polynomials in $N$ are uniquely
  determined. W.l.o.g.~we analyse probability that adversary is able to produce two
  (different) pairs of polynomials $(\p{h_2}, \p{g_2})$ and $(\p{h'_2}, \p{g'_2})$ such
  that
  \begin{align*}
    \p{h_2} (X) \ZERO_{\HHH} (X) + X \p{g_2} (X) & = \p{h_2} (X) \ZERO_{\HHH} (X) +
                                                   X \p{g_2} (X)\\
    (\p{h_2} (X) - \p{h'_2} (X)) \ZERO_{\HHH} (X) & = X (\p{g'_2} (X) - \p{g'_2}
    (X)).
  \end{align*}
  Since $\p{h_2}, \p{g_2} \in \FF^{< |\HHH| - 1} [X]$ and
  $\ZERO \in \FF^{|\HHH|} [X]$, LHS has different degree than RHS unless both
  sides have degree $0$. This happens when $\p{h_2} (X) = \p{h'_2} (X)$ and
  $\p{g_2} (X) - \p{g'_2} (X)$.
  Thus, for the adversary to be successful in this game it has to provide acceptable
  proofs where $(\p{h_2}, \p{g_2})$ and $(\p{h'_2}, \p{g'_2})$ differ. One of such pair
  has to be incorrect and will be accepted with probability at most $\epss$.
\end{proof}}

\subsection{Rewinding-Based Knowledge Soundness}
\begin{lemma}
	\label{lem:marlinprot_ss}
	$\marlinprotfs$ is $(2, \multconstr + 3)$-rewinding-based knowledge sound against algebraic adversaries who make up to $q$ random oracle queries with security loss 
	\[
	\epscss(\secpar,\accProb, q) \leq \left(1 - \frac{\accProb - (q + 1) \left( \frac{\multconstr + 2}{p}\right)}{1 - \frac{\multconstr + 2}{p}}\right) + (\multconstr + 3)\cdot\epsudlog (\secpar) + (\multconstr + 3) \cdot \epsid (\secpar)\,,
	\]
	Here $\accProb$ is a probability that the adversary outputs an acceptable proof, $\epsid(\secpar)$ is a soundness error of the ideal verifier for $\marlinprotfs$, and $\epsudlog(\secpar)$ is the security of $(\multconstr + 2, 1)$-$\udlog$ \hamid{2.5}{Not sure about $(\multconstr + 2, 1)$!}assumption.
\end{lemma}
\begin{proof}
The proof is similar to the proof of \cref{lem:plonkprot_ss} and \cref{lem:sonicprot_ss}. 
We use Attema et al.~\cite[Proposition 2]{EPRINT:AttFehKlo21} to bound the probability that the tree-building algorithm $\tdv$ does not obtain a tree of acceptable transcript in an expected number of runs. This happens with probability at most
\[
1 - \frac{\accProb - (q + 1) \left(\frac{ \noofc + 2}{p} \right)}{1 - \frac{\noofc + 2}{p}}
\]
Let $\tree$ be the tree outputted by $\tdv$. If one of the proofs in $\tree$ is not acceptable by the ideal verifier, one can break an instance of an updatable dlog assumption which happens with probability at most $(\noofc + 3)  \cdot \epsudlog (\secpar)$. In the case that all the transcripts are acceptable by the ideal verifier, but $\extss$ fails to extract a valid witness from $\tree$, one can break the soundness of the ideal verifier in one of the transcripts. That happens with probability at most $(\noofc + 3) \cdot \epsid (\secpar)$. Taking a union bound completes the proof.
\qed
\end{proof}


\COMMENT{\begin{lemma}\label{lem:marlinprot_ss}
	Assume that an idealised $\marlinprot$ verifier fails with probability at most
	$\epsid(\secpar)$ and probability that a $\ppt$ adversary breaks $\udlog$ is
	bounded by $\epsudlog(\secpar)$. Then $\marlinprotfs$ is
	$(2, d + 1)$-rewinding-based knowledge
	sound with security loss $\epsid (\secpar) + \epsudlog (\secpar)$.
\end{lemma}
\begin{proof}
	% \michals{8.9}{Need to check the degrees}
	The proof goes similarly to the respective proofs for $\plonk$ and
	$\sonic$. That is, let $\srs$ be $\marlinprot$'s finalized SRS and denote by $\srs_1$
	all SRS's $\GRP_1$-elements. Let $\tdv$ be an algebraic adversary that
	produces a statement $\inp$ and a $(1, \dconst + 1, 1, 1)$-tree of
	accepting transcripts $\tree$. Note that in all transcripts the instance
	$\inp$, proof elements
	$\sigma_1, \gone{\p{w}(\chi), \p{z_A}(\chi), \p{z_B}(\chi), \p{z_C}(\chi),
		\p{h_0}(\chi), \p{s}(\chi)}, \gone{\p{g_1}(\chi), \p{h_1}(\chi)}$
	and challenges $\alpha, \eta_1, \eta_2, \eta_3$ are common as the transcripts
	share the first $3$ messages. The tree branches after the third message of the
	protocol where the challenge $\beta_1$ is presented, thus tree $\tree$ is
	build using different values of $\beta_1$.
	
	We consider the following games.
	
	\ncase{Game 0} In this game the adversary wins if all the transcripts it
	produced are acceptable by the ideal verifier,
	i.e.~$\vereq_{\inp, \zkproof}(X) = 0$, cf.~\cref{eq:marlin_ver_eq}, yet the extractor
	fails to extract a valid witness out of them.
	
	Probability of $\tdv$ winning this game is $\epsid(\secpar)$ as the protocol
	$\marlinprot$, instantiated with the idealised verification equation, is
	perfectly sound except with negligible probability of the idealised verifier
	failure $\epsid(\secpar)$. Hence for a valid proof $\zkproof$ for a statement
	$\inp$ there exists a witness $\wit$, such that $\REL(\inp, \wit)$ holds. Note
	that since the $\tdv$ produces $(\dconst + 1)$ accepting transcripts for
	different challenges $\beta_1$, it obtains the same number of different
	evaluations of polynomials $\p{z_A}, \p{z_B}, \p{z_C}$.
	
	Since the transcripts are acceptable by an idealised verifier, the equality
	$\p{z_A} (X) \p{z_B} (X) - \p{z_C} (X) = \p{h_0} (X) \ZERO_\HHH (X)$ holds and
	each of $\p{z}_M$, $M \in \smallset{A, B, C}$, has been computed
	correctly. Hence, $\p{z_A}, \p{z_B}, \p{z_C}$ encodes the valid witness for
	the proven statement. Since $\p{z_A}, \p{z_B}, \p{z_C}$ are of degree at most
	$\dconst$ and there is more than $(\dconst + 1)$ their evaluations
	known, $\extt$ can recreate their coefficients by interpolation and reveal the
	witness with probability $1$. Hence, the probability that extraction fails in
	that case is upper-bounded by probability of an idealised verifier failing
	$\epsid(\secpar)$, which is negligible.
	
	\ncase{Game 1} In this game the adversary additionally wins if it produces a
	transcript in $\tree$ such that $\vereq_{\inp, \zkproof}(\chi) = 0$, but
	$\vereq_{\inp, \zkproof}(X) \neq 0$. That is, the ideal verifier does not
	accept the proof, but the real verifier does.
	
	\ncase{Game 0 to Game 1} Assume the adversary wins in Game 1, but
	does not win in Game 0. We show that such adversary may be used to break the
	$\udlog$ assumption. More precisely, let $\tdv$ be an adversary that for
	relation $\REL$ and randomly picked $\srs \sample \kgen(\REL)$ produces a tree
	of accepting transcripts such that the winning condition of the game
	holds. Let $\rdvdulog$ be a reduction that gets as input an
	$(\dconst, 1)$-$\udlog$ instance $\gone{1, \ldots, \chi^\dconst}, \gtwo{1, \chi}$ and
	is tasked to output $\chi$. The reduction proceeds as follows---it builds $\adv$'s SRS $\srs$ in the updatable setting using the input $\udlog$ instance. Namely it answers $\adv$'s queries for SRS updates and sets the honest update of the SRS to be $\srs$. Let $\srs'$ be the finalized SRS and $(1, \tree)$ be the output
	returned by $\adv$. Let $\inp$ be a relation proven in $\tree$.  Consider a
	transcript $\zkproof \in \tree$ such that $\vereq_{\inp, \zkproof}(X) \neq 0$,
	but $\vereq_{\inp, \zkproof}(\chi) = 0$. Since the adversary is algebraic, all
	group elements included in $\tree$ are extended by their representation as a
	combination of the input $\GRP_1$-elements. Hence all coefficients of the
	verification equation polynomial $\vereq_{\inp, \zkproof}(X)$ are known and
	$\rdvudlog$ can find its zero points. Since
	$\vereq_{\inp, \zkproof}(\chi) = 0$, the targeted discrete log value $\chi'$ is
	among them.  Let $\chi_1, \ldots, \chi_\ell$ be the partial trapdoors of $\adv$'s SRS updates,  extracted by the reduction from the update proofs given by $\adv$. Now $\rdvudlog$ returns $\chi = \chi' (\chi_1 \chi_2 \ldots \chi_\ell)^{-1}$ and breaks the $\udlog$ assumption. Hence, the probability that this event happens is upper-bounded
	by $\epsudlog(\secpar)$.
	
\end{proof}}

\subsection{Trapdoor-Less Zero-Knowledge of Marlin}
\begin{lemma}
  \label{lem:marlin_hvzk}
  $\marlinprotfs$ is 2-programmable trapdoor-less zero-knowledge.
\end{lemma}
\begin{proof}
The simulator follows the protocol except it picks the challenges $\alpha,
\eta_A, \eta_B, \eta_C, \beta_1, \beta_2, \beta_3$ before it picks polynomials
it sends.

First, it picks $\p{\tilde{z}}_A (X)$, $\p{\tilde{z}}_B (X)$ at random and
$\p{\tilde{z}}_C (X)$ such that
$\p{\tilde{z}}_A (\beta_1) \p{\tilde{z}}_B (\beta_1) = \p{\tilde{z}}_C
(\beta_1)$.  Given the challenges and polynomials $\p{\tilde{z}}_A (X)$,
$\p{\tilde{z}}_B (X)$, $\p{\tilde{z}}_C (X)$ the simulator computes
$\sigma_1 \gets \sum_{\kappa \in \HHH} \p{s}(\kappa) + \p{r}(\alpha, X) (\sum_{M
  \in \smallset{A, B, C}}\eta_M \p{\tilde{z}}_M(X)) - \sum_{M \in \smallset{A,
    B, C}} \eta_M \p{r}_M (\alpha, X) \p{\tilde{z}} (X)$.

Then the simulator starts the protocol and follows it, except it programs the
random oracle that on partial transcripts it returns the challenges picked by
$\simulator$.
\end{proof}

%\subsection{From Rewinding-Based Knowledge Soundness and Unique Response Property to\COMMENT{ forking}
%  Simulation Extractability of $\marlinprotfs$}
%\begin{corollary}
%  Assume that $\marlinprotfs$ is $\ur{2}$ with security
%  $\epsur(\secpar) = 6 \cdot (\epsbind + \epsop + \epsk)$, and rewinding-based knowledge sound
%  with security $\epss(\secpar)$. Let $\ro\colon \bin^* \to \bin^\secpar$ be a
%  random oracle. Let $\advse$ be an adversary that can make up to $q$
%  random oracle queries, and outputs an
%  acceptable proof for $\marlinprotfs$ with probability at least
%  $\accProb$. Then $\marlinprotfs$ is \COMMENT{forking }simulation-extractable with
%  extraction error $\eta = \epsur(\secpar)$. The extraction probability
%  $\extProb$ is at least
%  \[
%    \extProb \geq q^{-\dconst} (\accProb - 6 \cdot (\epsbind + \epsop +
%    \epsk))^{\dconst + 1} -\eps(\secpar)\,.
%\]
%	for some negligible $\eps(\secpar)$, $\dconst$ being, the upper bound of
%  constraints of the system.
%\end{corollary}

\subsection{Simulation Extractability of $\marlinprotfs$}
Since \cref{lem:marlinprot_ur,lem:marlinprot_ss,lem:marlin_hvzk} hold, $\marlinprotfs$ is $\ur{2}$, rewinding-based knowledge sound and trapdoor-less zero-knowledge. By making use
of \cref{thm:se}, we conclude that $\marlinprotfs$ is simulation-extractable as defined in \cref{def:simext}.

\begin{corollary}[Simulation extractability of $\marlinprotfs$]
	\label{thm:marlinprotfs_se}
	$\marlinprotfs$ is \emph{updatable simulation-extractable} against any $\ppt$ adversary $\advse$ who makes up to $q$ random oracle queries and returns an acceptable proof with probability at least $\accProb$ with extraction failure probability 
	\[
	\epsse(\secpar, \accProb, q) \leq \left(1 - \frac{\accProb - \epsur (\secpar) - (q + 1) \epserr (\secpar)} {1 - \epserr (\secpar)}\right) + (\multconstr  + 3) \cdot \epsudlog (\secpar) + (\multconstr + 3) \cdot \epsid (\secpar),
	\]
	where $\epserr (\secpar) = \frac{\multconstr + 2}{p}$, $p$ is the size of the field, and $\noofc$ is the number of constrains in the circuit. 
\end{corollary}


%\section{Further work}
%We identify a number of problems which we left as further work. First of all,
%the generalised version of the forking lemma presented in this paper can be
%generalised even further to include protocols where forking soundness holds for
%protocols where $\extt$ extracts a witness from a $(n_1, \ldots, n_\mu)$-tree of
%acceptable transcripts, where more than one $n_j > 1$. I.e.~to include
%protocols that for witness extraction require transcripts that branch at more
%than one point.
%
%Although we picked $\plonk$ and $\sonic$ as examples for our framework, it is
%not limited to SRS-based NIZKs. Thus, it would be interesting to apply it to
%known so-called transparent zkSNARKs like Bulletproofs \cite{SP:BBBPWM18},
%Aurora \cite{EC:BCRSVW19} or AuroraLight \cite{EPRINT:Gabizon19a}.
%
%Since the rewinding technique and the forking lemma used to show simulation
%extractability of $\plonkprotfs$ and $\sonicprotfs$ come with security loss,
%it would be interesting to show SE of these protocols directly in the
%algebraic group model.
%
%Although we focused here only on zkSNARKs, it is worth to
%investigating other protocols that may benefit from our framework, like
%e.g.~identification schemes.
%
%Last, but not least, this paper would benefit greatly if a more tight version
%of the generalised forking lemma was provided. However, we have to note here
%that some of the inequalities used in the proof are already tight, i.e.~for
%specific adversaries, some of the inequalities are already equalities.

%%% Local Variables:
%%% mode: latex
%%% TeX-master: "main"
%%% End:
