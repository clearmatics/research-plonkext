% !TEX root = main.tex
% !TEX spellcheck = en-US
\section{Simulation-extractability---the general result}
\label{sec:general}
Equipped with definitional framework of \cref{sec:se_definitions}, we now present the
main result of this paper---a proof of \COMMENT{forking }simulation extractability of
Fiat-Shamir compiled multi-message protocols.

The proof proceeds by game hopping. The games are controlled by an environment $\env$ that
internally runs a simulation extractability adversary $\advse$, provides it with access to
simulator $\simulator$ which also responds $\adv$'s random oracle queries. The environment
also rewinds $\adv$ when necessary. The games differ by various breaking points,
i.e.~points where the environment decides to abort the game.

The first game $\game{0}$ is a simulation-extractability game, where the extractor
gets as input
\begin{inparaenum}[(1)]
\item simulated proofs $Q$,
\item random oracle queries $Q_\ro$,
\item SRS updates $Q_\srs$
\end{inparaenum}
which were made by $\advse$.  We denote by $\radv$ randomness of $\advse$ and by
$\rsim$ randomness that was used by the simulator to create responses in the list $Q$
and $Q_\ro$. Formally, one could state that $\rsim$ could be parsed into $r_Q$ and
$r_\ro$ used to answer queries in $Q$ and $Q_\ro$. Let $r = (\radv,
\rsim)$. Randomness $r$ is used in the following games -- we pass it to a
unique-response property reduction $\rdvur$ and computational special soundness adversary
$\bdv$. Importantly, since we require that $\rdvur$ and $\bdv$ are successful (with
non-negligible probability) given uniformly random randomness and $r$ is picked
uniformly random we conclude that we do not change any success probability by running
$\rdvur$ and $\bdv$ on $r$. 

Denote by $\zkproof_{\advse}, \zkproof_{\simulator}$ proofs returned by the adversary
and the simulator respectively. We use $\zkproof[i]$ to denote $i$-th prover's
message. $\zkproof[i].\ch$ denotes the challenge that is given to the prover after
$\zkproof[i]$, and $\zkproof[i..j]$ denotes all messages of the proof exchanged
between $i$-th and $j$-th prover's messages, but not challenge
$\zkproof[j].\ch$. When it is not explicitly stated, we denote the proven instance
$\inp$ by $\zkproof[0]$. If the proof system starts with a message from the verifier,
we denote it by $\zkproof[0].\ch$; else, we state that $\zkproof[0].\ch$ is empty.

Without loss of generality, we assume that whenever the accepting proof contains a
response to a challenge from a random oracle, then the adversary queried the oracle
to get it. It is straightforward to transform any adversary that violates this
condition into an adversary that makes these additional queries to the random oracle
and wins with the same probability.

Finally, we note that for a proof system that is $k$-unique response, $k'$-programmable trapdoor-less simulatable, and $(1, \ldots, 1, n_{k''}, \ldots, n_\mu)$-computational special sound, the theorem requires that $k \leq k' \leq k''$. The first condition ensures that a $k'$-th move message $\zkproof_{\advse}[k']$ output by $\adv$ is fresh and not coming from the simulator. The second condition $k' \leq k''$ is to make the simulator able to program the random oracle for the challenges that $\advse$ needs to respond in the $k''$-th move.

%!TEX root=main.tex

\begin{theorem}[Simulation-extractable multi-message protocols]
	\label{thm:se}
	Let $\psfs = (\SRScer, \prover, \verifier, \simulator)$ be a  $(2\mu + 1)$-message FS-transformed NIZK proof system with an updatable SRS setup. If $\psfs$ is an
	updatable $k$-unique response protocol with security loss $\epsur$,
	updatable $k$-programmable trapdoor-less zero-knowledge, and %$(1, \ldots, 1, n_{k}, \ldots, n_\mu)$-
	updatable rewinding-based knowledge sound with security loss $\epscss$; 
	%
	Then $\psfs$ is \emph{updatable simulation-extractable} with security loss $$\epsse(\secpar,\accProb,q) \leq \epscss(\secpar,\accProb - \epsur(\secpar),q)$$ against any $\ppt$ adversary $\advse$ that makes up to $q$ random oracle queries and returns an accepting proof with probability at least $\accProb$.
\end{theorem}

\begin{proof}	
	Let $(\inp, \zkproof) \gets \advse^{\initU, \simOH, \simOP'}(r_{\advse})$ be the USE adversary. We show how to build an extractor $\extse (\srs, \advse, r_{\advse}, \Qsim, \Qro, \Qsrs)$ that outputs a witness $\wit$, such that $\REL(\inp, \wit)$ holds with high probability. To that end we define an algorithm $\advcss^{\initU,\ro}(r)$ against rewinding-based knowledge soundness of $\psfs$ that runs internally $\advse^{\initU, \simOH, \simOP'}(r_{\advse})$. Here $r = (\rsim, r_{\advse})$ with $\rsim$ the randomness that will be used to simulate $\simOP'$. 

	The code of $\advcss^{\initU,\ro}(r)$ hardcodes $\Qsim$ such that it does not use any randomness for proofs in $\Qsim$ as long as statements are queried in order. In this case it simply returns a proof $\zkproofs$ from $\Qsim$ but nevertheless queries $\simOHprog$ on $(\tzkproofs[0..k],\tzkproofs[k].ch)$, i.e. it programs the $k$-th challenge. While it is hard to construct such an adversary without knowing $\Qsim$, it clearly exists and $\extse$ has the necessary inputs to construct $\advcss$. This hardcoding guarantees that $\advcss$ returns the same $(\inp,\zkproof)$ as $\advse$ in the experiment.
	%
	Eventually, $\extse$ uses the tree builder $\tdv$ and extractor $\extcss$ for $\advcss$ to extract the witness for $\inp$. Both guaranteed to exist (and be successful with high probability) by rewinding-based knowledge soundness. This high-level argument shows that $\extse$ exists as well.
	
	We now give the details of the simulation that guarantees that $\advcss$ is successful whenever $\advse$ is---except with a small security loss that we will bound later:
	Since $\advcss$ runs $\advse$ internally, it needs to take care of $\advse$'s oracle queries.
	$\advcss$ passes on queries of $\advse$ to the update oracle $\initU$ to its own $\initU$ oracle and returns the result to $\advse$.
	$\advcss$ internally simulates (non-hardcoded) queries to the simulator $\simOP'$ by running the $\simulator$ algorithm on randomness $\rsim$ of its tape. $\simulator$ requires access to oracles $\simOH$ to compute a challenge honestly and $\simOHprog$ to program a challenge. Again $\advcss$ simulates both of these oracles internally, cf.~\cref{fig:simulator_oracles}, this time using the $\ro$ oracle of $\advcss$. 	Note that queries of $\advse$ to $\simOH$ are not programmed, but passed on to $\ro$. 
	
	Importantly, all challenges in simulated proofs, up to round $k$ are also computed honestly, i.e. $\tzkproof[i].\ch = \ro(\tzkproof[0..i])$, for $i < k$.
	%

	
	\begin{figure}
		\centering
			\begin{pcvstack}[center,boxed]
			\begin{pchstack}
				\procedure{$\simOH (x)$}
				{
				\pcif H[x] = \bot \pcthen \\
				\pcind H[x] \gets \ro(x) \\
				\pcreturn H[x]
		  		}
				%
				\pchspace
				%
				\procedure{$\simulator\oracleo.\prog(x, h)$}
				{ 
					\pcif H[x] = \bot \pcthen \\ 
					\pcind H[x] \gets h \\
					\pcind \Qprog \gets \Qprog \cup \{x\}\\
					\pcreturn H[x]
				}
			\end{pchstack}
		\end{pcvstack}
		\caption{Simulating random oracle calls.}
		\label{fig:simulator_oracles}
	\end{figure}	
%

	Eventually, $\advse$ outputs an instance and proof $(\inp, \zkproof)$. $\advcss$ returns the same values as long as $\tzkproof[0..i] \notin \Qprog$, $i\in[1,\mu]$. This models that the proof output by $\advcss$ must not contain any programmed queries as such a proof would not be consistent to~$\ro$ in the RBKS experiment. If $\advse$ outputs a proof that does contain programmed challenges, then $\advcss$ aborts. We denote this event by $\event{E}$.
	
	\begin{lemma}
		Probability that $\event{E}$ happens is upper-bounded by $\epsur(\secpar)$. 
	\end{lemma}
	\begin{proof}
	%Denote by $\zkproof_{\advse}, \zkproof_{\simulator}$ proofs returned by the adversary and the simulator respectively.

	We build an adversary $\advur^{\initU, \ro} (\secpar; r)$ that has access to the random oracle $\ro$ and update oracle $\initU$. $\advur$ uses $\advcss$ to break the $\ur{k}$ property of $\psfs$. 
	%Namely, for randomness $r = (r_{\advse}, \rsim)$, $\advur^{\ro, \initU} (\secpar; r)$ runs $\advse^{\initU, \simOH, \simOP'}(\secparam; r_{\advse})$ internally and answers its oracle queries as $\advcss$ did.

	When $\advcss$ outputs a proof $\zkproof$ for $\inp$ such that $\event{E}$ holds, $\advur$ looks through lists $Q$ and $\Qro$ until it finds $\tzkproofs[0..k]$ such that $\tzkproof[0..k] = \tzkproofs[0..k]$ and a programmed random oracle query $\tzkproofs[k].\ch$ on $\tzkproofs[0..k]$.	$\advur$ returns two proofs $\zkproof$ and $\zkproofs$ for $\inp$:
		% \begin{align*}
		% \zkproof_1 = \zkproofs =  (\zkproofs[1..k], \zkproofs[k + 1..\mu + 1])\\
		% \zkproof_2 = \zkproof \;\;\;\;\, = (\zkproofs[1..k], \;\;\;\;\zkproof[k + 1..\mu + 1])
		% \end{align*}
		and the challenge $\tzkproofs[k].\ch=\tzkproof[k].\ch$

	Importantly, both proofs are  w.r.t~the unique response verifier. The first, since it is a correctly computed simulated proof for which the unique response property definition allows any challenges at $k$. The latter, since it is an  proof produced by the adversary.
	We have that $\zkproof \neq \zkproofs$ as otherwise $\advse$ does not win the simulation extractability game as $\zkproof \in Q$. On the other hand, if the proofs are different, then $\advur$ breaks $\ur{k}$-ness of $\psfs$. This happens only with  probability $\epsur(\secpar)$. 
	\qed
	\end{proof}

	We denote by $\waccProb$ the probability that $\advcss$ outputs an  proof. We note that by up-to-bad reasoning $\waccProb$ is at most $\epsur (\secpar)$ far from the probability that $\advse$ outputs an  proof. Thus, the probability that $\advcss$ outputs an  proof is at least $\waccProb \geq \accProb - \epsur (\secpar)$. %\markulf{30.04}{I am ignorant. What is the role of the union bound here. Is this the same as up-to-bad reasoning?}
%
	Since $\psfs$ is $\epscss (\secpar, \waccProb,q)$ rewinding-based knowledge sound, there is a tree builder $\tdv$ and extractor $\extcss$ that rewinds $\advcss$ to obtain a tree of accepting transcripts $\tree$ and fails to extract the witness with probability at most $\epscss (\secpar, \waccProb, q)$. The extractor $\extse$ outputs the witness with the same probability.

	%\hamid{30.4}{We probably mean rewinding-based KS here. I think we need to also justify why we require this property with specific parameters $(1, \ldots, 1, n_{k}, \ldots, n_\mu)$ for the tree structure!}\markulf{30.04}{Fixed the typo, if that's what you meant. I don't think we need any specific tree structure. There is nevertheless more to say here, I agree.}
%
	Thus $\epsse(\secpar,\accProb,q) = \epscss (\secpar, \waccProb,q) \leq \epscss(\secpar,\accProb - \epsur,q)$.
	\qed
	\end{proof}

\begin{remark}
Observe that our theorem does not depend on $\epszk(\secpar)$. There is no real prover algorithm $\prover$ in the experiment. Only the $k$-programmability of TLZK matters. 
\end{remark}

\begin{remark}
Observe that the theorem does not prescribe a tree shape for the tree builder $\tdv$. Interestingly, in our concrete results $\tdv$ outputs a $(k, *)$-tree of accepting transcripts.
\end{remark}

%%% Local Variables:
%%% mode: latex
%%% TeX-master: "main"
%%% End:


%%!TEX root=main.tex

\begin{theorem}[Simulation-extractable multi-message protocols]
	\label{thm:se}
	Let $\ps = (\kgen, \prover, \verifier, \simulator)$ be an interactive
	$(2 \mu + 1)$-message zero-knowledge proof system for $\RELGEN(\secparam)$
	\changedm{with knowledge soundness error $\epsk(\secpar)$}. Let
	$\ro\colon \bin^{*} \to \bin^{\secpar}$ be a random oracle. If $\psfs$ has the
	$k$-\emph{unique response property} with security loss $\epsur(\secpar)$, is
	\emph{$k'$-programmable trapdoor-less simulatable}, is $(k'', n)$-\emph{forking
		special sound} with security loss $\epss(\secpar)$, and all of these properties
	hold in the updatable setting and $k \geq k' \geq k''$ then $\psfs$ is
	\emph{simulation-extractable in the updatable setting} with extraction error
	\textcolor{red}{$\epsur(\secpar)$} against $\ppt$ adversaries that make up to $q$ random oracle
	queries and returns an acceptable proof with probability at least $\accProb$.  The
	extraction probability $\extProb$ is at least
	\textcolor{red}{\( \extProb \geq \frac{1}{q^{n - 1}} (\accProb - \epsur(\secpar))^{n}
		-\eps(\secpar)\,, \)} for some negligible $\eps(\secpar)$.
\end{theorem}
\begin{proof}		
	
	\ngame{0} This is the simulation-extractability game played between an adversary
	$\advse$ who is given access to an oracle $\initU$ that defines an updatable SRS
	setup, and a simulation oracle $\simulator = (\simulator_1, \simulator_2')$ which provides the adversary with simulated
	proofs and random oracle functionality. We denote by $\rsim$
	randomness of $\simulator$ which is randomly sampled. Execution of $\advse$ results in
	lists of
	\begin{inparaenum}[(1)]
		\item simulated proofs $Q$,
		\item random oracle queries $Q_\ro$,
		\item SRS updates $Q_\srs$.
	\end{inparaenum}
	%
	Extractor $\ext$ takes as input a proof $\zkproof_{\advse}$ for instance
	$\inp_{\advse}$ output by the adversary and lists $Q, Q_\ro, Q_\srs$ and is tasked
	to extract a witness $\wit_{\advse}$ such that $\REL(\inp_{\advse}, \wit_{\advse})$
	holds. $\advse$ wins if it manages to produce an acceptable proof and the extractor
	fails to output a witness. In the following game hops we upper-bound the
	probability that this happens. Note that $\srs$ is with respect to the finalised
	SRS with respect to which $\adv$ is allowed to make simulation queries.
	
	\ngame{1} This is identical to $\game{0}$ except that now the game is aborted if
	there is $(\inp_\advse, \zkproof_{\simulator}) \in Q$ such that
	$\zkproof_\simulator[1..k] = \zkproof_{\advse}[1..k]$. That is, the adversary in
	its final proof reuses at least $k$ messages from a simulated proof, and the proof
	is accepting.  Denote this event by $\event{\errur}$.
	
	\ncase{Game 0 to Game 1} We have,
	\( \prob{\game{0} \land \nevent{\errur}} = \prob{\game{1} \land \nevent{\errur}} \)
	and, from the difference lemma, cf.~\cref{lem:difference_lemma},
	$ \abs{\prob{\game{0}} - \prob{\game{1}}} \leq \prob{\event{\errur}}\,$.  Thus, to
	show that the transition from one game to another introduces only minor change in
	probability of $\adv$ winning it should be shown that $\prob{\event{\errur}}$ is
	small.
	
	We can assume that $\adv$ queried the simulator on the instance it wishes to
	output, i.e.~$\inp_{\adv}$. We show a reduction $\rdvur$ that utilises $\adv$ to
	break the $\ur{k}$ property of $\ps$. That is, for randomness $r = (r_\adv, \rsim)$,
	$\rdvur (r)$ runs $\advse^{\initU, \simulator_1, \simulator_2'}(\secparam; r_\adv)$ internally as a black-box:
	\begin{compactitem}
		\item The reduction answers $\adv$'s update queries by asking the same query from the
		update oracle in the unique response experiment. The reduction finalises the same
		SRS $\srs$ as the one $\adv$ does.
		\item The reduction answers both queries to the simulator $\simulator$
		and to the random oracle using $\rsim$. It also keeps lists $Q$, for the simulated proofs,
		and $Q_\ro$ for the random oracle queries.
		\item When $\adv$ makes a fake proof $\zkproof_{\adv}$ for $\inp_{\adv}$, $\rdvur$
		looks through lists $Q$ and $Q_\ro$ until it finds $\zkproof_{\simulator}[0..k]$
		such that $\zkproof_{\adv}[0..k] = \zkproof_{\simulator}[0..k]$ and a random
		oracle query $\zkproof_{\simulator}[k].\ch$ on $\zkproof_{\simulator}[0..k]$.
		\item $\rdvur$ returns two proofs for $\inp_{\adv}$:
		\begin{align*}
		\zkproof_1 = (\zkproof_{\simulator}[1..k],
		\zkproof_{\simulator}[k].\ch, \zkproof_{\simulator}[k + 1..\mu + 1])\\
		\zkproof_2 = (\zkproof_{\simulator}[1..k],
		\zkproof_{\simulator}[k].\ch, \zkproof_{\adv}[k + 1..\mu + 1])
		\end{align*}
	\end{compactitem}
	If $\zkproof_1 = \zkproof_2$, then $\adv$ fails to break simulation soundness, as
	$\zkproof_2 \in Q$. On the other hand, if the proofs are not equal, then $\rdvur$
	breaks $\ur{k}$-ness of $\ps$. This happens only with negligible probability
	$\epsur(\secpar)$, hence \( \prob{\event{\errur}} \leq \epsur(\secpar)\,. \)
	
	\COMMENT{We have,
		\( \prob{\game{0} \land \nevent{\errur}} = \prob{\game{1} \land
			\nevent{\errur}} \) and, from the difference lemma,
		cf.~\cref{lem:difference_lemma},
		\[ \abs{\prob{\game{0}} - \prob{\game{1}}} \leq \prob{\event{\errur}}\,. \]
		Thus, to show that the transition from one game to another introduces only
		minor change in probability of $\advse$ winning it should be shown that
		$\prob{\event{\errur}}$ is small.
		
		We can assume that $\advse$ queried the simulator on the instance it wishes to
		output---$\inp_{\advse}$. We show a reduction $\rdvur$ that utilises $\advse$,
		who outputs a valid proof for $\inp_{\advse}$, to break the $\ur{k}$ property of
		$\ps$. Let $\rdvur$ run $\advse$ internally as a black-box:
		\begin{itemize}
			\item The reduction answers both queries to the simulator $\psfs.\simulator$ and to the random oracle. 
			It also keeps lists $Q$, for the simulated proofs, and $Q_\ro$ for the random oracle queries. 
			\item When $\advse$ makes a fake proof $\zkproof_{\advse}$ for $\inp_{\advse}$,
			$\rdvur$ looks through lists $Q$ and $Q_\ro$ until it finds
			$\zkproof_{\simulator}[0..k]$ such that
			$\zkproof_{\advse}[0..k] = \zkproof_{\simulator}[0..k]$
			and a random oracle query $\zkproof_{\simulator}[k].\ch$ on
			$\zkproof_{\simulator}[0..k]$.
			\item $\rdvur$ returns two proofs for $\inp_{\advse}$:
			\begin{align*}
			\zkproof_1 = (\zkproof_{\simulator}[1..k],
			\zkproof_{\simulator}[k].\ch, \zkproof_{\simulator}[k + 1..\mu + 1])\\
			\zkproof_2 = (\zkproof_{\simulator}[1..k],
			\zkproof_{\simulator}[k].\ch, \zkproof_{\advse}[k + 1..\mu + 1])
			\end{align*}
		\end{itemize}  
		If $\zkproof_1 = \zkproof_2$, then $\advse$ fails to break simulation
		extractability, as $\zkproof_2 \in Q$. On the other hand, if the proofs are
		not equal, then $\rdvur$ breaks $\ur{k}$-ness of $\ps$. This happens only with
		negligible probability $\epsur(\secpar)$, hence \( \prob{\event{\errur}} \leq
		\epsur(\secpar)\,. \)
	}
	%
	\ngame{2} Define an adversary $\bdv$ against forking special soundness such that,
	given access to oracles $\initU$ and $\ro$, and randomness $r = (r_\adv, \rsim)$,
	internally runs $\advse^{\initU, \simulator_1, \simulator_2'} (1^\secpar; r_\adv)$,
	where
	\begin{compactenum}
		\item $\bdv$ answers $\adv$ random oracle and update queries by passing the queries to the real
		oracles $\ro$ and $\initU$. When $\adv$ finalises an SRS $\srs$, $\bdv$ does the same.
		\item $\bdv$ answers simulator queries by using coins $\rsim$. $\bdv$ maintains a
		list of instance-proof pairs $Q$ consisting of all simulation queries made by
		$\adv$, and corresponding responses.
		\item Eventually when $\adv$ outputs $(\inp_\advse, \zkproof_\advse)$, $\bdv$ outputs
		the same $(\inp_\advse, \zkproof_\advse)$.
	\end{compactenum}
	% \michals{7.10}{Need forking soundness in the updatable setting. Alternatively we
	%   could show that we can move a proof from one SRS to another using partial trapdoor}
	
	In this game the environment additionally aborts if extractor $\ext$ fails to build a
	$(1, \ldots, 1, n, 1, \ldots, 1)$-tree of accepting transcripts $\tree$ by rewinding
	$\bdv$.
	
	$\ext$ proceeds as follows. First, it takes as input the final SRS $\srs$,
	$\bdv$'s code, its randomness $r$, the output instance $\inp_{\advse}$ and proof
	$\zkproof_{\advse}$, and the list of random oracle queries and responses
	$Q_\ro$. Then, $\ext$ starts a forking algorithm
	$\genforking^{n}_\zdv(y,h_1, \ldots, h_q)$ for
	$y = (\srs, \bdv, r, \inp_{\advse}, \zkproof_{\advse})$ where we set
	$h_1, \ldots, h_q$ to be the consecutive queries from list $Q_\ro$. We run $\bdv$
	internally in $\zdv$. \michals{3.12}{Note we will have a bit different forking
		algorithm here -- as described in Attema et al. Bottom line is -- the extractor in
		Attema et al.~rewinds the adversary $k + q (k + 1)$ times (expected number of
		rewinds). Since the number of rewinds is larger, probability of success is much
		higher. That is, they don't have our exponential security loss.}
	
	To assure that in the first execution of $\zdv$ the adversary $\bdv$ produces the
	same $(\inp_{\advse}, \zkproof_{\advse})$ as in the extraction game, $\zdv$ provides
	$\bdv$ with the same randomness $r$ and answers queries to the random oracle with
	pre-recorded responses in $Q_\ro$.
	%
	Note, that since the view of the adversary when run inside $\zdv$ is the same as its
	view with access to the real random oracle, it produces exactly the same
	output. After the first run, $\zdv$ outputs the index $i$ of a random oracle query
	that was used by $\bdv$ to compute the challenge
	$\zkproof[k''].\ch = \ro(\zkproof_{\advse}[0..k''])$ it had to answer as
	$(k'' + 1)$-th message and adversary's transcript, denoted by $s_1$ in
	$\genforking$'s description. If no such query took place or the proof output by
	$\advse$ is unacceptable, $\zdv$ outputs $i = 0$. In such case we state that $\zdv$
	was not successful.
	
	Then, new random oracle responses are picked for queries indexed by $i, \ldots, q$
	and the adversary is rewound to the point just prior to when it gets the response to
	random oracle query $\zkproof_{\advse}[0..k'']$. The adversary gets a random oracle response from
	a new set of responses $h^2_i, \ldots, h^2_q$. If the adversary requests a simulated
	proof after seeing $h^2_i$, then $\zdv$ computes the simulated proof on its
	own. Eventually, $\zdv$ outputs index $i'$ of a query that was used by the adversary
	to compute $\ro(\zkproof_{\advse}[0..k''])$, and a new transcript $s_2$. $\zdv$ is run
	$n$ times with different random oracle responses.  Eventually, if all runs of $\zdv$
	were successful, then tree of acceptable transcripts $\tree$ is built.
	% If a tree
	% $\tree$ of $n$ transcripts is built, then $\ext$ internally runs the tree extractor
	% $\extt(\tree)$ and outputs what it returns.
	
	We emphasize here the importance of the unique response property. If it does not hold
	then in some $j$-th execution of $\zdv$ the adversary $\adv$ (run internally in
	$\bdv$) could reuse a challenge that it learned from observing proofs in $Q$. In that
	case, $\bdv$ would output a proof that would make $\zdv$ output $i = 0$, making the
	extractor fail. Fortunately, the case that the adversary breaks the unique response
	property has already been covered by the abort condition in $\game{1}$.
	
	Denote by $\waccProb$ the probability that $\advse$ outputs a proof that is accepted
	and does not break $\ur{k}$-ness of $\ps$. With the same probability, an accepting
	proof is returned by $\bdv$. This comes since the proof system is
	$k'$-programmable. That is, $\bdv$ when providing $\adv$ with simulated proofs
	programs the random oracle only in moves from $k'$ on or, to put that differently,
	it does not program the random oracle for moves $1$ to $k' - 1$.  Hence, $\adv$
	outputs proof that does not contain any programmed random oracle output and $\bdv$
	can output that proof as its own. This holds since (1) for messages $1$ to $k' - 1$,
	the simulator does not program the oracle, (2) $\adv$ cannot use a part of a
	simulated proof from move $k'$ because of the unique response property.
	
	\ncase{Game 1 to Game 2} Note that for every accepting proof $\zkproof_{\advse}$, we
	may assume that whenever $\advse$ outputs a $k''$-th move message
	$\zkproof_{\advse}[k'']$, then the $(\inp_{\advse}, \zkproof_{\advse}[1..k''])$
	random oracle query was made by the adversary, not the
	simulator\footnote{\cite{INDOCRYPT:FKMV12} calls these queries \emph{fresh}.},
	i.e.~there is no simulated proof $\zkproof_\simulator$ on $\inp_\simulator$ such that
	$(\inp_{\advse}, \zkproof_{\advse} [1..k'']) = (\inp_\simulator,
	\zkproof_\simulator[1..k''])$. Otherwise, the game would be already interrupted by
	the error event in Game $\game{1}$.  As previously,
	\( \abs{\prob{\game{1}} - \prob{\game{2}}} \leq \prob{\event{\errfrk}}\,.  \)
	
	Denote by $\waccProb'$ the probability that algorithm $\zdv$, defined in the general
	forking lemma, produces an accepting proof with a fresh challenge after move
	$k''$. From the above argument, we have that $\waccProb = \waccProb'$.
	
	Next, from the generalised forking lemma, cf.~\cref{lem:generalised_forking_lemma},
	we get that
	\changedm{
		\begin{equation}
		\begin{split}
		\prob{\event{\errfrk}} \leq 1 - \frac{\waccProb - (q + 1) \epsk(\secpar)} {1 -
			\epsk(\secpar)}.
		\end{split}
		\end{equation}
	}
	
	
	\ngame{3} This game is identical to $\game{2}$ except that the extractor $\ext$,
	given tree of acceptable transcripts $\tree$ runs additionally $\extt (\tree)$ to
	learn the witness $\wit$. The environment additionally aborts this game if
	$\REL(\inp, \wit)$ does not hold.
	
	\ncase{Game 2 to Game 3}	
	Since $\ps$ is forking special sound the probability that $\extt(\tree)$
	fails is upper-bounded by $\epsss(\secpar)$.
	
	\ncase{Conclusion} Since Game $\game{3}$ is aborted when it is impossible to
	extract a witness from $\tree$ and $\bdv$ only passes proofs produced by $\adv$,
	the adversary $\advse$ cannot win. Thus, by the game-hopping argument,
	\[
	\abs{\prob{\game{0}} - \prob{\game{3}}} \leq 1 - \changedm{\frac{\waccProb - (q
			+ 1) \epsk(\secpar)}{1 - \epsk(\secpar)}} + \epsur(\secpar) +
	%q_{\ro}^{\mu} \epss +
	\epsss(\secpar)\,.
	\]
	Thus the probability that extractor $\extss$ succeeds is at least
	\[
	\changedm{\frac{\waccProb - (q + 1) \epsk(\secpar)}{1 - \epsk(\secpar)}} -
	\epsur(\secpar)
	%- q_{\ro}^{\mu} \epss
	- \epsss(\secpar)\,.
	\]
	Since $\waccProb$ is the probability of $\advse$ producing an accepting transcript
	that does not break $\ur{k}$-ness of $\ps$, then $\waccProb \geq \accProb -
	\epsur(\secpar)$, where $\accProb$ is the probability of $\advse$ outputting an accepting
	proof as defined in \cref{def:updsimext}. Thus, 
	\begin{equation}
	\label{eq:frk}
	\extProb \geq \changedm{\frac{\accProb - \epsur(\secpar) - (q + 1) \epsk(\secpar)}{1 - \epsk(\secpar)}} 
	- \epsur(\secpar) - \epsss(\secpar)\,.
	\end{equation}
	Hence $\psfs$ is \COMMENT{forking }simulation extractable with extraction error \hl{...}. 
	\qed
\end{proof}


%\hamid{7.3}{This should be rephrased or removed:}\michals{21.4}{Check now} 

% We conjecture that based on the recent results on state restoration soundness~\cite{C:GhoTes21}, which effectively allows to query the verifier multiple times on different overlapping transcripts, the $q$ loss could be avoided. However, this would reduce the class of protocols covered by our results. More precisely, we would require that the input interactive protocols are state-restoration sound \cite{TCC:BenChiSpo16}.

%%% Local Variables:
%%% mode: latex
%%% TeX-master: "main"
%%% End:
