% !TeX spellcheck = en_GB
\let\accentvec\vec
\documentclass[runningheads,11pt]{llncs}
 % \documentclass[runningheads]{amsart}
\let\spvec\vec
\let\vec\accentvec
\usepackage{amssymb,amsmath}
\let\vec\spvec

\usepackage[T1]{fontenc}

\newcommand{\iflipics}[1] {}
\newcommand{\iflncs}[1] {#1}

\def\vec#1{\mathchoice{\mbox{\boldmath$\displaystyle#1$}}
	{\mbox{\boldmath$\textstyle#1$}}
	{\mbox{\boldmath$\scriptstyle#1$}}
	{\mbox{\boldmath$\scriptscriptstyle#1$}}}

\DeclareFontFamily{U}{mathx}{\hyphenchar\font45}
\DeclareFontShape{U}{mathx}{m}{n}{<-> mathx10}{}
\DeclareSymbolFont{mathx}{U}{mathx}{m}{n}
\DeclareMathAccent{\widebar}{0}{mathx}{"73}

% lncs size (as printed in books, with small margins):
% \usepackage[paperheight=23.5cm,paperwidth=15.5cm,text={13.2cm,20.3cm},centering]{geometry}


% \newcommand{\ifamsart}[1] {}
% \ifamsart{
% 	\newtheorem{theorem}{Theorem}%[section]
% 	\newtheorem{proposition}[theorem]{Proposition}
% 	\newtheorem{lemma}[theorem]{Lemma}
% 	\newtheorem{corollary}[theorem]{Corollary}
% 	\theoremstyle{definition}
% 	\newtheorem{definition}[theorem]{Definition}
% 	\newtheorem{example}[theorem]{Example}
% }
\usepackage{fullpage}
\usepackage{soul}
\usepackage{soulutf8}
\soulregister\cite7
\soulregister\ref7
\soulregister\pageref7
\usepackage{hyperref}
\usepackage[color=yellow]{todonotes}
\hypersetup{final}
\usepackage{mathrsfs}
\usepackage[advantage,asymptotics,adversary,sets,keys,ff,lambda,primitives,events,operators,probability,logic,mm,complexity]{cryptocode}
\usepackage[capitalise]{cleveref}
\usepackage{cite}
\usepackage{booktabs}
\usepackage{paralist}
\usepackage[innerleftmargin=5pt,innerrightmargin=5pt]{mdframed}

% \usepackage{fouriernc}

\newcommand{\newdefs}[1] {\setlength{\fboxsep}{1pt}\colorbox{gray!20}{\(#1\)}}

\newcommand{\COMMENT}[1]  {}

%general formatting
\newcommand{\pcvarstyle}[1]{\mathsf{#1}}
\newcommand{\comment}[1]{{\color{lightgray}#1}}
\newcommand{\continue}{{\Huge{\hl{$\cdots$}}}}

% General mathematics
\newcommand{\range}[2] {[#1 \, .. \, #2]}
\newcommand{\SD}{\Delta}
\newcommand{\smallset}[1] {\{#1\}}
\newcommand{\bigset}[1] {\left\{#1\right\}}
\newcommand{\GRP} {\mathbb{G}}
\newcommand{\pair} {\hat{e}}
\newcommand{\brak}[1] {\left(#1\right)}
\newcommand{\sbrak}[1] {(#1)}
\newcommand{\alg}[1] {\pcalgostyle{#1}}
\newcommand{\image} {\operatorname{im}}
\newcommand{\myland} {\,\land\,}
\newcommand{\mylor} {\,\lor\,}
\newcommand{\vect}[1] {\operatorname{vect}(#1)}
\newcommand{\w}{\omega}
\newcommand{\const}{\pcpolynomialstyle{const}}
\newcommand{\p}[1]{\pcpolynomialstyle{#1}}
\newcommand{\ev}[1]{\tilde{\pcpolynomialstyle{#1}}}
\newcommand{\numberofconstrains}{\pcvarstyle{n}}
\newcommand{\expected}[1]{\mathbb{E}\left[#1\right]}
\newcommand{\infrac}[2]{#1 / #2}

% bilinear maps

\newcommand{\bmap}[2] {\left[#1\right]_{#2}}
\newcommand{\gone}[1] {\bmap{#1}{1}}
\newcommand{\gtwo}[1] {\bmap{#1}{2}}
\newcommand{\gi} {\iota}
\newcommand{\gtar}[1] {\bmap{#1}{T}}
\newcommand{\grpgi}[1] {\bmap{#1}{\gi}}


% zero knowledge
\newcommand{\oracleo}{\mathsf{O}}
\newcommand{\crs}{\pcvarstyle{crs}}
\newcommand{\td}{\pcvarstyle{td}}
\newcommand{\ip}[2]{\left\langle #1, #2\right\rangle}
\newcommand{\zkproof}{\pi}
\newcommand{\proofsystem}{\mathrm{\Psi}}
\newcommand{\ps}{\proofsystem}
\newcommand{\nuppt}{\pcmachinemodelstyle{NUPPT}}
\newcommand{\ro}{\mathcal{H}}
\newcommand{\rof}[2]{\mathbf{\Omega}_{#1, #2}}
\newcommand{\trans}{\pcvarstyle{trans}}
\newcommand{\tr}{\pcvarstyle{tr}}
\newcommand{\instsize}{\pcvarstyle{n}}
\newcommand{\KG} {\mathsf{K}}
\newcommand{\kcrs} {\KG_{\crs}}
\renewcommand{\dist}{\ddv}
\newcommand{\fs}{\pcalgostyle{FS}}
\newcommand{\sigmaprot}{\pcalgostyle{\Sigma}}
\newcommand{\se}{\pcvarstyle{se}}
\newcommand{\snd}{\pcvarstyle{snd}}
\newcommand{\zk}{\pcvarstyle{zk}}
\newcommand{\advse}{\adv_\se}

%rewinding---tree of transcripts
\newcommand{\pcboolstyle}[1]{\mathtt{#1}}
\newcommand{\treebuild}{\pcalgostyle{TreeBuild}}
\newcommand{\tree}{\pcvarstyle{tree}}
\newcommand{\counter}{\pcvarstyle{counter}}


%PLONK related
\newcommand{\plonkprot}{\mathbf{P}}
\newcommand{\plonkprotfs}{\mathbf{P}_\fs}
\newcommand{\selector}[1]{\pcvarstyle{q_{#1}}}
\newcommand{\selmulti}{\selector{M}}
\newcommand{\selleft}{\selector{L}}
\newcommand{\selright}{\selector{R}}
\newcommand{\seloutput}{\selector{O}}
\newcommand{\selconst}{\selector{C}}
\newcommand{\chz}{\mathfrak{z}}
\newcommand{\reduction}{\rdv}

\newcommand{\game}[1]{\pcalgostyle{G}_{#1}}

\newcommand{\lag}{\p{L}}
\newcommand{\pubinppoly}{\p{PI}}

% general complexity theory
% \newcommand{\RND}[1]{\pcalgostyle{RND}(#1)}
\newcommand{\RND}[1]{\pcvarstyle{R}(#1)}
\newcommand{\RELGEN}{\mathcal{R}}
\newcommand{\REL}{\mathbf{R}}
\newcommand{\LANG}{\mathcal{L}}
\newcommand{\inp}{\pcvarstyle{x}}
\newcommand{\wit}{\pcvarstyle{w}}
\newcommand{\class}[1]{\mathfrak{#1}}
\newcommand{\ig}{\pcalgostyle{IG}}
\newcommand{\accProb}{\event{acc}}
\newcommand{\frkProb}{\event{frk}}
\newcommand{\FS}{\pcalgostyle{FS}} % Fiat-Shamir transform
\newcommand{\aux}{\pcvarstyle{aux}} %auxiliary input

%Plonk and Sonic
\newcommand{\plonk}{\ensuremath{\textsc{Plonk}}}
\newcommand{\plonkmod}{\ensuremath{\plonk^\star}}
\newcommand{\plonkint}{\ensuremath{\plonk^\star}}
\newcommand{\polyprot}{\pcalgostyle{poly}}
\newcommand{\plonkintpoly}{\plonkint_\polyprot}
\newcommand{\sonic}{\textsc{Sonic}}
\newcommand{\maxdegree}{\pcvarstyle{N}}

\newcommand{\dlog}{\pcvarstyle{dlog}}

\newcommand{\ur}[1]{{#1\text{-}\mathsf{ur}}}

%forking
\newcommand{\forking}{\pcalgostyle{F}}
\newcommand{\genforking}{\pcalgostyle{GF}}

%colors
\definecolor{darkmagenta}{rgb}{0.5,0,0.5}
\definecolor{lightmagenta}{rgb}{1,0.85,1}
\definecolor{lightmagenta}{rgb}{0.9,0.9,0.9}
\definecolor{darkred}{rgb}{0.7,0,0}
\definecolor{blueish}{rgb}{0.1,0.1,0.5}
\definecolor{pinkish}{rgb}{0.9,0.8,0.8}
\definecolor{darkgreen}{rgb}{0,0.6,0}
\definecolor{lightgreen}{rgb}{0.85,1,0.85}
\definecolor{skyblue}{rgb}{0.3,0.9,0.99}

%comments
\DeclareRobustCommand{\markulf}[2]  {{\color{darkmagenta}\hl{\scriptsize\textsf{Markulf #1:} #2}}}
\DeclareRobustCommand{\michals}[2]  {{\color{blueish}\sethlcolor{pinkish}\hl{\scriptsize\textsf{Michal #1:} #2}}}
\newcommand{\task}[2]{\todo[author=\textbf{Task},inline]{({\textit{#1}}) #2}}
% \newcommand{\task}[2] {\xcommenti{Task}{#1}{#2}}
% \DeclareRobustCommand{\task}[2]  {{\color{black}\sethlcolor{yellow}\hl{\textsf{TASK #1:} #2}}}

%%% Local Variables:
%%% mode: latex
%%% TeX-master: "plonkext"
%%% End:


\title{On Simulation-Extractability of \textsc{Plonk}, abstract}

\author{Markulf Kohlweiss\inst{1,2} \and Michał Zając\inst{3}}
\iflncs{
  \institute{University of Edinburgh, Edinburgh, UK \and IOHK \\ \email{mkohlwei@inf.ed.ac.uk} \and Clearmatics, London UK \\ \email{m.p.zajac@gmail.com}}
}

\begin{document}
	\maketitle
	
	\subsubsection*{On the importance of simulation extractability.}
	Simulation extractability (SE) is a strong version of protocol's soundness.
	
	\subsubsection*{$\plonk$ as an efficient universal zkSNARK.}
	\michals{31.08}{importance of plonk}
	
	\subsubsection*{Preliminaries.}
	\michals{31.08}{Put a (informal?) definition of SE, unique resp., dlog assumption here}
	Before we continue we introduce definitions crucial to understand the following result. 
	\begin{description}
		\item[Algebraic Group Model] is a computational model introduced in \cite{C:FucKilLos18}, which assumes that each algorithm $\adv$ that takes a vector of group elements $\vec{g}$ and returns a vector of $n$ group elements $\vec{h}$ outputs also a matrix $\vec{A} = (\vec{a_1}, \ldots, \vec{a_n})$ such that $h_i = \vec{a_i}^\top \vec{g}$, for $i \in \range{1}{n}$.
		\item[zkSNARK] is a non-interactive zero-knowledge proof, which proof length is sublinear to the statement and witness size. Due to impossibility result of \cite{STOC:GenWic11}, zkSNARKs can be sound only under some non-falsifiable assumption, like the Algebraic Group Model.
		\item[Simulation extractability] is a strong notion of soundness, which states that for every adversary there exists an extractor able to extract a valid witness from a proven statement given an acceptable proof for that statement. This property holds even if the adversary has access to a simulator oracle that provides it simulated proofs for required statements. 
		\item[Unique response property] states that after some round (here, Round 3) prover's responses to verifier's challenges are deterministic. That is, no PPT adversary can produce two valid transcripts that have the same challenges yet different answers (after Round 3).
		\item[$q$-dlog assumption] states that it is infeasibly hard for a PPT adversary to output $\chi$ from $1, g^{\chi}, \ldots, g^{\chi^q}, h^{\chi}$, where $g \in \GRP_1$ and $h \in \GRP_2$.
	\end{description}
	
	\subsubsection*{Simulation-extractability of $\plonk$.}
	In the paper we show that $\plonk$ is (non black-box) simulation-extractable. Importantly, the security proof we show does not require any modification to the proof system. 
	Hence, there is no computational overhead compared to other works that took an existing zkSNARK and apply some existing transformation which inevitable lead to efficiency loss---either because the the final proof is longer \cite{EPRINT:BowGab18}, the prover has to prove a more complicated statement (\michals{31.08}{cite OR constructions}\cite{ASIACCS:DerSla18,EPRINT:AbdRamSla20}), or, in the context of showing correctness of computation, the arithmetisation of choice is less efficient \cite{C:GroMal17}.
	
	We were able to show simulation extractability of $\plonk$ by providing three reductions, each of which able to break some assumption if the proof system has not the desired security property. 
	More precisely, we show that if there exists a simulation extractability-breaking adversary $\advse$ then one can either break: 
	\begin{compactenum}
		\item \label{it:ks} knowledge soundness of the interactive protocol underlying $\plonk$; 
		\item \label{it:dlog} unique response property (after $3$-rd round) of $\plonk$; or 
		\item \label{it:ur} an $(\numberofconstrains + 2)$ discrete log assumption, for $\numberofconstrains$ being a number of constrain in the proven statement.
	\end{compactenum}
	
	\medskip
	We believe, although it is for the further work, that the result can be generalised to other protocols which relies on Fiat--Shamir transform and are knowledge sound in the Algebraic Group Model.
	
	\subsubsection*{Simulation-extractability of special-sound protocols. Generalised forking lemma}
	Independently on the previous result, we also show that simulation-extractability can 
	
	\bibliographystyle{abbrv}
	\bibliography{cryptobib/abbrev1,cryptobib/crypto,additional_bib}
\end{document}

