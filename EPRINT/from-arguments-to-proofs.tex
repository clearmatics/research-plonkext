% !TEX root = main.tex
% !TEX spellcheck = en-US

\section{Making Attema et al work in our setting}
\newcommand{\epsa}{\eps_{\mathsf{attema}}}

\subsection{Extractability of polynomial commitments}
\michals{27.01}{In this section we want to show how one could extract polynomials from a
  PCOM}

\begin{lemma}
  Let $\PCOM$ be a polynomial commitment scheme and $\proofsystem$ a $(2 \mu + 1)$-message
  proof system where the adversary $\adv$ sends polynomial commitments using $\PCOM$ for
  polynomials that are evaluated in the run of the protocol. Denote by $\accProb$
  probability that the adversary outputs an acceptable proof. Then there exists an
  extractor $\ext$ that for each commitment $c$ provided by $\adv$ outputs polynomial $f$
  such that $c = \com (f)$ with probability at least \hl{...}
\end{lemma}

\begin{proof}
  Denote by $k$ the number of polynomials that are committed during a run of
  $\proofsystem$ and denote by $d$ the maximum degree of the polynomials. We assume that
  each polynomial is evaluated at least at one point during the protocol.

  The extractor proceeds as follows. It gets as input the polynomial commitment SRS
  $\srs_{pcom}$ and build a proof system SRS $\srs$; then it runs internally $\adv (\srs)$
  and plays the role of the verifier. Each time the adversary outputs a commitment $c$, it
  continues the protocol, randomly picks the evaluation points $x$, adds $x$ to initially
  empty list $X$, and learns evaluation of $c$ and the corresponding opening. Then it
  rewinds the adversary to the point just before $x$ was presented to the adversary and
  picks a new challenge $x' \not \in X$.

  This procedure is repeated till the adversary learns $d + 1$ evaluations of $c$, what
  allows it to re-create the underlying polynomial $f$. Expected number of runs the
  adversary makes is 
\end{proof}

\subsection{From arguments to proofs}
\michals{25.01}{Here I describe how to make Attema et al result ``for proofs'' usable in
  our ``arguments''}
%
\michals{26.1}{Should we give adversary access to the simulator oracle? Probably not, as
  that would force us to write the whole SE proof here. Let's focus on a simpler case and
  introduce the simulator later on
}
%
\michals{26.1}{Should we show this result for every proof system separately?}
%
Let $\proofsystem'$ be a statistically sound idealised proof system and $\proofsystem$ a
computationally sound proof system (i.e.~an argument) obtained from $\proofsystem'$. We
show how to build a knowledge extractor for $\proofsystem$ using extractor for $\proofsystem'$.

\begin{lemma}
  Let $\proofsystem'$ be a knowledge sound proof system with knowledge error
  $\epsid$. Let $\proofsystem$ be a computationally sound proof system \changedm{compiled
    from $\proofsystem'$ with security loss $\eps$}. Let $\adv'$ be a $\proofsystem'$
  adversary that outputs an acceptable proof with probability $\accProb'$, then \hl{...}  
\end{lemma}


\ncase{Game 1} %
The $\proofsystem$ adversary $\adv$ is a PPT machine that gets as input an SRS $\srs$ --
that encodes relation $\REL$ --, oracle access to a random oracle $\ro$, and outputs an
instance--proof pair $(\inp, \zkproof)$. We denote by $\accProb$ probability that
$\verifier (\srs, \inp, \zkproof)$ accepts the proof. We build an extractor $\ext$ that
rewinds $\adv$ expected number of $N$ times each time providing it with a modified set of
random oracle responses $Q_i$ and builds a tree $\tree$ of acceptable transcripts.
Eventually the extractor outputs $\wit$ such that $\REL(\inp, \wit) = 1$ with
\changedm{non-negligible} probability.  Adversary $\adv$ wins whenever $\ext$ fails to
output the witness.

\ncase{Game $2.i$} %
This game is identical to Game 1, except now the environment aborts if the $i$-th proof
submitted by the adversary $\adv$ in not acceptable by the idealized verifier
$\verifier'$.

\michals{26.1}{Add more explanation of what the above means.}

\ncase{Game $1$ to Game $2.i$} %
\michals{26.1}{Here we are using the reduction known from, e.g.~plonk -- if the adversary
manages to provide a proof acceptable by $\verifier$, but not $\verifier'$ then we can
break the dlog assumption. This happens with probability at most $\epsdlog$.
}

\ncase{Game $3.i$}
\michals{26.1} {(this game was previously a part of Game 2, maybe that was a good idea.)}
  We now introduce $\proofsystem'$ adversary $\adv'$ that internally runs $\adv$ (provides
  it with the SRS, passes $\verifier'$ challenges as $\verifier$ challenges) and whenever
  $\adv$ outputs a polynomial commitment it extracts the underlying polynomial and sends
  it to $\prover'$.

\ncase{Game $2.i$ to Game $3.i$} %
The environment aborts Game $3.i$ when $\adv'$ fails to extract a polynomial from
a commitment submitted by $\adv$. This happens with probability at most $k \cdot \epsk$,
where $k$ is the number of polynomial commitments output by $\adv$ and $\epsk$ is a
knowledge error of polynomial commitment knowledge extractor.

Probability $\accProb'$ that $\adv'$ makes an acceptable proof is thus at least
$\accProb - \epsdlog - k \cdot \epsk$.

\ncase{Game $4$} %
\michals{26.1} { Here we run the Attema et al extractor on $\proofsystem'$ adversary
  $\adv'$. Probability that this extractor fails is upper-bounded by
  $\epsa (\secpar, \accProb')$.

  Note that $\epsa$ already includes probability that $\adv'$ breaks knowledge soundness
  of $\proofsystem'$, in our paper denoted by $\epsid$.  }

\ncase{Conclusion}
\michals{26.1}{Probability that $\adv$ outputs a valid proof that is acceptable by
  $\verifier$ but $\ext_\adv$ fails is upper-bounded by
  \[
    \epsa + N \cdot (\epsdlog + k \cdot \epsk).
  \]
  }


%%% Local Variables:
%%% mode: latex
%%% TeX-master: "main"
%%% End:
