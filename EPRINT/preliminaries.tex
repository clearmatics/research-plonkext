% !TEX root = main.tex
% !TEX spellcheck = en-US
%\section{Preliminaries}


\iffalse
\paragraph{Notation.} %Let $\ppt$ denote probabilistic polynomial-time and $\secpar \in \NN$ be the
%security parameter. %All adversaries are stateful. 
For a PPT algorithm $\adv$, 
%let
%$\image (\adv)$ be the image of $\adv$ (the set of valid outputs of $\adv$), 
let
$\RND{\adv}$ denote the set of random tapes of correct length for adversary $\adv$
(assuming the given value of security parameter $\secpar$), and let $r \sample \RND{\adv}$ denote
the random choice of tape $r$ from $\RND{\adv}$. We denote by $\negl$
($\poly$) an arbitrary negligible (resp.~polynomial) function.
%
%For probability ensembles $X = \smallset{X_\secpar}_\secpar$ and
%$Y = \smallset{Y_\secpar}_\secpar$, with distributions $X_\secpar, Y_\secpar$ that have
%\emph{statistical distance} $\SD(X_\secpar, Y_\secpar) = \epsilon(\secpar)$ 
%if
%$\sum_{a \in \supp{X_\secpar \cup Y_\secpar}} \abs{\prob{X_\secpar = a} -
%  \prob{Y_\secpar = a}} = \epsilon(\secpar)$. 
%we write $X \approx_\secpar Y$ if
%$\SD(X_\secpar, Y_\secpar) \leq \negl$. 
For functions $a(\secpar)$, $b(\secpar)$ and probability ensembles $X = \smallset{X_\secpar}_\secpar$,
$Y = \smallset{Y_\secpar}_\secpar$ we
write $a(\secpar) \approx_\secpar b(\secpar)$ if
$\abs{a(\secpar) - b(\secpar)} \leq \negl$ and $X \approx_\secpar Y$ if they have \emph{statistical distance} $\SD(X_\secpar, Y_\secpar) \leq \negl )$, respectively. \medskip
\fi
% For a probability space
% $(\samplespace, \eventspace, \probfunction)$ and event $\event{E} \in \eventspace$ we
% denote by $\nevent{E}$ an event that is complementary to $\event{E}$,
% i.e.~$\nevent{E} = \samplespace \setminus \event{E}$.

% \begin{lemma}[Difference lemma,~{\cite[Lemma 1]{EPRINT:Shoup04}}]
% 	\label{lem:difference_lemma}
% 	Let $\event{A}, \event{B}, \event{F}$ be events defined in some probability
% 	space, and suppose that $\event{A} \land \nevent{F} \iff \event{B}
% 		\land \nevent{F}$.  Then 
% 	$
% 		\abs{\prob{\event{A}} - \prob{\event{B}}} \leq \prob{\event{F}}\,.
% 	$
% \end{lemma}


\section{Definitions and Lemmas for Multi-message SRS-based Protocols}
\label{sec:se_definitions}
\label{sec:preliminaries}



\ourpar{Simulation-extractability for multi-message protocols.}
Most recent SNARK schemes follow the same blueprint of constructing an interactive information-theoretic proof system 
that is then compiled into a public coin computationally sound scheme using cryptographic tools such as polynomial commitments,
and finally made non-interactive via the Fiat--Shamir transformation.
Existing results on simulation extractability (for proof systems and
signatures of knowledge) for Fiat--Shamir transformed systems work for $3$-message protocols without reference string that
require two transcripts for standard model extraction, e.g.,
\cite{JC:PoiSte00,INDOCRYPT:FKMV12,C:RotSeg21}.

In this section, we define properties that are necessary for our
analysis of multi-message protocols with a universal updatable SRS.  In order to
prove simulation-extractability for such protocols, we require more than just two
transcripts for extraction. Moreover, in the updatable setting we consider protocols
that rely on an SRS where the adversary gets to contribute to the SRS. We first recall the updatable SRS setting and the Fiat-Shamir transform for $(2\mu+1)$ message protocols.
Next, we define trapdoor-less zero-knowledge and simulation-extractability
which we base on~\cite{INDOCRYPT:FKMV12} adapted to the updatable SRS setting. Then,
to support multi-message SRS-based protocols compiled using the Fiat--Shamir transform,
we generalize the unique response property, and define a notion of computational special
soundness called rewinding-based knowledge soundness.\medskip


%\subsection{Proof Systems With Updatable Setups and Random Oracles}
\noindent Let $\prover$ and $\verifier$ be $\ppt$ algorithms, the former called the \emph{prover}
and the latter the \emph{verifier} of a proof system. Both algorithms take a pre-agreed structured reference string $\srs$ as input. The structured reference strings we consider are (potentially) updatable, a notion we recall shortly.
%
We focus on proof systems made non-interactive via the multi-message Fiat--Shamir transform presented below where prover and
verifier are provided with a random oracle $\ro$. 
We denote by $\zkproof$ a proof created by $\prover$ on input
$(\srs, \inp, \wit)$. We say that proof is accepting if $\verifier(\srs, \inp,
\zkproof)$ accepts it.

Let
$\RND{\adv}$ denote the set of random tapes of correct length for adversary $\adv$
(assuming the given value of security parameter $\secpar$), and let $r \sample \RND{\adv}$ denote
the random choice of tape $r$ from $\RND{\adv}$.

%The simulator
%$\simulator$ is not only given access to $\ro$, but it can also \emph{program}
%it. That is, it can require that for $(x, y)$ of its choice, $\ro (x) = y$.


  
\iffalse
\subsection{Zero-Knowledge Proof Systems}\label{prelim:nizk}
In a zero-knowledge proof system, a prover convinces the verifier of the veracity of a statement
without leaking any other information. The zero-knowledge property is proven by constructing a
simulator that can simulate the view of a cheating verifier without knowing the secret
information -- witness -- of the prover. A proof system has to be sound as well, i.e.~for a
malicious prover it should be infeasible to convince a verifier of a false statement. Here, we
focus on proof systems, so-called arguments, that guarantee soundness against $\ppt$ malicious provers. 
\markulf{24.04}{Nice text, but maybe too basic? Also we don't really have views, but really reconstruct the non-interactive proof.}
Typically, a stronger notion of soundness is required -- besides requiring that the
verifier rejects proofs of statements outside the language, we request from the
prover to know a witness corresponding to the proven statement. This property is
called \emph{knowledge soundness}. In this work we investigate an even stronger notion of soundness, \emph{simulation extractability} for non-interactive protocols obtained from interactive arguments via the \emph{Fiat--Shamir heuristic} via random oracles in a setting with an \emph{updatable common reference string.} 

We now introduce these concepts.
\fi


\subsection{Updatable SRS Setup Ceremonies}\label{def:upd-scheme}

%Let $\prover$ and $\verifier$ be $\ppt$ algorithms, the former called \emph{prover}
%and the latter \emph{verifier}. We allow our proof system to have a setup, i.e.~there is a
%$\kgen$ algorithm that takes as input the relation description $\REL$ and outputs a common
%reference string $\srs$.

The definition of updatable SRS ceremonies of~\cite{C:GKMMM18} requires the following algorithms.

\begin{itemize} 
	\item
	$(\srs,\rho) \gets \kgen(\REL)$ is a PPT algorithm that takes a relation $\REL$ and outputs a reference string $\srs$, and correctness proof $\rho$.
	\item
	$ (\srs',\rho') \gets \upd(\srs, \{\rho_j \}_{j=1}^n)$ is a PPT algorithm that takes a $\srs$, a list of update proofs and outputs an updated $\srs'$ together with a proof of correct update~$\rho'$. 
	\item
	$b \gets \verifyCRS(\srs, \{\rho_j \}_{j=1}^n)$ takes a reference string $\srs$, a list of update proofs, and outputs a bit indicating acceptance or not.\footnote{For instance \plonk{} and \marlin{} will use the $\kgen$, $\upd$ and $\verifyCRS$ algorithms in~\cref{fig:upd-scheme}.}
\end{itemize}


In the next section, we define security notions in the updatable setting by giving the adversary access to an SRS update oracle $\initU$, defined in~\cref{fig:upd}. The oracle allows the adversary to control the SRS generation. A trusted setup can be expressed by the updatable setup definition simply by restricting the adversary to only call the oracle on $\intent = \setup$ and $\intent = \final$. Note that a soundness adversary now has access to both the random oracle $\ro$ and $\initU$:  $(\inp, \zkproof) \gets \adv^{\initU,\ro}(1^\secpar; r)$.

\newcommand*{\Scale}[2][4]{\scalebox{#1}{$#2$}}% 

\begin{figure}[t!]
	\centering
	\centerline{\fbox{
		\begin{minipage}[t]{1.15\linewidth}
			\begin{pchstack}
			\procedure{$\initU(\intent, \srs_n,\{\rho_j \}_{j=1}^n)$}{
				\pcif \srs \neq \bot: \pcreturn \bot \\
				\pcif (\intent = \setup): \\
				\t (\srs',\rho') \leftarrow \kgen(\REL)\\
				\t Q_\srs \gets Q_\srs \cup \{(\srs',\rho')\}\\
				\t \pcreturn (\srs',\rho')}
			%
			\procedure{}{
				\\
				\pcif (\intent = \update): \\
				\t b \gets \verifyCRS(\srs_n, \{\rho_j \}_{j=1}^{n})\\
				\t \pcif (b=0): \pcreturn \bot \\
				\t (\srs',\rho') \leftarrow \upd (\srs_n,\{\rho_j \}_{j=1}^n)\\
				\t Q_\srs \gets Q_\srs \cup \{(\srs',\rho')\}\\
		 	% \pccomment{$Q_\srs = (Q^{(1)}_\srs, Q^{(2)}_\srs) \text{ s.t. }   Q^{(2)}_\srs \text{ contains the update proofs in } Q_\srs$ } \\
				\t \pcreturn (\srs',\rho')}
				%
			\procedure{}{
				\\
				\pcif (\intent = \final): \\
				\t b \gets \verifyCRS(\srs_n,\{\rho_j \}_{j=1}^{n})\\
				\t \pcif (b=0) \vee Q^{(2)}_\srs \cap \{ \rho_j \}_i = \emptyset: \\
				\t \pcreturn \bot \\
				\t 
				\t \srs \gets \srs_n, \pcreturn \srs \\
				%
				\pcelse \pcreturn \bot
			}
			\end{pchstack}
		\end{minipage}
	}}
	\caption{The oracle defines the notion of updatable SRS setup.} 
		\label{fig:upd}
\end{figure}

\ourpar{Remark on universality of the SRS.} The proof systems we consider in this work are universal. This means that both the relation $\REL$ and the reference string $\srs$ allows to prove arithmetic constraints defined over a particular field up to some size bound. The public instance $\inp$ must determine the constraints. 
If $\REL$ comes with any auxiliary input, the latter is benign. 
We elide public preprocessing of constraint specific proving and verification keys. While important for performance, this modeling is not critical for security.

\subsection{Multi-message Fiat-Shamir Compiled Provers and Verifiers}
Given interactive prover and (public coin) verifier $\prover', \verifier'$ that exchange messages resulting in transcript $\tzkproof = (a_1, c_1, \ldots, a_{\mu}, c_{\mu}, a_{\mu + 1})$, where $a_i$ comes from
$\prover'$ and $c_i$ comes from $\verifier'$, the $(2\mu + 1)$-message Fiat-Shamir heuristic defines non-interactive provers and verifiers $\prover, \verifier$ as follows:

\begin{compactitem}
	\item $\prover$ behaves as $\prover'$ except after sending message
	  $a_i$, $i \in \range{1}{\mu}$, the prover does not wait for
	  the message from the verifier but computes it locally setting $c_i
	  = \ro(\tzkproof[0..i])$, where $\tzkproof[0..j] = (\inp, a_1, c_1, \ldots,
	  a_{j - 1}, c_{j - 1}, a_j)$.\footnote{For Fiat--Shamir based SoK the message signed $m$ is added to $\inp$ before hashing.} 
	  
	  $\prover$ output the non-interactive proof $\pi=(a_1,\ldots, a_{\mu}, a_{\mu + 1})$, that omits challenges as they can be recomputed using $\ro$.

	\item $\verifier$ takes $\inp$ and $\pi$ as input and behaves as $\verifier'$ would but does not provide
	  challenges to the prover. Instead it computes the
	  challenges locally as $\prover$ would, starting from $\tzkproof[0..1]=(\inp,a_1)$ which can be obtained from $\inp$ and $\pi$. Then it verifies the
	  resulting transcript $\tzkproof$ as the verifier $\verifier'$ would. 
	\end{compactitem}
	We note that since the verifier can compute the challenges by querying the random oracle, they do not need to be sent by the prover. Thus the $\zkproof$ - $\tzkproof$ notational distinction.
$(2\mu + 1)$-message FS-transformed NIZK proof system with an updatable SRS setup
\paragraph{Notation for $(2\mu + 1)$-message Fiat--Shamir transformed proof systems.}
Let $\SRScer= (\kgen,\upd, \verifyCRS)$ be the algorithm of an updatable SRS ceremony.
All our definitions and theorems are about non-interactive proof systems $\ps = (\SRScer, \prover, \verifier, \simulator)$ compiled via the $(2\mu + 1)$-message FS transform. 
%
That is $\pi = (a_1, \ldots, a_{\mu}, a_{\mu + 1})$ and $\tzkproof = (a_1, c_1, \ldots, a_{\mu}, c_{\mu}, a_{\mu + 1})$, with $c_i
= \ro(\tzkproof[0..i])$.
%
%\markulf{30.04}{This paragraph seems a bit of overkill, remove notation that we do not actually use}
We use $\tzkproof[0]$ for instance
$\inp$  and $\tzkproof[i]$, $\tzkproof[i].\ch$ to denote prover
message $a_i$ and challenge $c_i$ respectively. %$\tzkproof[i..j]$ denotes all messages of the transcript 
%between the $i$-th and $j$-th prover messages, but not $\tzkproof[j].\ch$.  

%If the proof system starts with a message from the verifier,
%we denote it by $\tzkproof[0].\ch$; else, we state that $\tzkproof[0].\ch$ is empty.

%%% Local Variables:
%%% mode: latex
%%% TeX-master: "main"
%%% End:
