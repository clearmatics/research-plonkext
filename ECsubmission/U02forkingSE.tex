% !TEX root = main.tex
% !TEX spellcheck = en-US
\section{Simulation soundness and forking simulation-extractability---the general result}
\label{sec:general}
Equipped with definitional framework of \cref{sec:se_definitions} we are ready
to present the main result of this paper---a proof of simulation soundness and
forking simulation extractability of Fiat-Shamir NIZK based on multi-round protocols.

The proofs go by game hopping. The games are controlled by an environment $\env$
that internally runs a simulation extractability adversary $\advse$, provides it
with access to a random oracle and simulator, and when necessary, rewinds it. The
games differ by various breaking points, i.e.~points where the environment
decides to abort the game.

Denote by $\zkproof_{\advse}, \zkproof_{\simulator}$ proofs returned by the
adversary and the simulator respectively. We use $\zkproof[i]$ to denote
prover's message in the $i$-th round of the proof (counting from 1),
i.e.~$(2i - 1)$-th message exchanged in the protocol. $\zkproof[i].\ch$ denotes
the challenge that is given to the prover after $\zkproof[i]$, and
$\zkproof[i..j]$ to denote all messages of the proof including challenges
between rounds $i$ and $j$, but not challenge $\zkproof[j].\ch$. When it is not
explicitly stated, we denote the proven instance $\inp$ by $\zkproof[0]$
(however, there is no following challenge $\zkproof[0].\ch$).

Without loss of generality, we assume that whenever the accepting proof contains
a response to a challenge from a random oracle, then the adversary queried the
oracle to get it. It is straightforward to transform any adversary that violates
this condition into an adversary that makes these additional queries to the
random oracle and wins with the same probability.

\begin{theorem}[Simulation soundness]
  \label{thm:simsnd}
  Assume that $\ps$ is $k$-programmable HVZK in the standard model, that is
  $\epss(\secpar)$-sound and $\ur{k}$ with security $\epsur(\secpar)$. Then, the
  probability that a $\ppt$ adversary $\adv$ breaks simulation soundness of
  $\ps_{\fs}$ is upper-bounded by
  \(
    \epsur(\secpar) + q_\ro^\mu  \epss(\secpar)\,,
  \)
  where $q$ is the total number of queries made by the adversary $\adv$ to a
  random oracle $\ro\colon \bin^{*} \to \bin^{\secpar}$.
\end{theorem}

\begin{proof}
  \ngame{0} This is a simulation soundness game played between an adversary
  $\adv$ who is given access to a random oracle $\ro$ and simulator
  $\psfs.\simulator$. $\adv$ wins if it manages to produce an accepting proof
  for a false statement. In the following game hops, we upper-bound the
  probability that this happens.

  \ngame{1} This is identical to $\game{0}$ except that the game is aborted if
  there is a simulated proof $\zkproof_\simulator$ for $\inp_{\adv}$ such that
  $(\inp_{\adv}, \zkproof_\simulator[1..k]) = (\inp_{\adv},
  \zkproof_{\adv}[1..k])$. That is, the adversary in its final proof reuses at
  least $k$ messages from a simulated proof it saw before and the proof is
  accepting.  Denote this event by $\event{\errur}$.

  \ncase{Game 0 to Game 1} We have, \( \prob{\game{0} \land
    \nevent{\errur}} = \prob{\game{1} \land \nevent{\errur}} \) and, from the
  difference lemma, cf.~\cref{lem:difference_lemma},
  $ \abs{\prob{\game{0}} - \prob{\game{1}}} \leq \prob{\event{\errur}}\,$.
  Thus, to show that the transition from one game to another introduces only
  minor change in probability of $\adv$ winning it should be shown that
  $\prob{\event{\errur}}$ is small.

  We can assume that $\adv$ queried the simulator on the instance it wishes to
  output, i.e.~$\inp_{\adv}$. We show a reduction $\rdvur$ that utilises $\adv$
  to break the $\ur{k}$ property of $\ps$. Let $\rdvur$ run $\advse$ internally
  as a black-box:
  \begin{compactitem}
	\item The reduction answers both queries to the simulator $\psfs.\simulator$
    and to the random oracle.  It also keeps lists $Q$, for the simulated
    proofs, and $Q_\ro$ for the random oracle queries.
  \item When $\adv$ makes a fake proof $\zkproof_{\adv}$ for $\inp_{\adv}$,
    $\rdvur$ looks through lists $Q$ and $Q_\ro$ until it finds
    $\zkproof_{\simulator}[0..k]$ such that
    $\zkproof_{\adv}[0..k] = \zkproof_{\simulator}[0..k]$ and a random oracle
    query $\zkproof_{\simulator}[k].\ch$ on $\zkproof_{\simulator}[0..k]$.
	\item $\rdvur$ returns two proofs for $\inp_{\adv}$:
	\begin{align*}
		\zkproof_1 = (\zkproof_{\simulator}[1..k],
		\zkproof_{\simulator}[k].\ch, \zkproof_{\simulator}[k + 1..\mu + 1])\\
		\zkproof_2 = (\zkproof_{\simulator}[1..k],
		\zkproof_{\simulator}[k].\ch, \zkproof_{\adv}[k + 1..\mu + 1])
	\end{align*}
	\end{compactitem}  
	If $\zkproof_1 = \zkproof_2$, then $\adv$ fails to break simulation soundness,
  as $\zkproof_2 \in Q$. On the other hand, if the proofs are not equal, then
  $\rdvur$ breaks $\ur{k}$-ness of $\ps$. This happens only with negligible
  probability $\epsur(\secpar)$, hence
  \( \prob{\event{\errur}} \leq \epsur(\secpar)\,. \)
	
  \ngame{2} This is identical to $\game{1}$ except that now the environment
  aborts if the instance the adversary proves is not in the language.

  \ncase{Game 1 to Game 2} 
  % REDUCTION TO INTERACTIVE SOUNDNESS:
  We show that
  $\abs{\prob{\game{1}} - \prob{\game{2}}} \leq q^{\mu} \cdot \epss(\secpar)$,
  where $\epss(\secpar)$ is the probability of breaking soundness of the underlying
  \emph{interactive} protocol $\ps$. Note that
  $\abs{\prob{\game{1}} - \prob{\game{2}}}$ is the probability that $\adv$
  outputs an acceptable proof for a false statement which does not break the
  unique response property (such proofs have been excluded by
  $\game{1}$). Consider a soundness adversary $\adv'$ who initiates a proof with
  $\ps$'s verifier $\ps.\verifier$, internally runs $\adv$ and proceeds as
  follows:
  \begin{compactitem}
  \item It guesses indices $i_1, \ldots, i_\mu$ such that random oracle queries
    $h_{i_1}, \ldots, h_{i_\mu}$ are the queries used in the $\zkproof_\adv$
    proof eventually output by $\adv$. This is done with probability at least
    $1/q^\mu$ (since there are $\mu$ challenges from the verifier in
    $\ps$).
  \item On input $h$ for the $i$-th,
    $i \not\in \smallset{{i_1}, \ldots, {i_\mu}}$, random oracle query, $\adv'$
    returns randomly picked $y$, sets $\ro(h) = y $ and stores $(h, y)$ in
    $Q_\ro$ if $h$ is sent to $\ro$ the first time. If that is not the case,
    $\adv$ finds $h$ in $Q_\ro$ and returns the corresponding $y$.
  \item On input $h_{i_j}$ for the $i_j$-th,
    $i_j \in \smallset{{i_1}, \ldots, {i_\mu}}$, random oracle query, $\adv'$
    parses $h_{i_j}$ as a partial proof transcript $\zkproof_\adv[1..j]$ and
    runs $\ps$ using $\zkproof_\adv[j]$ as a $\ps.\prover$'s $j$-th message to
    $\ps.\verifier$. The verifier responds with a challenge
    $\zkproof_\adv[j].\ch$. $\adv'$ sets $\ro(h_{i_j}) =
    \zkproof_\adv[j].\ch$. If we guessed the indices correctly we have that
    $h_{i_{j'}}$, for $j' \leq j$, parsed as $\zkproof_\adv[1..j']$ is a prefix
    of $\zkproof_\adv[1..j]$.
  \item On query $\inp_\simulator$ to $\simulator$, $\adv'$ runs the simulator
    $\ps.\simulator$ internally. Note that we require a simulator that only
    programs the random oracle for $j \geq k$.  If the simulator makes a
    previously unanswered random oracle query with input
    $\zkproof_\simulator[1..j]$, $1 \leq j < k$, and this is the $i_j$-th query,
    it generates $\zkproof_\simulator[j].\ch$ by invoking $\ps.\verifier$ on
    $\zkproof_\simulator[j]$ and programs
    $\ro(h_{i_j}) = \zkproof_\simulator[j].\ch$.  It returns
    $\zkproof_\simulator$.
  \item Answers $\ps.\verifier$'s final challenge $\zkproof_\adv[\mu].\ch$ using the
    answer given by $\adv$, i.e.~$\zkproof_\adv[\mu]$.
  \end{compactitem}
  That is, $\adv'$ manages to break soundness of $\ps$ if $\adv$ manages to
  break simulation soundness without breaking the unique response property and
  $\adv'$ correctly guesses the indices of $\adv$ random oracle queries. This
  happens with probability upper-bounded by $\abs{\prob{\game{1}} -
    \prob{\game{2}}} \cdot \infrac{1}{q^{\mu}}$. Hence $\abs{\prob{\game{1}} -
    \prob{\game{2}}} \leq q^{\mu} \cdot \epss(\secpar)$.

  Note that in $\game{2}$ the adversary cannot win. Thus the probability
  that $\advss$ is successful is upper-bounded by
  $\epsur(\secpar) + q^{\mu} \cdot \epss(\secpar)$.  \qed
\end{proof}


We conjecture that based on the recent results on state restoration soundness~\cite{cryptoeprint:2020:1351}, which effectively allows to query the verifier multiple times on different overlapping transcripts, the $q^{\mu}$ loss could be avoided. However, this would reduce the class of protocols covered by our results. 

\hamid{The following theorem is changed from the original to the one in the updatable setting!}
\begin{theorem}[Updatable Forking simulation-extractable multi-message protocols]
  \label{thm:se}
  Let $\ps = (\kgen, \prover, \verifier, \simulator)$ be an interactive
  $(2 \mu + 1)$-message proof system for $\RELGEN(\secparam)$ that is honest
  verifier zero-knowledge in the standard model\hamid{Is this different from Trapdoor-less simulatable proof system (def. 1)?}\footnote{Crucially, we require
    that one can provide an indistinguishable simulated proof without any
    additional knowledge, as e.g~knowledge of a SRS trapdoor.}, has $\ur{k}$
  property \hamid{in the updatable setting} with security $\epsur(\secpar)$, and is $(\epss(\secpar), k, n)$-forking sound \hamid{in the updatable setting}.
  % for
  % $n_i = 1, i \in \range{1}{\mu} \setminus \smallset{k}$ and $n_k = n$.
%
Let $\ro\colon \bin^{*} \to \bin^{\secpar}$ be a random oracle. 
Then $\psfs$ is \hamid{updatable} forking simulation-extractable with extraction error $\epsur(\secpar)$
against $\ppt$ algebraic adversaries that makes up to $q$ random oracle queries and
returns an acceptable proof with probability at least $\accProb$. 
The extraction probability $\extProb$ is at least
\(
	\extProb \geq \frac{1}{q^{n - 1}} (\accProb - \epsur(\secpar))^{n} -\eps(\secpar)\,,
\)
for some negligible $\eps(\secpar)$.	
\end{theorem}
\begin{proof}		

  \ngame{0} This is \hamid{the updatable forking} simulation extraction game played between an adversary
  $\advse$ who is given access to \hamid{an oracle $\initU$ that defines an updatable SRS setup,} a random oracle $\ro$ and simulator
  $\psfs.\simulator$. There is also an extractor $\ext$ that, from a proof
  $\zkproof_{\advse}$ for instance $\inp_{\advse}$ output by the adversary and from
   transcripts of $\advse$'s operations is tasked to extract a witness
  $\wit_{\advse}$ such that $\REL(\inp_{\advse}, \wit_{\advse})$ holds. $\advse$ wins
  if it manages to produce an acceptable proof and the extractor fails to reveal
  the corresponding witness. In the following game hops we upper-bound the
  probability that this happens.

  \ngame{1} This is identical to $\game{0}$ except that now the game is aborted
  if there is a simulated proof $\zkproof_\simulator$ for $\inp_{\advse}$ such
  that $(\inp_{\advse}, \zkproof_\simulator[1..k]) = (\inp_{\advse},
  \zkproof_{\advse}[1..k])$. That is, the adversary in its final proof
  reuses at least $k$ messages from a simulated proof it saw before and the proof is acceptable.
  Denote that event by $\event{\errur}$.

  \ncase{Game 0 to Game 1} $\prob{\event{\errur}} \leq \epsur(\secpar)$. The
  proof goes exactly as in \cref{thm:simsnd}.

  \COMMENT{We have,
    \( \prob{\game{0} \land \nevent{\errur}} = \prob{\game{1} \land
      \nevent{\errur}} \) and, from the difference lemma,
    cf.~\cref{lem:difference_lemma},
  \[ \abs{\prob{\game{0}} - \prob{\game{1}}} \leq \prob{\event{\errur}}\,. \]
  Thus, to show that the transition from one game to another introduces only
  minor change in probability of $\advse$ winning it should be shown that
  $\prob{\event{\errur}}$ is small.

  We can assume that $\advse$ queried the simulator on the instance it wishes to
  output---$\inp_{\advse}$. We show a reduction $\rdvur$ that utilises $\advse$,
  who outputs a valid proof for $\inp_{\advse}$, to break the $\ur{k}$ property of
  $\ps$. Let $\rdvur$ run $\advse$ internally as a black-box:
\begin{itemize}
	\item The reduction answers both queries to the simulator $\psfs.\simulator$ and to the random oracle. 
	It also keeps lists $Q$, for the simulated proofs, and $Q_\ro$ for the random oracle queries. 
\item When $\advse$ makes a fake proof $\zkproof_{\advse}$ for $\inp_{\advse}$,
  $\rdvur$ looks through lists $Q$ and $Q_\ro$ until it finds
  $\zkproof_{\simulator}[0..k]$ such that
  $\zkproof_{\advse}[0..k] = \zkproof_{\simulator}[0..k]$
  and a random oracle query $\zkproof_{\simulator}[k].\ch$ on
  $\zkproof_{\simulator}[0..k]$.
	\item $\rdvur$ returns two proofs for $\inp_{\advse}$:
	\begin{align*}
		\zkproof_1 = (\zkproof_{\simulator}[1..k],
		\zkproof_{\simulator}[k].\ch, \zkproof_{\simulator}[k + 1..\mu + 1])\\
		\zkproof_2 = (\zkproof_{\simulator}[1..k],
		\zkproof_{\simulator}[k].\ch, \zkproof_{\advse}[k + 1..\mu + 1])
	\end{align*}
	\end{itemize}  
	If $\zkproof_1 = \zkproof_2$, then $\advse$ fails to break simulation
  extractability, as $\zkproof_2 \in Q$. On the other hand, if the proofs are
  not equal, then $\rdvur$ breaks $\ur{k}$-ness of $\ps$. This happens only with
  negligible probability $\epsur(\secpar)$, hence \( \prob{\event{\errur}} \leq
  \epsur(\secpar)\,. \)
}

  \ngame{2} This is identical to $\game{1}$ except that now the environment
  aborts also when it fails to build a $(1, \ldots, 1, n, 1, \ldots, 1)$-tree
  of accepting transcripts $\tree$ by rewinding $\advse$. Denote that event by
  $\event{\errfrk}$. 

  \ncase{Game 1 to Game 2} Note that for every acceptable proof
  $\zkproof_{\advse}$, we may assume that whenever $\advse$ outputs in Round $k$
  message $\zkproof_{\advse}[k]$, then the
  $(\inp_{\advse}, \zkproof_{\advse}[1..k])$ random oracle query was made
  by the adversary, not the simulator\footnote{\cite{INDOCRYPT:FKMV12} calls
    these queries \emph{fresh}.}, i.e.~there is no simulated proof
  $\zkproof_\simulator$ on $\inp_\simulator$ such that
  $(\inp_{\advse}, \zkproof_{\advse} [1..k]) = (\inp_\simulator,
  \zkproof_\simulator[1..k])$. Otherwise, the game would be already interrupted
  by the error event in Game $\game{1}$.  As previously,
\(
  \abs{\prob{\game{1}} - \prob{\game{2}}} \leq \prob{\event{\errfrk}}\,.
\)

We describe our extractor $\ext$ here. The extractor takes as input relation
$\REL$, SRS $\srs$, $\advse$'s code, its randomness $r$, the output instance
$\inp_{\advse}$ and proof $\zkproof_{\advse}$, as well as the list $Q$ of
simulated proofs (and their instances) and the list of random oracle queries and
responses $Q_\ro$. Then, $\ext$ starts a forking algorithm
$\genforking^{n}_\zdv(y,h_1, \ldots, h_q)$ for
$y = (\srs, \advse, r, \inp_{\advse}, \zkproof_{\advse}, Q)$ where we set
$h_1, \ldots, h_q$ to be the consecutive queries from list $Q_\ro$. We run
$\advse$ internally in $\zdv$.% which returns the proof $\zkproof$ and index $i$
%of the random oracle query that $\advse$ used to answer $\zkproof$'s $k$-th challenge. 

To assure that in the first execution of $\zdv$ the adversary $\advse$ produce
the same $(\inp_{\advse}, \zkproof_{\advse})$ as in the extraction game, $\zdv$
provides $\advse$ with the same randomness $r$ and answers queries to the random
oracle and simulator with pre-recorded responses in $Q_\ro$ and $Q$.
%
Note, that since the view of the adversary when run inside $\zdv$ is the same as its
view with access to the real random oracle and simulator, it produces exactly the
same output. After the first run, $\zdv$ outputs the index $i$ of a random oracle
query that was used by $\advse$ to compute the challenge $\zkproof[k].\ch =
\ro(\zkproof_{\advse}[0..k])$ it had to answer in the $(k + 1)$-th round and
adversary's transcript, denoted by $s_1$ in $\genforking$'s description. If no
such query took place $\zdv$ outputs $i = 0$.

Then new random oracle responses are picked for queries indexed by
$i, \ldots, q$ and the adversary is rewound to the point just prior to when it gets the
response to RO query $\zkproof_{\advse}[0..k]$. The adversary gets a random
oracle response from a new set of responses $h^2_i, \ldots, h^2_q$. If the
adversary requests a simulated proof after seeing $h^2_i$ then $\zdv$ computes
the simulated proof on its own. Eventually, $\zdv$ outputs index $i'$ of a query
that was used by the adversary to compute $\ro(\zkproof_{\advse}[0..k])$, and a
new transcript $s_2$. $\zdv$ is run $n$ times with different random oracle
responses. If a tree $\tree$ of $n$ transcripts is built then $\ext$
runs internally the tree extractor $\extt(\tree)$ and outputs what it returns.

We emphasize here the importance of the unique response property. If it does not
hold then in some $j$-th execution of $\zdv$ the adversary could reuse a
challenge that it learned from observing proofs in $Q$. In that case, $\zdv$
would output $i = 0$, making the extractor fail. Fortunately, the case that the
adversary breaks the unique response property has already been covered by the
abort condition in $\game{1}$.

Denote by $\waccProb$ the probability that $\advse$ outputs a proof that is
accepted and does not break $\ur{k}$-ness of $\ps$.  Denote by $\waccProb'$ the
probability that algorithm $\zdv$, defined in the lemma, produces an accepting
proof with a fresh challenge after Round $k$. Given the discussion above, we can
state that $\waccProb = \waccProb'$.

Next, from the generalised forking lemma, cf.~\cref{lem:generalised_forking_lemma}, we get that
\begin{equation}
  \begin{split}
    \prob{\event{\errfrk}} \leq 1 - \waccProb \cdot \left(\infrac{\waccProb^{n -
          1}}{q^{n - 1}} + \infrac{(2^\secpar) !}{((2^\secpar - n)! \cdot
        (2^\secpar)^{n})} - 1\right).
    % = \\
    % 1 - \left(\frac{\waccProb^{n}}{q^{n - 1}} + 
    %   \waccProb \cdot \frac{(2^\secpar) !}{(2^\secpar - n)! \cdot
    %     (2^\secpar)^{n}} - \waccProb\right)\,.
\end{split}
\end{equation}

% \ngame{3} This game is identical to $\game{2}$ except that the environment
% aborts it when the adversary manages to break simulation soundness in one the
% transcripts in tree $\tree$.

% \ncase{$\game{2} \mapsto \game{3}$} From \cref{thm:simsnd} we have that
% probability that $\adv$ breaks simulation soundness, while not breaking the
% unique response property, is upper-bounded by $q_\ro^{\mu} \epss$. Since all the
% branches of the tree are either for a valid or invalid statement, then the
% probability that $\adv$ breaks simulation soundness in one of the tree branches
% is upper-bounded by $q_{\ro}^{\mu} \epss$ as well.

\ngame{3} This game is identical to $\game{2}$ except that it aborts if
$\extt(\tree)$ run by $\ext$ fails to extract the witness. 

\ncase{Game 2 to Game 3}	
Since $\ps$ is forking-sound the probability that $\extt(\tree)$
fails is upper-bounded by $\epsss(\secpar)$. 

Since Game $\game{3}$ is aborted when it is impossible to extract the correct witness
from $\tree$, hence the adversary $\advse$ cannot win.  Thus, by the game-hopping
argument,
\[
	\abs{\prob{\game{0}} - \prob{\game{4}}} \leq 1 -
  \left(\frac{\waccProb^{n}}{q^{n - 1}} + \waccProb \cdot \frac{(2^\secpar)
      !}{(2^\secpar - n)! \cdot (2^\secpar)^{n}} - \waccProb\right) + \epsur(\secpar) +
  %q_{\ro}^{\mu} \epss +
  \epsss(\secpar)\,.
\]
Thus the probability that extractor $\extss$ succeeds is at least
\[
	\frac{\waccProb^{n}}{q^{n - 1}} + \waccProb \cdot \frac{(2^\secpar)
    !}{(2^\secpar - n)! \cdot (2^\secpar)^{n}} - \waccProb - \epsur(\secpar) 
  %- q_{\ro}^{\mu} \epss
  - \epsss(\secpar)\,.
\]
Since $\waccProb$ is probability of $\advse$ outputting acceptable transcript
that does not break $\ur{k}$-ness of $\ps$, then $\waccProb \geq \accProb -
\epsur(\secpar)$, where $\accProb$ is the probability of $\advse$ outputing an acceptable
proof as defined in \cref{def:simext}. It thus holds
\begin{equation}
	\label{eq:frk}
	\extProb \geq \frac{(\accProb - \epsur(\secpar))^{n}}{q^{n - 1}} -
  \underbrace{(\accProb - \epsur(\secpar)) \cdot \left( 1 - \frac{(2^\secpar)
        !}{(2^\secpar - n)! \cdot (2^\secpar)^{n}}\right) - \epsur(\secpar) -
    % q_{\ro}^{\mu} \epss -
    \epsss(\secpar)}_{\eps(\secpar)}\,.
\end{equation}
Note that the part of \cref{eq:frk} denoted by $\eps(\secpar)$ is negligible as
$\epsur(\secpar), \epsss(\secpar)$ are negligible, and
$\frac{(2^\secpar) !}{(2^\secpar - n)! \cdot (2^\secpar)^{n}} \geq
\left(\infrac{(2^\secpar - n)}{2^\secpar}\right)^{n}$ is overwhelming.  Thus,
\[
	\extProb \geq q^{-(n - 1)} (\accProb - \epsur(\secpar))^{n} -\eps(\secpar)\,.
\] 
and $\psfs$ is forking simulation extractable with extraction error $\epsur(\secpar)$.
\qed
\end{proof}
 