accepting%!TEX root=main.tex

\begin{theorem}[Simulation-extractable multi-message protocols]
	\label{thm:se}
	Let $\psfs = (\SRScer, \prover, \verifier, \simulator)$ be a  $(2\mu + 1)$-message FS-transformed NIZK proof system with an updatable SRS setup. If $\psfs$ is an
	updatable $k$-unique response protocol with security loss $\epsur$, is
	updatable $k$-programmable trapdoor-less zero-knowledge, and %$(1, \ldots, 1, n_{k}, \ldots, n_\mu)$-
	updatable rewinding-based knowledge sound with security loss $\epscss$; 
	%
	Then $\psfs$ is \emph{updatable simulation-extractable} with security loss $$\epsse(\secpar,\accProb,q) \leq \epscss(\secpar,\accProb - \epsur(\secpar),q)$$ against any $\ppt$ adversary $\advse$ that makes up to $q$ random oracle queries and returns an  proof with probability at least $\accProb$.
\end{theorem}

\begin{proof}	
	Let $(\inp, \zkproof) \gets \advse^{\initU, \simOH, \simOP'}(r_{\advse})$ be the USE adversary. We show how to build an extractor $\extse (\srs, \advse, r_{\advse}, \Qsim, \Qro, \Qsrs)$ that outputs a witness $\wit$, such that $\REL(\inp, \wit)$ holds with high probability. To that end we define an algorithm $\advcss^{\initU,\ro}(r)$ against rewinding-based knowledge soundness of $\psfs$ that runs internally $\advse^{\initU, \simOH, \simOP'}(r_{\advse})$. Here $r = (\rsim, r_{\advse})$ with $\rsim$ the randomness that will be used to simulate $\simOP'$. 

	The code of $\advcss^{\initU,\ro}(r)$ hardcodes $\Qsim$ such that it does not use any randomness for proofs in $\Qsim$ as long as statements are queried in order. In this case it simple returns a proof $\zkproofs$ from $\Qsim$ but nevertheless queries $\simOHprog$ on $(\tzkproofs[0..k],\tzkproofs[k].ch]$, i.e. it programs the $k$-th challenge. While it is hard to construct such an adversary without knowing $\Qsim$, it clearly exists and $\extse$ has the necessary inputs to construct $\advcss$. This hardcoding guarantees that $\advcss$ returns the same $(\inp,\zkproof)$ as $\advse$ in the experiment.
	%
	Eventually, $\extse$ uses the tree builder $\tdv$ and extractor $\extcss$ for $\advcss$ to extract the witness for $\inp$. Both guaranteed to exist (and be successful with high probability) by rewinding-based knowledge soundness. This high-level argument shows that $\extse$ exists as well.
	
	We now give the details of the simulation that guarantees that $\advcss$ is successful whenever $\advse$ is---except with a small security loss that we will bound late:
	Since $\advcss$ runs $\advse$ internally, it needs to take care of $\advse$'s oracle queries.
	$\advcss$ passes on queries of $\advse$ to the update oracle $\initU$ to its own $\initU$ oracle and returns the result to $\advse$.
	$\advcss$ internally simulates (non-hardcoded) queries to the simulator $\simOP'$ by running the $\simulator$ algorithm on randomness $\rsim$ of its tape. $\simulator$ requires access to oracles $\simOH$ to compute a challenge honestly and $\simOHprog$ to program a challenge. Again $\advcss$ simulates both of these oracles internally, cf.~\cref{fig:simulator_oracles}, this time using the $\ro$ oracle of $\advcss$. 	Note that queries of $\advse$ to $\simOH$ are not programmed, but passed on to $\ro$. 
	
	Importantly, all challenges in simulated proofs, up to round $k$ are also computed honestly, i.e. $\tzkproof[i].\ch = \ro(\tzkproof[0..i])$, for $i < k$.
	%

	
	\begin{figure}
		\centering
			\begin{pcvstack}[center,boxed]
			\begin{pchstack}
				\procedure{$\simOH (x)$}
				{
				\pcif H[x] = \bot \pcthen \\
				\pcind H[x] \gets \ro(x) \\
				\pcreturn H[x]
		  		}
				%
				\pchspace
				%
				\procedure{$\simulator\oracleo.\prog(x, h)$}
				{ 
					\pcif H[x] = \bot \pcthen \\ 
					\pcind H[x] \gets h \\
					\pcind \Qprog \gets \Qprog \cup \{x\}\\
					\pcreturn H[x]
				}
			\end{pchstack}
		\end{pcvstack}
		\caption{Simulating random oracle calls.}
		\label{fig:simulator_oracles}
	\end{figure}	
%

	Eventually, $\advse$ outputs an instance and proof $(\inp, \zkproof)$. $\advcss$ returns the same values as long as $\tzkproof[0..i] \notin \Qprog$, $i\in[1,\mu]$. This models that the proof output by $\advcss$ must not contain any programmed queries as such a proof would not  w.r.t~$\ro$ in the RBKS experiment. If $\advse$ outputs a proof that does contain programmed challenges, then $\advcss$ aborts. We denote this event by $\event{E}$.
	
	\begin{lemma}
		Probability that $\event{E}$ happens is upper-bounded by $\epsur(\secpar)$. 
	\end{lemma}
	\begin{proof}
	%Denote by $\zkproof_{\advse}, \zkproof_{\simulator}$ proofs returned by the adversary and the simulator respectively.

	We build an adversary $\advur^{\initU, \ro} (\secpar; r)$ that has access to the random oracle $\ro$ and update oracle $\initU$. $\advur$ uses $\advcss$ to break the $\ur{k}$ property of $\psfs$. 
	%Namely, for randomness $r = (r_{\advse}, \rsim)$, $\advur^{\ro, \initU} (\secpar; r)$ runs $\advse^{\initU, \simOH, \simOP'}(\secparam; r_{\advse})$ internally and answers its oracle queries as $\advcss$ did.

	When $\advcss$ outputs a proof $\zkproof$ for $\inp$ such that $\event{E}$ holds, $\advur$ looks through lists $Q$ and $\Qro$ until it finds $\tzkproofs[0..k]$ such that $\tzkproof[0..k] = \tzkproofs[0..k]$ and a programmed random oracle query $\tzkproofs[k].\ch$ on $\tzkproofs[0..k]$.	$\advur$ returns two proofs $\zkproof$ and $\zkproofs$ for $\inp$:
		% \begin{align*}
		% \zkproof_1 = \zkproofs =  (\zkproofs[1..k], \zkproofs[k + 1..\mu + 1])\\
		% \zkproof_2 = \zkproof \;\;\;\;\, = (\zkproofs[1..k], \;\;\;\;\zkproof[k + 1..\mu + 1])
		% \end{align*}
		and the challenge $\tzkproofs[k].\ch=\tzkproof[k].\ch$

	Importantly, both proofs are  w.r.t~the unique response verifier. The first, since it is a correctly computed simulated proof for which the unique response property definition allows any challenges at $k$. The latter, since it is an  proof produced by the adversary.
	We have that $\zkproof \neq \zkproofs$ as otherwise $\advse$ does not win the simulation extractability game as $\zkproof \in Q$. On the other hand, if the proofs are different, then $\advur$ breaks $\ur{k}$-ness of $\psfs$. This happens only with  probability $\epsur(\secpar)$. 
	\qed
	\end{proof}

	We denote by $\waccProb$ the probability that $\advcss$ outputs an  proof. We note that by up-to-bad reasoning $\waccProb$ is at most $\epsur (\secpar)$ far from the probability that $\advse$ outputs an  proof. Thus, the probability that $\advcss$ outputs an  proof is at least $\waccProb \geq \accProb - \epsur (\secpar)$. %\markulf{30.04}{I am ignorant. What is the role of the union bound here. Is this the same as up-to-bad reasoning?}
%
	Since $\psfs$ is $\epscss (\secpar, \waccProb,q)$ rewinding-based knowledge sound, there is a tree builder $\tdv$ and extractor $\extcss$ that rewinds $\advcss$ to obtain a tree of accepting transcripts $\tree$ and fails to extract the witness with probability at most $\epscss (\secpar, \waccProb, q)$. The extractor $\extse$ outputs the witness with the same probability.

	%\hamid{30.4}{We probably mean rewinding-based KS here. I think we need to also justify why we require this property with specific parameters $(1, \ldots, 1, n_{k}, \ldots, n_\mu)$ for the tree structure!}\markulf{30.04}{Fixed the typo, if that's what you meant. I don't think we need any specific tree structure. There is nevertheless more to say here, I agree.}
%
	Thus $\epsse(\secpar,\accProb,q) = \epscss (\secpar, \waccProb,q) \leq \epscss(\secpar,\accProb - \epsur,q)$.
	\qed
	\end{proof}

\begin{remark}
Observe that our theorem does not depend on $\epszk(\secpar)$. There is no real prover algorithm $\prover$ in the experiment. Only the $k$-programmability of TLZK matters. 
\end{remark}

\begin{remark}
Observe that the theorem does not prescribe a tree shape for the tree builder $\tdv$. Interestingly, in our concrete results $\tdv$ outputs a $(k, *)$-tree of accepting transcripts.
\end{remark}

%%% Local Variables:
%%% mode: latex
%%% TeX-master: "main"
%%% End:

