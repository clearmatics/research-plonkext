% !TeX spellcheck = en_US
\let\accentvec\vec \documentclass[runningheads,10pt]{llncs}
 % \documentclass[runningheads]{amsart}
\let\spvec\vec \let\vec\accentvec

\usepackage{amssymb,amsmath} \let\vec\spvec \usepackage{lmodern}

\usepackage[T1]{fontenc}

\newcommand{\iflipics}[1] {} \newcommand{\iflncs}[1] {#1}

\def\vec#1{\mathchoice{\mbox{\boldmath$\displaystyle#1$}}
{\mbox{\boldmath$\textstyle#1$}} {\mbox{\boldmath$\scriptstyle#1$}}
{\mbox{\boldmath$\scriptscriptstyle#1$}}}

\DeclareFontFamily{U}{mathx}{\hyphenchar\font45}
\DeclareFontShape{U}{mathx}{m}{n}{<-> mathx10}{}
\DeclareSymbolFont{mathx}{U}{mathx}{m}{n}
\DeclareMathAccent{\widebar}{0}{mathx}{"73}

% lncs size (as printed in books, with small margins):
\usepackage[paperheight=23.5cm,paperwidth=15.5cm,text={13.2cm,20.3cm},centering]{geometry}
% \usepackage{fullpage}

\newcommand{\ifamsart}[1] {} \ifamsart{ \newtheorem{theorem}{Theorem}%[section]
		\newtheorem{proposition}[theorem]{Proposition}
		\newtheorem{lemma}[theorem]{Lemma}
		\newtheorem{corollary}[theorem]{Corollary} \theoremstyle{definition}
		\newtheorem{definition}[theorem]{Definition}
\newtheorem{example}[theorem]{Example} } \usepackage{soul} \usepackage{soulutf8}
\soulregister\cite7 \soulregister\ref7 \soulregister\pageref7
\usepackage{hyperref} \usepackage[color=yellow]{todonotes} \hypersetup{final}
\usepackage{mathrsfs}
\usepackage[advantage,asymptotics,adversary,sets,keys,ff,lambda,primitives,events,operators,probability,logic,mm,complexity]{cryptocode}
\pcbodylinesep=0.15\baselineskip

\usepackage[capitalise]{cleveref} \usepackage{cite} \usepackage{booktabs}
\usepackage{paralist}
\usepackage[innerleftmargin=5pt,innerrightmargin=5pt]{mdframed}
%\usepackage{setspace}
\usepackage{caption}
\captionsetup{belowskip=0pt}
%\captionsetup[figure]{font={stretch=2}}
% \usepackage{subcaption}
\usepackage{bm}

\newcommand{\newdefs}[1] {\setlength{\fboxsep}{1pt}\colorbox{gray!20}{\(#1\)}}

\newcommand{\COMMENT}[1]  {}

%general formatting
\newcommand{\pcvarstyle}[1]{\mathsf{#1}}
\newcommand{\comment}[1]{{\color{lightgray}#1}}
\newcommand{\continue}{{\Huge{\hl{$\cdots$}}}}

% General mathematics
\newcommand{\range}[2] {[#1 \, .. \, #2]}
\newcommand{\SD}{\Delta}
\newcommand{\smallset}[1] {\{#1\}}
\newcommand{\bigset}[1] {\left\{#1\right\}}
\newcommand{\GRP} {\mathbb{G}}
\newcommand{\pair} {\hat{e}}
\newcommand{\brak}[1] {\left(#1\right)}
\newcommand{\sbrak}[1] {(#1)}
\newcommand{\alg}[1] {\pcalgostyle{#1}}
\newcommand{\image} {\operatorname{im}}
\newcommand{\myland} {\,\land\,}
\newcommand{\mylor} {\,\lor\,}
\newcommand{\vect}[1] {\operatorname{vect}(#1)}
\newcommand{\w}{\omega}
\newcommand{\const}{\pcpolynomialstyle{const}}
\newcommand{\p}[1]{\pcpolynomialstyle{#1}}
\newcommand{\ev}[1]{\tilde{\pcpolynomialstyle{#1}}}
\newcommand{\numberofconstrains}{\pcvarstyle{n}}
\newcommand{\expected}[1]{\mathbb{E}\left[#1\right]}
\newcommand{\infrac}[2]{#1 / #2}

% bilinear maps

\newcommand{\bmap}[2] {\left[#1\right]_{#2}}
\newcommand{\gone}[1] {\bmap{#1}{1}}
\newcommand{\gtwo}[1] {\bmap{#1}{2}}
\newcommand{\gi} {\iota}
\newcommand{\gtar}[1] {\bmap{#1}{T}}
\newcommand{\grpgi}[1] {\bmap{#1}{\gi}}


% zero knowledge
\newcommand{\oracleo}{\mathsf{O}}
\newcommand{\crs}{\pcvarstyle{crs}}
\newcommand{\td}{\pcvarstyle{td}}
\newcommand{\ip}[2]{\left\langle #1, #2\right\rangle}
\newcommand{\zkproof}{\pi}
\newcommand{\proofsystem}{\mathrm{\Psi}}
\newcommand{\ps}{\proofsystem}
\newcommand{\nuppt}{\pcmachinemodelstyle{NUPPT}}
\newcommand{\ro}{\mathcal{H}}
\newcommand{\rof}[2]{\mathbf{\Omega}_{#1, #2}}
\newcommand{\trans}{\pcvarstyle{trans}}
\newcommand{\tr}{\pcvarstyle{tr}}
\newcommand{\instsize}{\pcvarstyle{n}}
\newcommand{\KG} {\mathsf{K}}
\newcommand{\kcrs} {\KG_{\crs}}
\renewcommand{\dist}{\ddv}
\newcommand{\fs}{\pcalgostyle{FS}}
\newcommand{\sigmaprot}{\pcalgostyle{\Sigma}}
\newcommand{\se}{\pcvarstyle{se}}
\newcommand{\snd}{\pcvarstyle{snd}}
\newcommand{\zk}{\pcvarstyle{zk}}
\newcommand{\advse}{\adv_\se}

%rewinding---tree of transcripts
\newcommand{\pcboolstyle}[1]{\mathtt{#1}}
\newcommand{\treebuild}{\pcalgostyle{TreeBuild}}
\newcommand{\tree}{\pcvarstyle{tree}}
\newcommand{\counter}{\pcvarstyle{counter}}


%PLONK related
\newcommand{\plonkprot}{\mathbf{P}}
\newcommand{\plonkprotfs}{\mathbf{P}_\fs}
\newcommand{\selector}[1]{\pcvarstyle{q_{#1}}}
\newcommand{\selmulti}{\selector{M}}
\newcommand{\selleft}{\selector{L}}
\newcommand{\selright}{\selector{R}}
\newcommand{\seloutput}{\selector{O}}
\newcommand{\selconst}{\selector{C}}
\newcommand{\chz}{\mathfrak{z}}
\newcommand{\reduction}{\rdv}

\newcommand{\game}[1]{\pcalgostyle{G}_{#1}}

\newcommand{\lag}{\p{L}}
\newcommand{\pubinppoly}{\p{PI}}

% general complexity theory
% \newcommand{\RND}[1]{\pcalgostyle{RND}(#1)}
\newcommand{\RND}[1]{\pcvarstyle{R}(#1)}
\newcommand{\RELGEN}{\mathcal{R}}
\newcommand{\REL}{\mathbf{R}}
\newcommand{\LANG}{\mathcal{L}}
\newcommand{\inp}{\pcvarstyle{x}}
\newcommand{\wit}{\pcvarstyle{w}}
\newcommand{\class}[1]{\mathfrak{#1}}
\newcommand{\ig}{\pcalgostyle{IG}}
\newcommand{\accProb}{\event{acc}}
\newcommand{\frkProb}{\event{frk}}
\newcommand{\FS}{\pcalgostyle{FS}} % Fiat-Shamir transform
\newcommand{\aux}{\pcvarstyle{aux}} %auxiliary input

%Plonk and Sonic
\newcommand{\plonk}{\ensuremath{\textsc{Plonk}}}
\newcommand{\plonkmod}{\ensuremath{\plonk^\star}}
\newcommand{\plonkint}{\ensuremath{\plonk^\star}}
\newcommand{\polyprot}{\pcalgostyle{poly}}
\newcommand{\plonkintpoly}{\plonkint_\polyprot}
\newcommand{\sonic}{\textsc{Sonic}}
\newcommand{\maxdegree}{\pcvarstyle{N}}

\newcommand{\dlog}{\pcvarstyle{dlog}}

\newcommand{\ur}[1]{{#1\text{-}\mathsf{ur}}}

%forking
\newcommand{\forking}{\pcalgostyle{F}}
\newcommand{\genforking}{\pcalgostyle{GF}}

%colors
\definecolor{darkmagenta}{rgb}{0.5,0,0.5}
\definecolor{lightmagenta}{rgb}{1,0.85,1}
\definecolor{lightmagenta}{rgb}{0.9,0.9,0.9}
\definecolor{darkred}{rgb}{0.7,0,0}
\definecolor{blueish}{rgb}{0.1,0.1,0.5}
\definecolor{pinkish}{rgb}{0.9,0.8,0.8}
\definecolor{darkgreen}{rgb}{0,0.6,0}
\definecolor{lightgreen}{rgb}{0.85,1,0.85}
\definecolor{skyblue}{rgb}{0.3,0.9,0.99}

%comments
\DeclareRobustCommand{\markulf}[2]  {{\color{darkmagenta}\hl{\scriptsize\textsf{Markulf #1:} #2}}}
\DeclareRobustCommand{\michals}[2]  {{\color{blueish}\sethlcolor{pinkish}\hl{\scriptsize\textsf{Michal #1:} #2}}}
\newcommand{\task}[2]{\todo[author=\textbf{Task},inline]{({\textit{#1}}) #2}}
% \newcommand{\task}[2] {\xcommenti{Task}{#1}{#2}}
% \DeclareRobustCommand{\task}[2]  {{\color{black}\sethlcolor{yellow}\hl{\textsf{TASK #1:} #2}}}

%%% Local Variables:
%%% mode: latex
%%% TeX-master: "plonkext"
%%% End:


\title{On Simulation-Extractability of Updatable Universal zkSNARKs}

\author{Markulf Kohlweiss\inst{1,2} \and Michał Zając\inst{3}} \iflncs{
\institute{University of Edinburgh, Edinburgh, UK \and IOHK \\
\email{mkohlwei@inf.ed.ac.uk} \and Clearmatics, London UK \\
\email{m.p.zajac@gmail.com}} }

\allowdisplaybreaks

\begin{document} \sloppy \maketitle

\begin{abstract} 
	In this paper we prove that two efficient updatable universal
	zkSNARK---\plonk{}~\cite{EPRINT:GabWilCio19} and
	$\sonic$~\cite{CCS:MBKM19}---are (non black-box) simulation extractable.  To
	that end, we generalise the result by Faust et al.~\cite{INDOCRYPT:FKMV12}
	(INDOCRYPT 2012) and show that any (computationally) special-sound proof of
	knowledge that has the unique response property and is zero-knowledge in the
	standard model is also simulation-extractable if made non-interactive by the
	Fiat--Shamir transform.  We then explain why \plonk{} and \sonic{} meet
	these requirements and conclude by showing its simulation-extractability.
	Unfortunately, as a side result we also show that relying on rewinding and
	Fiat--Shamir transform often comes at a great price of inefficient (yet
	still polynomial) knowledge extraction and the security loss introduced by
	these techniques should always be taken into account.  Because of the
	inefficiency and loosenes of the reduction we independently show that
	$\plonk{}$ is simulation-sound.  
\end{abstract}

\section{Introduction} \subsection{Motivation} \paragraph{The rise of updatable
zkSNARKs.} \cite{C:GKMMM18} \cite{EC:CHMMVW20} \cite{CCS:MBKM19}
\cite{EPRINT:GabWilCio19} \cite{EPRINT:Gabizon19c} \cite{EPRINT:Lipmaa19a}

\paragraph{On the importance of the simulation extractability.}
Although zkSNARKs satisfy the knowledge soundness definition,
simulation-extractability is the property that should be required from
zkSNARKs used in practice---as efficient zero-knowledge proofs deployed in the
real-life cryptosystems. This is since in the real life one simply cannot
assume that the adversary who tries to break security of the system does not
have access to any proofs provided by other parties using the same
zero-knowledge system. On contrary, in the most popular applications of
zkSNARKs, like privacy-preserving blockchains, proofs made by all
blockchain-participants are public. Thus it is only reasonable to require from
a zero-knowledge proof system to be resilient to attacks that utilise proofs
performed by different parties.

There are many results regarding obtaining simulation extractability for
NIZKs.  First, Groth \cite{AC:Groth07} noticed that a (black-box) simulation
extractable NIZK is UC-secure. Then Dodis et al.~\cite{AC:DHLW10} introduced a
notion of (black-box) \emph{true simulation extractability} and shown that no
NIZK can be UC-secure if it does not have this property. 

In the context of zkSNARKs it is important to mention such works as Kosba's et
al.~\cite{EPRINT:KZMQCP15} who show a general transformation from a NIZK
to a black-box SE NIZK. Although their transformation works for zkSNARKs
as well, succinctness of the proof system is not preserved as the witness
is encrypted. 
Abdolmaleki et al.~\cite{EPRINT:AbdRamSla20} shown another transformation that
obtains non-black-box simulation extractability but also preserves succinctness
of the argument.

Independently, some authors focused on obtaining simulation extractability of
 known zkSNARKs, like $\groth$, by introducing minor modifications and using
stronger assumptions \cite{EPRINT:BowGab18,EPRINT:AtaBag19}. Interestingly,
although such modifications hurt performance of the proof system the resulting
zkSNARKs are still more efficient than the first SE zkSNARK proposed in
\cite{C:GroMal17}, see \cite{EPRINT:AtaBag19}. 

\paragraph{State of the art---simulation-extractable updatable zkSNARKs.} 
Up to our best knowledge, there are zkSNARKs that are simulation-extractable
and zkSNARKs that are updatable, however there is no known zkSNARKs that enjoy
both of these properties out-of-the-box. Obviously, given an updatable zkSNARK
one could lift it to be simulation-extractable using techniques described
e.g.~in \cite{EPRINT:KZMQCP15,EPRINT:AbdRamSla20}, but such lift comes with
inevitably efficiency loss.  On the other hand, there is no known lift that
would take a zkSNARK and make it updatable. Since updatable zkSNARKs are quite
restrictive regarding format of their CRSs, making such lift seems as a very
difficult task. 

\subsection{Our contribution} 
First of all, we prove that $\plonk$ and $\sonic$ are simulation-extractable.
Although we do not change these protocols at all, we had to solve a number of
problems in order to show that. The idea of the proof is presented below. 

Note that these protocols, as originally presented in
\cite{EPRINT:GabWilCio19} and \cite{CCS:MBKM19}, are non-interactive protocols
built using the Fiat--Shamir transform on an interactive proof of knowledge.
In the following, we denote the underlying protocols by $\plonkprot$ (for
$\plonk$) and $\sonicprot$ (for $\sonic$) and the resulting, non-interactive
ones, by $\plonkprotfs$ and $\sonicprotfs$, respectively. 

\subsubsection{Special soundness.} 
To be able to follow \cite{INDOCRYPT:FKMV12} it had to be shown that
$\plonkprot$ and $\sonicprot$ are special-sound.  However the standard definition of special
soundness could not be met.  First of
all, the standard definition requires extraction of a witness from just two
transcripts, each containing three messages, that share the first message. For
$\plonkprot$ and $\sonicprot$ that gives nothing. The definition had to be tuned to cover
protocols that have more rounds than just three.  Furthermore, the number of
transcripts required is much greater---$\numberofconstrains + 3$, where
$\numberofconstrains$ is the number of constrains in the proven circuit, for
$\plonkprot$ and $\multconstr + \linconstr + 1$, where $\multconstr$ and
$\linconstr$ are the numbers of multiplicative and linear constraints,
for$\sonicprot$. Hence, we do not have a \emph{pair of transcripts}, but \emph{tree of transcripts}.

Secondly,both protocols rely on common reference strings which have trapdoors
that allow an adversary who knows them to
produce multiple proofs even without knowing the witness. Recall that the
standard special soundness definition requires witness extraction from
\emph{any} pair of acceptable transcripts that share a common root. Thus, the
definition cannot be met. Fortunately, one could define a weaker version of special
soundness, i.e.~a computational special soundness. More precisely, we show that
either it is possible to extract a witness from a tree of acceptable transcripts
or one could use the adversary that produced the tree to break some underlying
computational assumption.

% \subsubsection{Unique opening property of polynomial commitment schemes}%
% \label{ssub:unique_opening_property_of_pcom}
% Another

\subsubsection{Unique response property.} Another property that
has to be proven for the protocols is the unique response property
which states, as expressed in \cite{C:Fischlin05}, that except the first round, all messages sent
by the prover are deterministic. As previously, we also could not use this
definition right out of the box as $\plonkprot$ does not follow it---both first
and the second prover's messages are randomised. We thus propose a
generalisation of the definition which states that a protocol is $\ur{i}$ if the
prover is deterministic after sending its $i$-th message. This property is
fulfilled by $\plonkprot$ with $i = 3$.

To be able to show the unique response property (for both of he protocols) we
also had to show that the modifications of the KZG \cite{AC:KatZavGol10}
proposed in \cite{EPRINT:GabWilCio19,CCS:MBKM19} have a \emph{unique opening
property}. More precisely, for a polynomial $\p{f}(X)$ evaluated at some
point $z$ it should be infeasible for any PPT adversary to provide
two different but acceptable openings of the commitment.

\subsubsection{HVZK.}%
In order to show our result we also show that (interactive) $\plonkprot$ and
$\sonicprot$ are
honest verifier zero-knowledge in the standard model, i.e.~the simulator is
able to produce a transcript indistinguishable from a transcript produced by a
honest prover and verifier without any additional knowledge.
Although both $\sonic$ and $\plonk$ are shown to be zero-knowledge, the proofs
provided by their authors rely on CRS's trapdoors. For our reduction to work,
we cannot assume any additional power of the simulator. That is, any PPT
party should be able to produce a simulated proof. (Note that this property
does not necessary break soundness of the protocol as the simulator is
required only to produce a transcript and is not involved in a real
conversation with a real verifier).

\subsubsection{Generalisation of the general forking lemma.}
Consider an interactive $3$-round special-sound protocol $\ps$ and its
non-interactive version $\ps_\fs$ obtained by using the Fiat--Shamir
transform.  The general forking lemma provides an instrumental lower bound for
probability of extracting a witness from two proofs for the same statement
that share the first message.  Since $\plonkprot$ and $\sonicprot$ have more
than $3$ rounds and are not special-sound, the forking lemma cannot be used
directly. We thus propose its modification that covers protocols that have
more rounds and require more transcript than merely two.  Unfortunately, the
security gap grows with the number of transcript and probability that the
extractor succeeds diminishes significantly. (That said, we have to note that
the security loss is polynomial, albeit big.)

We observe here that some modern zkSNARKs rely on the Fiat--Shamir transform
and the forking lemma heavily. First, an interactive protocol is proposed and
its security and special-soundness analysed; second, one uses an argument that
the Fiat--Shamir transform can be used to get a protocol that is
non-interactive and shares the same security properties.  We claim here that
such an analysis is not enough and one has to consider the security loss
implied by the generalisation of the forking lemma or disclose a
transformation that does not suffer from the generalisation inefficiency.

\subsubsection{Towards simulation-extractability.} Given modified, less
restrictive, definition for special-soundness and the unique response property, and
generalised forking lemma we are able to show the announced result---simulation
extractability of $\plonkprotfs$ and $\sonicprotfs$. The proof is inspired by
simulation-extractability and simulation-soundness proofs from
\cite{INDOCRYPT:FKMV12}, with major modifications, which were required as
\cite{INDOCRYPT:FKMV12} considers (Fiat--Shamir transformed) sigma-protocols
that are undoubtedly simpler protocols than the considered proof systems.
% Since the proof highly relies on the (generalised) forking lemma, the security
% lost it introduces is considerable.

% \subsubsection{Efficient simulation-soundness.}
% Given that the security reduction for simulation-extractability introduces a security gap we also present a proof for $\plonkprot_\fs$ simulation soundness which utilises the algebraic group model and is tight. 
% It remains an open question how to show simulation extractability tightly, e.g.~using AGM.

\subsubsection{Interactive zero knowledge vs non-interactive zero knowledge.}
Another issue we tackle with is a question whether NIZK proof systems are in
fact zero-knowledge, see e.g.~\cite{C:Pass03}. This problem was raised as one
could observe that a NIZK proof system has a property alien to interactive
proofs---a verifier who obtains a proof $\zkproof$ for a statement $\inp \in
\LANG$ inevitably learns how to prove this statement. More precisely, he can
just reuse the obtained proof $\zkproof$. This makes the verifier learn
undoubtedly more than simply the veracity of the proven statement. On the other
hand, the verifier learns a particular proof $\zkproof$ for a concrete CRS
$\crs$ only. However, the CRS generator is considered trusted, thus the proof
$\zkproof$ could be considered as a proof for $\inp$ generally, regardless the
CRS.

Simulation-extractable updatable NIZKs---and zkSNARKs in particular---tighten
the gap between interactive zero knowledge and non-interactive zero knowledge in
the CRS model. First, simulation-extractability assures that no adversary can
maul an existing proof and make a fresh one, what limits replay attacks.
Second, updatable NIZKs does not assume that there is a trusted party that
provides a CRS, but rather a sequence of parties (called updaters) that modify
it. The security model is radically different as now one believes in the
veracity of a proof only if she trusts that there is at least one honest party
between the updaters. 

We believe that our result may be useful for designing new zkSNARKs, especially those based on a polynomial commitment schemes~\cite{AC:KatZavGol10}, as it shows that a careful protocol design may give it a strong security notion for free.

\section{Preliminaries}
Let $\ppt$ denote probabilistic polynomial-time and $\secpar \in \NN$ be the
security parameter.  All adversaries are stateful.  For an algorithm $\adv$,
let $\image (\adv)$ be the image of $\adv$ (the set of valid outputs of
$\adv$), let $\RND{\adv}$ denote the random tape of $\adv$ (assuming the given
value of $\secpar$), and let $r \sample \RND{\adv}$ denote the random choice
of the randomiser $r$ from $\RND{\adv}$.  We denote by $\negl$ ($\poly$) an
arbitrary negligible (resp.~polynomial) function.

Distributions $X$ and $Y$ have \emph{statistical distance} $\SD$ equal
$\epsilon$ if $\sum_{a \in \supp{X \cup Y}} \abs{\prob{X = a} - \prob{Y = a}}
= \epsilon$.  We write $X \approx_\secpar Y$ if $\SD(X, Y) \leq \negl$, where
$\SD$ is the statistical distance between the distributions.  For values $a$
and $b$ we write $a \approx_\secpar b$ if $\abs{a - b} \leq \negl$.

Denote by $\RELGEN = \smallset{\REL}$ a family of relations. We assume that if
$\REL$ comes with any auxiliary input, it is benign. Directly from the
description of $\REL$ one learns security parameter $\secpar$ and other
necessary information like public parameters $\pp$ containing description of a
group $\GRP$, if the relation is a relation of group elements (as it usually
is in case of zkSNARKs).

\paragraph{Bilinear groups.}
A bilinear group generator $\pgen (\secparam)$ returns public parameters $ \pp
= (p, \GRP_1, \GRP_2, \GRP_T, \pair, \gone{1}, \gtwo{1})$, where $\GRP_1$,
$\GRP_2$, and $\GRP_T$ are additive cyclic groups of prime order $p =
2^{\Omega (\secpar)}$, $\gone{1}, \gtwo{1}$ are generators of $\GRP_1$,
$\GRP_2$, resp., and $\pair: \GRP_1 \times \GRP_2 \to \GRP_T$ is a
non-degenerate $\ppt$-computable bilinear pairing.  We assume the bilinear
pairing to be Type-3, i.e., that there is no efficient isomorphism from
$\GRP_1$ to $\GRP_2$ or from $\GRP_2$ to $\GRP_1$.  We use the by now standard
bracket notation, i.e., we write $\bmap{a}{\gi}$ to denote $a g_{\gi}$ where
$g_{\gi}$ is a fixed generator of $\GRP_{\gi}$.  We denote $\pair (\gone{a},
\gtwo{b})$ as $\gone{a} \bullet \gtwo{b}$.  Thus, $\gone{a} \bullet \gtwo{b} =
\gtar{a b}$.  We freely use the bracket notation with matrices, e.g., if
$\vec{A} \vec{B} = \vec{C}$ then $\vec{A} \grpgi{\vec{B}} = \grpgi{\vec{C}}$
and $\gone{\vec{A}}\bullet \gtwo{\vec{B}} = \gtar{\vec{C}}$.
Since every algorithm $\adv$ takes as input the public parameters we
skip them when describing $\adv$'s input. Similarly, we do not explicitly
state that each protocol starts with generating these parameters by $\pgen$.

\paragraph{Computational assumptions.}
Security of $\plonk$ and $\sonic$ rely on two discrete-log based security
assumptions---$(q_1, q_2)$-dlog assumption and its extended with
negative exponents version $(q_1,
q_2)$-ldlog assumption\footnote{Note that \cite{CCS:MBKM19} dubs their
	assumption \emph{a dlog assumption}. We changed that name to distinct it
from the more standard dlog assumption used in \cite{EPRINT:GabWilCio19}.
``l'' in \emph{ldlog} relates to use of Laurent polynomials in the assumption.}.
\begin{definition}[$(q_1, q_2)\mhyph\dlog$ assumption]
	Let $\adv$ be a $\ppt$ adversary that gets as input $\gone{1, \chi, \ldots, \chi^{q_1}}, \gtwo{1, \chi, \ldots, \chi^{q_2}}$, for some randomly picked $\chi \in \FF_p$, then
	\[
		\condprob{\chi \gets \adv(\gone{1, \chi, \ldots, \chi^{q_1}}, \gtwo{1, \chi, \ldots, \chi^{q_2} })}{\chi \sample \FF_p} \leq \negl.
	\]
\end{definition}

\begin{definition}[$(q_1, q_2)\mhyph\ldlog$ assumption]
		Let $\adv$ be a $\ppt$ adversary that gets as input $\gone{\chi^{-q_1},
		\ldots, 1, \chi, \ldots, \chi^{q_1}}, \gtwo{\chi^{-q_2}, \ldots, 1, \chi, \ldots, \chi^{q_2}}$, for some randomly picked $\chi \in \FF_p$, then
	\[
			\condprob{\chi \gets \adv(\gone{\chi^{-q_1}, \ldots, 1, \chi, \ldots,
			\chi^{q_1}}, \gtwo{\chi^{-q_2}, \ldots, 1, \chi, \ldots, \chi^{q_2} })}{\chi \sample \FF_p} \leq \negl.
	\]
\end{definition}

\paragraph{Proofs by Game-Hoping.}
Proofs by \emph{game hoping} is a method of writing proofs popularised by e.g.~Shoup \cite{EPRINT:Shoup04} and Dent \cite{EPRINT:Dent06c}. The method relies on the following lemma.

\begin{lemma}[Difference lemma,  cf.~{\cite[Lemma 1]{EPRINT:Shoup04}}]
	\label{lem:difference_lemma}
	Let $\event{A}, \event{B}, \event{F}$ be events defined in some probability distribution, and suppose that $\event{A} \land \nevent{F} \iff \event{B} \land \nevent{F}$. 
	Then 
	\[
		\abs{\prob{\event{A}} - \prob{\event{B}}} \leq \prob{\event{F}}\,.
	\]
\end{lemma}

\subsection{Polynomial commitment.}
\label{sec:poly_com}
In the polynomial commitment scheme $\PCOM = (\kgen, \com, \open, \verify)$
the committer $\committer$ can prove to the receiver $\receiver$ that some
polynomial $\p{f}$ that $\committer$ committed to evaluates to $s$ as some point
$z$ chosen by $\receiver$.

$\plonk$ and $\sonic$ use variants of the KZG polynomial commitment scheme. We denote the first by $\PCOMp$ and present in \cref{fig:pcomp} and the latter by $\PCOMs$ and present in \cref{fig:pcoms}.

\begin{figure}[t!]
	\begin{pcvstack}[center,boxed]
		\begin{pchstack}
			\procedure{$\kgen(\secparam)$}
			{
			\chi \sample \FF^2_p \\ [\myskip]
			\pcreturn \gone{1, \ldots, \chi^{\numberofconstrains + 2}}, \gtwo{\chi}\\ [\myskip]
				\hphantom{\hspace*{5.5cm}}	
				%\hphantom{\pcind \p{o}_i(X) \gets \sum_{j = 1}^{t_i} \gamma_i^{j - 1} \frac{\p{f}_{i,j}(X) - \p{f}_{i, j}(z_i)}{X - z_i}}
			}
			
			\pchspace
			
			\procedure{$\com(\crs, \vec{\p{f}}(X))$}
			{ 
				\pcreturn \gone{\vec{c}} = \gone{\vec{\p{f}}(\chi)}\\ [\myskip]
				\hphantom{\pcind \pcif 
					\sum_{i = 1}^{\abs{\vec{z}}} r_i \cdot \gone{\sum_{j = 1}^{t_j}
					\gamma_i^{j - 1} c_{i, j} - \sum{j = 1}^{t_j} s_{i, j}} \bullet
				\gtwo{1} + }
			}
		\end{pchstack}
		% \pcvspace
		\begin{pchstack}
			\procedure{$\open(\crs, \vec{\gamma}, \vec{z}, \vec{s}, \vec{f}(X))$}
			{
			\pcfor i \in \range{1}{\abs{\vec{z}}} \pcdo\\ [\myskip]
			\pcind \p{o}_i(X) \gets \sum_{j = 1}^{t_i} \gamma_i^{j - 1} \frac{\p{f}_{i,j}(X) - \p{f}_{i, j}(z_i)}{X - z_i}\\ [\myskip]
				\pcreturn \vec{o} = \gone{\vec{\p{o}}(\chi)}\\ [\myskip]
				\hphantom{\hspace*{5.5cm}}	
			}
			
			\pchspace
			
			\procedure{$\verify(\crs, \gone{c}, \vec{z}, \vec{s}, \gone{\p{o}(\chi)})$}
			{
				\vec{r} \gets \FF_p^{\abs{\vec{z}}}\\ [\myskip]
				\pcfor i \in \range{1}{\abs{\vec{z}}} \pcdo \\ [\myskip]
				\pcind \pcif 
					\sum_{i = 1}^{\abs{\vec{z}}} r_i \cdot \gone{\sum_{j = 1}^{t_j} \gamma_i^{j - 1} c_{i, j} - \sum{j = 1}^{t_j} s_{i, j}} \bullet \gtwo{1} + \\ [\myskip]
					\pcind \sum_{i = 1}^{\abs{\vec{z}}} r_i z_i o_i
					\bullet \gtwo{1} \neq 
					\gone{- \sum_{i = 1}^{\abs{\vec{z}}} r_i o_i } \bullet \gtwo{\chi} \pcthen  \\
					\pcind \pcreturn 0\\ [\myskip]
					\pcreturn 1.
			}
		\end{pchstack}
	\end{pcvstack}
	\caption{$\PCOMp$ polynomial commitment scheme}
	\label{fig:pcomp}
\end{figure}

\begin{figure}[t!]
	\begin{pcvstack}[center,boxed]
		\begin{pchstack}
			\procedure{$\kgen(\secparam)$}
			{
				\alpha, \chi \sample \FF^2_p \\ [\myskip]
				\pcreturn \\
			%	\hphantom{\pcind \p{o}_i(X) \gets \sum_{j = 1}^{t_i} \gamma_i^{j - 1} \frac{\p{f}_{i,j}(X) - \p{f}_{i, j}(z_i)}{X - z_i}}
				\hphantom{\hspace*{5.5cm}}	
		}
			
			\pchspace
			
			\procedure{$\com(\crs, \p{f}(X))$}
			{
				\p{c}(X) \gets \alpha \cdot X^{\dconst - \maxconst} \p{f}(X) \\ [\myskip]
				\pcreturn \gone{c} = \gone{\p{c}(\chi)}\\ [\myskip]
				\hphantom{\pcind \pcif 
					\sum_{i = 1}^{\abs{\vec{z}}} r_i \cdot \gone{\sum_{j = 1}^{t_j}
					\gamma_i^{j - 1} c_{i, j} - \sum{j = 1}^{t_j} s_{i, j}} \bullet
				\gtwo{1} + }
			}
		\end{pchstack}
		%\pcvspace
		\begin{pchstack}
			\procedure{$\open(\crs, z, s, f(X))$}
			{
				\p{o}(X) \gets \frac{\p{f}(X) - \p{f}(z)}{X - z}\\ [\myskip]
				\pcreturn \gone{\p{o}(\chi)}\\ [\myskip]
				\hphantom{\hspace*{5.5cm}}	
			}
			
			\pchspace
			
			\procedure{$\verify(\crs, \gone{c}, z, s, \gone{\p{o}(\chi)})$}
			{
				\pcif \gone{\p{o}(\chi)} \bullet \gtwo{\alpha \chi} + \gone{s - z \p{o}(\chi)} \bullet \gtwo{\alpha} = \\ [\myskip]
				\pcind \gone{c} \bullet \gtwo{\chi^{- \dconst + \maxconst}} \pcthen  \pcreturn 1\\ [\myskip]
				\rlap{\pcelse \pcreturn 0.}
				\hphantom{\pcind \pcif 
					\sum_{i = 1}^{\abs{\vec{z}}} r_i \cdot \gone{\sum_{j = 1}^{t_j}
					\gamma_i^{j - 1} c_{i, j} - \sum{j = 1}^{t_j} s_{i, j}} \bullet
				\gtwo{1} + }
			}
		\end{pchstack}
	\end{pcvstack}
	
	\caption{$\PCOMs$ polynomial commitment scheme}
	\label{fig:pcoms}
\end{figure}

% We require $\PCOM$ to have the following properties:
We emphasize the following properties of a secure polynomial commitment
$\PCOM$:
\begin{description}
	\item[Evaluation binding] a PPT adversary $\adv$ who output a commitment $c$
		to a polynomial $\p{f}$ and is given an evaluation point $z$ has at most
negligible chances to correctly open the commitment to a value different than
$\p{f}(z)$. 
\end{description}
	
We say that $\PCOM$ has unique opening property if the following holds:
\begin{description}
	\item[Opening uniqueness] 
	Let 
	$k \in \NN$ be the number of committed polynomials,
	$l \in \NN$ number of evaluation points,
	$\vec{c} \in \GRP^k$ be the commitments,  
	$\vec{z} \in \FF_p^l$ be the attributes the polynomials are evaluated at, $\vec{s} \in \FF_p^k$ the evaluations, and 
	$\vec{o} \in \FF_p^l$ be the commitment openings. 
	Then for every $\ppt$ adversary $\adv$	
	\[
		\Pr
			\left[
			\begin{aligned}
				& \verify(\crs, \vec{c}, \vec{z}, \vec{s}, \vec{o}),  \\ 
				& \verify(\crs, \vec{c}, \vec{z}, \vec{s'}, \vec{o'}), \\
				& \vec{o} \neq \vec{o'}
			\end{aligned}
			\,\left|\,
			\begin{aligned}
				& \crs \gets \kcrs(\secparam),\\
				& (\vec{c}, \vec{z}, \vec{s}, \vec{s'}, \vec{o}, \vec{o'}) \gets \adv(\crs)
			\end{aligned}
			\right.\right] \leq \negl\,.
	\]
\end{description}
Intuitively, opening uniqueness assures that there is only one valid opening
for the committed polynomial and given evaluation point. This property is
crucial in showing simulation-extractability of $\plonk$ and $\sonic$. We show
that the $\plonk$'s and $\sonic$'s polynomial commitment schemes satisfy this
requirement in \cref{lem:pcomp_unique_op} and \cref{lem:pcoms_unique_op}
respectively. 

\subsection{Algebraic Group Model}
The algebraic group model (AGM) introduced in \cite{C:FucKilLos18} lies
between the standard model and generic bilinear group model. In the AGM it is
assumed that an adversary $\adv$ can output a group element $\gnone{y} \in
\GRP$ if $\gnone{y}$ has been computed by applying group operations to group
elements given to $\adv$ as input. It is further assumed, that $\adv$ knows
how to ``build'' $\gnone{y}$ from that elements. More precisely, the AGM
requires that whenever $\adv(\gnone{\vec{x}})$ outputs a group element
$\gnone{y}$ then it also outputs $\vec{c}$ such that $\gnone{y} = \vec{c}^\top
\cdot \gnone{\vec{x}}$.  It is worth to note that both $\plonk$ and $\sonic$
have been shown secure using the AGM. 

\subsection{Zero knowledge}
In a zero-knowledge proof system, a prover convinces the verifier of
the veracity of a statement without leaking any side information except that the
statement is true.
Here, we focus on proof systems that guarantee soundness against a $\ppt$
cheating prover.
The zero-knowledge property is proven by constructing a simulator that can
simulate the view of a cheating verifier without knowing the secret
information---witness---of the prover.

More precisely, let $\RELGEN(\secparam) = \smallset{\REL}$ be a family of
$\npol$ relations.
Denote by $\LANG_\REL$ the language determined by $\REL$.
Let $\prover$ and $\verifier$ be I$\ppt$ algorithms, the former called \emph{prover}
and the latter \emph{verifier}. We allow our proof system to have a setup,
i.e.~there is a $\kgen$ algorithm that takes as input the relation $\REL$ and
outputs a common reference string $\crs$.
We denote by $\ip{\prover(\REL, \crs, \inp, \wit)}{\verifier(\REL,
\crs,\inp)}$ a transcript $\trans$ (or, proof $\pi$--we use these two
alternatively) of a conversation between a $\prover$ with input
$(\REL, \crs, \inp, \wit)$ and $\verifier$ with input $(\REL, \crs, \inp)$.
We write $\ip{\prover (\REL, \crs, \inp, \wit)}{\verifier(\REL, \crs, \inp)} =
1$ if in the end of the transcript the verifier $\verifier$ returns $1$ and
say that $\verifier$ accepts the transcript. We sometimes abuse notation and
write $\verifier(\trans) = 1$ to denote a fact that $\trans$ is accepted by
the verifier. 

\markulf{09.07.20}{Could also all quantify for Soundness and Zero knowledge?}
\michals{09.07.20}{I am not sure what you meant here. You want to have for all $\inp, \wit \in \REL$ or you doubt these properties make sense for all $\REL \gets \RELGEN$?}
A proof system $\proofsystem = (\kgen, \prover, \verifier, \simulator)$ for $\RELGEN$ is required to have three properties: completeness, soundness and zero knowledge, which are defined as follows:
\begin{description}
	\item[Completeness] An interactive proof system $\proofsystem$ is
			\emph{complete} if an honest prover always convinces an honest verifier,
			that is for all $\REL \in \RELGEN(\secparam)$ and $(\inp, \wit) \in \REL$
	\[
		\condprob{\ip{\prover (\REL, \crs, \inp, \wit)}{\verifier (\REL, \crs,
		\inp)} = 1}{\crs \gets \kgen(\REL)} = 1\,.
	\]
	\item[Soundness] We say that $\proofsystem$ for $\RELGEN$ is \emph{sound} if
			no $\ppt$ prover $\adv$ can convince an honest verifier $\verifier$ to
			accept a proof for a false statement, i.e~for $\inp \not\in\LANG$. More
			precisely, for all $\REL \in \RELGEN(\secparam)$
	\[
		\condprob{\ip{\adv(\REL, \crs, \inp)}{\verifier(\REL, \crs, \inp)} =
		1}{\crs \gets \kgen(\REL), \inp \gets \adv(\REL, \crs); \inp \not\in \LANG_\REL} \leq \negl\,;
	\]
	\item[Zero knowledge] We call an interactive proof system $\proofsystem$
		\emph{zero-knowledge} if for any $\REL \in \RELGEN(\secparam)$, $(\inp,
		\wit) \in \REL$, and adversary $\adv$ there exists a $\ppt$ simulator $\simulator$ such that
	\begin{multline*}
	  \left\{\ip{\prover(\REL, \crs, \inp, \wit)}{\adv(\REL, \crs, \inp, \wit)}
	\,\left|\, \crs \gets \kgen(\REL)\COMMENT{, (\inp, \wit) \gets \adv(\REL,
\crs)}\vphantom{\simulator^\adv}\right.\right\} \approx_\secpar
		\\
		\left\{\simulator^{\adv}(\REL, \crs, \inp)\,\left|\, \crs \gets
\kgen(\REL)\COMMENT{, (\inp, \wit) \gets \adv(\REL,
\crs)}\vphantom{\simulator^\adv}\right.\right\}\,.  
\end{multline*}
	%
	We call zero knowledge \emph{perfect} if the distributions are equal and
	\emph{computational} if they are indistinguishable for any $\nuppt$ distinguisher.

	Occasionally, a weaker version of zero knowledge is required, so called
	\emph{honest verifier zero knowledge} (HVZK), where it is assumed that the
	verifier's challenges are picked at random from some predefined set.
	Although weaker, this definition suffices in many applications. Especially,
	an interactive zero-knowledge proof that is HVZK and \emph{public-coin}
	(i.e.~the verifier outputs as challenges its random coins) can be made
	non-interactive in the random oracle model by using the Fiat--Shamir
	transformation.
\end{description}
	
	Sometimes a stronger notion of soundness is required---except requiring that the verifier rejects proofs of statements outside the language, we request from the prover to know a witness corresponding to the proven statement. This property is formalised by the following notion:
\begin{description}
	\item[Knowledge soundness] We call an interactive proof system $\proofsystem$
			\emph{knowledge-sound} if for any $\REL \in \RELGEN(\secparam)$ and a $\ppt$ adversary $\adv$
	% \begin{multline*}
	\[
	\Pr\left[
		\begin{aligned}
			& \verifier(\REL, \crs, \inp, \trans) = 1 \\
			& \land \REL(\inp, \wit) = 0
	 \end{aligned}
	  \,\left|\,
	 \begin{aligned}
		 & \crs \gets \kcrs(\REL), \inp \gets \adv(\REL, \crs), \\
		 & (\wit, \trans) \gets \ext^{\ip{\adv(\REL, \crs, \inp)}{\verifier(\REL, \crs, \inp)}}(\REL, \inp)
	 \end{aligned}
	 \vphantom{\begin{aligned}
		 \adv (\trans) = 1, \\
		 \text{if $\trans{}$ is accepting} \\
		 \pcind \text{then $\REL(\inp, \wit)$}
	 \end{aligned}}\right.
	 \right] \leq \negl\,,
 % \end{multline*}
 \]
\end{description}

\paragraph{NIZKs in the Random Oracle Model.}
In NIZKs in the Random Oracle Model we distinguish, for the sake of clarity, two simulators, one denoted by $\simulator_\zkproof$ that is responsible for providing simulated proofs and $\simulator_\ro$ that picks a random oracle instantiation and takes care of all parties' queries to $\ro$.
% \michals{9.06}{Should we distinguish two simulators or just pack everything into a single one?}
For the sake of consistency (with random oracle-free NIZKs) we use $\simulator$
to denote the pair of state-sharing simulators $\simulator_\zkproof,
\simulator_\ro$.

\paragraph{Sigma protocols.}
A sigma protocol $\sigmaprot = (\prover, \verifier, \simulator)$  for a relation
$\REL \in \RELGEN(\secparam)$  is a special case of an interactive proof which transcript compounds of three messages $(a, b, z)$, the middle being a challenge provided by the verifier.
Sigma protocols are honest verifier zero-knowledge and specially-sound. That
is, there exists an extractor $\ext$ which given two accepting transcripts $(a, b, z)$, $(a, b', z')$ for a statement $\inp$ can recreate the corresponding witness if $b \neq b'$. Formally,
\begin{description}
	\item[Special soundness] A sigma protocol $\sigmaprot$ is \emph{specially-sound} if for any adversary $\adv$ the probability
	\[
		\Pr\left[
		\begin{aligned}
				& \wit \gets \ext(\REL, \inp, (a, b, z), (a, b', z')),\\
				& \REL(\inp, \wit) = 0
		\end{aligned}
		\,\left|\,
		\begin{aligned}
			& (\inp, (a, b, z), (a, b', z')) \gets \adv(\REL), \\
			& \verifier(\REL, \inp, (a, b, z)) = \\
			& \qquad = \verifier(\REL, \inp, (a, b', z')) = 1, \\
		\end{aligned}
		\right.\right]
	\]
	is negligible in $\secpar$.
\end{description}

% Furthermore sigma protocols are honest verifier zero-knowledge.
%That is the zero-knowledge property holds only for honest verifiers, what is
%formalized as follows: \begin{description}
%	\item[Honest verifier zero knowledge] A sigma protocol $\sigmaprot$ is \emph{honest verifier zero-knowledge} if for all adversaries $\adv$ holds
%	\begin{multline*}
%		\left\{\ip{\prover(\REL, \crs, \inp, \wit)}{\verifier(\REL, \inp)} \,\left|\, (\inp, \wit) \gets \adv(\REL)\vphantom{\simulator_\zkproof^\adv}\right.\right\} \approx_\secpar
%		\left\{\simulator_\zkproof^{\verifier}(\REL, \inp)\,\left|\, (\inp, \wit) \gets \adv(\REL)\vphantom{\simulator^\adv}\right.\right\}\,.
%	\end{multline*}
%\end{description}
%Although this notion is weaker than a standard zero knowledge it is often sufficient. Furthermore, a HVZK interactive proof system transformed by the Fiat--Shamir transformation is zero-knowledge.

Another property that sigma protocols sometimes have is, introduced by Fischlin \cite{C:Fischlin05}, a unique response property which states that no $\ppt$ adversary  can produce two accepting transcripts that differ only on the last element.
More precisely,
\begin{description}
	\item[Unique response property] Let $\sigmaprot = (\prover, \verifier,
			\simulator)$ be a sigma-protocol for $\REL \in \RELGEN(\secparam)$ which
			proofs compound of three messages $(a, b, z)$. We say that
			$\sigmaprot$ is has a unique response property if for all $\ppt$
			algorithms $\adv$ holds 
			\[
				\condprob{\verifier (a, b, z) = \verifier (a, b, z')  = 1}{(a, b, z,
				z') \gets \adv(\REL)} \leq \negl\,.
			\]
\end{description}
(If the unique response property holds even against unbounded adversaries, we call it \emph{strict}, cf.~\cite{INDOCRYPT:FKMV12}.)
Later on we often call protocols that follows this notion \emph{ur-protocols}.
For the sake of completeness we note that many sigma-protocols, like e.g.~Schnorr's protocol \cite{C:Schnorr89}, fulfil this requirement.

% \paragraph{Zero knowledge proof system in the CRS model.}
% Many proof systems additionally compounds of a setup algorithm $\kcrs$ that on input $\REL$ outputs a common reference string (CRS) $\crs$. The common reference string comes with a corresponding trapdoor $\td$ that allows the simulator to simulate a proof.

\subsection{Simulation extractable NIZKs from sigma protocols}
Real life applications often require from a NIZK proof system to be
non-malleable. That is, no adversary seeing a proof $\zkproof$ for a statement
$\inp$ should be able to provide a new proof $\zkproof'$ related to
$\zkproof$.  A strong version of non-malleability is formalised by simulation
extractability.  This notion states that no adversary can produce valid proof
without knowing the corresponding witness. This must hold even if the
adversary is allowed to see polynomially many simulated proofs for any
statements it wishes.

\begin{definition}[Simulation-extractable NIZK]
	\label{def:simext}
	Let $\ps = (\kgen, \prover, \verifier, \simulator)$ be a
	computationally special-sound HVZK proof and $\ps_\fs = (\kgen_\fs,
	\prover_\fs, \verifier_\fs, \simulator_\fs)$ be $\ps$ transformed by the
	Fiat--Shamir transform.  We say that $\ps_\fs$ is simulation extractable
	with \emph{extraction error} $\nu$ if for any $\ppt$ adversary $\adv$ that
	is given oracle access to a random oracle $\ro$ and simulator
	$\simulator_\fs$, and produces an accepting transcript of $\ps$ with
	probability
	$\waccProb$, that is
	\[
		\waccProb = \Pr \left[
		\begin{aligned}
			& \verifier_\fs(\REL, \crs, \inp_\advse, \zkproof_\advse) = 1,\\
			& (\inp_\advse, \zkproof_\advse) \not\in Q
		\end{aligned}
		\, \left| \,
		\begin{aligned}
			& \crs \gets \kgen(\REL),\\
			& (\inp_\advse, \zkproof_\advse) \gets \advse^{\simulator_\fs, \ro} (\REL, \crs) 
		\end{aligned}
		\right.\right]\,,
	\]
	probability
	\[
		\frkProb = \Pr \left[
		\begin{aligned}
			& \verifier_\fs(\REL, \crs, \inp_\advse, \zkproof_\advse) = 1,\\
			& (\inp_\advse, \zkproof_\advse) \not\in Q,\\
			& \REL(\inp_\advse, \wit_\advse) = 0
		\end{aligned}
		\, \left| \,
		\begin{aligned}
			& \crs \gets \kgen(\REL),\\
			& (\inp_\advse, \zkproof_\advse) \gets \advse^{\simulator_\fs, \ro} (\REL, \crs) \\
			& \wit_\advse \gets \ext_\ss (\trans_{\advse}) 
		\end{aligned}
		\right.\right]
	\]
	is at at least 
	\[
		\frkProb \geq \frac{1}{\poly} (\waccProb - \nu)^d - \eps(\secpar)\,,
	\]
	for some polynomial $\poly$, constant $d$ and negligible $\eps$ whenever $\waccProb \geq \nu$.
\end{definition}

Consider a sigma protocol $\sigmaprot = (\prover, \verifier, \simulator)$ that
is specially sound and has a unique response property. Let $\sigmaprot_\fs = (\prover_\fs, \verifier_\fs, \simulator_\fs)$ be a NIZK obtained by applying the Fiat--Shamir transform to $\sigmaprot$.
Faust et al.~\cite{INDOCRYPT:FKMV12} show that every such $\sigmaprot_\fs$ is simulation-extractable.

\begin{theorem}[Simulation extractability of the Fiat--Shamir transform \cite{INDOCRYPT:FKMV12}]
	Let $\sigmaprot = (\prover, \verifier, \simulator_\zkproof)$ be a non-trivial sigma protocol with unique responses for a language $\LANG \in \npol$.
	In the random oracle model, the NIZK proof system $\sigmaprot_\fs = (\prover_\fs, \verifier_\fs, \simulator_{\fs})$ resulting by applying the Fiat--Shamir transform to $\sigmaprot$ is simulation extractable with extraction error $\eta = q/h$ for the simulator $\simulator$. Here, $q$ is the number of random oracle queries and $h$ is the number of elements in the range of $\ro$.
	% Furthermore, the extractor $\ext_\adv$ needs to run $\adv^{\simulator_{\fs, \ro}}, \adv^{\simulator_{\fs, \zkproof}}$ twice.
\end{theorem}

The theorem relies on the following classical lemma, called \emph{General forking lemma} \cite{JC:PoiSte00}.

\begin{lemma}[General forking lemma, cf.~\cite{INDOCRYPT:FKMV12,CCS:BelNev06}]
	\label{lem:forking_lemma}
	Fix $q \in \ZZ$ and a set $H$ of size $h > 2$. Let $\adv$ be a $\ppt$ algorithm that on input $y, h_1, \ldots, h_q$ returns $(i, s)$, where $i \in\range{0}{q}$ and $s$ is called a \emph{side output}.
	Denote by $\ig$ a randomised instance generator.
	We denote by $\accProb$ the probability
	\[
		\condprob{i > 0}{y \gets \ig; h_1, \ldots, h_q \sample H; (i, s) \gets \adv(y, h_1, \ldots, h_q)}\,.
	\]
	Let $\forking_\adv(y)$ denote the algorithm described in \cref{fig:forking_lemma}, then the probability $\frkProb$ defined as
	$
		\frkProb := \condprob{b = 1}{y \gets \ig; (b, s, s') \gets \forking_{\adv}(y)}
	$
	holds
	\[
		\frkProb \geq \accProb \brak{\frac{\accProb}{q} - \frac{1}{h}}\,.
	\]
	%
	\begin{figure}[t]
		\centering
		\fbox{
		\procedure{$\forking_\adv (y)$}
		{
			r \sample \RND{\adv}\\
			h_1, \ldots, h_q \sample H\\
			(i, s) \gets \adv(y, h_1, \ldots, h_q; r)\\
			\pcif i = 0\ \pcreturn (0, \bot, \bot)\\
			h'_{i}, \ldots, h'_{q} \sample H\\
			(i', s') \gets \adv(y, h_1, \ldots, h_{i - 1}, h'_{i} h'_{q}; r)\\
			\pcif (i = i') \land (h_{i} \neq h'_{i})\ \pcreturn (1, s, s')\\
			\pcind \pcelse \pcreturn (0, \bot, \bot)
		}}
		\caption{Forking algorithm $\forking_\adv$}
		\label{fig:forking_lemma}
\end{figure}
\end{lemma}
%
In case of a sigma protocol, the probability $\frkProb$ can be interpreted as
a lower bound for a successful witness extraction from two transcripts.  Let
$\trans_1 = (\inp, a, b, z)$ and $\trans_2 = (\inp, a, b', z')$ be the
transcripts.  Both $\trans_1$ and $\trans_2$ have to be \emph{acceptable},
i.e.~$i > 0$ and the probability that $\adv$ makes an acceptable transcript is
denoted by $\accProb$.  Index $i$ can be interpreted as an index of $h_i$
which was sent as a challenge for $(\inp, a)$, this index has to be guessed by
the security reduction.  For the sake of extractability, both transcripts have
to have the same index $i$, i.e.~the same instance $\inp$ and the first
message $a$, but the actual challenges $b = h_i$ and $b' = h'_{i}$ have to
differ.

\section{Towards simulation extractability for multi-round protocols---definitions and lemmas}
Unfortunately, Faust et al.'s result cannot be directly applied in our case since the protocols we consider have more than three rounds of interaction, require more than just two transcript for the extractor to work and $\plonkprot$ is not special sound.

\subsection{Generalised forking lemma.}
First of all, although dubbed ``general'', \cref{lem:forking_lemma} is not
general enough for our purpose as it useful only for protocols that extract
witness from two transcripts.  To be able to extract a witness from, say, a
$\plonkprot$ execution we need to obtain at least $\numberofconstrains + 3$
transcripts.  Fortunately, since the witness can be extracted by repeating
only one round of the protocol, we can proceed similarly as in a protocol that
utilises only one challenge.

Here we propose a generalisation of the general forking lemma that given probability $\accProb$ gives a lower bound on the probability of generating a \emph{tree of accepting transcripts}, which could be used to extract a witness. 

\begin{lemma}[General forking lemma II]
	\label{lem:generalised_forking_lemma}
	Fix $q \in \ZZ$ and set $H$ of size $h \geq m$. 
	Let $\adv$ be a $\ppt$ algorithm that on input $y, h_1, \ldots, h_q$ returns $(i, s)$ where $i \in \range{0}{q}$ and $s$ is called a side output. 
	Denote by $\ig$ a randomised instance generator. We denote by $\accProb$ the probability
	\[
		\condprob{i \neq 0}{ y \gets \ig;\ h_1, \ldots, h_q \sample H;\ (i, s) \gets \adv(y, h_1, \ldots, h_q)}\,.
	\]
	Let $\genforking_\adv$ denote the algorithm described in \cref{fig:genforking_lemma} then the probability $\frkProb := \condprob{b = 1}{y \gets \ig;\ (b, \vec{s}) \gets \genforking_\adv(y)}$ is at least 
	\[
		\frac{\accProb^m}{q^{m - 1}} - \frac{(m - 1) \cdot \accProb}{h}\,.
	\]
		
	\begin{figure}
		\centering
		\fbox{
		\procedure{$\genforking_\adv (y)$}
		{
			r \sample \RND{\adv}\\
			h_1, \ldots, h_{i - 1} \sample H\\
			\pcfor j \in \range{1}{m}\\
			\pcind h_{i_j}, \ldots, h_{q_j} \sample H\\
			\pcind (i_j, s_j) \gets \adv(y, h_1, \ldots, h_{i - 1}, h_{i_j}, \ldots, h_{q_j}; r)\\
			\pcind \pcif i = 0\ \pcreturn (0, \bot, \bot)\\
			\pcif \forall j, j'  \in \range{1}{m}\colon (i_{j} = i_{j'}) \land (h_{i_j} \neq h_{i_{j'}})\ \pcreturn (1, \vec{s})\\
			\pcind \pcelse \pcreturn (0, \bot, \bot)
		}}
		\caption{Generalised forking algorithm $\genforking_\adv$}
		\label{fig:genforking_lemma}
\end{figure}
\end{lemma}
\begin{proof}
	\michals{3.08.20}{DISCLAIMER: This proof is a naive generalisation of the general forking lemma from Bellare and Neven 06. Need to check that all (in)equalities hold!}
	We proceed similarly as in \cite{CCS:BelNev06} (with some parts taken almost verbatim).
	
	First let denote by $\accProb(y)$ and $\frkProb(y)$ the following probabilities
	\begin{align*}
		\accProb(y) & =  \condprob{i \neq 0}{h_1, \ldots, h_q \sample H;\ (i, s) \gets \adv(y, h_1, \ldots, h_q)}\,.\\
		\frkProb(y) & = \condprob{b = 1}{(b, \vec{s}) \gets \genforking_\adv(y)}\,.
	\end{align*}
	
	We start by claiming that for all $y$ 
	\begin{equation}\label{eq:frkProb_y}
		\frkProb(y) \geq 
			\frac{\accProb(y)^m}{q^{m - 1}} - \frac{(m - 1) \cdot  \accProb(y)}{h}\,.
	\end{equation}
	Then with the expectation taken over $y \sample \ig$, we have
	\begin{align}
		\frkProb & = \expected{\frkProb(y)} \geq \expected{\frac{\accProb(y)^m}{q^{m - 1}} - \frac{(m - 1) \cdot  \accProb(y)}{h}} \label{eq:use_eq1}\\
		& = \frac{\expected{\accProb(y)^m}}{q^{m - 1}} - \frac{(m - 1) \cdot \expected{\accProb(y)}}{h} \\
		& \geq \frac{\expected{\accProb(y)}^m}{q^{m - 1}} - \frac{(m - 1) \cdot \expected{\accProb(y)}}{h} \label{eq:by_lemma_jensen}\\
		& = \frac{\accProb^m}{q^{m - 1}} - \frac{(m - 1) \cdot  \accProb}{h}\label{eq:by_accProb}\,.
	\end{align}
	Where Ineq.~(\ref{eq:use_eq1}) comes from \cref{eq:frkProb_y};   Ineq.~(\ref{eq:by_lemma_jensen}) comes from \cref{lem:jensen}; and (\ref{eq:by_accProb}) holds by the fact that $\expected{\accProb(y)} = \accProb$.
	
	We now show \cref{eq:frkProb_y}.
	Denote by $J = \range{1}{m}^2 \setminus \smallset{(j, j)}_{j \in \range{1}{m}}$. 
	For any input $y$, with probabilities taken over the coin tosses of $\genforking_\adv$ we have
	\begin{align*}
		\frkProb (y) & = \prob{i_j = i_{j'} \land i_j \geq 1 \land h_{i_j} \neq h_{i_{j'}} \text{ for } (j, j') \in J}	\\
		& \geq \prob{i_j = i_{j'} \land i_j \geq 1 \text{ for } (j, j') \in J} \\
		& \qquad - \prob{i_j \geq 1 \land h_{i_j} = h_{i_{j'}} \text{ for some } (j, j') \in J}\\
		& = \prob{i_j = i_{j'} \land i_j \geq 1 \text{ for } (j, j') \in J} - \prob{i_j \geq 1} \cdot \frac{m - 1}{h} \\
		& = \prob{i_j = i_{j'} \land i_j \geq 1 \text{ for } (j, j') \in J} - \accProb(y) \cdot \frac{m - 1}{h}\,.
	\end{align*}
	It remains to show that $\prob{i_j = i_{j'} \land i_j \geq 1 \text{ for } (j, j') \in J} \geq \infrac{\accProb(y)^m}{q^{m - 1}}$.
	
	Let $\RND{\adv}$ denote the set from which $\adv$ picks its coins at random. For each $\iota \in \range{1}{q}$ let $X_\iota \colon \RND{\adv} \times H^{\iota - 1} \to [0, 1]$ be defined by setting $X_\iota(\rho, h_1, \ldots, h_\iota)$ to
	\[
		\condprob{i = \iota}{h_\iota, \ldots, h_q \sample H; (i, \vec{s}) \gets \adv(y, h_1, \ldots, h_q; \rho)} 
	\] 
	for all $\rho \in \RND{\adv}$ and $h_1, \ldots, h_{\iota - 1} \in H$. Consider $X_\iota$ as a random variable over the uniform distribution on its domain. 
	Then
	\begin{align*}
		& \prob{i_j = i_{j'} \land i_j \geq 1 \text{ for } (j, j') \in J} \\
		& = \sum_{\iota = 1}^{q} \prob{i_1 = \iota \land \ldots \land i_m = \iota} \\
		& = \sum_{\iota = 1}^{q} \prob{i_1 = \iota} \cdot \condprob{i_2 = \iota}{i_1 = \iota} \cdot \ldots \cdot \condprob{i_m = \iota}{i_1 = \ldots = i_{m - 1} = \iota} \\
		& = \sum_{\iota = 1}^{q} \sum_{\rho, h_1, \ldots, h_{\iota - 1}} X_{\iota} (\rho, h_1, \ldots, h_{\iota - 1})^{m} \cdot \frac{1}{\abs{\RND{\adv}} \cdot \abs{H}^{\iota - 1}}\\
		& = \sum_{\iota = 1}^{q} \expected{X_\iota^m} \,.
	\end{align*}
	Then by \cref{lem:jensen} we get
	\[
		\sum_{\iota = 1}^{q} \expected{X_\iota^m} \geq \sum_{\iota = 1}^{q} \expected{X_\iota}^m\,.
	\]
	We note that for e.g.~$X_i = 1$, $i \in \range{1}{q}$ the inequality becomes equality, that is, it is tight.
	 
	We now use the H\"older inequality from \cref{lem:holder} where we set  $x_i = \expected{X_i}$, $y_i = 1$, $p = m$, and $q = m/(m - 1)$ obtaining
	\begin{gather}
		\sum_{i = 1}^{q} \expected{X_i}  \leq \left(\sum_{i = 1}^{q} \expected{X_i}^m\right)^{\frac{1}{m}} \cdot \left(\sum_{i = 1}^{q} 1^\frac{m}{m - 1}\right)^{\frac{m - 1}{m}} \label{eq:tightness} \\
		\left(\sum_{i = 1}^{q} \expected{X_i}\right)^{m}  \leq \left(\sum_{i = 1}^{q} \expected{X_i}^m\right) \cdot q^{m - 1}\\
		\frac{1}{q^{m - 1}} \cdot \accProb(y)^{m} \leq \sum_{i = 1}^{q} \expected{X_i}^m\,.
	\end{gather}
	Finally, we get
	\[
		\frkProb(y) \geq \frac{\accProb(y)^m}{q^{m - 1}} - 
		\frac{(m - 1) \cdot \accProb(y)}{h}\,.
	\]
	\qed
\end{proof}

\begin{remark}[Tightness of \cref{eq:tightness}]
	In is important to note that Inequality (\ref{eq:tightness}) is tight. More precisely, for $\expected{X_i} = x$, $i \in \range{1}{q}$ we have
	\begin{gather*}
		\sum_{i = 1}^q x = \left(\sum_{i = 1}^{q} x^m\right)^\frac{1}{m} \cdot \left(\sum_{i = 1}^{q} 1^{\frac{m}{m - 1}}\right)^{\frac{m - 1}{m}} \\
		qx = \left(qx^m\right)^\frac{1}{m} \cdot q^{\frac{m - 1}{m}} \\
		(qx)^m = qx^m \cdot q^{m - 1} \\
		(qx)^m = (qx)^m\,.
	\end{gather*}
\end{remark}

\begin{lemma}\label{lem:jensen}
	Let $\RND{\adv}$ denote the set from which $\adv$ picks its coins at random. For each $\iota \in \range{1}{q}$ let $X_\iota \colon \RND{\adv} \times H^{\iota - 1} \to [0, 1]$ be defined by setting $X_\iota(\rho, h_1, \ldots, h_\iota)$ to 
	\[
		\condprob{i = \iota}{h_\iota, \ldots, h_q \sample H; (i, \vec{s}) \gets \adv(y, h_1, \ldots, h_q; \rho)} 
	\] 
	for all $\rho \in \RND{\adv}$ and $h_1, \ldots, h_{\iota - 1} \in H$. Consider $X_\iota$ as a random variable over the uniform distribution on its domain. 
	Then $\expected{X_\iota^m} \geq \expected{X_\iota}^m$.
\end{lemma}
\begin{proof}
	First we recall the Jensen inequality \cite{W:Weissten20}, if for some random variable $X$ holds $\abs{\expected{X}} \leq \infty$ and $f$ is a Borel convex function then 
	\[
		f(\expected{X}) \leq \expected{f(X)}\,.
	\] 
	Finally, we note that $\abs{\expected{X}} \leq \infty$ and taking to thee $m$-th power is a Borel convex function on $[0, 1]$ interval. 
	\qed
\end{proof}

\begin{lemma}[H\"older's inequality. Simplified.]\label{lem:holder}
	Let $x_i, y_i$, for $i \in \range{1}{q}$, and $p, q$ be real numbers such that $1/p + 1/q = 1$. Then
	\[
		\sum_{i = 1}^{q} x_i y_i \leq \left(\sum_{i = 1}^{q} x_i^p\right)^{\frac{1}{p}} \cdot \left(\sum_{i = 1}^{q} y_i^p\right)^{\frac{1}{q}}\,.
	\]
\end{lemma}

\subsection{Unique-response protocols.}
Another problem comes with another assumption required by Faust et al. That is, the unique response property of the transformed sigma protocol.
Fischlin's formulation, although perfectly fine for applications presented in \cite{C:Fischlin05}, is not enough in our case.
First of all, the property assumes that the protocol has three rounds, with the middle being the challenge from the verifier. That is not the case we consider here. Second, it is not entirely clear how to generalize the property. Should one require that after the first challenge from the verifier the responses are fixed? That could not work since if there is more challenges then they are random.
Another problem rises when the protocol contains some round---obviously, except the first one---where the prover randomises his message. In that case unique-responsiveness can not hold as well.
Last but not least, the protocol we consider here most, \plonk, is not in a standard-model, but utilises CRS. That also complicates things considerably.

We walk around these obstacles by providing a generalised notion of the unique response property.
More precisely, we say that a $(2\mu + 1)$-round protocol has \emph{unique responses after $i$} and is called a $\ur{i}$-protocol if
\begin{definition}[$\ur{i}$-protocol]
	\label{def:wiur}
	Let $\proofsystem$ be a multi-round proof system.
	Denote by $a_0, b_0, \ldots, a_{\mu - 1}, b_{\mu - 1}, a_{\mu}$ the consecutive messages exchanged in the protocol, where messages $a_i$ come from the prover and $b_i$ from the verifier.
	We say that $\proofsystem$ has \emph{unique responses after $i$}
	if after submitting its $i$-th message the prover is a deterministic function. That is, it does not use his randomness tape and deterministically answers verifier's challenges.
\end{definition}
\begin{example}
	The Schnorr protocol is $\ur{1}$. That is, after submitting his first message $a$, the prover is a deterministic function of the instance, $a$, and the verifier's challenge.
\end{example}

We note that the definition above is independent on whether the proof system $\proofsystem$ utilises CRS (and compounds of the CRS-generating $\kgen$ algorithm) or not.
% \michals{08.07.20}{Should we change it to "deterministic prover property"?}

\subsection{Computational special soundness}
Note that the special soundness property (as usually defined) holds for all--- even computationally unbounded---adversaries. Since a simulation trapdoor for $\plonkprot$ exists, it is not special sound in that regard---as an unbounded adversary could reveal the trapdoor and build a number of simulated proofs for a fake statement. 
Hence, we provide a weaker, yet sufficient, definition of \emph{computational special soundness}. More precisely, we state that an adversary that is able to answer correctly multiple challenges either knows the witness or can be used to break some computational assumption. 

\begin{definition}[Computational special soundness]
	Let $\proofsystem$ be an $2 \mu$-round zero-knowledge proof system for a relation $\REL$. 
	We say that $\proofsystem$ is $(n_1, \ldots, n_\mu)$-\emph{special sound} if for every $\ppt$ adversary $\adv$ that produces an accepting $(n_1, \ldots, n_\mu)$-tree of transcripts $\tree$ for a statement $\inp$ there exists an extractor $\ext$ that given $\tree$ extracts $\wit$ such that $\REL(\inp, \wit) = 1$ with an overwhelming probability.
\end{definition}

Since we do not utilise the classical special soundness (that holds for all, even unbounded, adversaries) all references to that property should be understood as references to its computational version.

% \subsection{Unique opening property}
% here we show that both the polynomial commitment scheme $\PCOMp$ used in
% $\plonk$ and the polynomial commitment scheme $\PCOMs$ used in $\sonic$ have
% the unique opening property.

\section{Simulation-extractability---the general result}
\michals{25.09}{The generalisation of the forking lemma we have may work only for schemes that require rewinding of a single round. That limits the generalisation of the result}

\begin{theorem}[Simulation-extractable multi-round protocols]
	\label{thm:se}
	Let $\ps = (\kgen, \prover, \verifier, \simulator)$ be an interactive $2
	\mu$-round proof system that is honest verifier zero-knowledge in the
	standard model\footnote{Crucially, we require that one can provide an
			indistinguishable simulated proof without any additional knowledge, as
	e.g~knowledge of a CRS trapdoor.}, has $\ur{k}$ property with security $\epsur$ that is $(1, \ldots, 1, n_j, 1, \ldots, 1)$-special sound, where $j > k$.
	Assume that the simulator $\simulator$ does not get as input any trapdoor related to the CRS.
	Let $\ro\colon \bin^{*} \to \bin^{\secpar}$ be a random oracle. 
	Then $\psfs$ is simulation-extractable with extraction error $\epsur$ against $\ppt$ adversaries that makes up up to $q$ random oracle queries and returns an acceptable proof with probability at least $\waccProb$. 
	The extraction probability $\extProb$ is at least
	\[
		\extProb \geq \frac{1}{q^{n_j - 1}} (\waccProb - \epsur)^{n_j} -\eps\,,
	\]
	for some negligible $\eps$.	
\end{theorem}
\begin{proof}		
	The proof goes by game hoping. The games are controlled by an environment $\env$ that internally runs a simulation extractability adversary $\advse$,  provides it with access to a random oracle and simulator, and when necessary rewinds it.
	The games differ by various breaking points, i.e.~points where the environment decides to abort the game. 

	Denote by $\zkproof_{\advse}, \zkproof_{\simulator}$ proofs
	returned by the adversary and the simulator respectively. We use $\zkproof[i]$
	to denote prover's message in the $i$-th round of the proof, $\zkproof[i].\ch$
	to denote the challenge that is given to the prover after $\zkproof[i]$, and
	$\zkproof\range{i}{j}$ to denote all messages of the proof including challenges between rounds $i$ and $j$.
	
	Without loss of generality, we assume that whenever the accepting proof contains a response to a challenge from a random oracle, we assume that the adversary queried the oracle to get it. 
	It is straightforward to transform any adversary that violates this condition into an adversary that makes these additional queries to the random oracle and wins with the same probability.
	
	\ngame{0} 
	This is a simulation extraction game played between an adversary $\advse$ who has given access to a random oracle $\ro$ and simulator $\psfs.\simulator$. 
	There is also an extractor $\ext$ that, from the proof $\zkproof_\advse$ for instance $\inp_\advse$ output by the adversary and from a transcripts of $\advse$'s operations, is tasked to extract a witness $\wit_\advse$ such that $\REL(\inp_\advse, \wit_\advse)$ holds.
	$\advse$ wins if it manages to produce an acceptable proof and the extractor fails to reveal the corresponding witness.
	In the following game hops we upper-bound probability that this happens.
	
	\ngame{1}
	This is identical to $\game{0}$ except that now the game is aborted if there is a simulated proof $\zkproof_\simulator$ such that $\zkproof_\simulator\range{1}{k} = \zkproof_\advse\range{1}{k}$. That is, the adversary in its final proof reuses a part of a simulated proof it saw before and the proof is acceptable.
	Denote that event by $\event{\errur}$.
	
	\ncase{$\game{0} \mapsto \game{1}$}	
	We have, 
	\[
		\prob{\game{0} \land \nevent{\errur}} = \prob{\game{1} \land \nevent{\errur}}
	\]
	and, from the difference lemma, cf.~\cref{lem:difference_lemma}
	\[
		\abs{\prob{\game{0}} - \prob{\game{1}}} \leq \event{\errur}\,.
	\]
	Thus, to show that the transition from one game to another introduces only minor change in probability of $\advse$ winning it should be shown that $\prob{\event{\errur}}$ is small.
	
	Assume that $\advse$ queried the simulator on $\inp_{\advse}$---the instance which $\advse$ outputs. 
	We show a reduction $\rdvur$ that utilises $\advse$, who outputs a valid proof for $\inp_\advse$, to break the $\ur{3}$ property of $\ps$. 

	Consider an algorithm $\rdvur$ that runs $\advse$ internally as a black-box:
	\begin{itemize}
		\item The reduction answers both queries to the simulator $\psfs.\simulator$ and to the random oracle. 
		It also keeps lists $Q$, for the simulated proofs, and $Q_\ro$ for the random oracle queries. 
		\item When $\advse$ outputs a fake proof $\zkproof_{\advse}$ for  $\inp_\advse$, $\rdvur$ looks through lists $Q$ and $Q_\ro$ until it finds 
		$\zkproof_{\simulator}\range{1}{k}$ such that $\zkproof_{\advse}\range{1}{k} = \zkproof_{\simulator}\range{1}{k}$ and a random oracle query $\zkproof_{\simulator}[k].\ch$ on $\zkproof_{\simulator}\range{1}{k}$.
		\item $\rdvur$ returns two proofs for $\inp_\advse$:
		\begin{align*}
			\zkproof_1 = (\zkproof_{\simulator}\range{1}{k}, \zkproof_{\simulator}[k].\ch, \zkproof_{\simulator}\range{k + 1}{\mu})\\
			\zkproof_2 = (\zkproof_{\simulator}\range{1}{k}, \zkproof_{\simulator}[k].\ch, \zkproof_{\advse}\range{k + 1}{\mu})
		\end{align*}
		\end{itemize}  
		If $\zkproof_1 = \zkproof_2$, then $\advse$ fails to break simulation extractability, as $\zkproof_2 \in Q$.
		On the other hand, if the proofs are not equal, then $\rdvur$ breaks $\ur{k}$-ness of $\ps$, what may happen with some negligible probability $\epsur$ only, hence
		\[
			\prob{\event{\errur}} \leq \epsur\,.
		\]
		
	\ngame{2}
	This is identical to $\game{1}$ except that now the environment aborts also when it fails to build a $(1, 1, 1, n_j, 1)$-tree of accepting transcripts $\tree$ by rewinding $\advse$. Denote that event 
	by $\event{\errfrk}$.
	Note that for every acceptable proof $\zkproof_{\advse}$, we may
	assume that whenever $\advse$ outputs in Round $i$, for $i > k$, a message $\zkproof_{\advse}[i]$, then
	$\zkproof_{\advse}\range{1}{i}$ is a query to the random oracle that was made by the adversary, not the simulator\footnote{\cite{INDOCRYPT:FKMV12} calls these queries \emph{fresh}.}. 
	That is, assume that is not true and for some query $\zkproof_{\advse}\range{1}{i}$ holds $\zkproof_{\advse}\range{1}{i} = \zkproof_\simulator\range{1}{i'}$, for $i, i' > k$, the proof is acceptable and not in $Q$, then the unique response property would be broken.
	
	\ncase{$\game{1} \mapsto \game{2}$}	
	As previously, 
	\[
		\abs{\prob{\game{1}} - \prob{\game{2}}} \leq \event{\errfrk}\,.
	\]
	Denote by $\accProb$ the probability that $\advse$ outputs a proof such that is acceptable and does not break $\ur{k}$-ness of $\ps$. 
	From the generalised forking lemma, cf.~\cref{lem:generalised_forking_lemma}, 
	\[
		\prob{\event{\errfrk}} \leq 1 - \left(\frac{\accProb^{n_j}}{q^{n_j - 1}} - \frac{\accProb \cdot (n_j - 1)}{2^\secpar}\right)\,.
	\]
	
	\ngame{3}
	This is identical to $\game{2}$ except that now the environment uses the tree $\tree$ to extract the witness for the proven statement and aborts when it fails. Denote that event by $\event{\errss}$.
	
	\ncase{$\game{2} \mapsto \game{3}$}	
	As previously, 
	\[
		\abs{\prob{\game{2}} - \prob{\game{3}}} \leq \event{\errss}\,.
	\]
	Since $\ps$ is special-sound the probability that $\env$ fails in extracting the witness is upper-bounded by some negligible $\eps_\ss$.
	
	In the last game, Game $\game{3}$, the environment aborts when it fails to extract the correct witness, hence the adversary $\advse$ cannot win. 
	Thus, by the game-hoping argument, 
	\[
		\abs{\prob{\game{0}} - \prob{\game{3}}} \leq 1 - \left(\frac{\accProb^{n_j}}{q^{n_j - 1}} - \frac{\accProb \cdot (n_j - 1)}{2^\secpar}\right) + \epsur + \epsss\,.
	\]
	Thus the probability that extractor $\extss$ succeeds is at least
	\[
		\left(\frac{\accProb^{n_j}}{q^{n_j - 1}} - \frac{\accProb \cdot (n_j - 1)}{2^\secpar}\right) - \epsur - \epsss\,.
	\]
	Since $\accProb$ is probability of $\advse$ outputting acceptable transcript that does not break $\ur{3}$-ness of $\ps$, then $\waccProb \leq \accProb + \epsur$, where $\waccProb$ is the probability of $\advse$ outputing an acceptable proof as defined in \cref{def:simext}. It thus holds
	\[
 		\label{eq:frk}
 		\extProb \geq \frac{(\waccProb - \epsur)^{n_j}}{q^{n_j}} - \underbrace{\frac{(\waccProb - \epsur) \cdot (n_j - 1)}{2^\secpar} - \epsur - \epsss}_{\eps}\,.
 	\]
 	Note that the part of \cref{eq:frk} denoted by $\eps$ is negligible and 
 	\[
 		\extProb \geq \frac{1}{q^{n_j - 1}} (\waccProb - \epsur)^{n_j} -\eps\,.
 	\] 
 	thus
 	$\psfs$ is simulation extractable with extraction error $\epsur$.
 	\qed
\end{proof}

\section{Simulation extractability of $\plonkprotfs$} 
In this section we show that $\plonkprotfs$
is not only simulation-sound, but also simulation-extractable. To that end, we
proceed as follows. 
First we show that the version of the KZG polynomial commitment scheme that is proposed in the \plonk{} paper has the unique opening property, cf.~\cref{sec:poly_com} and \cref{lem:pcomp_unique_op}. This is next used to show that $\plonkprot$ has $\ur{3}$ property, cf.~\cref{lem:plonkprot_ur}.

Then, we show that 
$\plonkprot$ is (kind of) special-sound. That is, given a number of acceptable
transcripts which match on the first 3 rounds of the protocol we can either
reveal a correct witness for the proven statement or use one of the transcripts
to break the dlog assumption. The latter requires use of the AGM. More
precisely, we assume that each group element that is published as a part of the
transcripts comes with a vector of coefficients that represents the element in
the basis compound of the input group elements, i.e.~$\plonkprot$'s CRS. See \cref{lem:plonkprot_ss}.

Given special-soundness of $\plonkprot$, we use the fact that it is also $\ur{3}$ and show, in a similar fashion to \cite{INDOCRYPT:FKMV12}, that it is simulation-extractable. That is, we build reductions that given a simulation extractability adversary $\advse$ either breaks the protocol's unique response property or breaks the dlog assumption, if extracting a valid witness from a tree of transcripts is impossible. See \cref{thm:plonkprotfs_se}.

\subsection{Unique opening property of $\PCOMp$}
\begin{lemma}
	\label{lem:pcomp_unique_op}
	Let $\PCOMp$ be a batched version of a KZG polynomial commitment
	\cite{AC:KatZavGol10} as described in \cite{EPRINT:GabWilCio19} then $\PCOMp$ has the unique opening property. 
\end{lemma}
\begin{proof}
	Let 
	$\vec{z} = (z, z') \in \FF_p^2$ be the two attributes the polynomials are evaluated at,
	$k \in \NN$ be the number of the committed polynomials to be evaluated at $z$, and $k' \in \NN$ be the number of the committed polynomials to be evaluated at  $z'$,
	$\vec{c} \in \GRP^k, \vec{c'} \in \GRP^{k'}$ be the commitments,  
	$\vec{s} \in \FF_p^k, \vec{s'} \in \FF_p^{k'}$ the evaluations, and 
	$\vec{o} = (o, o') \in \FF_p^2$ be the commitment openings. We need to show that for every $\ppt$ adversary $\adv$ probability
	\[
		\Pr
			\left[
				\begin{aligned}
					& \verify(\crs, \vec{c}, \vec{c'}, (z, z'), \vec{s}, \vec{s'}, \vec{o}), \\
					& \verify(\crs, \vec{c}, \vec{c'}, (z, z'), \vec{\tilde{s}}, \vec{\tilde{s}'}, \vec{\tilde{o}}) %\\
					% &(o, o') \neq (\tilde{o}, \tilde{o}')
				\end{aligned}
			\,\left|\,
			\vphantom{\begin{aligned}
				& \verify(\crs, \vec{c}, \vec{c}, \vec{z}, \vec{s}, \vec{s'}, \vec{o}), \\
				& \verify(\crs, \vec{c}, \vec{c}, \vec{z}, \vec{s}, \vec{s'}, \vec{\tilde{o}}) \\
				&\vec{o} \neq \vec{\tilde{o}})
			\end{aligned}}
			\begin{aligned}
				& \crs \gets \kcrs(\secparam), \\
				&	(\vec{c}, \vec{c'}, \vec{z}, \vec{s}, \vec{s'}, \vec{\tilde{s}}, \vec{\tilde{s}'}, \vec{o}, \vec{\tilde{o}}) \gets \adv(\crs)
			\end{aligned}
			\right.\right]
		 % \leq \negl.
	\]
	is at most negligible.
	
	\ncase{Step 1} First, consider a case where the commitment is limited to commit to multiple polynomials which are evaluated at the same point $z$. 
	As noted in \cite[Lemma 2.2]{EPRINT:GabWilCio19} it is enough to upper bound
  the probability of the adversary succeeding using the idealised verification equation---which considers equality between polynomials---instead of the real verification equation---which consider equality of the polynomials' evaluations.
	
	For polynomials $\vec{f} = f_1, \ldots, f_k$, evaluation point $z$, evaluation result $\vec{s} = s_1, \ldots, s_k$, random $\gamma$, and opening $o(X)$ the idealised check verifies that
	\begin{equation}
		\sum_{i = 1}^k \gamma^{i - 1} f_i(X) - \sum_{i = 1}^{k} \gamma^{i - 1} s_i \equiv o(X) (X - z)\,.
		\label{eq:pcom_idealised_check}
	\end{equation}
	Since $o(X)(X - z) \in \FF_p[X]$ then from the uniqueness of polynomial composition, there is only one $o(X)$ that fulfils the equation above.

	\ncase{Step 2} Second, consider a case when the polynomials are evaluated on two points $\vec{z} = (z, z')$ and the adversary is asked to provide two openings $\vec{o} = (o, o')$.
	Similarly, we analyse the case of the ideal verification. In that scenario, the verifier checks whether the following equality, for $\gamma, r'$ picked at random, holds:
	\begin{multline}
		\label{eq:ver_eq_poly}
		\sum_{i = 1}^{k} \gamma^{i - 1} \cdot f_i(X) - \sum_{i = 1}^{k} \gamma^{i - 1} \cdot s_i  + r' \left(\sum_{i = 1}^{k'} \gamma'^{i - 1} \cdot f'_i(X) - \sum_{i = 1}^{k'} \gamma'^{i - 1} \cdot s'_i \right)\\
		\equiv o(X)(X - z) + r' o'(X)(X- z')
	\end{multline}
	Since $r'$ has been picked at random, \cref{eq:ver_eq_poly} holds while either
	\[
		\sum_{i = 1}^{k} \gamma^{i - 1} \cdot f_i(X) - \sum_{i = 1}^{k}  \gamma^{i - 1} \cdot s_i \equiv o(X)(X - z)
	\]
	or 
	\[
		\sum_{i = 1}^{k'} \gamma'^{i - 1} \cdot f'_i(X) - \sum_{i = 1}^{k'} \gamma'^{i - 1} \cdot s'_i \equiv o'(X)(X - z')
	\]
	does not is negligible~\cite{EPRINT:GabWilCio19}. This brings the proof back to Step 1 above. 
	\qed
\end{proof}

\subsection{Unique response property}

\begin{lemma}
	\label{lem:plonkprot_ur}
	If a polynomial commitment scheme $\PCOM$ is evaluation binding with parameter $\epsbind$ and has unique openings property with parameter $\epsop$, then $\plonkprot$ is $\ur{3}$\footnote{An attentive reader may note that $\plonkprot$ is $\ur{2}$. However, in the case presented here, less is required.} 
	with parameter $\epsur \leq \epsbind + \epsop$.
\end{lemma}
\begin{proof}
	Let $\adv$ be an adversary that breaks $\ur{3}$-ness of
  $\plonkprot$. 
	% \markulf{30.08}{This is fine. But sounds as if the law of the
    % excluded middle is needed here. I am not sure it is. :) Mostly a
    % philosophical point.}
	We consider two cases, depending on which round $\adv$ is able to provide at
  least two different outputs such that the resulting transcripts are acceptable.
  % \markulf{30.08}{Rewrote this text: that in each of them $\adv$ breaks the
    % evaluation binding property of $\PCOM$.}
  For the first case we show that $\adv$ breaks the evaluation binding property of $\PCOM$, while for the
  second case we show that it breaks the unique opening property of $\PCOM$.
	
	\case{1}
	In Round 4 the prover is asked to give evaluations of predefined polynomials at some point $\chz$. Naturally, for the given polynomials only one value at $\chz$ is correct.
	Assume $\adv$ is able to produce two different outputs in that round: $\vec{r_4} = (\ev{\p{a}}, \ev{\p{b}}, \ev{\p{c}}, \ev{\p{S_{\sigma 1}}}, \ev{\p{S_{\sigma 2}}}, \ev{\p{r}}, \ev{\p{z}})$ and 
	$\vec{r_4} = (\ev{\p{a}}', \ev{\p{b}}', \ev{\p{c}}', \ev{\p{S_{\sigma 1}}}', \ev{\p{S_{\sigma 2}}}', \ev{\p{r}}', \ev{\p{z}}')$
	which suppose to be evaluations at $\chz$ of polynomials $\p{a}, \p{b}, \p{c}, \p{S_{\sigma 1}}, \p{S_{\sigma 2}}, \p{r}$ and an evaluation at $\chz \omega$ of $\p{z}$.
	Clearly, at least one of $\vec{r_4}$, $\vec{r'_4}$ has to be incorrect, thus if both evaluations are acceptable by the $\PCOM.\verify$ then the evaluation binding property of $\PCOM$ is broken. This happens with probability upper-bounded by $\epsbind$.
	
	\case{2}
	In the last round of the protocol the prover provides openings for the polynomial commitments done before. 
	Assume $\adv$ is able to produce two different polynomial commitment openings pairs: 
	$\vec{r_5} = (\ev{\p{W_\chz}}, \ev{\p{W_{\chz \omega}}})$ and 
	$\vec{r'_5} = (\ev{\p{W_\chz}}', \ev{\p{W_{\chz \omega}}}')$.
	% Since \cref{lem:pcomp_unique_op}, 
	Since $\PCOM$ has unique opening property, 
	one of the openings has to be incorrect and should be rejected by the polynomial commitment verifier. This happens except probability $\epsop$
	
	\conclude
	Hence the probability that $\adv$ breaks $\ur{3}$-property of $\PCOM$ is upper-bounded by $\epsbind + \epsop$.
	\qed
\end{proof}


\subsection{Special soundness}
\begin{lemma}
	\label{lem:plonkprot_ss}
	Let $\adv$ be a $\ppt$ algebraic adversary. The probability $\epsss$ that $\adv$ breaks 
	 $(1, 1, 1, \numberofconstrains + 3, 1)$-computational special soundness of $\plonkprot$ is upper-bounded as
	 \[
	 	\epsss \leq \epsbatch + \epsdlog\,,
	 \] 
	 where $\epsbatch$ is (negligible) probability that $\plonkprot$'s idealised verification equation $\vereq(X)$ accepts an invalid proof because of batching and $\epsdlog$ is a probability that a $\ppt$ algorithm can break $(\numberofconstrains + 2, 1)$-dlog assumption.
\end{lemma}
\begin{proof}
	Let $\crs$ be $\plonkprot$'s CRS and denote by $\crs_1$ all CRS's $\GRP_1$-elements; that is, $\crs_1 = \gone{1, \chi, \ldots, \chi^{\numberofconstrains + 2}}$. 
	Let $\adv$ be an algebraic adversary that for a statement $\inp$ produces a $(1, 1, 1, \numberofconstrains + 3, 1)$-tree of acceptable transcripts $\tree$. % with non-negligible probability $\eta_\tree$. 
	Note that in all transcripts the instance $\inp$, proof elements $\gone{\p{a}(\chi), \p{b}(\chi), \p{c}(\chi), \p{z}(\chi), \p{t}(\chi)}$ and challenges $\alpha, \beta, \gamma$ are common as the transcripts share the first three rounds. 
	
	We consider two mutually disjunctive events. 
	First, $\event{E}$ holds when all of the transcripts are acceptable by the idealised verification equation, i.e.~$\vereq(X) = 0$, cf.~\cref{eq:ver_eq}.
	Second, $\nevent{E}$ holds when there is a transcript that is acceptable, yet 
	for some transcript $\vereq(\chi) = 0$, but $\vereq(X) \neq 0$.
	We build a special extractor $\extss$ which given the tree of transcripts $\tree$ reveals the witness with an overwhelming probability when $\event{E}$ happens. 
	We show a reduction $\rdvdlog$ that, when $\nevent{E}$ happens, breaks the dlog assumption. 
	
	\ncase{When $\event{E}$ happens}  Since the protocol $\plonkprot$,
	instantiated with the idealised verification equation, is perfectly sound,
	except probability of batching failure $\epsbatch$, for a valid proof
	$\zkproof$ of a statement $\inp$ there exists a witness $\wit$, such that
	$\REL(\inp, \wit)$ holds.  Note that witness-carrying polynomials $\p{a}(X),
	\p{b}(X), \p{c}(X)$ have degree $(\numberofconstrains + 2)$ and since $\adv$
	answered honestly on $(\numberofconstrains + 3)$ different challenges $\chz$
	then $(\numberofconstrains + 3)$ evaluations of these polynomials (at
	different points) are known. The extractor $\extss$ interpolates the
	polynomials and reveals the corresponding witness $\wit$. 
	
	\ncase{When $\nevent{E}$ happens} Consider a transcript that such that
	$\vereq(X) \neq 0$, but $\vereq(\chi) = 0$.
	Since the adversary is algebraic, all group elements included in the tree of
  transcripts are extended by their representation as a combination of the input
  $\GRP_1$-elements i.e.~$\gone{1, \chi, \ldots, \chi^{\numberofconstrains +
      2}}$. Hence all coefficients of the verification equation polynomial
  $\vereq(X)$ are known and $\rdvdlog$ can find its zero points. Since
  $\vereq(\chi) = 0$, the targeted discrete log value $\chi$ is among them.
	\qed
\end{proof}

\subsection{Zero knowledge}
\begin{lemma}
	$\plonk$ is honest verifier zero-knowledge in the standard model.	
\end{lemma}
% Here we show that $\sonic$ is trapdoor-less zero-knowledge.
\begin{proof}
Since the simulator $\simulator$ does not know a witness $\wit$ for the proven
statement $\inp$, it cannot compute the output of Round 1 accordingly to the
protocol. Instead, it picks randomly both the ``blinders'' $b_1, \ldots, b_6$
and polynomials $\p{a}, \p{b}, \p{c}$ by picking their
coefficients randomly. Then $\simulator$ outputs $\gone{\p{a}(\chi),
\p{b}(\chi), \p{c}(\chi)}$. 
For the first round challenge, the simulator picks permutation argument
challenges $\beta, \gamma$ randomly.

In the second round, similarly as in the previous one, the simulator cannot
evaluate the requested polynomial $\p{z}$ honestly as it does not know the
witness. Thus it picks the polynomial's coefficients randomly and outputs
$\gone{\p{z}(\chi)}$. Challenge $\alpha$ that should be sent by the verifier
after Round 2 is picked by the simulator at random.

The next round starts by the simulator picking at random a challenge $\chz$,
which in the real proof comes as a challenge from the verifier sent after
Round 3. 
Then $\simulator$ computes evaluations
\[\p{a}(\chz), \p{b}(\chz), \p{c}(\chz), \p{S_{\sigma 1}}(\chz), \p{S_{\sigma
2}}(\chz), \pubinppoly(\chz), \lag_1(\chz), \p{Z_H}(\chz),\allowbreak
\p{z}(\chz\omega)\]
and computes $\p{\tilde{t}}(X)$ the way an honest prover would compute $\p{t}(X)$. Since for a random
$\p{a}, \p{b}, \p{c}, \p{z}$ the constraint system is not satisfied
$\p{\tilde{t}}(X)$
is a rational function, not a polynomial (as constraint polynomials are not
divisible by $\p{Z_H}(X)$. The simulator evaluates $\p{\tilde{t}}(X)$ at
$\chz$ and picks randomly a degree-$(\numberofconstrains + 2)$ polynomial
$\p{t}(X)$ such that $\p{t}(\chz) = \p{\tilde{t}}(\chz)$ and publishes a
commitment to $\p{t}(X)$ as an honest prover would.
After this round the simulator outputs $\chz$ as a challenge.

In the next round, the simulator computes polynomial $\p{r}(X)$ as an honest
prover would, i.e.  
\[
		\p{r}(X) = 
		\begin{aligned}
			& \p{a}(\chz) \p{b}(\chz) \selmulti(X) + \p{a}(\chz) \selleft(X) + \p{b}(\chz) \selright(X) + \p{c}(\chz) \seloutput(X) + \selconst(X) \\
			& + \alpha \cdot \left( (\p{a}(\chz) + \beta \chz + \gamma) (\p{b}(\chz) + \beta k_1 \chz + \gamma)(\p{c}(\chz) + \beta k_2 \chz + \gamma) \cdot \p{z}(X)\right) \\
			& - \alpha \cdot \left( (\p{a}(\chz) + \beta \p{S_{\sigma 1}}(\chz) + \gamma) (\p{b}(\chz) + \beta \p{S_{\sigma 2}}(\chz) + \gamma)\beta \p{z}(\chz\omega) \cdot \p{S_{\sigma 3}}(X)\right) \\
			& + \alpha^2 \cdot \lag_1(\chz) \cdot \p{z}(X)
		\end{aligned}
\]
and evaluates $\p{r}(X)$ at $\chz$. 

The rest of the evaluations are already computed, thus 
$\simulator$ simply outputs 
\[
		\p{a}(\chz), \p{b}(\chz), \p{c}(\chz), \p{S_{\sigma 1}}(\chz), \p{S_{\sigma 2}}(\chz), \p{t}(\chz), \p{z}(\chz \omega)\,.
\]
After that it picks randomly the challenge $v$, proceeds in the last round as
an honest prover would proceed and outputs the final challenge, $u$, by
picking it at random as well.

%\michals{28.10}{Add explanation why the proof generated by the simulator is
%indistinguishable from a proof generated by a real prover}

The proof provided by the simulator is indistinguishable from the real proof
since the simulator picks the messages as an honest prover or verifier would
except how the message output in Round 3 are computed. However,
real-proof polynomial $\p{t}(X)$ is random since random are polynomials
$\p{a}(X), \p{b}(X), \p{c}(X)$ and $\p{z}(X)$. Hence, a commitment to
$\p{t}(X)$ picked randomly by the simulator is indistinguishable from a
commitment to a real-proof $\p{t}(X)$. 
Furthermore, since $\p{t}(X)$ is evaluated at $\chz$ according to the
prover's, the rest of the proof that utilises $\p{t}(\chz)$ can be computed
honestly as well.
\qed
\end{proof}

\subsection{From special-soundness and unique response property to simulation extractability of $\plonkprotfs$}

Since \cref{lem:plonkprot_ur,lem:plonkprot_ss} hold, $\plonkprot$ is $\ur{3}$ and computationally special sound we are able to follow \cite{INDOCRYPT:FKMV12} and show that $\plonkprot_\fs$ is simulation-extractable as defined in \cref{def:simext}.

% \michals{13.10}{The theorem below is going to have a much shorter proof -- for it being just a corollary of Sec.~4}
\begin{corollary}[Simulation extractability of $\plonkprot_\fs$]
	\label{thm:plonkprotfs_se}
	Assume that  $\plonkprot$ is $\ur{3}$ with security $\epsur(\secpar)$, and computational special-sound with security $\epsss(\secpar)$.
	Let $\ro\colon \bin^* \to \bin^\secpar$ be a random oracle. 
	Let $\advse$ be a $\ppt$ adversary that can make up to $q$ random oracle queries and outputs an acceptable proof for $\plonkprotfs$ with probability at least $\waccProb$.
	Then $\plonkprotfs$ is simulation-extractable with extraction error $\eta = \epsur$. The extraction probability $\extProb$ is at least
	\[
		\extProb \geq \frac{1}{q^{\numberofconstrains + 2}} (\waccProb - \epsur)^{\numberofconstrains + 3} -\eps\,,
	\]
	for some negligible $\eps$ and $\numberofconstrains$ being the number of contrains in the proven circuit.
\end{corollary}
% \begin{proof}
% 	The theorem holds as a corollary to \cref{thm:se}.
	% The proof goes by game hoping. The games are controlled by an environment $\env$ that internally runs a simulation extractability adversary $\advse$, simulates its with access to a random oracle and simulator, and when necessary rewinds it.
	% The games differ by various breaking points, i.e.~points where the environment decides to abort the game. 
	% 
	% Denote by $\zkproof_{\advse}, \zkproof_{\simulator}$ proofs
	% returned by the adversary and the simulator respectively. We use $\zkproof[i]$
	% to denote prover's message in the $i$-th round of the proof, $\zkproof[i].\ch$
	% to denote the challenge that is given to the prover after $\zkproof[i]$, and
	% $\zkproof\range{i}{i'}$ to denote all messages of the proof exchanged between rounds $i$ and $i'$, i.e.~$\zkproof[i], \zkproof[i].\ch, \ldots, \zkproof[i']$.
	% 
	% Without loss of generality, we assume that whenever the accepting proof contains a response to a challenge from a random oracle, we assume that the adversary queried the oracle to get it. 
	% It is straightforward to transform any adversary that violates this condition into an adversary that makes these additional queries to the random oracle and wins with the same probability.
	% 
	% \ngame{0}
	% This is a simulation extraction game played between an adversary $\advse$ who
	% has given access to a random oracle $\ro$ and simulator
	% $\plonkprotfs.\simulator$.  There is also an extractor $\ext$ that, from the
	% proof $\zkproof_\advse$ for instance $\inp_\advse$ output by the adversary and
	% from a transcripts of $\advse$'s operations, is tasked to extract a witness
	% $\wit_\advse$ such that $\REL(\inp_\advse, \wit_\advse)$ holds. $\advse$ wins
	% if it manages to produce an acceptable proof and the extractor fails to reveal
	% the corresponding witness. In the following game hops we upper-bound
	% probability of this happening.
	% 
	% \ngame{1}
	% This is identical to $\game{0}$ except that now the game is aborted if there
	% is a simulated proof $\zkproof_\simulator$ such that
	% $\zkproof_\simulator\range{1}{3} = \zkproof_\advse\range{1}{3}$. That is, the
	% adversary in its final proof reuses a part of a simulated proof it saw before
	% and the proof is acceptable. Denote that event by $\event{\errur}$.
	% 
	% \ncase{$\game{0} \mapsto \game{1}$}	
	% We have, 
	% \[
	% 	\prob{\game{0} \land \nevent{\errur}} = \prob{\game{1} \land \nevent{\errur}}
	% \]
	% and, from \cref{lem:difference_lemma}
	% \[
	% 	\abs{\prob{\game{0}} - \prob{\game{1}}} \leq \event{\errur}\,.
	% \]
	% Thus, to show that the transition from one game to another introduces only minor change in probability of $\advse$ winning it should be shown that $\prob{\event{\errur}}$ is small.
	% 
	% Assume that $\advse$ queried the simulator on $\inp_{\advse}$---the instance which $\advse$ outputs. 
	% We show a reduction $\rdvur$ that utilises $\advse$, who outputs a valid proof for $\inp_\advse$, to break the $\ur{3}$ property of $\plonkprot$. 
	% 
	% Consider an algorithm $\rdvur$ that runs $\advse$ internally as a black-box:
	% \begin{itemize} \item The reduction answers both queries to the simulator
	% $\plonkprotfs.\simulator$ and to the random oracle.  It also keeps lists $Q$,
	% for the simulated proofs, and $Q_\ro$ for the random oracle queries.  \item
	% When $\advse$ outputs a fake proof $\zkproof_{\advse}$ for  $\inp_\advse$,
	% $\rdvur$ looks through lists $Q$ and $Q_\ro$ until it finds
	% $\zkproof_{\simulator}\range{1}{3}$ such that $\zkproof_{\advse}\range{1}{3} =
	% \zkproof_{\simulator}\range{1}{3}$ and a random oracle query
	% $\zkproof_{\simulator}[3].\ch$ on $\zkproof_{\simulator}\range{1}{3}$. \item
	% $\rdvur$ returns two proofs for $\inp_\advse$: \begin{align*} \zkproof_1 =
	% (\zkproof_{\simulator}\range{1}{3}, \zkproof_{\simulator}[3].\ch,
	% \zkproof_{\simulator}\range{4}{5})\\ \zkproof_2 =
	% (\zkproof_{\simulator}\range{1}{3}, \zkproof_{\simulator}[3].\ch,
	% \zkproof_{\advse}\range{4}{5}) \end{align*} \end{itemize}   If $\zkproof_1 =
	% \zkproof_2$, then $\advse$ fails to break simulation extractability, as
	% $\zkproof_2 \in Q$. On the other hand, if the proofs are not equal, then
	% $\rdvur$ breaks $\ur{3}$-ness of $\plonkprot$, what may happen with some
	% negligible probability $\epsur$ only, hence 
	% \[ 
	% \prob{\event{\errur}} \leq \epsur\,. 
	% \]
	% 
	% \ngame{2}
	% This is identical to $\game{1}$ except that now the environment aborts also when it fails to build a $(1, 1, 1, \numberofconstrains + 3, 1)$-tree of accepting transcripts $\tree$ by rewinding $\advse$. Denote that event 
	% by $\event{\errfrk}$.
	% Note that for every acceptable proof $\zkproof_{\advse}$, we may
	% assume that whenever $\advse$ outputs in Round $i$, for $i \in \smallset{4, 5}$, a message $\zkproof_{\advse}[i]$, then
	% $\zkproof_{\advse}\range{1}{i}$ is a query to the random oracle that was made by the adversary, not the simulator\footnote{\cite{INDOCRYPT:FKMV12} calls these queries \emph{fresh}.}. 
	% That is, assume that is not true and for some query $\zkproof_{\advse}\range{1}{i}$ holds $\zkproof_{\advse}\range{1}{i} = \zkproof_\simulator\range{1}{i'}$, for $i, i' \in \smallset{4, 5}$, the proof is acceptable and not in $Q$, then the unique response property would be broken.
	% 
	% \ncase{$\game{1} \mapsto \game{2}$}	
	% As previously, 
	% \[
	% 	\abs{\prob{\game{1}} - \prob{\game{2}}} \leq \event{\errfrk}\,.
	% \]
	% Denote by $\accProb$ the probability that $\advse$ outputs a proof which is acceptable and does not break $\ur{3}$-ness of $\plonkprot$. 
	% From the generalised forking lemma, cf.~\cref{lem:generalised_forking_lemma}, 
	% \[
	% 	\prob{\event{\errfrk}} \leq 1 - \left(\frac{\accProb^{\numberofconstrains + 3}}{q^{\numberofconstrains + 2}} - \frac{\accProb \cdot (\numberofconstrains + 2)}{2^\secpar}\right)\,.
	% \]
	% 
	% \ngame{3}
	% This is identical to $\game{2}$ except that now the environment uses the tree $\tree$ to extract the witness for the proven statement and aborts when it fails. Denote that event by $\event{\errss}$.
	% 
	% \ncase{$\game{2} \mapsto \game{3}$}	
	% As previously, 
	% \[
	% 	\abs{\prob{\game{2}} - \prob{\game{3}}} \leq \event{\errss}\,.
	% \]
	% Since $\plonkprot$ is special-sound the probability that $\env$ fails in extracting the witness is upper-bounded by some negligible $\eps_\ss$.
	% 
	% In the last game, Game $\game{3}$, the environment aborts when it fails to extract the correct witness, hence the adversary $\advse$ cannot win. 
	% Thus, by the game-hoping argument, 
	% \[
	% 	\abs{\prob{\game{0}} - \prob{\game{3}}} \leq 1 - \left(\frac{\accProb^{\numberofconstrains + 3}}{q^{\numberofconstrains + 2}} - \frac{\accProb \cdot (\numberofconstrains + 2)}{2^\secpar}\right) + \epsur + \epsss\,.
	% \]
	% Thus the probability that extractor $\extss$ succeeds is at least
	% \[
	% 	\left(\frac{\accProb^{\numberofconstrains + 3}}{q^{\numberofconstrains + 2}} - \frac{\accProb \cdot (\numberofconstrains + 2)}{2^\secpar}\right) - \epsur - \epsss\,.
	% \]
	% Since $\accProb$ is probability of $\advse$ outputting acceptable transcript that does not break $\ur{3}$-ness of $\plonkprot$, then $\waccProb \leq \accProb + \epsur$, where $\waccProb$ is the probability of $\advse$ outputing an acceptable proof as defined in \cref{def:simext}. It thus holds
	% \[
 	% 	\label{eq:frk}
 	% 	\extProb \geq \frac{(\waccProb - \epsur)^{\numberofconstrains + 3}}{q^{\numberofconstrains + 2}} - \underbrace{\frac{(\waccProb - \epsur) \cdot (\numberofconstrains + 2)}{2^\secpar} - \epsur - \epsss}_{\eps}\,.
 	% \]
 	% Note that the part of \cref{eq:frk} denoted by $\eps$ is negligible and 
 	% \[
 	% 	\extProb \geq \frac{1}{q^{\numberofconstrains + 2}} (\waccProb - \epsur)^{\numberofconstrains + 3} -\eps\,.
 	% \] 
 	% thus
 	% $\plonkprot_\fs$ is simulation extractable with extraction error $\epsur$.
 % 	\qed
 % \end{proof}

\section{Simulation extractability of $\sonicprotfs$}
\subsection{Unique opening property of $\PCOMs$}
\begin{lemma}
	\label{lem:pcoms_unique_op}
	Let $\PCOMs$ be a batched version of a KZG polynomial commitment
	\cite{AC:KatZavGol10} as described in \cite{CCS:MBKM19} then $\PCOMs$ has the unique opening property. 
\end{lemma}
\begin{proof}
	Let 
	$z \in \FF_p$ be the attribute the polynomial is evaluated at,
	$c \in \GRP$ be the commitment,  
	$s \in \FF_p$ the evaluation, and 
	$o \in \FF_p$ be the commitment opening. 
	We need to show that for every $\ppt$ adversary $\adv$ probability
	\[
		\Pr
			\left[
				\begin{aligned}
					& \verify(\crs, c, z, s, o), \\
					& \verify(\crs, c, z, \tilde{s}, \tilde{o})
				\end{aligned}
			\,\left|\,
			\vphantom{\begin{aligned}
				& \verify(\crs, c, z, s, o),\\
				& \verify(\crs, c, z, \tilde{s}, \tilde{o}) \\
				&o \neq \tilde{o})
			\end{aligned}}
			\begin{aligned}
				& \crs \gets \kcrs(\secparam), \\
				&	(c, z, s, \tilde{s}, o, \tilde{o}) \gets \adv(\crs)
			\end{aligned}
			\right.\right]
		 % \leq \negl.
	\]
	is at most negligible.
	
	As noted in \cite[Lemma 2.2]{EPRINT:GabWilCio19} it is enough to upper bound
  the probability of the adversary succeeding using the idealised verification equation---which considers equality between polynomials---instead of the real verification equation---which consider equality of the polynomials' evaluations.
	
	For a polynomial $f$, evaluation point $z$, evaluation result $s$, and opening $o(X)$ the idealised check verifies that
	\begin{equation}
		\alpha (X^{\dconst - \maxconst}f(X) \cdot X^{-\dconst + \maxconst} -  s) \equiv \alpha \cdot o(X) (X - z)\,,
	\end{equation}
	what is equivalent to 
	\begin{equation}
		f(X) -  s \equiv o(X) (X - z)\,.
		\label{eq:pcoms_idealised_check}
	\end{equation}
	Since $o(X)(X - z) \in \FF_p[X]$ then from the uniqueness of polynomial composition, there is only one $o(X)$ that fulfils the equation above.

	\qed
\end{proof}


\subsection{Unique response property}
The unique response property of $\sonicprot$ follows from the unique opening
property of the used polynomial commitment scheme $\PCOMs$.
\begin{lemma}
	\label{lem:sonicprot_ur}
	If a polynomial commitment scheme $\PCOMs$ is evaluation binding with
	parameter $\epsbind$ and has unique openings property with parameter
	$\epsop$, then $\sonicprot$ is $\ur{1}$ with parameter $\epsur \leq
	\epsbind + \epsop$.  
\end{lemma}
\begin{proof}
	Let $\adv$ be an adversary that breaks $\ur{1}$-ness of $\sonicprot$. 
	We consider two cases, depending on which round $\adv$ is able to provide at
	least two different outputs such that the resulting transcripts are
	acceptable.  For the first case we show that $\adv$ breaks the evaluation
	binding property of $\PCOMs$, while for the
  second case we show that it breaks the unique opening property of $\PCOMs$.
	
	The proof goes similarly to the proof of \cref{lem:plonkprot_ur} thus we
	provide only draft of it here. 
	In each Round $i$, for $i > 1$, the prover either commits to some
	well-defined polynomials (deterministically), evaluates these on some
	randomly picked points, or shows that the evaluations were performed
	correctly. 
	Naturally for a committed polynomial $\p{p}$ evaluated at point $x$ only one
	value $y = \p{p}(x)$ is correct. If the adversary was able to provide two
	different values $y$ and $\tilde{y}$ that would be accepted as an evaluation
	of $\p{p}$ at $x$ then the $\PCOMs$'s evaluation binding would be broken.
	Alternatively, if $\adv$ was able to provide two openings $\p{W}$ and
	$\p{\tilde{W}}$ for $y = \p{x}$ then the unique opening property would be
	broken.

	Hence the probability that $\adv$ breaks $\ur{1}$-property of $\PCOMs$ is
	upper-bounded by $\epsbind + \epsop$. \michals{22.X}{Do we need to multiply
	the epsilons by some constant that relates to the number of chances
	the adversary can break the binding and op properties?} 
	\qed
	
\end{proof}

\subsection{Special soundness}
\begin{lemma}
		\label{lem:sonicprot_ss}
		Let $\adv$ be a $\ppt$ algebraic adversary. The probability $\epsss$ that
		$\adv$ breaks computational special soundness of $\sonicprot$ is upper-bounded as
		\[
				\epsss \leq \epss + \epsldlog\,,
		\]
		where $\epss$ is a soundness error of the protocol, and $\epsldlog$ is a probability that a $\ppt$ algorithm can break the
		$(\dconst, \dconst)$-ldlog assumption.
\end{lemma}
\begin{proof}
		The proof goes similarly to the proof of \cref{lem:plonkprot_ss}.
%
		Let $\adv$ be an adversary that produces a $(1, 1, \multconstr + \linconstr + 1, 1, 1)$-tree of acceptable
		transcripts $\tree$ for a statement $\inp$. We consider two disjunctive
		events $\event{E}$ and $\nevent{E}$. The first corresponds to a case when
		all transcripts in $\tree$ are acceptable for the ideal verifier,
		i.e.~$\vec{\vereq}(X) = \vec{0}$. In that case we show an extractor $\extss$ that
		from $\tree$ extracts a valid witness $\wit$. The second, corresponds to
		a case when $\tree$ contains a transcript that is acceptable by the
		real verifier but is not acceptable by the ideal verifier. In that case we
		show a reduction $\rdvldlog$ that uses $\adv$ to break the $(\dconst,
		\dconst)$-ldlog
		assumption with at least the same probability.

		\ncase{When $\event{E}$ happens}
		Since $\sonicprot$ is perfectly sound regarding the ideal verifier, for an
		acceptable proof $\zkproof$ for a statement $\inp$ there exists a witness
		$\wit$ such that $\REL(\inp, \wit)$ holds and the polynomial $\p{r}(X, Y)$
		contains witness at its coefficients.  Note that the witness-carrying
		polynomial $\p{r}(X, y)$ has degree at most $(\multconstr + \linconstr)$
		and since $\adv$ answered correctly on $(\multconstr + \linconstr + 1)$
		different challenges $z$ (for the sake of concreteness let us call them
		$z_1, \ldots, z_{\multconstr + \linconstr + 1}$) then $(\multconstr +
		\linconstr + 1)$ evaluations $\p{r}(z_1, y), \ldots, \p{r}(z_{\multconstr
		+ \linconstr + 1}, y)$ of $\p{r}(X, y)$ are known. The extractor $\extss$
		interpolates $\p{r}(X, y)$ and reveals the corresponding witness $\wit$.

		\ncase{When $\nevent{E}$ happens} Consider a transcript such that for some verification equation
		$\vereq_i(X) \neq 0$, but $\vereq_i(\chi) = 0$.
		Since the adversary is algebraic, all group elements included in the tree of
  	transcripts are extended by their representation as a combination of the input
  	$\GRP_1$ or $\GRP_2$-elements. Hence all coefficients of the verification equation polynomial
  	$\vereq_i(X)$ are known and $\rdvdlog$ can find its zero points. Since
  	$\vereq_i(\chi) = 0$, the targeted discrete log value $\chi$ is among them.
		\qed		
\end{proof}

\subsection{Zero knowledge}
\begin{lemma}
	$\sonic$ is honest verifier zero-knowledge in the standard model.	
\end{lemma}
% Here we show that $\sonic$ is trapdoor-less zero-knowledge.
\begin{proof}
The simulator proceeds as follows.
In the first round, it picks randomly vectors $\vec{a}$, $\vec{b}$ and sets
\begin{equation}
		\label{eq:ab_eq_c}
		\vec{c} = \vec{a} \cdot \vec{b}. 
\end{equation}
Then it honestly computes polynomials
$\p{r}(X, Y), \p{r'}(X, Y), \p{s}(X, Y)$ and $\p{t}(X, Y)$ and concludes the
first round as an honest prover would. 

$\simulator$ computes the first verifier's challenge $y$ such that $\p{t}(X,
y)$ was a polynomial that has $0$ as a coefficient next to $X^0$.
As noted in \cite{CCS:MBKM19}, the coefficient next to $X^0$ in $\p{t}(X, Y)$
equals
\begin{equation}
		\label{eq:x0}
		\ptx (Y) = 
		\vec{a} \cdot \vec{\p{u}}(Y) + 
		\vec{b} \cdot \vec{\p{v}}(Y) + 
		\vec{c} \cdot \vec{\p{w}}(Y) +	
		\sum_{i = 1}^{\multconstr} a_i b_i (Y^i + Y^{-i}) - \p{k}(Y), 
\end{equation}
for public $\vec{\p{u}}(Y), \vec{\p{v}}(Y), \vec{\p{w}}(Y)$ defined as 
\begin{align*}
		\p{u_i}(Y) & = \sum_{q = 1}^\linconstr Y^{q + \multconstr} u_{q, i}\\
		\p{v_i}(Y) & = \sum_{q = 1}^\linconstr Y^{q + \multconstr} v_{q, i}\\
		\p{w_i}(Y) & = -Y^i - Y^{-i}  + \sum_{q = 1}^\linconstr Y^{q +
		\multconstr} w_{q, i}\\
		\p{k}(Y) & = \sum_{q = 1}^\linconstr Y^{q + \multconstr} k_{q}.
\end{align*}
Vectors $\vec{u_q}, \vec{v_q}, \vec{w_q}$ are $\multconstr$-elements long
and correspond to the linear constrains of the proof system. Field element
$k_{q}$ is
the instance value. When the proven instance is correct, $\ptx$ is a zero
polynomial. See \cite{CCS:MBKM19} for  details.
Also, when \cref{eq:ab_eq_c} holds, that polynomial got simplified to
\begin{equation}
		\label{eq:x0_simpl}
		\ptxsim(Y) = \vec{a} \cdot \vec{\p{u}}(Y) + 
		\vec{b} \cdot \vec{\p{v}}(Y) + 
		\vec{c} \cdot \vec{\p{\tilde{w}}}(Y)	
		- \p{k}(Y), 
\end{equation}
where $\vec{\p{\tilde{w}}}(Y)$ is defined as
\[
		\p{\tilde{w}_i}(Y) = \sum_{q = 1}^\linconstr Y^{q +
		\multconstr} w_{q, i}\,.
\]
Note that the polynomial from \cref{eq:x0_simpl} is a ``classical'',
i.e.~non-Laurent, polynomial. Also, since $\vec{a}, \vec{b}$ were picked at
random, $\p{t}(X, Y)$ is a random polynomial and also $\ptx$ and $\ptxsim$
are. Probability that a random degree-$\dconst$ polynomial over $\FF_p[Y]$ has
a root is at least $\infrac{1}{\dconst !}$, see \cref{lem:root_prob} for a
proof of that bound. Since we assume that $\dconst =
\poly$, we can say that the polynomial $\ptxsim$ picked by the simulator
has roots with non-negligible probability. Furthermore, these roots can be
found and are random $\FF_p$ elements. Hence, $y$ picked by the simulator has
comes from the same distribution as it was picked by an honest verifier.

The simulator continues building the transcript by honestly computing prover's
messages and by picking verifier's challenges at random. This and the fact
that $\ptxsim(y) = 0$ assures that the transcript
provided by the simulator is acceptable and comes from the same distribution
as a transcript prepared by an honest prover and verifier.
\qed
\end{proof}

\subsection{From special-soundness and unique response property to simulation extractability of $\sonicprotfs$}
Since \cref{lem:sonicprot_ur,lem:sonicprot_ss} hold, $\sonicprot$ is $\ur{1}$ and computationally special sound we are able to follow \cite{INDOCRYPT:FKMV12} and show that $\sonicprotfs$ is simulation-extractable as defined in \cref{def:simext}.

\begin{corollary}[Simulation extractability of $\sonicprotfs$]
	\label{thm:sonicprotfs_se}
	Assume that  $\sonicprot$ is $\ur{1}$ with security $\epsur(\secpar)$, and
	computational special-sound with security $\epsss(\secpar)$.  Let $\ro\colon
	\bin^* \to \bin^\secpar$ be a random oracle. 
	Let $\advse$ be a $\ppt$ adversary that can make up to $q$ random oracle
	queries and outputs an acceptable proof for $\sonicprotfs$ with probability
	at least $\waccProb$.  Then $\sonicprotfs$ is simulation-extractable with
	extraction error $\eta = \epsur$. The extraction probability $\extProb$ is at least
	\[
			\extProb  \geq \frac{1}{q^{\multconstr + \linconstr}} (\waccProb - \epsur)^{\multconstr +
			\linconstr + 1} - \eps.
	\]
	for some negligible $\eps$, $\multconstr$ and $\linconstr$ being,
	respectively, the multiplicative and linear constrains of the system.
\end{corollary}

\bibliographystyle{alpha}
\bibliography{cryptobib/abbrev1,cryptobib/crypto,additional_bib}

\appendix

\section{$\plonk$ protocol rolled out}
\label{sec:plonk_explained}
\ncase{$\plonk$ prover $\prover(\REL, \crs, \inp, \wit = (\wit_i)_{i \in \range{1}{3 \cdot \numberofconstrains}})$}
\begin{description}
	\item[Round 1] 
	Sample $b_1, \ldots, b_9 \sample \FF_p$; compute $\p{a}(X), \p{b}(X), \p{c}(X)$ as 
	\begin{align*}t t
		\p{a}(X) &= (b_1 X + b_2)\p{Z_H}(X) + \sum_{i = 1}^{\numberofconstrains} \wit_i \lag_i(X) \\
		\p{b}(X) &= (b_3 X + b_4)\p{Z_H}(X) + \sum_{i = 1}^{\numberofconstrains} \wit_{\numberofconstrains + i} \lag_i(X) \\
		\p{c}(X) &= (b_5 X + b_6)\p{Z_H}(X) + \sum_{i = 1}^{\numberofconstrains} \wit_{2 \cdot \numberofconstrains + i} \lag_i(X) 
	\end{align*}
	Output $\gone{\p{a}(\chi), \p{b}(\chi), \p{c}(\chi)}$.
	
	\item[Round 2]
	Get challenges $\beta, \gamma \in \FF_p$
	\[
		\beta = \ro(\trans, 0)\,, \qquad \gamma = \ro(\trans, 1)\,.
	\]
	Compute permutation polynomial $\p{z}(X)$
	\begin{multline*}
		\p{z}(X) = (b_7 X^2 + b_8 X + b_9)\p{Z_H}(X) + \lag_1(X) + \\
			+ \sum_{i = 1}^{\numberofconstrains - 1} 
			\left(\lag_{i + 1} (X) \prod_{j = 1}^{i} 
			\frac{
			(\wit_j +\beta \omega^{j - 1} + \gamma)(\wit_{\numberofconstrains + j} + \beta k_1 \omega^{j - 1} + \gamma)(\wit_{2 \numberofconstrains + j} +\beta k_2 \omega^{j- 1} + \gamma)}
			{(\wit_j+\sigma(j) \beta + \gamma)(\wit_{\numberofconstrains + j} + \sigma(\numberofconstrains + j)\beta + \gamma)(\wit_{2 \numberofconstrains + j} + \sigma(2 \numberofconstrains + j)\beta + \gamma)}\right)
	\end{multline*}
	Output $\gone{\p{z}(\chi)}$
		
	\item[Round 3]
	Get the challenge $\alpha = \ro(\trans)$, compute the quotient polynomial 
	\begin{align*}
	& \p{t}(X)  = \\
	& (\p{a}(X) \p{b}(X) \selmulti(X) + \p{a}(X) \selleft(X) + 
	\p{b}(X)\selright(X) + \p{c}(X)\seloutput(X) + \pubinppoly(X) + \selconst(X)) 
	\frac{1}{\p{Z_H}(X)} +\\
	& + ((\p{a}(X) + \beta X + \gamma) (\p{b}(X) + \beta k_1 X + \gamma)(\p{c}(X) 
	+ \beta k_2 X + \gamma)\p{z}(X)) \frac{\alpha}{\p{Z_H}(X)} \\
	& - (\p{a}(X) + \beta \p{S_{\sigma 1}}(X) + \gamma)(\p{b}(X) + \beta 
	\p{S_{\sigma 2}}(X) + \gamma)(\p{c}(X) + \beta \p{S_{\sigma 3}}(X) + 
	\gamma)\p{z}(X \omega))  \frac{\alpha}{\p{Z_H}(X)} \\
	& + (\p{z}(X) - 1) \lag_1(X) \frac{\alpha^2}{\p{Z_H}(X)}
	\end{align*}
	Split $\p{t}(X)$ into degree less then $\numberofconstrains$ polynomials $\p{t_{lo}}(X), \p{t_{mid}}(X), \p{t_{hi}}(X)$, such that
	\[
		\p{t}(X) = \p{t_{lo}}(X) + X^{\numberofconstrains} \p{t_{mid}}(X) + X^{2 \numberofconstrains} \p{t_{hi}}(X)\,.
	\]
	Output $\gone{\p{t_{lo}}(\chi), \p{t_{mid}}(\chi), \p{t_{hi}}(\chi)}$.
	
	\item[Round 4]
	Get the challenge $\chz \in \FF_p$, $\chz = \ro(\trans)$.
	Compute opening evaluations
	\begin{align*}
			\p{a}(\chz), \p{b}(\chz), \p{c}(\chz), \p{S_{\sigma 1}}(\chz), \p{S_{\sigma 2}}(\chz), \p{t}(\chz), \p{z}(\chz \omega),
	\end{align*}
	Compute the linearisation polynomial
	\[
		\p{r}(X) = 
		\begin{aligned}
			& \p{a}(\chz) \p{b}(\chz) \selmulti(X) + \p{a}(\chz) \selleft(X) + \p{b}(\chz) \selright(X) + \p{c}(\chz) \seloutput(X) + \selconst(X) \\
			& + \alpha \cdot \left( (\p{a}(\chz) + \beta \chz + \gamma) (\p{b}(\chz) + \beta k_1 \chz + \gamma)(\p{c}(\chz) + \beta k_2 \chz + \gamma) \cdot \p{z}(X)\right) \\
			& - \alpha \cdot \left( (\p{a}(\chz) + \beta \p{S_{\sigma 1}}(\chz) + \gamma) (\p{b}(\chz) + \beta \p{S_{\sigma 2}}(\chz) + \gamma)\beta \p{z}(\chz\omega) \cdot \p{S_{\sigma 3}}(X)\right) \\
			& + \alpha^2 \cdot \lag_1(\chz) \cdot \p{z}(X)
		\end{aligned}
	\]
	Output $\p{a}(\chz), \p{b}(\chz), \p{c}(\chz), \p{S_{\sigma 1}}(\chz), \p{S_{\sigma 2}}(\chz), \p{t}(\chz), \p{z}(\chz \omega), \p{r}(\chz).$
	
	\item[Round 5]
	Compute the opening challenge $v \in \FF_p$, $v = \ro(\trans)$.
	Compute the openings for the polynomial commitment scheme 
	\begin{align*}
	& \p{W_\chz}(X) = \frac{1}{X - \chz} \left(
	\begin{aligned}
		& \p{t_{lo}}(X) + \chz^\numberofconstrains \p{t_{mid}}(X) + \chz^{2 \numberofconstrains} \p{t_{hi}}(X) - \p{t}(\chz)\\
		& + v(\p{r}(X) - \p{r}(\chz)) \\
		& + v^2 (\p{a}(X) - \p{a}(\chz))\\
		& + v^3 (\p{b}(X) - \p{b}(\chz))\\
		& + v^4 (\p{c}(X) - \p{c}(\chz))\\
		& + v^5 (\p{S_{\sigma 1}}(X) - \p{S_{\sigma 1}}(\chz))\\
		& + v^6 (\p{S_{\sigma 2}}(X) - \p{S_{\sigma 2}}(\chz))
	\end{aligned}
	\right)\\
	& \p{W_{\chz \omega}}(X) = \frac{\p{z}(X) - \p{z}(\chz \omega)}{X - \chz \omega}
\end{align*}
	Output $\gone{\p{W_{\chz}}(\chi), \p{W_{\chz \omega}}(\chi)}$.
\end{description}

\ncase{$\plonk$ verifier $\verifier(\REL, \crs, \inp, \zkproof)$}\ \newline
The \plonk{} verifier works as follows
\begin{description}
	\item[Step 1] Validate all obtained group elements.
	\item[Step 2] Validate all obtained field elements.
	\item[Step 3] Validate the instance $\inp = \smallset{\wit_i}_{i = 1}^\instsize$.
	\item[Step 4] Compute challenges $\beta, \gamma, \alpha, \alpha', \chz, v, u$ from the transcript.
	\item[Step 5] Compute zero polynomial evaluation $\p{Z_H} (\chz)  =\chz^\numberofconstrains - 1$.
	\item[Step 6] Compute Lagrange polynomial evaluation $\lag_1 (\chz) = \frac{\chz^\numberofconstrains -1}{\numberofconstrains (\chz - 1)}$.
	\item[Step 7] Compute public input polynomial evaluation $\pubinppoly (\chz) = \sum_{i \in \range{1}{\instsize}} \wit_i \lag_i(\chz)$.
	\item[Step 8] Compute quotient polynomials evaluations
	\begin{multline*}
		\p{t} (\chz)  = \frac{1}{\p{Z_H}(\chz)}
		\Big(
			\p{r} (\chz) + \pubinppoly(\chz) - (\p{a}(\chz) + \beta \p{S_\sigma 1}(\chz) + \gamma) (\p{b}(\chz) + \beta \p{S_\sigma 2}(\chz) + \gamma) \\
			(\p{c}(\chz) +
			\gamma)\p{z}(\chz \omega) \alpha - \lag_1 (\chz) \alpha^2
		\Big) \,.
	\end{multline*}
	\item[Step 9] Compute batched polynomial commitment
	$\gone{D} = v \gone{r} + u \gone {z}$ that is
	\begin{align*}
		\gone{D} & = v
		\left(
		\begin{aligned}
			& \p{a}(\chz)\p{b}(\chz) \cdot \gone{\selmulti} + \p{a}(\chz)  \gone{\selleft} + \p{b}  \gone{\selright} + \p{c}  \gone{\seloutput} + \\
			& + (	(\p{a}(\chz) + \beta \chz + \gamma) (\p{b}(\chz) + \beta k_1 \chz + \gamma) (\p{c} + \beta k_2 \chz + \gamma) \alpha  + \lag_1(\chz) \alpha^2)  + \\
			% &   \\
			& - (\p{a}(\chz) + \beta \p{S_{\sigma 1}}(\chz) + \gamma) (\p{b}(\chz) + \beta \p{S_{\sigma 2}}(\chz) + \gamma) \alpha  \beta \p{z}(\chz \omega) \gone{\p{S_{\sigma 3}}(\chi)})
		\end{aligned}
		\right) + \\
		& + u \gone{\p{z}(\chi)}\,.
	\end{align*}
	\item[Step 10] Computes full batched polynomial commitment $\gone{F}$:
	\begin{align*}
		\gone{F} & = \left(\gone{\p{t_{lo}}(\chi)} + \chz^\numberofconstrains \gone{\p{t_{mid}}(\chi)} + \chz^{2 \numberofconstrains} \gone{\p{t_{hi}}(\chi)}\right) + u \gone{\p{z}(\chi)} + \\
		& + v
		\left(
		\begin{aligned}
			& \p{a}(\chz)\p{b}(\chz) \cdot \gone{\selmulti} + \p{a}(\chz)  \gone{\selleft} + \p{b}(\chz)   \gone{\selright} + \p{c}(\chz)  \gone{\seloutput} + \\
			& + (	(\p{a}(\chz) + \beta \chz + \gamma) (\p{b}(\chz) + \beta k_1 \chz + \gamma) (\p{c}(\chz)  + \beta k_2 \chz + \gamma) \alpha  + \lag_1(\chz) \alpha^2)  + \\
			% &   \\
			& - (\p{a}(\chz) + \beta \p{S_{\sigma 1}}(\chz) + \gamma) (\p{b}(\chz) + \beta \p{S_{\sigma 2}}(\chz) + \gamma) \alpha  \beta \p{z}(\chz \omega) \gone{\p{S_{\sigma 3}}(\chi)})
		\end{aligned}
		\right) \\
		& + v^2 \gone{\p{a}(\chi)} + v^3 \gone{\p{b}(\chi)} + v^4 \gone{\p{c}(\chi)} + v^5 \gone{\p{S_{\sigma 1}(\chi)}} + v^6 \gone{\p{S_{\sigma 2}}(\chi)}\,.
	\end{align*}
	\item[Step 11] Compute group-encoded batch evaluation $\gone{E}$
	\begin{align*}
		\gone{E}  = \frac{1}{\p{Z_H}(\chz)} & \gone{
		\begin{aligned}
			& \p{r}(\chz) + \pubinppoly(\chz) +  \alpha^2  \lag_1 (\chz) + \\
			& - \alpha \left( (\p{a}(\chz) + \beta \p{S_{\sigma 1}} (\chz) + \gamma) (\p{b}(\chz) + \beta \p{S_{\sigma 2}} (\chz) + \gamma) (\p{c}(\chz) + \gamma) \p{z}(\chz \omega) \right)
		\end{aligned}
		}\\
		 + & \gone{v \p{r}(\chz) + v^2 \p{a}(\chz) + v^3 \p{b}(\chz) + v^4 \p{c}(\chz) + v^5 \p{S_{\sigma 1}}(\chz) + v^6 \p{S_{\sigma 2}}(\chz) + u \p{z}(\chz \omega) }\,.
	\end{align*}
	\item[Step 12] Check whether the verification $\vereq(\chi)$ equation holds
	\begin{multline}
		\label{eq:ver_eq}
		\left(
		\gone{\p{W_{\chz}}(\chi)} + u \cdot \gone{\p{W_{\chz \omega}}(\chi)}
		\right) \bullet
		\gtwo{\chi} - \\
		\left(
			\chz \cdot \gone{\p{W_{\chz}}(\chi)} + u \chz \omega \cdot \gone{\p{W_{\chz \omega}}(\chi)} + \gone{F} - \gone{E}
		\right) \bullet
		\gtwo{1} = 0\,.
	\end{multline}
The verification equation is a batched version of the verification equation from \cite{AC:KatZavGol10} which allows the verifier to check openings of multiple polynomials in two points (instead of checking an opening of a single polynomial at one point).
\end{description}

Since the original paper \cite{EPRINT:GabWilCio19} lacks of explanation how the simulator of \plonk{} works, it is presented here.

\ncase{$\plonk$ simulator $\simulator(\REL, \crs, \td, \inp)$}
% \paragraph{Simulation in \plonk.}
% The simulator $\simulator$ in $\plonk$ proceeds according to the following steps:
\begin{description}
	\item[Round 1]
	Since the simulator does not know a witness $\wit$ for the proven statement $\inp$, $\simulator$ cannot compute the output of this round accordingly to the protocol. Instead, it picks randomly both the "blinders" $b_1, \ldots, b_6$ and evaluations of polynomials $\p{a}, \p{b}, \p{c}$ by picking their coefficients randomly and outputting $\gone{\p{a}(\chi), \p{b}(\chi), \p{c}(\chi)}$.
	\item[Round 2]
	The simulator takes permutation argument challenges $\beta, \gamma$ as a random oracle output in the ongoing proof.
	Similarly as in the previous round, the simulator cannot evaluate the requested polynomial $\p{z}$ honestly as it does not know the witness, picks its coefficients randomly and outputs $\gone{\p{z}(\chi)}$.
	\item[Round 3]
	In this round the simulator starts by picking at random a challenge $\chz$ that will be later used to program a random oracle.
	Then it computes evaluations $\p{a}(\chz), \p{b}(\chz), \p{c}(\chz), \p{S_{\sigma 1}}(\chz), \p{S_{\sigma 2}}(\chz), \pubinppoly(\chz), \lag_1(\chz), \p{Z_H}(\chz),\allowbreak \p{z}(\chz\omega)$
	
	Given the evaluations $\simulator$ computes polynomial $\p{r}(X)$ honestly, i.e.
	\[
		\p{r}(X) = 
		\begin{aligned}
			& \p{a}(\chz) \p{b}(\chz) \selmulti(X) + \p{a}(\chz) \selleft(X) + \p{b}(\chz) \selright(X) + \p{c}(\chz) \seloutput(X) + \selconst(X) \\
			& + \alpha \cdot \left( (\p{a}(\chz) + \beta \chz + \gamma) (\p{b}(\chz) + \beta k_1 \chz + \gamma)(\p{c}(\chz) + \beta k_2 \chz + \gamma) \cdot \p{z}(X)\right) \\
			& - \alpha \cdot \left( (\p{a}(\chz) + \beta \p{S_{\sigma 1}}(\chz) + \gamma) (\p{b}(\chz) + \beta \p{S_{\sigma 2}}(\chz) + \gamma)\beta \p{z}(\chz\omega) \cdot \p{S_{\sigma 3}}(X)\right) \\
			& + \alpha^2 \cdot \lag_1(\chz) \cdot \p{z}(X)
		\end{aligned}
	\]
	and evaluates $\p{r}(X)$ at $\chz$.
	
	In the next step the simulator computes $\ev{t}$ as the verifier would compute in Step 8.
	Next, $\simulator$ picks randomly a polynomial $\p{t}$ such that $\p{t} (\chz) = \ev{t}$.
	The simulator concludes this round as an honest prover would by dividing $\p{t}$ into $\p{t_{hi}}, \p{t_{mid}}, \p{t_{lo}}$ and outputting $\gone{\p{t_{hi}}(\chi), \p{t_{mid}}(\chi), \p{t_{lo}}(\chi)}$. 
	\item[Round 4]
	The simulator program random oracle to return $\chz$ when queried on the current state of the transcript. 
	Since the necessary evaluations at $\chz$ are already computed, $\simulator$ simply outputs 
	\[
		\p{a}(\chz), \p{b}(\chz), \p{c}(\chz), \p{S_{\sigma 1}}(\chz), \p{S_{\sigma 2}}(\chz), \p{t}(\chz), \p{z}(\chz \omega)\,.
	\]
	\item[Round 5]
	In this round the simulator proceeds as an honest prover would.
	\end{description}

	\section{$\sonic$ protocol rolled out}
	\label{sec:sonic}
	%\ncase{$\sonic$ simulator $\simulator(\REL, \crs, \inp, \zkproof)$}
	%$\sonic$'s simulator proceeds as an honest prover except its behaviour in
	%Round 1 and setting a challenge after that round. Then it picks verifier's
	%challenges randomly and answers them honestly.
	%\begin{description}
	%		\item[Round 1] Pick $\vec{a}, \vec{b}$ randomly and set $\vec{c} =
	%				\vec{a} \cdot \vec{b}$. Honestly computes polynomials $\p{r}(X, Y),
	%				\p{r'}(X, Y), \p{s}(X, Y)$ and $\p{t}(X, Y)$. Commits to $\p{r}(X,
	%				1)$ and publishes the commitment. 
	%		\item[Round 1 challenge] compute and output as a challenge $y$ such that
	%				$\p{t}(X, y)$ is a polynomial with $0$ coefficient next to $X^0$. If
	%				$\simulator$ fails to find appropriate $y$, e.g.~because the created
	%				polynomial $\p{t}$ has no zero points, then it reverts to Round 1
	%				above and picks fresh $\vec{a}$ and $\vec{b}$.
	%\end{description}

	\section{Omitted lemmas and proofs}
	\begin{lemma}
			\label{lem:root_prob}
			Let $\p{f}(X)$ be a random degree-$d$ polynomial over $\FF_p[X]$. Then
			the probability that $\p{f}(X)$ has roots in $\FF_p$ is at least
			$\infrac{1}{d!}$.
	\end{lemma}
	\begin{proof}
			First observe that there is $p^{d}$ canonical polynomials in $\FF_p[X]$.
			Each of the polynomials may have up to $d$ roots. Consider polynomials
			which are reducible to polynomials of degree $1$, i.e.~polynomials that
			have all $d$ roots. The roots can be picked in $\bar{C}^{p}_{d}$
			ways, where $\bar{C}^{n}_{k}$ is the number of
			$k$-elements combinations with 
			repetitions from $n$-element set. That is,
			\[
					\bar{C}^n_k = \binom{n + k - 1}{k}\,.
			\]
			Thus, the probability that a randomly picked polynomial has all $d$
			roots is 
			\begin{multline*}
					\frac{\bar{C}^p_d}{p^d} = \frac{\binom{p + d - 1}{d}}{p^d} =
					\frac{\frac{(p + d - 1)!}{(p + d - 1 - d)! \cdot d!}}{p^d} =
					\frac{\frac{(p + d - 1) \cdot \ldots \cdot p \cdot (p - 1)!}{(p -
					1)! \cdot d!}}{p^d} = \frac{\frac{(p + d - 1)\cdot \ldots \cdot
	p}{d!}}{p^d}
					 \geq \frac{\frac{p^d}{d!}}{p^d} = \frac{1}{d!}
			\end{multline*}
			\qed
	\end{proof}

	\section{(Tight) simulation soundness of $\plonk$}
	\task{28.07}{All proofs in this section should be verified}
	% \michals{15.09}{As MK noted, this proof holds only for simulation soundness. For extractability we use special soundness (cf the next section)}

	\begin{theorem}[Simulation soundness]
		% \michals{16.09}{To be change to sim snd proof}
		Assume that $(\numberofconstrains + 2, 1)$-dlog is $\eps_\dlog(\secpar)$-hard, $\plonkprot$ is sound and $\ur{3}$ with security $\epss(\secpar)$ and $\epsur(\secpar)$ respectively. 
		Then the probability that an algebraic, $\ppt$ adversary $\advss$ breaks simulation soundness of $\plonkprotfs$ is upper-bounded by 
		\[
			\epsur(\secpar) + q_\ro^6 (\eps_{\dlog}(\secpar) + \epss(\secpar))\,,
		\]
		where $q_\ro$ is the total number of queries required by the adversary $\advss$.
	\end{theorem}
	\begin{proof}
		We proceed by contradiction. Suppose there exists a $\ppt$ adversary $\advss$ that breaks simulation soundness with non-negligible probability
		\[
		\eps := \Pr
			\left[
			\begin{aligned}
				& \plonkprotfs.\verifier(\REL, \crs, \inp, \zkproof_{\advss}),\\
				& (\inp_{\advss}, \zkproof_{\advss}) \not\in Q,\\
				& \inp_\advss \not\in \LANG_\REL 
			\end{aligned}
			\,\left|\, 
			\vphantom{\begin{aligned}
				& \plonkprotfs.\verifier(\REL, \crs, \inp, \zkproof_{\advss}),\\
				& (\inp_{\advss}, \zkproof_{\advss}) \not\in Q,\\
				& \inp_\advss \not\in \LANG_\REL 
			\end{aligned}}
			\begin{aligned}
				& \crs \gets \plonkprotfs.\kcrs(\REL, \secparam)\\
				& (\inp_{\advss}, \zkproof_{\advss}) \gets \advss^{\simulator, \ro} (\REL, \crs),		
			\end{aligned}
			\right.\right].
		\]

	In such case, we are able to build reductions $\rdvs$, $\rdvur$, $\rdvdlog$ which using $\advss$ as a black-box, violate either the soundness, unique response properties of the underlying interactive protocol $\plonkprot$, or the $(\numberofconstrains + 2, 1)$-dlog assumption.

	In the following we denote by $\zkproof_{\advss}, \zkproof_{\simulator}$ proofs
	returned by the adversary and the simulator respectively. We use $\zkproof[i]$
	to denote prover's message in the $i$-th round of the proof, $\zkproof[i].\ch$
	to denote the challenge that is given to the prover after $\zkproof[i]$, and
	$\zkproof\range{i}{j}$ to denote all messages of the proof including challenges between rounds $i$ and $j$.

	% Without loss of generality, for every acceptable proof $\zkproof_{\advss}$, we
	% assume that whenever $\advss$ outputs in Round $i$, $i \geq 3$, a message $m$, then
	% $\zkproof_{\advss}\range{1}{i}$ is a fresh query to the random oracle.
	% \markulf{30.08}{What does fresh mean here? What about messages coming from the simulator?} 
	% \michals{30.08}{I thought that with this assumption we may just consider only queries from $\advss$}
	Without loss of generality, we assume that whenever the accepting proof contains a response to a challenge from a random oracle, we assume that the adversary queried the oracle to get it. 
	It is straightforward to transform any adversary that violates this condition into an adversary that makes these additional queries to the random oracle and wins with the same probability.

	A crucial observation is that the adversary $\advss$ may have learned $\zkproof_{\advss}\range{1}{3}$ by querying the simulator on $\inp_{\advss}$ or might have computed it itself. We denote the first event by $\event{E}$ and the second by $\nevent{E}$. 
	%
	Additionally, we divide event $\nevent{E}$ into two disjunctive subevents: $\nevent{E}_0$ and $\nevent{E}_1$. 
	Event $\nevent{E}_0$ considers a case when the final proof provided by the adversary $\advss$ is accepted by the idealised verification equation, i.e.~for that proof $\vereq(X) = 0$. 
	Alternatively, event $\nevent{E}_1$ covers a case when for $\zkproof_\advss$ it
	holds that $\vereq(\chi) = 0$, but $\vereq(X) \neq 0$, where $\chi$ is $\plonkprotfs$'s trapdoor.
	%
	As all these events are mutually exclusive and exhaustive, we have
	\[
		\eps = \prob{\advss \text{ wins}} = \prob{\advss \text{ wins}, \event{E}} + \prob{\advss \text{ wins}, \nevent{E}_0} + \prob{\advss \text{ wins}, \nevent{E}_1}\,.
	\]


	Before analysing the events, we make the following observation.
	First of all, we allow reductions $\rdv_\dlog, \rdvur, \rdvs$ to simulate the random oracle and simulator for the adversary $\advss$. We argue that since the reductions in their simulation behaves as real random oracle or simulator would, the chances the adversary breaks simulation soundness does not change. 

	Furthermore, we note that since $\advss$ is algebraic, it outputs a proof $\zkproof_\advss$ that can be written as 
	\[
		\zkproof_\advss = \vec{M} \cdot (\underbrace{1 \| \chi \| \ldots \| \chi^{\numberofconstrains + 2}}_{\crs} \| \vec{\tilde{\zkproof}_\simulator}_1^{\top} \| \ldots \| \vec{\tilde{\zkproof}_\simulator}_{q_\simulator}^\top)^\top\,,
	\]
	where $\gone{1, \chi, \ldots, \chi^{\numberofconstrains + 2}}$ are all $\GRP_1$-elements from the CRS of $\plonkprotfs$, $\vec{M}$ is a matrix of coefficients output by $\advss$ aside the proof, $\vec{\tilde{\zkproof}_\simulator}_i$ denote all $\GRP_1$-elements from the simulated proof ${\zkproof_\simulator}_i$, and the adversary makes $q_\simulator$ queries to the simulator. 
	Since the reduction itself provides the simulated proofs, it knows a matrix $\vec{M'}$ such that
	\begin{equation}
		\label{eq:M_prim}
		\zkproof_\advss = \vec{M'} \cdot (1 \| \chi \| \ldots \| \chi^{\numberofconstrains + 2})^\top\,.
	\end{equation}
	We use this property when analysing the success probability of reductions $\rdvs$ and $\rdvdlog$.

	% Before analyzing the events $\nevent{E}_0$ and $\nevent{E}_1$, 
	Also note that a proof $\zkproof$ could be accepted only if the verification equation $\vereq(\chi)$ holds. That is, the verifier plugs-in elements of $\zkproof$ into $\vereq(\chi)$ and checks whether it equals $0$. That is what is called a \emph{real check} in \cite{EPRINT:GabWilCio19}. 
	On the other hand there is an \emph{idealised check}, which verifies whether $\vereq(X) = 0$ \emph{as a polynomial}---with proof elements being polynomials as well.

	\ncase{When $\event{E}$ happens}
	We assume that $\inp_{\advss}$ is submitted to the simulator $\simulator$. 
	We show how $\rdvur$ utilizes $\advss$, that makes use of $\inp_\advss, \zkproof_{\advss}\range{1}{3}$, to break the $\ur{3}$ property of $\plonkprot$. 
	This way we bound the probability $\prob{\adv \text{ wins}, \event{E}}$ by the probability of $\rdvur$ being able to win in the $\ur{3}$ game.

	Consider an algorithm $\rdvur$ that runs $\advss$ internally as a black-box:
	\begin{itemize}
		\item The reduction answers both queries to the simulator $\plonkprotfs.\simulator$ and to the random oracle. 
		It also keeps lists $Q$, for the simulated proofs, and $Q_\ro$ for the random oracle queries. 
		\item When $\advss$ outputs a fake proof $\zkproof_{\advss}$ for  $\inp_\advss$, $\rdvur$ looks through lists $Q$ and $Q_\ro$ until it finds 
		$\zkproof_{\simulator}\range{1}{3}$ such that $\zkproof_{\advss}\range{1}{2} = \zkproof_{\simulator}\range{1}{3}$ and a random oracle query $\zkproof_{\simulator}[3].\ch$ on $\zkproof_{\simulator}\range{1}{3}$.
		\item $\rdvur$ returns two proofs for $\inp_\advss$:
		\begin{align*}
			\zkproof_1 = (\zkproof_{\simulator}\range{1}{3}, \zkproof_{\simulator}[3].\ch, \zkproof_{\simulator}\range{4}{5})\\
			\zkproof_2 = (\zkproof_{\simulator}\range{1}{3}, \zkproof_{\simulator}[3].\ch, \zkproof_{\advss}\range{4}{5})
		\end{align*}
		\end{itemize}  
		If $\zkproof_1 = \zkproof_2$, then $\advss$ fails to break simulation extractability, as $\zkproof_2 \in Q$.
		On the other hand, if the proofs are not equal, then $\rdvur$ breaks $\ur{3}$-ness of $\plonkprot$. Thus 
		\[
			\prob{\advss \text{ wins}, \event{E}} \leq \epsur(\secpar).
		\]

	\ncase{When $\nevent{E}_0$ happens}
	In this case the reduction $\rdvs$ uses $\advss$ to break soundness of $\plonkprot$ with probability $\epss / q_{\ro}^6$, where $q_\ro$ is the number of total random oracle queries performed by the adversary or by $\rdvs$ on behalf of the simulator.
	As previously, $\rdvs$ runs $\advss$ internally and simulates its environment by answering to its queries to $\plonkprotfs.\simulator$ and $\ro$. The reduction works as follows:
	\begin{itemize}
	\item It guesses indices $i_1, \ldots, i_6$ such that random oracle queries $h_{i_1}, \ldots, h_{i_6}$ are the queries used in $\zkproof_\advss$. This is done with probability at least $1/q_{\ro}^6$ (since there are $6$ challenges from the verifier in $\plonkprot$).
		\item On input $h$ for the $i$-th, $i \not\in \smallset{{i_1}, \ldots, {i_6}}$, random
	    oracle query, $\rdvs$ returns randomly picked $
	    y$, sets $\ro(h) = y $ and stores $(h, y)$ in $Q_\ro$ if $h$ is sent to $\ro$ the first time. If that is not the case, $\rdv$ finds $h$ in $Q_\ro$ and returns the corresponding $y$.
		\item On input $h_{i_j}$ for the $i_j$-th, $i_j \in \smallset{{i_1}, \ldots, {i_6}}$,
	    random oracle query, $\rdvs$ parses $h_{i_j}$ as a partial proof transcript
	    $\zkproof[1..j]$ and runs $\plonkprot$ using $\zkproof[j]$ as a $\plonkprot.\prover$'s $j$-th message to $\plonkprot.\verifier$. The verifier responds with a challenge $\zkproof[j].\ch$. The reduction sets $\ro(h_{i_j}) = \zkproof[j].\ch$.
		\item On query $\inp_\simulator$ to $\simulator$ it runs a simulator
	    $\plonkprotfs.\simulator$ internally and returns $\zkproof_\simulator$. If
	    the random oracle query with input $\zkproof_\simulator[j]$, $1 \leq j \leq 2$, of the simulator is the $i_j$-th query,
	    generate $\zkproof_\simulator[j].\ch$ by invoking $\plonkprot.\verifier$ on
	    $\zkproof_\simulator[j]$ and programming $\ro(h_{i_j}) = \zkproof_\simulator[j].\ch$.
		\item Answers $\plonkprot.\verifier$'s challenge $\zkproof[j].\ch$ using the answer given by $\advss$, i.e.~$\zkproof_\advss[j + 1]$.
	\end{itemize}

	Assume that the $\plonkprot.\verifier$ accepts $\zkproof_\advss$. We consider a case when the idealised verification equation accepts. (Thus, the real verification accepts as well.) 
	In that case $\rdvs$ extracts from $\vec{M'}$ coefficients of $1, \chi, \ldots, \chi^{\numberofconstrains + 2}$ for polynomials $\p{a}(X), \p{b}(X)$, and $\p{c}(X)$ and reveals the witness $\wit_\advss$ (as it is encoded in theses polynomials' coefficients).
	If $\REL(\inp_\advss, \wit_\advss)$ holds then $\advss$ failed to break simulation-soundness of $\plonkprotfs$. On the other hand, if that is not the case, then $\rdvs$ breaks soundness of $\plonkprot$.
	%
	Since the reduction guesses queries $h_{i_1}, \ldots, h_{i_6}$ with probability $1/q_\ro^6$, then 
	\[
		\prob{\rdvs \text{ wins}} = \prob{\advss \text{ wins}, h_{i_1}, \ldots, h_{i_6} \text{ are guessed correctly}, \nevent{E}_0}\,.
	\] 
	Hence,
	\[
		\prob{\advss \text{ wins}, \nevent{E}_0} \leq q_\ro^6 \cdot \epsss(\secpar).
	\]

	\ncase{When $\nevent{E}_1$ happens}
	The reduction $\rdvdlog$ runs internally a protocol $\plonkprotfs$, which CRS is computed from the challenge $\gone{1, \chi, \ldots, \chi^{\numberofconstrains + 2}}, \gtwo{\chi}$ from the $(\numberofconstrains + 2, 1)$-dlog assumption challenger. 
	Then it proceeds as $\rdvs$ does, except in the last part, when the adversary provided its proof $\zkproof_\advss$, $\rdvdlog$ uses the fact that the real verification equation holds, but the ideal verification equation does not to break the dlog assumption. 

	Since $\vereq(X) \neq 0$, but $\vereq(\chi) = 0$ and $\rdvdlog$ knows
	$\vec{M'}$, as defined in \cref{eq:M_prim}, it can recreate all the polynomials
	submitted by $\advss$ as part of the proof and included in $\vereq(X)$. This
	way, it knows all coefficients of $\vereq(X)$. Thus it can factorise it and find
	its roots, one of them is the required $\chi$. Hence it holds, by the analogous analysis as in the previous case, that
	\[
		\prob{\advss \text{ wins}, \nevent{E}_1} \leq q_\ro^6 \cdot \eps_{\dlog}(\secpar).
	\]

	The proof is concluded by observing that the analysis of events $\event{E}, \nevent{E}_0, \nevent{E}_1$ gives
	\[
		\eps \leq \epsur(\secpar) + q_\ro^6 (\eps_{\dlog}(\secpar) + \epsss(\secpar))\,,
	\]
	hence $\eps$ is negligible if dlog is hard and $\plonkprot$ is sound and $\ur{3}$.
	\qed
	\end{proof}
\end{document}

