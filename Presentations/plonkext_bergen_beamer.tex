\documentclass[aspectratio=169]{beamer}
\usetheme{boxes}
\usefonttheme{professionalfonts,structurebold}
\usecolortheme{dove}

\usepackage[advantage,asymptotics,adversary,sets,keys,ff,lambda,primitives,events,operators,probability,logic,mm,complexity]{cryptocode}
\usepackage[capitalise]{cleveref}
\usepackage{scrextend}
\changefontsizes{9pt}

\title{Non malleability of the Fiat-Shamir transformation}
\author{\scriptsize{Michal Zajac \inst{1} \and Markulf Kohlweiss \inst{2}}}
\institute{\inst{1} Clearmatics \inst{2}IOHK}
\date{}

\newcommand{\newdefs}[1] {\setlength{\fboxsep}{1pt}\colorbox{gray!20}{\(#1\)}}

\newcommand{\COMMENT}[1]  {}

%general formatting
\newcommand{\pcvarstyle}[1]{\mathsf{#1}}
\newcommand{\comment}[1]{{\color{lightgray}#1}}
\newcommand{\continue}{{\Huge{\hl{$\cdots$}}}}

% General mathematics
\newcommand{\range}[2] {[#1 \, .. \, #2]}
\newcommand{\SD}{\Delta}
\newcommand{\smallset}[1] {\{#1\}}
\newcommand{\bigset}[1] {\left\{#1\right\}}
\newcommand{\GRP} {\mathbb{G}}
\newcommand{\pair} {\hat{e}}
\newcommand{\brak}[1] {\left(#1\right)}
\newcommand{\sbrak}[1] {(#1)}
\newcommand{\alg}[1] {\pcalgostyle{#1}}
\newcommand{\image} {\operatorname{im}}
\newcommand{\myland} {\,\land\,}
\newcommand{\mylor} {\,\lor\,}
\newcommand{\vect}[1] {\operatorname{vect}(#1)}
\newcommand{\w}{\omega}
\newcommand{\const}{\pcpolynomialstyle{const}}
\newcommand{\p}[1]{\pcpolynomialstyle{#1}}
\newcommand{\ev}[1]{\tilde{\pcpolynomialstyle{#1}}}
\newcommand{\numberofconstrains}{\pcvarstyle{n}}
\newcommand{\expected}[1]{\mathbb{E}\left[#1\right]}
\newcommand{\infrac}[2]{#1 / #2}

% bilinear maps

\newcommand{\bmap}[2] {\left[#1\right]_{#2}}
\newcommand{\gone}[1] {\bmap{#1}{1}}
\newcommand{\gtwo}[1] {\bmap{#1}{2}}
\newcommand{\gi} {\iota}
\newcommand{\gtar}[1] {\bmap{#1}{T}}
\newcommand{\grpgi}[1] {\bmap{#1}{\gi}}


% zero knowledge
\newcommand{\oracleo}{\mathsf{O}}
\newcommand{\crs}{\pcvarstyle{crs}}
\newcommand{\td}{\pcvarstyle{td}}
\newcommand{\ip}[2]{\left\langle #1, #2\right\rangle}
\newcommand{\zkproof}{\pi}
\newcommand{\proofsystem}{\mathrm{\Psi}}
\newcommand{\ps}{\proofsystem}
\newcommand{\nuppt}{\pcmachinemodelstyle{NUPPT}}
\newcommand{\ro}{\mathcal{H}}
\newcommand{\rof}[2]{\mathbf{\Omega}_{#1, #2}}
\newcommand{\trans}{\pcvarstyle{trans}}
\newcommand{\tr}{\pcvarstyle{tr}}
\newcommand{\instsize}{\pcvarstyle{n}}
\newcommand{\KG} {\mathsf{K}}
\newcommand{\kcrs} {\KG_{\crs}}
\renewcommand{\dist}{\ddv}
\newcommand{\fs}{\pcalgostyle{FS}}
\newcommand{\sigmaprot}{\pcalgostyle{\Sigma}}
\newcommand{\se}{\pcvarstyle{se}}
\newcommand{\snd}{\pcvarstyle{snd}}
\newcommand{\zk}{\pcvarstyle{zk}}
\newcommand{\advse}{\adv_\se}

%rewinding---tree of transcripts
\newcommand{\pcboolstyle}[1]{\mathtt{#1}}
\newcommand{\treebuild}{\pcalgostyle{TreeBuild}}
\newcommand{\tree}{\pcvarstyle{tree}}
\newcommand{\counter}{\pcvarstyle{counter}}


%PLONK related
\newcommand{\plonkprot}{\mathbf{P}}
\newcommand{\plonkprotfs}{\mathbf{P}_\fs}
\newcommand{\selector}[1]{\pcvarstyle{q_{#1}}}
\newcommand{\selmulti}{\selector{M}}
\newcommand{\selleft}{\selector{L}}
\newcommand{\selright}{\selector{R}}
\newcommand{\seloutput}{\selector{O}}
\newcommand{\selconst}{\selector{C}}
\newcommand{\chz}{\mathfrak{z}}
\newcommand{\reduction}{\rdv}

\newcommand{\game}[1]{\pcalgostyle{G}_{#1}}

\newcommand{\lag}{\p{L}}
\newcommand{\pubinppoly}{\p{PI}}

% general complexity theory
% \newcommand{\RND}[1]{\pcalgostyle{RND}(#1)}
\newcommand{\RND}[1]{\pcvarstyle{R}(#1)}
\newcommand{\RELGEN}{\mathcal{R}}
\newcommand{\REL}{\mathbf{R}}
\newcommand{\LANG}{\mathcal{L}}
\newcommand{\inp}{\pcvarstyle{x}}
\newcommand{\wit}{\pcvarstyle{w}}
\newcommand{\class}[1]{\mathfrak{#1}}
\newcommand{\ig}{\pcalgostyle{IG}}
\newcommand{\accProb}{\event{acc}}
\newcommand{\frkProb}{\event{frk}}
\newcommand{\FS}{\pcalgostyle{FS}} % Fiat-Shamir transform
\newcommand{\aux}{\pcvarstyle{aux}} %auxiliary input

%Plonk and Sonic
\newcommand{\plonk}{\ensuremath{\textsc{Plonk}}}
\newcommand{\plonkmod}{\ensuremath{\plonk^\star}}
\newcommand{\plonkint}{\ensuremath{\plonk^\star}}
\newcommand{\polyprot}{\pcalgostyle{poly}}
\newcommand{\plonkintpoly}{\plonkint_\polyprot}
\newcommand{\sonic}{\textsc{Sonic}}
\newcommand{\maxdegree}{\pcvarstyle{N}}

\newcommand{\dlog}{\pcvarstyle{dlog}}

\newcommand{\ur}[1]{{#1\text{-}\mathsf{ur}}}

%forking
\newcommand{\forking}{\pcalgostyle{F}}
\newcommand{\genforking}{\pcalgostyle{GF}}

%colors
\definecolor{darkmagenta}{rgb}{0.5,0,0.5}
\definecolor{lightmagenta}{rgb}{1,0.85,1}
\definecolor{lightmagenta}{rgb}{0.9,0.9,0.9}
\definecolor{darkred}{rgb}{0.7,0,0}
\definecolor{blueish}{rgb}{0.1,0.1,0.5}
\definecolor{pinkish}{rgb}{0.9,0.8,0.8}
\definecolor{darkgreen}{rgb}{0,0.6,0}
\definecolor{lightgreen}{rgb}{0.85,1,0.85}
\definecolor{skyblue}{rgb}{0.3,0.9,0.99}

%comments
\DeclareRobustCommand{\markulf}[2]  {{\color{darkmagenta}\hl{\scriptsize\textsf{Markulf #1:} #2}}}
\DeclareRobustCommand{\michals}[2]  {{\color{blueish}\sethlcolor{pinkish}\hl{\scriptsize\textsf{Michal #1:} #2}}}
\newcommand{\task}[2]{\todo[author=\textbf{Task},inline]{({\textit{#1}}) #2}}
% \newcommand{\task}[2] {\xcommenti{Task}{#1}{#2}}
% \DeclareRobustCommand{\task}[2]  {{\color{black}\sethlcolor{yellow}\hl{\textsf{TASK #1:} #2}}}

%%% Local Variables:
%%% mode: latex
%%% TeX-master: "plonkext"
%%% End:

\renewcommand{\emph}[1]{\textbf{#1}}

\begin{document}

\begin{frame}
\titlepage
\end{frame}

\begin{frame}
  % \frametitle{Prior our work}
  \begin{block}{Some background}
    Most efficient \emph{universal} or \emph{updatable} zkSNARKs use random-oracle
  \end{block}\pause  

  \begin{block}{Prior state-of-the-art}
    %\begin{itemize}
    %\item
    Proven security of the \emph{interactive} proof system
    % \item

    Security of the \emph{non-interactive} variant conjectured
      relying on Fiat--Shamir transformation
    %\end{itemize}
  \end{block}\pause

  \begin{block}{Problems}
    Fiat--Shamir transformation relies on \emph{forking lemma}

    Forking lemma shown secure only for a narrow class of protocols
    that \emph{requires only a single rewinding}

    Not a case for any known zkSNARK
  \end{block}\pause

  \begin{block}{The result}
    Generalized Fiat--Shamir transformation to work with wider class of protocols
    
     Shown that a wide class of RO-based zkSNARKs are
     simulation-extractable \emph{out-of-the-box}
     
     This includes: Plonk, Sonic, some variants of Marlin
  \end{block}
\end{frame}

\section{Fiat--Shamir transformation}
\begin{frame}[t]
  \frametitle{Fiat--Shamir transformation}
  \begin{block}{}
    \begin{columns}
      \begin{column}{.4\linewidth}
        Let $\ps = (\prover, \verifier)$ be a $\Sigma$-protocol which
        transcript consists of
        \begin{itemize}
        \item instance $\inp$
        \item prover's message $a$
        \item \fbox{verifier's challenge $b$}
        \item prover's answer $c$
        \end{itemize}
      \end{column}\pause
      \begin{column}{.4\linewidth}
        Let $\psfs$ be a variant of $\ps$ which transcript consists of
        \begin{itemize}
        \item instance $\inp$
        \item prover's message $a$
        \item \fbox{\parbox{3cm}{verifier's challenge $b = \ro(\inp, a)$}}
        \item prover's answer $c$
        \end{itemize}
      \end{column}
    \end{columns}\pause
    
    Then $\psfs$ is sound and zero-knowledge.
  \end{block}

  \begin{block}{}
    \centering
    \fbox{The above says nothing about \emph{multiround} protocols!}
  \end{block}
\end{frame}

\begin{frame}
  \frametitle{Security of FS transformation---forking lemma}
  \begin{lemma}[General forking lemma, cf.~\cite{INDOCRYPT:FKMV12,CCS:BelNev06}]
    \label{lem:forking_lemma}
    \begin{columns}
      \begin{column}{.5\linewidth}
	Fix $q \in \ZZ$ and a set $H$ of size $h > 2$. Let $\zdv$ be a $\ppt$
  algorithm that on input $y, h_1, \ldots, h_q$ returns $(i, s)$, where $i
  \in\range{0}{q}$ and $s$ is called a \emph{side output}. Denote by $\ig$ a
  randomised instance generator. We denote by $\accProb$ the probability
	\[
		\condprob{i > 0}{y \gets \ig; h_1, \ldots, h_q \sample H; (i, s) \gets
		\zdv(y, h_1, \ldots, h_q)}\,.
	\]
	Let $\forking_\zdv(y)$ denote the algorithm described in
  \cref{fig:forking_lemma}, then the probability $\frkProb$ defined as $
  \frkProb := \condprob{b = 1}{y \gets \ig; (b, s, s') \gets \forking_{\zdv}(y)}
  $ holds
	\[
		\frkProb \geq \accProb \brak{\frac{\accProb}{q} - \frac{1}{h}}\,.
	\]
      \end{column}
      \begin{column}{.4\linewidth}
	\begin{figure}
		\centering
		\fbox{
		\procedure{$\forking_\zdv (y)$}
		{
			\rho \sample \RND{\zdv}\\
			h_1, \ldots, h_q \sample H\\
			(i, s) \gets \zdv(y, h_1, \ldots, h_q; \rho)\\
			\pcif i = 0\ \pcreturn (0, \bot, \bot)\\
			h'_{i}, \ldots, h'_{q} \sample H\\
			(i', s') \gets \zdv(y, h_1, \ldots, h_{i - 1}, h'_{i}, \ldots,  h'_{q};
			\rho)\\
			\pcif (i = i') \land (h_{i} \neq h'_{i})\ \pcreturn (1, s, s')\\
			\pcind \pcelse \pcreturn (0, \bot, \bot)
		}}
		\caption{Forking algorithm $\forking_\zdv$}
		\label{fig:forking_lemma}
              \end{figure}
            \end{column}
            \end{columns}
\end{lemma}

\end{frame}

\end{document}