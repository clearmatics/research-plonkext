% !TeX spellcheck = en_GB
\let\accentvec\vec
% \documentclass[runningheads]{llncs}
 \documentclass[runningheads]{amsart}
\let\spvec\vec
\let\vec\accentvec
\usepackage{amssymb,amsmath}
\let\vec\spvec

\usepackage[T1]{fontenc}

\newcommand{\iflipics}[1] {}
\newcommand{\iflncs}[1] {#1}

\def\vec#1{\mathchoice{\mbox{\boldmath$\displaystyle#1$}}
	{\mbox{\boldmath$\textstyle#1$}}
	{\mbox{\boldmath$\scriptstyle#1$}}
	{\mbox{\boldmath$\scriptscriptstyle#1$}}}

\DeclareFontFamily{U}{mathx}{\hyphenchar\font45}
\DeclareFontShape{U}{mathx}{m}{n}{<-> mathx10}{}
\DeclareSymbolFont{mathx}{U}{mathx}{m}{n}
\DeclareMathAccent{\widebar}{0}{mathx}{"73}

% lncs size (as printed in books, with small margins):
%\usepackage[paperheight=23.5cm,paperwidth=15.5cm,text={12.2cm,19.3cm},centering]{geometry}

\newcommand{\ifamsart}[1] {#1}
\ifamsart{
	\newtheorem{theorem}{Theorem}%[section]
	\newtheorem{proposition}[theorem]{Proposition}
	\newtheorem{lemma}[theorem]{Lemma}
	\newtheorem{corollary}[theorem]{Corollary}
	\theoremstyle{definition}
	\newtheorem{definition}[theorem]{Definition}
	\newtheorem{example}[theorem]{Example}
}

\usepackage{soulutf8}
\soulregister\cite7
\soulregister\ref7
\soulregister\pageref7
\usepackage{hyperref}
\hypersetup{final}
\usepackage{mathrsfs}
\usepackage[advantage,asymptotics,adversary,sets,keys,ff,lambda,primitives,events,operators,probability,logic,mm,complexity]{cryptocode}
\usepackage[capitalise]{cleveref}
\usepackage{cite}
\usepackage{booktabs}
\usepackage{paralist}
\usepackage[innerleftmargin=5pt,innerrightmargin=5pt]{mdframed} 

\newcommand{\newdefs}[1] {\setlength{\fboxsep}{1pt}\colorbox{gray!20}{\(#1\)}}

\newcommand{\COMMENT}[1]  {}

%general formatting
\newcommand{\pcvarstyle}[1]{\mathsf{#1}}
\newcommand{\comment}[1]{{\color{lightgray}#1}}
\newcommand{\continue}{{\Huge{\hl{$\cdots$}}}}

% General mathematics
\newcommand{\range}[2] {[#1 \, .. \, #2]}
\newcommand{\SD}{\Delta}
\newcommand{\smallset}[1] {\{#1\}}
\newcommand{\bigset}[1] {\left\{#1\right\}}
\newcommand{\GRP} {\mathbb{G}}
\newcommand{\pair} {\hat{e}}
\newcommand{\brak}[1] {\left(#1\right)}
\newcommand{\sbrak}[1] {(#1)}
\newcommand{\alg}[1] {\pcalgostyle{#1}}
\newcommand{\image} {\operatorname{im}}
\newcommand{\myland} {\,\land\,}
\newcommand{\mylor} {\,\lor\,}
\newcommand{\vect}[1] {\operatorname{vect}(#1)}
\newcommand{\w}{\omega}
\newcommand{\const}{\pcpolynomialstyle{const}}
\newcommand{\p}[1]{\pcpolynomialstyle{#1}}
\newcommand{\ev}[1]{\tilde{\pcpolynomialstyle{#1}}}
\newcommand{\numberofconstrains}{\pcvarstyle{n}}
\newcommand{\expected}[1]{\mathbb{E}\left[#1\right]}
\newcommand{\infrac}[2]{#1 / #2}

% bilinear maps

\newcommand{\bmap}[2] {\left[#1\right]_{#2}}
\newcommand{\gone}[1] {\bmap{#1}{1}}
\newcommand{\gtwo}[1] {\bmap{#1}{2}}
\newcommand{\gi} {\iota}
\newcommand{\gtar}[1] {\bmap{#1}{T}}
\newcommand{\grpgi}[1] {\bmap{#1}{\gi}}


% zero knowledge
\newcommand{\oracleo}{\mathsf{O}}
\newcommand{\crs}{\pcvarstyle{crs}}
\newcommand{\td}{\pcvarstyle{td}}
\newcommand{\ip}[2]{\left\langle #1, #2\right\rangle}
\newcommand{\zkproof}{\pi}
\newcommand{\proofsystem}{\mathrm{\Psi}}
\newcommand{\ps}{\proofsystem}
\newcommand{\nuppt}{\pcmachinemodelstyle{NUPPT}}
\newcommand{\ro}{\mathcal{H}}
\newcommand{\rof}[2]{\mathbf{\Omega}_{#1, #2}}
\newcommand{\trans}{\pcvarstyle{trans}}
\newcommand{\tr}{\pcvarstyle{tr}}
\newcommand{\instsize}{\pcvarstyle{n}}
\newcommand{\KG} {\mathsf{K}}
\newcommand{\kcrs} {\KG_{\crs}}
\renewcommand{\dist}{\ddv}
\newcommand{\fs}{\pcalgostyle{FS}}
\newcommand{\sigmaprot}{\pcalgostyle{\Sigma}}
\newcommand{\se}{\pcvarstyle{se}}
\newcommand{\snd}{\pcvarstyle{snd}}
\newcommand{\zk}{\pcvarstyle{zk}}
\newcommand{\advse}{\adv_\se}

%rewinding---tree of transcripts
\newcommand{\pcboolstyle}[1]{\mathtt{#1}}
\newcommand{\treebuild}{\pcalgostyle{TreeBuild}}
\newcommand{\tree}{\pcvarstyle{tree}}
\newcommand{\counter}{\pcvarstyle{counter}}


%PLONK related
\newcommand{\plonkprot}{\mathbf{P}}
\newcommand{\plonkprotfs}{\mathbf{P}_\fs}
\newcommand{\selector}[1]{\pcvarstyle{q_{#1}}}
\newcommand{\selmulti}{\selector{M}}
\newcommand{\selleft}{\selector{L}}
\newcommand{\selright}{\selector{R}}
\newcommand{\seloutput}{\selector{O}}
\newcommand{\selconst}{\selector{C}}
\newcommand{\chz}{\mathfrak{z}}
\newcommand{\reduction}{\rdv}

\newcommand{\game}[1]{\pcalgostyle{G}_{#1}}

\newcommand{\lag}{\p{L}}
\newcommand{\pubinppoly}{\p{PI}}

% general complexity theory
% \newcommand{\RND}[1]{\pcalgostyle{RND}(#1)}
\newcommand{\RND}[1]{\pcvarstyle{R}(#1)}
\newcommand{\RELGEN}{\mathcal{R}}
\newcommand{\REL}{\mathbf{R}}
\newcommand{\LANG}{\mathcal{L}}
\newcommand{\inp}{\pcvarstyle{x}}
\newcommand{\wit}{\pcvarstyle{w}}
\newcommand{\class}[1]{\mathfrak{#1}}
\newcommand{\ig}{\pcalgostyle{IG}}
\newcommand{\accProb}{\event{acc}}
\newcommand{\frkProb}{\event{frk}}
\newcommand{\FS}{\pcalgostyle{FS}} % Fiat-Shamir transform
\newcommand{\aux}{\pcvarstyle{aux}} %auxiliary input

%Plonk and Sonic
\newcommand{\plonk}{\ensuremath{\textsc{Plonk}}}
\newcommand{\plonkmod}{\ensuremath{\plonk^\star}}
\newcommand{\plonkint}{\ensuremath{\plonk^\star}}
\newcommand{\polyprot}{\pcalgostyle{poly}}
\newcommand{\plonkintpoly}{\plonkint_\polyprot}
\newcommand{\sonic}{\textsc{Sonic}}
\newcommand{\maxdegree}{\pcvarstyle{N}}

\newcommand{\dlog}{\pcvarstyle{dlog}}

\newcommand{\ur}[1]{{#1\text{-}\mathsf{ur}}}

%forking
\newcommand{\forking}{\pcalgostyle{F}}
\newcommand{\genforking}{\pcalgostyle{GF}}

%colors
\definecolor{darkmagenta}{rgb}{0.5,0,0.5}
\definecolor{lightmagenta}{rgb}{1,0.85,1}
\definecolor{lightmagenta}{rgb}{0.9,0.9,0.9}
\definecolor{darkred}{rgb}{0.7,0,0}
\definecolor{blueish}{rgb}{0.1,0.1,0.5}
\definecolor{pinkish}{rgb}{0.9,0.8,0.8}
\definecolor{darkgreen}{rgb}{0,0.6,0}
\definecolor{lightgreen}{rgb}{0.85,1,0.85}
\definecolor{skyblue}{rgb}{0.3,0.9,0.99}

%comments
\DeclareRobustCommand{\markulf}[2]  {{\color{darkmagenta}\hl{\scriptsize\textsf{Markulf #1:} #2}}}
\DeclareRobustCommand{\michals}[2]  {{\color{blueish}\sethlcolor{pinkish}\hl{\scriptsize\textsf{Michal #1:} #2}}}
\newcommand{\task}[2]{\todo[author=\textbf{Task},inline]{({\textit{#1}}) #2}}
% \newcommand{\task}[2] {\xcommenti{Task}{#1}{#2}}
% \DeclareRobustCommand{\task}[2]  {{\color{black}\sethlcolor{yellow}\hl{\textsf{TASK #1:} #2}}}

%%% Local Variables:
%%% mode: latex
%%% TeX-master: "plonkext"
%%% End:


\title{On Simulation-Extractability of \textsc{Plonk}}
% \titlerunning{Achieving additional security in LegoSNARK}

\author{Michał Zając}
% \institute{ \email{}
% }

\allowdisplaybreaks

\begin{document}
	\sloppy
	\maketitle

\begin{abstract}
	In this paper we show that \plonk{}~\cite{EPRINT:GabWilCio19} is (non black-box) simulation extractable.
\end{abstract}

\section{Introduction}

\section{Preliminaries}
\subsection{Zero knowledge}
\begin{definition}[Simulation extractability]
	Proof system $\Psi$ is computationally (adaptively) \emph{simulation-extractable for $\RELGEN$}, if for every NUPPT $\adv$, there exists a NUPPT extractor $\ext_\adv$, s.t.
  \[
	\condprob{
  \begin{aligned}
    &(\inp, \wit) \not\in \REL \land \inp\not\in Q \\
    & \land \verifier (\REL, \crs_{\verifier}, \inp, \pi) = 1
  \end{aligned}
  }
  {
		\begin{aligned}
		& \REL \gets \RELGEN (\secparam),
		(\crs, \td) \gets \kgen (\REL), r \sample \RND{\adv},
		\\ &
		((\inp, \pi)  \|  \wit) \gets (\adv^{\oracleo}  \|  \ext_\adv) (\REL, \crs; r)
		\end{aligned}
	} \leq \negl \enspace,
	\]
	where $\oracleo$ on input $\inp'$ returns $\pi' \gets \simulator(\REL, \crs, \td, \cdot)$ and writes $\inp'$ to a list $Q$.
\end{definition}

The definition of simulation extractability as stated above allows the malicious prover to maul proofs output by the simulator. This possibility can be ruled out by requiring a stronger property as defined below.

\begin{definition}[Strong simulation extractability]
	Proof system $\Psi$ is computationally (adaptively) \emph{strongly simulation-extractable for $\RELGEN$}, if for every NUPPT $\adv$, there exists a NUPPT extractor $\ext_\adv$, s.t.
	\[
	\condprob{
  \begin{aligned}
    &(\inp, \wit) \not\in \REL \land (\inp, \pi) \not\in Q \\
    & \land \verifier (\REL, \crs_{\verifier}, \inp, \pi) = 1
  \end{aligned}
  }
  {
		\begin{aligned}
		& \REL \gets \RELGEN (\secparam),
		(\crs, \td) \gets \kgen (\REL), r \sample \RND{\adv},
		\\ &
		((\inp, \pi)  \|  \wit) \gets (\adv^{\oracleo}  \|  \ext_\adv) (\REL, \crs; r)
		\end{aligned}
	} \leq \negl \enspace,
	\]
	where $\oracleo$ on input $\inp'$ returns $\pi' \gets \simulator(\REL, \crs, \td, \cdot)$ and writes $(\inp', \pi')$ to a list $Q$.
\end{definition}

\subsection{Interactive zero knowledge}
Interactive proofs are often required to have the special soundness property which states that if for a given statement $\inp$ and first message $a$ the prover can respond to many verifier's challenges then she knows the witness. Moreover, a PPT algorithm with access to a number of prover--verifier conversations that shares the same first

\subsection{Interpolation of rational functions}
Let $F(X) = P(X) / Q(X)$ be a rational function defined over $\FF(X)$ such that $\deg P = m_a$ and $\deg Q = m_b$. We call a set $V = \smallset{(x_i, y_i)}_{i \in I}$ a \emph{support set} of $F$ if for all $i \in I$ holds $F(x_i) = y_i$. 
As shown by Minsky et al.~in \cite{TIT:MinTraZip03} if $P$ and $Q$ are monic, then a support set $V$ of size at least $m_a + m_b$ unambiguously determines $F$. 

\begin{lemma}[Rational function interpolation]
  Let $F(X) = P(X) / Q(X)$ be a rational function and let $P(X) = X^{m_a} + p_1 X^{m_a - 1} + \ldots p_{m_a}$, $Q(X) = X^{m_b} + q_1 X^{m_a - 1} + \ldots q_{m_b}$. Let $V_F$ be a support set of size $m = m_a + m_b$, then $V_F$ determines $F$ unambiguously.
\end{lemma}

We note that when $P$ and $Q$ are not monic, i.e~$P(X) = p_0 X^{m_a} + p_1 X^{m_a - 1} + \ldots p_{m_a}$, $Q(X) = q_0 X^{m_b} + q_1 X^{m_a - 1} + \ldots q_{m_b}$, for $p_0, q_0$ not equal $0$ or $1$, and $p_0$ and $q_0$ are known, then $F$ can be recreated anyway. 
To that end we define monic $P'(X) = P(X) / p_0$ and $Q'(X) =  Q(X) / q_0$, and recreate $F'(X) = P'(X) / Q'(X) = q_0 /p_0 \cdot F(X)$.


\subsection{Generalised forking lemma}
\subsection{Simulation extractability from special soundness}
\section{Simulation extractability of \plonk}
The main idea behind showing that \plonk{} is simulation extractable 

\bibliographystyle{alpha}
\bibliography{cryptobib/abbrev1,cryptobib/crypto,additional_bib}

\end{document}