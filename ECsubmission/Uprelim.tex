% !TEX root = main.tex
% !TEX spellcheck = en-US
\section{Preliminaries}
\label{sec:preliminaries}
\paragraph{Notation.} Let $\ppt$ denote probabilistic polynomial-time and $\secpar \in \NN$ be the
security parameter. All adversaries are stateful. For an algorithm $\adv$, let
$\image (\adv)$ be the image of $\adv$ (the set of valid outputs of $\adv$), let
$\RND{\adv}$ denote the set of random tapes of correct length for $\adv$
(assuming the given value of $\secpar$), and let $r \sample \RND{\adv}$ denote
the random choice of the randomiser $r$ from $\RND{\adv}$. We denote by $\negl$
($\poly$) an arbitrary negligible (resp.~polynomial) function.

Probability ensembles $X = \smallset{X_\secpar}_\secpar$ and $Y =
\smallset{Y_\secpar}_\secpar$, for distributions $X_\secpar, Y_\secpar$, have
\emph{statistical distance} $\SD$ equal $\epsilon(\secpar)$ if $\sum_{a \in
  \supp{X_\secpar \cup Y_\secpar}} \abs{\prob{X_\secpar = a} - \prob{Y_\secpar =
    a}} = \epsilon(\secpar)$. We write $X \approx_\secpar Y$ if $\SD(X_\secpar,
Y_\secpar) \leq \negl$. For values $a(\secpar)$ and $b(\secpar)$ we write
$a(\secpar) \approx_\secpar b(\secpar)$ if $\abs{a(\secpar) - b(\secpar)} \leq
\negl$.

For a probability space $(\samplespace, \eventspace, \probfunction)$ and event
$\event{E} \in \eventspace$ we denote by $\nevent{E}$ an event that is
complementary to $\event{E}$,
i.e.~$\nevent{E} = \samplespace \setminus \event{E}$.

\chaya{revisit after agreeing on how we deal with relations.}
Denote by $\RELGEN = \smallset{\REL}$ a family of relations. We assume that if
$\REL$ comes with any auxiliary input, the latter is benign. Directly from the
description of $\REL$ one learns security parameter $\secpar$ and description of the
group $\GRP$, if the relation is a relation of group elements (as it usually is
in case of zkSNARKs).

\ourpar{Bilinear groups.}
A bilinear group generator $\pgen (\secparam)$ returns public parameters $ \pp =
(p, \GRP_1, \GRP_2, \GRP_T, \pair, \gone{1}, \gtwo{1})$, where $\GRP_1$,
$\GRP_2$, and $\GRP_T$ are additive cyclic groups of prime order $p = 2^{\Omega
  (\secpar)}$, $\gone{1}, \gtwo{1}$ are generators of $\GRP_1$, $\GRP_2$, resp.,
and $\pair: \GRP_1 \times \GRP_2 \to \GRP_T$ is a non-degenerate
$\ppt$-computable bilinear pairing. We assume the bilinear pairing to be Type-3,
i.e., that there is no efficient isomorphism from $\GRP_1$ to $\GRP_2$ or from
$\GRP_2$ to $\GRP_1$. We use the by now standard bracket notation, i.e., we
write $\bmap{a}{\gi}$ to denote $a g_{\gi}$ where $g_{\gi}$ is a fixed generator
of $\GRP_{\gi}$. We denote $\pair (\gone{a}, \gtwo{b})$ as $\gone{a} \bullet
\gtwo{b}$. Thus, $\gone{a} \bullet \gtwo{b} = \gtar{a b}$. We freely use the
bracket notation with matrices, e.g., if $\vec{A} \vec{B} = \vec{C}$ then
$\vec{A} \grpgi{\vec{B}} = \grpgi{\vec{C}}$ and $\gone{\vec{A}}\bullet
\gtwo{\vec{B}} = \gtar{\vec{C}}$. Since every algorithm $\adv$ takes as input
the public parameters we skip them when describing $\adv$'s input. Similarly, we
do not explicitly state that each protocol starts with generating these
parameters by $\pgen$.

\begin{lemma}[Difference lemma,~{\cite[Lemma 1]{EPRINT:Shoup04}}]
	\label{lem:difference_lemma}
	Let $\event{A}, \event{B}, \event{F}$ be events defined in some probability
	space, and suppose that $\event{A} \land \nevent{F} \iff \event{B}
		\land \nevent{F}$.  Then 
	$
		\abs{\prob{\event{A}} - \prob{\event{B}}} \leq \prob{\event{F}}\,.
	$
\end{lemma}
\subsection{Algebraic Group Model}
The algebraic group model (AGM) introduced in \cite{C:FucKilLos18} lies between
the standard model and generic bilinear group model. In the AGM it is assumed
that an adversary $\adv$ can output a group element $\gnone{y} \in \GRP$ if
$\gnone{y}$ has been computed by applying group operations to group elements
given to $\adv$ as input. It is further assumed, that $\adv$ knows how to
``build'' $\gnone{y}$ from that elements. More precisely, the AGM requires that
whenever $\adv(\gnone{\vec{x}})$ outputs a group element $\gnone{y}$ then it
also outputs $\vec{c}$ such that $\gnone{y} = \vec{c}^\top \cdot
\gnone{\vec{x}}$. Both $\plonk$ and $\sonic$ have been shown secure using the
AGM. An adversary that works in the AGM is called \emph{algebraic}.
 
\subsection{Signatures of Knowledge}
\label{sec:sok}
  
Signatures of Knowledge (SoKs) \cite{C:ChaLys06} are closely related to simulation-extractable SNARKs: A signer can generate a valid signature for a message only if she has a valid witness for the statement. 

 The notion of SoKs mimics digital signatures with strong existential unforgeability: 
even if the adversary has seen many
signatures on arbitrary messages under arbitrary statement,  it cannot create a new signature (not seen before) without knowing the witness for the statement in question.

To define Signatures of Knowledge we use a recent version \cite{C:GroMal17} that implicitly considers only compact (succinct) signatures. 
     
A $\SoK$ for an efficiently decidable binary relation $\REL$ is defined as a tuple of PPT algorithms $\SoK = (\signsetup,  \sign, \allowbreak \verify,  \simsetup, \simsign)$:

\begin{description}
    \item[$\signsetup(1^\secpar, \REL) \rightarrow  \param$:]
	The setup algorithm takes a security parameter $\secpar$ and a binary relation $\REL$
	and returns public parameters $ \param$.  The input $ \param$ is implicit to all subsequent algorithms. 

    \item[$\sign(\mesage, \inp, \wit)  \rightarrow \signature$:]
The signing algorithm takes as input a message $ \msg \in \{0,1\}^{*}$, 
a statement $\inp$, and a witness $\wit$.
	Outputs a signature $\signature$.

    \item[$\verify(\mesage, \inp, \signature) \rightarrow 1/ 0$:]
The verification algorithm takes as input 
 a message $\mesage$,  a statement $\inp$ 
 together with a signature $\signature$,
	outputs $1$ if the the signature is valid, $0$ otherwise.
	
    \item[$\simsetup(\REL) \rightarrow (\param, \td)$:]
    	A simulated setup algorithm which takes as input a relation $\REL$ and returns public parameters $\param$ and a trapdoor $\td$. 
    	
   \item[$\simsign(\td, \mesage, \inp) \rightarrow \signature'$:]
   	A simulated signing  that takes as input  a trapdoor $\td$, a message $\mesage$ and a statement $\inp$ and returns a simulated signature $\signature'$.
\end{description}
 
A SoK scheme should satisfy correctness, extractability and simulatability:
   
\begin{description}
\item[Perfect Correctness.] This guarantees that a signer with a valid witness can always produce a signature that
will convince the verifier: 
For all $\secpar\in \NN$, for
all efficiently decidable binary relation $\REL$,  
for all  $(\inp, \wit) \in \REL$, and for all $ \mesage \in \{0,1\}^{*}$:
   \[
  \condprob{
	  \begin{matrix}
~\verify(\mesage, \inp, \signature) = 1   
	 \end{matrix}
}{
	  \begin{matrix}
~\param \gets \signsetup(1^\secpar, \REL)\\
~ \signature \gets  \sign(\mesage, \inp, \wit)
 \end{matrix} }  =1. 
\]
%
\item[Simulation Extractability:] This guarantees that an adversary is not able to issue a new signature
unless it knows a witness. This should hold even if the adversary gets to see signatures on
arbitrary messages under arbitrary statements. We model this notion in a strong sense, by
letting the adversary see simulated signatures for arbitrary messages and statements, which
potentially includes false statements. Even under this strong attack model, we require that
whenever the adversary outputs a valid signature not queried before, it is possible to extract a
witness for the signature. More formally,  for any PPT adversary $\adv$ there exists a PPT extractor $\ext_\adv$ such that:
   \[
  \condprob{
	  \begin{matrix}
~ \verify(\mesage, \inp, \signature) = 1   \\
\land (\inp, \wit) \notin \REL \\
\land (\mesage, \inp, \signature) \notin \text{Queries}^{\simsign_\td}
	 \end{matrix}
}{
	  \begin{matrix}
~\param \gets \signsetup(1^\secpar, \REL)\\
~ (\mesage, \inp, \signature) \gets \adv^{\simsign_\td} (\param)\\
\wit \gets \ext_\adv({\sf trans}_\adv)
 \end{matrix}
} \leq \negl
\]
where the adversary has access to simulated signatures via the oracle $\simsign_\td(\inp_i, \mesage_i) \coloneq \simsign(\td, \inp_i, \mesage_i)$ and the extractor $\ext_\adv$ takes as input the transcript ${\sf trans}_\adv$ of all queries made by $\adv$.

\vspace{4pt}
%
\item[Perfect Simulatability:] The verifier should not learn anything  about the witness from the signature.  The secrecy of the witness is modelled by the ability to
simulate signatures without the witness. More precisely,  we say the signatures of knowledge
are simulatable if there is a simulator that can create public parameters together with an associated trapdoor that enables producing signatures without a witness that are indistinguishable from real ones.  More formally, for any number of queried $\mesage_i\in \{0,1\}^{*}$ and $(\inp_i, \wit_i) \in \REL$:

$
		   \condprob{
		\begin{matrix}
		\adv(\param) = 1 
		\end{matrix}
		}{
		\begin{matrix}
	\param \gets  \signsetup(1^\secpar, \REL)\\
	~\signature_i \gets \sign(\mesage_i, \inp_i, \wit_i)
		\end{matrix}
		 }
		- $
\[	  \condprob{
		\begin{matrix}
		\adv (\param) = 1 
		\end{matrix}
}{
		\begin{matrix}
	~(\param,\td) \gets  \simsetup(1^\secpar, \REL) \\
	~\signature_i \gets   \simsign(\td,\mesage_i, \inp_i)
		\end{matrix}}
		 \leq \negl.
		\]
\end{description}
  

\subsection{Zero-Knowledge Proof Systems}
Let $\RELGEN(\secparam) = \smallset{\REL}$ be a family of
$\npol$ relations. Denote by $\LANG_\REL$ the language determined by $\REL$.
Let $\prover$ be a \emph{prover} and $\verifier$ be the \emph{verifier}, both $\ppt$ algorithms. We allow our proof system to have
a setup, i.e.~there is a $\kgen$ algorithm that takes as input the relation
description $\REL$ and outputs a common reference string $\srs$. We assume that
the $\srs$ defines the relation and for universal prove systems, such as Plonk
and Sonic, we treat both the reference string and the relation as universal.

We denote by $\ip{\prover(\srs, \inp, \wit)}{\verifier(\srs,\inp)}$ a
\emph{transcript} (also called \emph{proof}) $\zkproof$ of a conversation
between $\prover$ with input $(\srs, \inp, \wit)$ and $\verifier$ with input
$(\srs, \inp)$. We write
$\ip{\prover (\srs, \inp, \wit)}{\verifier(\srs, \inp)} = 1$ if in the end of
the transcript the verifier $\verifier$ returns $1$ and say that $\verifier$
accepts it.  For non-interactive proof systems we abuse notation and write
$\verifier(\srs, \inp, \zkproof) = 1$ to denote a fact that $\zkproof$ is
accepted by the verifier.  

A proof system $\proofsystem = (\kgen, \prover, \verifier, \simulator)$ for
$\RELGEN$ is required to have three properties: completeness, soundness and zero
knowledge, which are defined as follows:
% \begin{description}

\ourpar{Completeness.}
%\item[Completeness]
  An interactive proof system $\proofsystem$ is
  \emph{complete} if an honest prover always convinces an honest verifier, that
  is for all $\REL \in \RELGEN(\secparam)$ and $(\inp, \wit) \in \REL$
	\[
		\condprob{\ip{\prover (\srs, \inp, \wit)}{\verifier (\srs,
        \inp)} = 1}{\srs \gets \kgen(\REL)} = 1\,.
	\]
    % \item[Soundness]
\ourpar{Soundness.}
    We say that $\proofsystem$ for $\RELGEN$ is \emph{sound} if no
  $\ppt$ prover $\adv$ can convince an honest verifier $\verifier$ to accept a
  proof for a false statement $\inp \not\in\LANG$. More precisely, for
  all $\REL \in \RELGEN(\secparam)$
	\[
    \condprob{\ip{\adv(\srs, \inp)}{\verifier(\srs, \inp)} = 1 \land \inp
      \not\in \LANG_\REL}{\srs \gets \kgen(\REL), \inp \gets \adv(\srs)} \leq
    \negl.
	\]
%\end{description}
Sometimes a stronger notion of soundness is required---except requiring that the
verifier rejects proofs of statements outside the language, we request from the
prover to know a witness corresponding to the proven statement. This property is
called \emph{knowledge soundness}.%\markulf{Commented out the formal definition as we don't use it.}
 
\ourpar{Zero knowledge.}  We call a proof system $\proofsystem$
\emph{zero-knowledge} if for any $\REL \in \RELGEN(\secparam)$, and adversary
$\adv$ there exists a $\ppt$ simulator $\simulator$ such that for any
$(\inp, \wit) \in \REL$
\begin{multline*}
\hspace{-10pt}	  \left\{\ip{\prover(\srs, \inp, \wit)}{\adv(\srs, \inp, \wit)}
      \,\left|\, \srs \gets \kgen(\REL)\COMMENT{, (\inp, \wit) \gets \adv(\REL,
          \srs)}\vphantom{\simulator^\adv}\right.\right\} \approx_\secpar
		%\\
		\left\{\simulator^{\adv}(\srs, \inp)\,\left|\, \srs \gets
        \kgen(\REL)\COMMENT{, (\inp, \wit) \gets \adv(\REL,
          \srs)}\vphantom{\simulator^\adv}\right.\right\}.
\end{multline*}
	%
We call zero knowledge \emph{perfect} if the distributions are equal and
\emph{computational} if they are indistinguishable for any $\ppt$ distinguisher.

% \end{description}
Alternatively, zero-knowledge can be defined by allowing the simulator to use
the trapdoor $\td$ that is generated along the $\srs$. In this paper we distinguish
simulators that requires a trapdoor to simulate and those that do not. We call
the former \emph{SRS-simulators}. We say that a protocol is zero knowledge in
the standard model or  \emph{trapdoor-less simulatable} (TLS) if its simulator does not require the trapdoor.

% Occasionally, a weaker version of zero knowledge is sufficient. So called
% \emph{honest verifier zero knowledge} (HVZK) assumes that the verifier's
% challenges are picked at random from some predefined set. Although weaker, this
% definition suffices in many applications. Especially, an interactive
% zero-knowledge proof that is HVZK and \emph{public-coin} (i.e.~the verifier
% outputs as challenges its random coins) can be made non-interactive and
% zero-knowledge in the random oracle model by using the Fiat--Shamir
% transformation.

\subsection{From interactive to non-interactive---the Fiat--Shamir transform}
Consider a $(2\mu + 1)$-message, public-coin, honest verifier zero-knowledge
interactive proof system
$\proofsystem = (\kgen, \prover, \verifier, \simulator)$ for
$\REL \in \RELGEN(\secparam)$.  Let $\zkproof$ be a proof performed by the
prover $\prover$ and verifier $\verifier$ compound of messages
$(a_1, b_1, \ldots, a_{\mu}, b_{\mu}, a_{\mu + 1})$, where $a_i$ comes from
$\prover$ and $b_i$ comes from $\verifier$.  Denote by $\ro$ a random oracle.
Let $\proofsystem_\fs = (\kgen_\fs, \prover_\fs, \verifier_\fs, \simulator_\fs)$
be a proof system such that
\begin{compactitem}
  \item $\kgen_\fs$ behaves as $\kgen$.
  \item $\prover_\fs$ behaves as $\prover$ except after sending message
    $a_i$, $i \in \range{1}{\mu}$, the prover does not wait for
    the message from the verifier but computes it locally setting $b_i
    = \ro(\zkproof[0..i])$, where $\zkproof[0..j] = (\inp, a_1, b_1, \ldots,
    a_{j - 1}, b_{j - 1}, a_j)$. (Importantly, $\zkproof[0..\mu + 1] =
    (\inp, \zkproof)$).
  \item $\verifier_\fs$ behaves as $\verifier$ but does not provide
    challenges to the prover's proof. Instead it computes the
    challenges locally as $\prover_\fs$ does. Then it verifies the
    resulting transcript $\zkproof$ as the verifier $\verifier$ would. 
  \item $\simulator_\fs$ behaves as $\simulator$, except when
    $\simulator$ picks challenge $b_i$ before computing message $\zkproof[0, i]$, $\simulator_\fs$ programs the
    random oracle to output $b_i$ on $\zkproof[0, i]$.
  \end{compactitem}

\noindent
The Fiat--Shamir heuristic states that $\proofsystem_\fs$ is a zero-knowledge
non-interactive proof system for $\REL \in \RELGEN(\secparam)$.

%\subsection{Non-malleability definitions for NIZKs}
%\label{sec:simext_def}
%Real life applications often require a NIZK proof system to be
%non-malleable. That is, no adversary seeing a proof $\zkproof$ for a statement
%$\inp$ should be able to provide a new proof $\zkproof'$ related to $\zkproof$.
%\emph{Simulation extractability} formalizes a strong version of non-malleability
%by requiring that no adversary can produce a valid proof without knowing the
%corresponding witness. This must hold even if the adversary is allowed to see
%polynomially many simulated proofs for any statements it wishes.
%
%%\chaya{remove reference to forking soundness. quantify for $\ext_\se$}
%\begin{definition}[Forking simulation-extractable NIZK, \cite{INDOCRYPT:FKMV12}]
%	\label{def:simext}
%  Let $\ps_\fs = (\kgen_\fs, \prover_\fs, \verifier_\fs, \simulator_\fs)$ be a
%  HVZK proof system\hamid{$\ps_\fs$ is the Fiat-Shamir variant of the underlying proof system. So maybe we mean the underlying proof system is HVZK?}. We say that $\ps_\fs$ is \emph{forking
%    simulation-extractable} with \emph{extraction error} $\nu$ if for any $\ppt$
%  adversary $\adv$ that is given oracle access to a random oracle $\ro$ and
%  simulator $\simulator_\fs$, and produces an accepting transcript of $\ps$ with
%  probability $\accProb$, where
%	\[
%		\accProb = \Pr \left[
%		\begin{aligned}
%			& \verifier_\fs(\srs, \inp_{\advse}, \zkproof_{\advse}) = 1,\\
%			& (\inp_{\advse}, \zkproof_{\advse}) \not\in Q
%		\end{aligned}
%		\, \left| \,
%		\begin{aligned}
%			& \srs \gets \kgen_\fs(\REL), r \sample \RND{\advse}, \\
%			& (\inp_{\advse}, \zkproof_{\advse}) \gets \advse^{\simulator_\fs,
%			\ro} (\srs; r)
%		\end{aligned}
%		\right.\right]\,,
%	\]
%	there exists an extractor $\extse$ such that
%	\[
%		\extProb = \Pr \left[
%		\begin{aligned}
%			& \verifier_\fs(\srs, \inp_{\advse}, \zkproof_{\advse}) = 1,\\
%			& (\inp_{\advse}, \zkproof_{\advse}) \not\in Q,\\
%			& \REL(\inp_{\advse}, \wit_{\advse}) = 1
%		\end{aligned}
%		\, \left| \,
%		\begin{aligned}
%			& \srs \gets \kgen_\fs(\REL), r \sample \RND{\advse},\\
%			& (\inp_{\advse}, \zkproof_{\advse}) \gets \advse^{\simulator_\fs,
%			\ro} (\srs; r) \\
%			& \wit_{\advse} \gets \ext_\se (\srs, \advse, r, \inp_{\advse}, \zkproof_{\advse},
%			Q, Q_\ro) 
%		\end{aligned}
%		\right.\right]
%	\]
%	is at at least 
%	\[
%		\extProb \geq \frac{1}{\poly} (\accProb - \nu)^d - \eps(\secpar)\,,
%	\]
%	for some polynomial $\poly$, constant $d$ and negligible $\eps(\secpar)$ whenever
%  $\accProb \geq \nu$. List $Q$ contains all $(\inp, \zkproof)$ pairs where
%  $\inp$ is an instance provided to the simulator by the adversary and
%  $\zkproof$ is the simulator's answer. List $Q_\ro$ contains all $\advse$'s
%  queries to $\ro$ and $\ro$'s answers.
%\end{definition}


\subsection{Updatable SRS schemes}\label{def:upd-scheme}
We recall the definition of an updatable SRS scheme from~\cite{C:GKMMM18} which consists of the following algorithms.

\begin{itemize}
	\item
	$(\srs,\rho) \leftarrow \kgen(\secparam)$ is a PPT algorithm that takes a security parameter $\secpar$ and outputs a SRS $\srs$, and correctness proof $\rho$.
	\item
	$ (\srs',\rho') \leftarrow \upd(1^\secpar,\srs,\{\rho_i \}_{i=1}^n)$ is a PPT algorithm that takes as input the security parameter $\secpar$, a SRS $\srs$, a list of update proofs and outputs an updated SRS together with a proof of correct update. 
	\item
	$b \leftarrow \verifyCRS(1^\secpar,\srs,\{\rho_i \}_{i=1}^n)$ is a DPT algorithm that takes the security parameter $\secpar$, a SRS $\srs$, a list of update proofs, and outputs a bit indicating acceptance or not.
\end{itemize}

In the next section, we define security notions in the updatable setting. To this end, we define an SRS update oracle $\initU$ in~\cref{fig:upd} by which the adversary updates the SRS. We also define the simulation oracle $\simO$ in~\cref{fig:upd} that is the simulator w.r.t. the SRS finalised by the adversary using $\initU$. This simulation oracle will be used in the definition of forking simulation extractability.

\newcommand*{\Scale}[2][4]{\scalebox{#1}{$#2$}}% 

\begin{figure}[t]
	\centering
	\fbox{
		\begin{minipage}[t]{0.58\linewidth}
			\procedure{$\initU(\intent, \srs_n,\{\rho_i \}_{i=1}^n)$}{
				\pcif \srs \neq \bot: \pcreturn \bot \\
				\pcif (\intent = \setup): \\
				\t (\srs',\rho') \leftarrow \kgen(\REL)\\
				\t Q_\srs \gets Q_\srs \cup \{(\srs',\rho')\}\\
				\t \pcreturn (\srs',\rho')\\
				%
				\pcif (\intent = \update): \\
				\t b \gets \verifyCRS(1^\secpar,\srs_n,\{\rho_i \}_{i=1}^{n})\\
				\t \pcif (b=0): \pcreturn \bot \\
				\t (\srs',\rho') \leftarrow \upd(1^\secpar,\srs_n,\{\rho_i \}_{i=1}^n)\\
				\t Q_\srs \gets Q_\srs \cup \{(\srs',\rho')\}\\
		 	\Scale[0.75]{ // Q_\srs = (Q^{(1)}_\srs, Q^{(2)}_\srs) \text{ s.t. }   Q^{(2)}_\srs \text{ contains the update proofs in } Q_\srs }\\
				\t \pcreturn (\srs',\rho')\\
				%
				\pcif (\intent = \final): \\
				\t b \gets \verifyCRS(1^\secpar,\srs_n,\{\rho_i \}_{i=1}^{n})\\
				\t \pcif (b=0) \vee Q^{(2)}_\srs \cap \{ \rho_i \}_i = \emptyset: \pcreturn \bot \\
				\t 
				\td \gets \ext_\srs(\srs_n,, Q_\srs, \{\rho_i \}_{i=1}^{n}) \\
				\t \srs \gets \srs_n, \pcreturn \srs \\
				%
				\pcelse \pcreturn \bot
			}
		\end{minipage}
		\begin{minipage}[t]{0.3\linewidth}
			\procedure{$\simO(\inp')$}{
				\pcif (\srs = \bot): \pcreturn \bot \\
				% \hamid{should we formalize this extractor?}\\
				\zkproof'  \gets  \simulator_\FS(\srs, \td, \inp')\\
				Q = Q \cup \{(\inp', \zkproof')\}\\
				\pcreturn  \zkproof' \\	
			}
	\end{minipage}}
	\caption{The left oracle defines the notion of updatable SRS setup. The right oracle is the simulation oracle.} 
	\label{fig:upd}
\end{figure}


\hamid{TODO: add a paragraph here about the universality!}
% Consider a sigma protocol $\sigmaprot = (\prover, \verifier, \simulator)$ that
% is special-sound and has a unique response property. Let $\sigmaprot_\fs =
% (\prover_\fs, \verifier_\fs, \simulator_\fs)$ be a NIZK obtained by applying the
% Fiat--Shamir transform to $\sigmaprot$. Faust et al.~\cite{INDOCRYPT:FKMV12}
% show that every such $\sigmaprot_\fs$ is forking simulation-extractable. This result is
% presented in \cref{sec:forking_lemma} along with the instrumental forking lemma,
% cf.~\cite{CCS:BelNev06}.

\iffalse
\noindent \textbf{Simulation sound NIZKs.}
Another notion for non-malleable NIZKs is \emph{simulation soundness}. It allows the adversary to see simulated proof, however, in contrast to simulation
extractability it does not require an extractor to provide a witness for the
proven statement. Instead, it is only necessary, that an adversary who sees
simulated proofs cannot make the verifier accept a proof of an incorrect
statement. More precisely,
\chaya{this definition will go}
\begin{definition}[Simulation soundness]
  	\label{def:simsnd}
    Let $\ps = (\kgen, \prover, \verifier, \simulator)$ be a NIZK proof and
    $\ps_\fs = (\kgen_\fs, \prover_\fs, \verifier_\fs, \simulator_\fs)$ be $\ps$
    transformed by the Fiat--Shamir transform. We say that $\ps_\fs$ is
    \emph{simulation-sound}
    for any $\ppt$ adversary $\adv$ that is given oracle access to a random
    oracle $\ro$ and simulator $\simulator_\fs$, probability
    \[
      \ssndProb =
      \Pr\left[
        \begin{aligned}
          & \verifier_\fs(\srs, \inp_{\adv}, \zkproof_{\adv}) = 1,\\
          & (\inp_{\advse}, \zkproof_{\advse}) \not\in Q,\\
          & \neg \exists \wit_{\adv}: \REL(\inp_{\adv}, \wit_{\adv}) = 1
        \end{aligned}
        \, \left| \,
          \vphantom{\begin{aligned}
          & \verifier_\fs(\srs, \inp_{\adv}, \zkproof_{\adv}) = 1,\\
          & (\inp_{\advse}, \zkproof_{\advse}) \not\in Q,\\
          & \neg \exists \wit_{\adv}: \REL(\inp_{\adv}, \wit_{\adv}) = 1
        \end{aligned}}
      \begin{aligned}
        & \srs \gets \kgen(\REL), r \sample \RND{\advse},\\
        & (\inp_{\advse}, \zkproof_{\advse}) \gets \advse^{\simulator_\fs,
          \ro} (\srs; r)
      \end{aligned}
		\right.  \right]
    \]
    is at most negligible.  List $Q$ contains all $(\inp, \zkproof)$ pairs where
  $\inp$ is an instance provided to the simulator by the adversary and
  $\zkproof$ is the simulator's answer. 
\end{definition}

  \label{rem:simext_to_simsnd}
  We note that the probability $\ssndProb$ \cref{def:simsnd} can be expressed in
  terms of simulation-extractability. More precisely, the
  condition $\neg \exists \wit: \REL(\inp_\adv, \wit_\adv) = 1$ can be substituted with
  $\REL(\inp_\adv, \wit_\adv) = 0$, where $\wit_\adv$, returned by a possibly unbounded
  extractor, is either a witness to $\inp_\adv$ (if there exists any) or $\bot$ (if
  there is none). More precisely,
\[
      \ssndProb =
      \Pr\left[
        \begin{aligned}
          & \verifier_\fs(\srs, \inp_{\adv}, \zkproof_{\adv}) = 1,\\
          & (\inp_{\advse}, \zkproof_{\advse}) \not\in Q,\\
          & \REL(\inp_{\adv}, \wit_{\adv}) = 0
        \end{aligned}
        \, \left| \,
      \begin{aligned}
        & \srs \gets \kgen(\REL), r \sample \RND{\advse},\\
        & (\inp_{\advse}, \zkproof_{\advse}) \gets \advse^{\simulator_\fs,
          \ro} (\srs; r)\\
        & \wit_{\adv} \gets \ext(\srs, \advse, r, \inp_{\advse}, \zkproof_{\advse},
			Q, Q_\ro,) 
      \end{aligned}
		\right.  \right].
\]
The only necessary input to the unbounded extractor $\ext$ is the instance
$\inp_\adv$ (the rest is given for the consistency with the simulation extractability
definition). 
%
With the probabilities in \cref{def:simext} holding regardless of whether the extractor
is unbounded or not, we obtain the following equality
$ \ssndProb = \accProb - \extProb$.

% In \cref{cor:simext_to_ssnd} we show that (under some mild conditions) this is enough
% to conjecture that probability $\ssndProb$ is not only at most negligible, but
% also, in some parameters, exponentially smaller than $(1 - \extProb)$
% (probability of extraction failure in \cref{def:simext}).

\fi

%%% Local Variables:
%%% mode: latex
%%% TeX-master: "main"
%%% End:
