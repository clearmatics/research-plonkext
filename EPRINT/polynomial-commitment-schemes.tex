% !TEX root = main.tex
% !TEX spellcheck = en-US



\section{Polynomial Commitment Schemes}
\label{sec:pcom}
A polynomial commitment scheme $\PCOM = (\kgen, \com, \open, \verify)$ consists of four
algorithms and allows to commit to a polynomial $\p{f}$ and later open the evaluation in a
point $z$ to some value $s=\p{f}(z)$. More formally:
\begin{description}
\item[$\kgen(1^\secpar, \maxdeg)$:] The key generation algorithm takes in a security
  parameter $\secpar$ and a parameter $\maxdeg$ which determines the maximal degree of the
  committed polynomial. It outputs a structured reference string $\srs$ (the commitment
  key). Note that $\srs$ implicitly determines $\secpar$ and $\maxdeg$.
\item[$\com(\srs, \p{f})$:] The commitment algorithm $\com(\srs, \p{f})$ takes
  in $\srs$ and a polynomial $\p{f}$ with maximum degree $\maxdeg$, and outputs
  a commitment $c$.
\item[$\open(\srs, z, s, \p{f})$:] The opening algorithm
  takes as input $\srs$, an evaluation point $z$, a
  value $s$ and the polynomial $\p{f}$. It outputs an opening $o$.
\item[$\verify(\srs, c, z, s, o)$:] The verification algorithm takes in $\srs$,
  a commitment $c$, an evaluation point $z$, a value $s$ and an opening $o$. It
  outputs 1 if $o$ is a valid opening for $(c, z, s)$ and 0 otherwise.
\end{description} 

A secure polynomial commitment $\PCOM$ should satisfy correctness, evaluation binding,
opening uniqueness, hiding and knowledge-binding.  Note that since we are in the updatable
setting, $\srs$ in these security definitions is the SRS that $\advse$ finalizes using the
update oracle $\initU$ (See~\cref{fig:upd}).

\begin{description}
\item[Evaluation binding:] A $\ppt$ adversary $\adv$ which outputs a commitment
  $\vec{c}$ and evaluation points $\vec{z}$ has at most negligible chances to open
  the commitment to two different evaluations $\vec{s}, \vec{s'}$. That is, let
  $k \in \NN$ be the number of committed polynomials, $l \in \NN$ number of
  evaluation points, $\vec{c} \in \GRP^k$ be the commitments, $\vec{z} \in
  \FF_p^l$ be the arguments the polynomials are evaluated at, $\vec{s},\vec{s}'
  \in \FF_p^k$ the evaluations, and $\vec{o},\vec{o}' \in \FF_p^l$ be the
  commitment openings. Then for every $\ppt$ adversary $\adv$
	\[
		\condprob{
			\begin{matrix}
				  \verify(\srs, \vec{c}, \vec{z}, \vec{s}, \vec{o}) = 1,  \\ 
				  \verify(\srs, \vec{c}, \vec{z}, \vec{s}', \vec{o}') = 1, \\
				  \vec{s} \neq \vec{s}'
			\end{matrix}}
			{
			\begin{matrix}
%				& \srs \gets \kgen(\secparam, \maxdeg),\\
				 (\vec{c}, \vec{z}, \vec{s}, \vec{s}', \vec{o}, \vec{o}') \gets \adv^{\initU}(\maxdeg)
			\end{matrix}
		} \leq \negl\,.
	\]

\end{description}
	
\begin{description}
\item[Hiding:] We also formalize notion of $k$-hiding property of a polynomial commitment scheme. Let $\HHH$ be a set of size $\maxdeg + 1$ and $\ZERO_\HHH$ its
  vanishing polynomial. We say that a polynomial scheme is \emph{hiding} with
  security $\epsh(\secpar)$ if for every $\ppt$ adversary $\adv$, $k \in \NN$,
  probability
  \begin{align*}
    \condprob
   { b' = b}{
    (f_0, f_1, c, k, b') \gets \adv^{\initU, \oraclec}(\maxdeg, c), f_0, f_1 \in \FF^{\maxdeg}
    [X]}
\leq \frac{1}{2} + \eps(\secpar).
  \end{align*}
  Where $c = f'_b (\chi)$, for a random bit $b$ and the polynomial
      $f'_b (X) = f_b + \ZERO_\HHH (X) (a_0 + a_1 X + \ldots a_{k - 1} X^{k -
        1})$,
and the oracle $\oraclec$ on adversary's evaluation query $x$ it adds $x$ to initially empty set
      $Q_x$ and if $|Q_x| \leq k$, it provides $f'_b (x)$.
 
  \end{description}

\begin{description}
\item[Commitment of knowledge:] Intuitively, when a commitment scheme is ``of knowledge'' then if an
adversary produces a (valid) commitment $c$, which it can open correctly in an evaluation point, then it must
know the underlying polynomial $\p{f}$ which commits to that value.  For every $\ppt$ adversary $\adv$ who produces
  commitment $c$, evaluation $s$ and opening $o$ there
  exists a $\ppt$ extractor $\ext$ such that
\[
  \condprob{
    \begin{matrix}
       \deg \p{f} \leq \maxdeg,
       c = \com(\srs, \p{f}),\\
       \verify(\srs, c, z, s, o) = 1
    \end{matrix}
        }{
    \begin{matrix}
      %
     % & \srs \gets \kgen(\secparam, \maxdeg),\\
      c \gets \adv^{\initU}(\maxdeg),
      z \sample \FF_p \\
      (s, o) \gets \adv(c, z), \\
   \p{f} = \ext_\adv(\srs, c)\\
    \end{matrix}}
  \geq 1 - \epsk(\secpar).
\]
In that case we say that $\PCOM$ is $\epsk(\secpar)$-knowledge.
\end{description}

%%% Local Variables:
%%% mode: latex
%%% TeX-master: "main"
%%% End:
