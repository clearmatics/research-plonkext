\documentclass[a4paper, 11pt]{article}
\usepackage[utf8]{inputenc}
%\usepackage[T2A]{fontenc}
%\usepackage[utf8]{inputenc}
%\usepackage[russian]{babel}


\usepackage{appendix}
%% Sets page size and margins
% \usepackage[a4paper,top=2cm,bottom=2.5cm,left=1.8cm,right=1.8cm,marginparwidth=1.75cm]{geometry}
\usepackage{amsthm}
\usepackage{amsmath,amssymb,amsfonts}
\usepackage{amsfonts}
\usepackage{bbm}
\usepackage{cancel}
\usepackage{framed}
\usepackage[backend=bibtex,style=alphabetic,maxnames=99,minalphanames=3,maxalphanames=4]{biblatex}
\usepackage{hyperref}
\usepackage{datetime}
\usepackage{stmaryrd}
\usepackage{algorithmic}
\usepackage[ruled,vlined,linesnumbered]{algorithm2e}
\usepackage{graphicx}
\usepackage{textcomp}
\usepackage{url}
\usepackage{msc,multicol}
\usepackage{mathtools}
\usepackage{wrapfig}
\usepackage{tabularx}
\usepackage{bm}
\usepackage{cleveref}
\usepackage{listings}


\definecolor{grey04}{rgb}{0.4, 0.4, 0.4}
\lstset{
    columns=fullflexible,
    showspaces=false,
    showtabs=false,
    breaklines=true,
    showstringspaces=false,
    breakatwhitespace=true,
    escapeinside={(*@}{@*)},
    commentstyle=\color{grey04},
    keywordstyle=\color{grey04},
    stringstyle=\color{grey04},
    numberstyle=\color{grey04},
    basicstyle=\ttfamily\footnotesize,
    frame=l,
    framesep=12pt,
    xleftmargin=12pt,
    tabsize=4,
    captionpos=b
}

\addbibresource{weakse.bib}

\setlength{\abovecaptionskip}{0pt}

% Editing and writing
\newcommand{\TODO}[1]{\textcolor{orange}{TODO: #1}}
\newcommand{\FIXME}[1]{\textcolor{red}{FIXME: #1}}
% \newcommand{\MK}[1]{\textcolor{purple}{MK: #1}}
\newcommand{\MK}[1]{}
\newcommand{\MV}[1]{\textcolor{olive}{MV: #1}}
\newcommand{\Gro}{\textsf{Groth16}}

% Maths
\newtheorem{theorem}{Theorem}[section]
\newtheorem{definition}{Definition}[section]
\newtheorem{corollary}{Corollary}[theorem]
\newtheorem{lemma}[theorem]{Lemma}

\newcommand{\qdlog}[1]{$#1$\textbf{-dlog}}
\newcommand{\getsrand}{\xleftarrow{{\scriptscriptstyle \$}}}
\newcommand{\cf}[2]{#1_{[#2]}}
\newcommand{\xagm}{\mathcal{X}^{\textsf{alg}}}
\newcommand{\advers}{\mathcal{A}}
\newcommand{\Equation}[1]{Equation~(#1)}

\makeatletter
\newcommand*\bigcdot{\mathpalette\bigcdot@{.7}}
\newcommand*\bigcdot@[2]{\mathbin{\vcenter{\hbox{\scalebox{#2}{$\m@th#1\bullet$}}}}}
\makeatother
\renewcommand{\vec}[1]{\bm{#1}}

\DeclareMathOperator{\negl}{negl}
\DeclareMathOperator{\rank}{rank}
\DeclareMathOperator{\G}{\mathbb{G}}
\DeclareMathOperator{\Z}{\mathbb{Z}}
\DeclareMathOperator{\spn}{Span}

\newcommand{\email}[1]{\texttt{\href{mailto:#1}{#1}}}

%\title{A Note on Randomizable Simulation-Extractability in zk-SNARKs}
\title{Groth16 SNARKs are \\ Randomizable and (Weakly) Simulation Extractable}

%\author[1]{Mikhail Volkhov}
%\author[2]{Markulf Kohlweiss}
%\affil[1,2]{The University of Edinburgh \\ Some other text}
%\email{mikhail.volkhov@ed.ac.uk}
%\hspace{1em}
%\email{mkohlwei@ed.ac.uk}

\makeatletter
\renewcommand\@date{{%
  \vspace{-\baselineskip}%
  \large\centering
  \begin{tabular}{@{}c@{}}
    Markulf Kohlweiss$^{1,2}$
  \end{tabular}
  \quad and\quad
  \begin{tabular}{@{}c@{}}
    Mikhail Volkhov$^1$
  \end{tabular}%
  \\
  \bigskip
  \normalsize
  $^1$The University of Edinburgh and $^2$IOHK \\
   \email{mkohlwei@ed.ac.uk} \hspace{1em} \email{mikhail.volkhov@ed.ac.uk}
}}
\makeatother

% \date{ \currenttime \ \today }
%\date{\today}
%\setcounter{Maxaffil}{0}
%\renewcommand\Affilfont{\itshape\small}


\begin{document}

\maketitle

\begin{abstract}
  Due to its simplicity, succinctness, and performance, \Gro{} is
  currently the most widely deployed succinct (zero-knowledge) argument of
  knowledge (SNARK) system. \Gro{} is
  known to be perfectly zero-knowledge and knowledge sound in the generic (and
  algebraic) group model. However, the existing security arguments for \Gro{} are
  silent about the soundness of the proof system in the presence of simulated
  proofs---a common requirement for both the composable security and game-hopping
  style security analysis of
  protocols built using such argument systems. This important gap let to a line of work on
  simulation-extractable, also called simulation knowledge sound, succinct proof systems.
  \Gro{} itself cannot satisfy the strongest
  notion of simulation-extractability that implies proof
  non-malleability---in fact proofs are perfectly randomizable.

  Surprisingly, in this short note we show that \Gro{} does satisfy a
  weaker notion of simulation-extractability implying statement
  non-malleability. This property is often sufficient for typical
  applications that motivate the use of strong
  simulation-extractability. Notably, one can obtain UC security using
  efficient compilers.
\end{abstract}

\section{Introduction}

% 1. SNARKS and Groth16 important
% - zcash, etherium, most efficient

Succinct non-interactive arguments of knowledge (SNARK) have
revolutionized the deployment of zero-knowledge proofs, particularly
in the blockchain and cryptographic currency
space~\cite{DBLP:conf/sp/Ben-SassonCG0MTV14, kosba2016hawk,
  DBLP:journals/iacr/KerberKK20a, DBLP:journals/iacr/BoweCGMMW18}. The
ready availability of cryptographic libraries implementing SNARKs has
also inspired other
applications~\cite{DBLP:conf/sp/NavehT16,DBLP:conf/sp/Delignat-Lavaud16}. See
also the application chapter of~\cite{zkproof}.

Due to its exceptional performance and simplicity the most widely
deployed SNARK proof system is perhaps \Gro{}~\cite{gro16}. Naturally,
this is susceptible to change, especially if its security is
undermined by quantum attacks. That withstanding, however, due to its
near optimal proof size and verification performance, \Gro{} is likely
to be a mainstay of cryptographic deployments, maybe comparable to
ElGamal encryption~\cite{DBLP:journals/tit/Elgamal85} and
Schnorr~\cite{DBLP:journals/joc/Schnorr91} or
DSA~\cite{dsa} signature schemes.
%
In this short note we identity and close what we believe is a small
but important gap in the security analysis of \Gro{}, namely its
malleability and the limits of said malleability.

%
% 2. simulation-extractability important
%  - 1. many proofs need it
%  - 2. needed for UC security
%

Arguably, the strongest non-malleability property for SNARK systems is
\emph{simulation-extractability} (SE) \cite{sahai1999non,
  de2001robust}, a security notion that extends knowledge-soundness
(KS) by allowing the adversary to also access the simulaton
oracle. One of the important properties of this notion is that its
straight-line, blackbox variant is necessary to achieve Universally Composable~(UC~\cite{uc}) security,
as shown more generally in
\cite{canetti2002universally,groth2006perfect,groth2006simulation} for
non-interactive zero-knowledge (NIZK) proof systems. Moreover, it is
also needed in game-hopping style proofs~\cite{DBLP:journals/iacr/Shoup04} in which one game hop
introduces the simulator and a subsequent game hop relies on knowledge
soundness~\cite{kosba2016hawk,
  DBLP:conf/ccs/CamenischDD17}.

% 3. weak simulation-extractability enough
%  - hawk, kachina
%  - analogy to CMA signatures


In this work we focus on the weaker notion of
simulation-extractability, that allows for the limited malleability of
proofs, which we call weak simulation-extractability (weak-SE). Note
that the notion of weak simulation extractability from
\cite{DBLP:conf/indocrypt/FaustKMV12} is unrelated. The ``weakening''
there refers to a non-blackbox and non-straight line flavor of
simulation extraction in the Random Oracle model.  Rather, weak-SE and
SE of \emph{proof systems} are in analogy to chosen message attack
(CMA) and strong CMA unforgeability of \emph{signatures}. Indeed, in weak-SE it is
the statement rather than the proof that cannot be mauled, a weak-SE based
SNARKY signature scheme for a hard key-pair relation, give rise to CMA, rather
than strong CMA secure, signature scheme.\MK{We could define and prove this.}

% 4. out contribution
  \paragraph{Our contribution.}

  In this short note we show that \Gro{} is both weakly-simulation
  extractable and perfectly and efficiently randomizable. As in \Gro{}
  the randomization of proofs obtained from the simulator are
  distributed just like freshly generated proofs, this is seemingly
  the strongest extractability property that one can hope for. In the
  algebraic group model, however, we can show something even stronger,
  namely the extractor can either obtain a witness or point to the
  unique simulated proof that was randomized to obtain the proof
  produced by the adversary. Consequently, even if the adversary
  queries multiple proofs for the same statement, they cannot be
  combined to form a new proof of the same statement. Therefore,
  weak-SE \Gro{} can be directly lifted to (weak) blackbox SE, which
  is required by UC, using the technique explained
  in~\cite{baghery2019efficiency}, improving the performance of the
  resulting SNARK compared to Groth and Maller SNARK used in
  \cite{baghery2019efficiency}.


% 5. related work
%   - RO, CRS distraction, focus on Groth16
%   - Table with SE sizes. Behzad, Karim's thesis.
%   - transformation to SE


\paragraph{Related Work.}
 Simulation-extractability applies both to CRS-based and random-oracle (RO)
 based NIZKs. NIZKs obtained from
 $\Sigma$-protocols using Fiat-Shamir heuristic in the random oracle
 (RO) model, are showed to always satisfy
 simulation-extractability~\cite{DBLP:conf/indocrypt/FaustKMV12}.
 \MK{I suspect both Mary and I are working on this already, no need to make it
   even more apparent that this is interesting territory :): Although
 this result does not immediately extend to other FS-transformed
 NIZKs,}
 In this work we focus on simulation-extractability of
 CRS-based NIZKs, and on SNARKs in particular.

 SE SNARKs have been discovered only recently. Groth and
 Maller~\cite{gm17} presented the first construction in 2017,
 targeting SAP, together with a lower bound of three group elements for
 the proof size, and two equations for verification, for all
 \emph{non-interactive linear proofs} (NILP) based SNARKs, which covers many
 previously known pairing-based SNARKs, including~\cite{gro16,gm17}. Bowe and Gabizon~\cite{bowe2018making} provide a
 RO-based variant of \Gro{} for QAP that is simulation-extractable,
 and has five group elements and two verification
 equations. Lipmaa~\cite{lip19} presents a different technique that
 allows to lift known SNARKs for QAP and the three other arithmetisation
 techniques from the QAP family (namely, SAP, SSP, and QSP), together
 with a simpler security proof.\MK{Doesn't he do QAP as well?} Kim, Lee, and Oh~\cite{kim2019qap}
 present a SNARK for QAP with three elements but just a single
 verification equation, avoiding the lower bound of Groth and Maller
 by using a random oracle in addition to a knowledge extraction
 assumptions and a CRS.\smallskip

 \noindent\emph{General transformations and UC.} A generic
 transformation that lifts ordinary NIZKs to be simulation extractable
 has been known since \cite{de2001robust} at least. Along this
 direction, \cite{kosba2015use, kosba2015c} extend, analyse, and
 optimise this transformation technique, while Atapoor and
 Baghery~\cite{atapoor2019simulation} apply it directly to \Gro{} and
 evaluate the efficiency of the resulting strong SE argument. The
 transformation from non-blackbox SE to blackbox SE is analysed by
 Baghery~\cite{baghery2019efficiency}, with particular focus on
 (strong-SE) SNARK by Groth and Maller, although this technique should
 also work for lifting non-blackbox weak-SE to blackbox weak-SE. Other
 generic transformations take into account CRS subversion and
 updatability~\cite{abdolmaleki2020lift, bagherytiramisu}.

Regarding UC functionalities for NIZKs, it has been shown that a (non-malleable)
$\mathcal{F}_{\mathsf{NIZK}}$ functionality can be realised using (strong)
blackbox-extractable SE NIZKs~\cite{canetti2002universally, groth2006simulation}
assuming static corruption.
%
Kosba et al.~\cite{kosba2015use,kosba2015c} suggest their own variant of
$\mathcal{F}_{\mathsf{WEAK-NIZK}}$ without proving that a weak-SE NIZK
can realise it.\smallskip

\noindent\emph{Weaker simulation extraction notions.} Although
(strong) SE is sometimes a desirable property, weak-SE can be
sufficient for UC applications, for instance in
Hawk~\cite{kosba2016hawk}, as argued in \cite{kosba2015use,kosba2015c}.

Hawk uses SE NIZKs directly as a raw primitive
(without employing a functionality), and it suggests to use
non-succinct strong SE NIZK, since no other candidates were known at that
time. Kosba et al.~\cite{kosba2015use,kosba2015c} point out that
weak-SE NIZK can be used instead, without providing a formal proof.
%
% Other works use either strong definition or another
%variants of it: NIZKs of \cite{groth2006simulation, gm17,
%  bowe2018making, kim2019qap} are strongly simulation-extractable, and
%
Lipmaa's~\cite{lip19} presents an SE notion that is tag-based, although the
construction presented prevents standard randomization.

% \printbibliography
\end{document}