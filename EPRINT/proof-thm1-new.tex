%!TEX root=main.tex

\begin{theorem}[Simulation-extractable multi-message protocols]
	\label{thm:se}
	Let $\ps = (\kgen, \prover, \verifier, \simulator)$ be an interactive
	$(2 \mu + 1)$-message HVZK proof system for $\RELGEN(\secparam)$
	with knowledge soundness error $\epsk(\secpar)$. Let
	$\ro\colon \bin^{*} \to \bin^{\secpar}$ be a random oracle. If $\psfs$ has the
	$k$-\emph{unique response property} with security loss $\epsur(\secpar)$, is
	\emph{$k'$-programmable trapdoor-less simulatable}, and \chb{$(1, \ldots, 1, n_{k''}, \ldots, n_\mu)$-\emph{computational
		special sound}} with security loss $\epss(\secpar)$, where $k, k' \leq k''$, and all these properties
	hold in the updatable setting, then $\psfs$ is
	\emph{simulation-extractable in the updatable setting} against $\ppt$ adversaries that make up to $q$ random oracle
	queries and returns an acceptable proof with probability at least $\accProb$.  The
	extraction probability $\extProb$ is at least
	$\extProb \geq \frac{\accProb - \epsur(\secpar) - (q + 1) \epsk(\secpar)}{1 - \epsk(\secpar)} 
	- \epsur(\secpar) - \epss(\secpar)\,$.
\end{theorem}

%\hamid{7.3}{I think somewhere in the proof, we still need to justify and explain explicitly why it's important to have concrete parameters $k, k', k''$ with relation $k \geq k' \geq k''$!}

\begin{proof}		
	
	\ngame{0} This is the simulation-extractability game played between an adversary
	$\advse$ who is given access to an oracle $\initU$ that defines an updatable SRS
	setup, and a simulation oracle $\simulator = (\simulator_1, \simulator_2')$ which provides the adversary with simulated
	proofs and random oracle functionality. We denote by $\rsim$
	randomness of $\simulator$ which is randomly sampled. Execution of $\advse$ results in
	lists of
	\begin{inparaenum}[(1)]
		\item simulated proofs $Q$,
		\item random oracle queries $Q_\ro$,
		\item SRS updates $Q_\srs$.
	\end{inparaenum}
	%
	Extractor $\extse$ takes as input a proof $\zkproof_{\advse}$ for instance
	$\inp_{\advse}$ output by the adversary and lists $Q, Q_\ro, Q_\srs$ and is tasked
	to extract a witness $\wit_{\advse}$ such that $\REL(\inp_{\advse}, \wit_{\advse})$
	holds. $\advse$ wins if it manages to produce an acceptable proof and the extractor
	fails to output a witness. In the following game hops we upper-bound the
	probability that this happens. Note that $\srs$ is with respect to the finalised
	SRS with respect to which $\adv$ is allowed to make simulation queries.
	
	\ngame{1} This is identical to $\game{0}$ except that now the game is aborted if
	there is $(\inp_\advse, \zkproof_{\simulator}) \in Q$ such that
	$\zkproof_\simulator[1..k] = \zkproof_{\advse}[1..k]$. That is, the adversary in
	its final proof reuses at least $k$ messages from a simulated proof, and the proof
	is accepting.  Denote this event by $\event{\errur}$.
	
	\ncase{Game 0 to Game 1} We have,
	\( \prob{\game{0} \land \nevent{\errur}} = \prob{\game{1} \land \nevent{\errur}} \)
	and, from the difference lemma, cf.~\cref{lem:difference_lemma},
	$ \abs{\prob{\game{0}} - \prob{\game{1}}} \leq \prob{\event{\errur}}\,$.  Thus, to
	show that the transition from one game to another introduces only minor change in
	probability of $\adv$ winning it should be shown that $\prob{\event{\errur}}$ is
	small.
	
	We can assume that $\adv$ queried the simulator on the instance it wishes to
	output, i.e.~$\inp_{\adv}$. We show a reduction $\rdvur$ that utilises $\adv$ to
	break the $\ur{k}$ property of $\ps$. That is, for randomness $r = (r_\adv, \rsim)$,
	$\rdvur (r)$ runs $\advse^{\initU, \simulator_1, \simulator_2'}(\secparam; r_\adv)$ internally as a black-box:
	\begin{compactitem}
		\item The reduction answers $\adv$'s update queries by asking the same query from the
		update oracle in the unique response experiment. The reduction finalises the same
		SRS $\srs$ as the one $\adv$ does.
		\item The reduction answers both queries to the simulator $\simulator$
		and to the random oracle using $\rsim$. It also keeps lists $Q$, for the simulated proofs,
		and $Q_\ro$ for the random oracle queries.
		\item When $\adv$ makes a fake proof $\zkproof_{\adv}$ for $\inp_{\adv}$, $\rdvur$
		looks through lists $Q$ and $Q_\ro$ until it finds $\zkproof_{\simulator}[0..k]$
		such that $\zkproof_{\adv}[0..k] = \zkproof_{\simulator}[0..k]$ and a random
		oracle query $\zkproof_{\simulator}[k].\ch$ on $\zkproof_{\simulator}[0..k]$.
		\item $\rdvur$ returns two proofs for $\inp_{\adv}$:
		\begin{align*}
		\zkproof_1 = (\zkproof_{\simulator}[1..k],
		\zkproof_{\simulator}[k].\ch, \zkproof_{\simulator}[k + 1..\mu + 1])\\
		\zkproof_2 = (\zkproof_{\simulator}[1..k],
		\zkproof_{\simulator}[k].\ch, \zkproof_{\adv}[k + 1..\mu + 1])
		\end{align*}
	\end{compactitem}
	If $\zkproof_1 = \zkproof_2$, then $\adv$ fails to break simulation soundness, as
	$\zkproof_2 \in Q$. On the other hand, if the proofs are not equal, then $\rdvur$
	breaks $\ur{k}$-ness of $\ps$. This happens only with negligible probability
	$\epsur(\secpar)$, hence \( \prob{\event{\errur}} \leq \epsur(\secpar)\,. \)
	
	\COMMENT{We have,
		\( \prob{\game{0} \land \nevent{\errur}} = \prob{\game{1} \land
			\nevent{\errur}} \) and, from the difference lemma,
		cf.~\cref{lem:difference_lemma},
		\[ \abs{\prob{\game{0}} - \prob{\game{1}}} \leq \prob{\event{\errur}}\,. \]
		Thus, to show that the transition from one game to another introduces only
		minor change in probability of $\advse$ winning it should be shown that
		$\prob{\event{\errur}}$ is small.
		
		We can assume that $\advse$ queried the simulator on the instance it wishes to
		output---$\inp_{\advse}$. We show a reduction $\rdvur$ that utilises $\advse$,
		who outputs a valid proof for $\inp_{\advse}$, to break the $\ur{k}$ property of
		$\ps$. Let $\rdvur$ run $\advse$ internally as a black-box:
		\begin{itemize}
			\item The reduction answers both queries to the simulator $\psfs.\simulator$ and to the random oracle. 
			It also keeps lists $Q$, for the simulated proofs, and $Q_\ro$ for the random oracle queries. 
			\item When $\advse$ makes a fake proof $\zkproof_{\advse}$ for $\inp_{\advse}$,
			$\rdvur$ looks through lists $Q$ and $Q_\ro$ until it finds
			$\zkproof_{\simulator}[0..k]$ such that
			$\zkproof_{\advse}[0..k] = \zkproof_{\simulator}[0..k]$
			and a random oracle query $\zkproof_{\simulator}[k].\ch$ on
			$\zkproof_{\simulator}[0..k]$.
			\item $\rdvur$ returns two proofs for $\inp_{\advse}$:
			\begin{align*}
			\zkproof_1 = (\zkproof_{\simulator}[1..k],
			\zkproof_{\simulator}[k].\ch, \zkproof_{\simulator}[k + 1..\mu + 1])\\
			\zkproof_2 = (\zkproof_{\simulator}[1..k],
			\zkproof_{\simulator}[k].\ch, \zkproof_{\advse}[k + 1..\mu + 1])
			\end{align*}
		\end{itemize}  
		If $\zkproof_1 = \zkproof_2$, then $\advse$ fails to break simulation
		extractability, as $\zkproof_2 \in Q$. On the other hand, if the proofs are
		not equal, then $\rdvur$ breaks $\ur{k}$-ness of $\ps$. This happens only with
		negligible probability $\epsur(\secpar)$, hence \( \prob{\event{\errur}} \leq
		\epsur(\secpar)\,. \)
	}
	%
	\ngame{2} In this game the environment additionally aborts if $\extse$ fails to build a
	$(1, \ldots, 1, n_{k''}, \ldots, n_\mu)$-tree of accepting transcripts $\tree$ by rewinding
	$\adv$.
	
	\ncase{Game 1 to Game 2} \chb{Note that for every accepting proof $\zkproof_{\advse}$, since $k \leq k''$, we
	may assume that whenever $\advse$ outputs a $k''$-th move message
	$\zkproof_{\advse}[k'']$, then the $(\inp_{\advse}, \zkproof_{\advse}[1..k''])$
	random oracle query was made by the adversary, not the
	simulator\footnote{\cite{INDOCRYPT:FKMV12} calls these queries \emph{fresh}.},
	i.e.~there is no simulated proof $\zkproof_\simulator$ on $\inp_\simulator$ such that
	$(\inp_{\advse}, \zkproof_{\advse} [1..k'']) = (\inp_\simulator,
	\zkproof_\simulator[1..k''])$. Otherwise, the game would be already interrupted by
	the error event in Game $\game{1}$. As previously, we have $\abs{\prob{\game{1}} - \prob{\game{2}}} \leq \prob{\event{\errfrk}}$, where $\event{\errfrk}$ denotes the event of $\extse$ failing to build a
	$(1, \ldots, 1, n_{k''}, \ldots, n_\mu)$-tree of accepting transcripts.}
%	\fbox{\begin{minipage}{45em}
%	 \hamid{28.2}{Is this needed?:}\michals{3.3.22}{I don't think so, we needed that to make a reduction to special soundness, but since we don't do that anymore (at least here) we don't need this part.} We build an adversary $\bdv$ against forking special soundness such that,
%	given access to oracles $\initU$ and $\ro$, and randomness $r = (r_\adv, \rsim)$,
%	internally runs $\advse^{\initU, \simulator_1, \simulator_2'} (1^\secpar; r_\adv)$ as follows:
%	\begin{compactenum}
%		\item $\bdv$ answers $\adv$ random oracle and update queries by passing the queries to the real
%		oracles $\ro$ and $\initU$. When $\adv$ finalises an SRS $\srs$, $\bdv$ does the same.
%		\item $\bdv$ answers simulator queries by using coins $\rsim$. $\bdv$ maintains a
%		list of instance-proof pairs $Q$ consisting of all simulation queries made by
%		$\adv$, and corresponding responses.
%		\item Eventually when $\adv$ outputs $(\inp_\advse, \zkproof_\advse)$, $\bdv$ outputs
%		the same $(\inp_\advse, \zkproof_\advse)$.
%	\end{compactenum}	
%	Denote by $\waccProb$ the probability that $\advse$ outputs a proof that is accepted
%	and does not break $\ur{k}$-ness of $\ps$. With the same probability, an accepting
%	proof is returned by $\bdv$. This comes since the proof system is
%	$k'$-programmable. That is, $\bdv$ when providing $\adv$ with simulated proofs
%	programs the random oracle only in moves from $k'$ on or, to put that differently,
%	it does not program the random oracle for moves $1$ to $k' - 1$.  Hence, $\adv$
%	outputs proof that does not contain any programmed random oracle output and $\bdv$
%	can output that proof as its own. This holds since (1) for messages $1$ to $k' - 1$,
%	the simulator does not program the oracle, (2) $\adv$ cannot use a part of a
%	simulated proof from move $k'$ because of the unique response property.
%	
%	Denote by $\waccProb'$ the probability that algorithm $\zdv$, defined in the general
%	forking lemma, produces an accepting proof with a fresh challenge after move
%	$k''$. From the above argument, we have that $\waccProb = \waccProb'$.
%\end{minipage}}
	Denote by $\waccProb$ the probability that $\advse$ outputs a proof that is accepted
	and does not break $\ur{k}$-ness of $\ps$. By~\cref{lemma:Attema}, we have 
		\begin{equation}
		\abs{\prob{\game{1}} - \prob{\game{2}}} \leq \prob{\event{\errfrk}} \leq 1 - \frac{\waccProb - (q + 1) \epsk(\secpar)} {1 -
			\epsk(\secpar)}.
		\end{equation}
	
	\ngame{3} This game is identical to $\game{2}$ except that the extractor $\extse$,
	given tree of acceptable transcripts $\tree$ runs additionally $\extt (\tree)$ to
	learn the witness $\wit$. The environment additionally aborts this game if
	$\REL(\inp, \wit)$ does not hold.
	
	\ncase{Game 2 to Game 3}
    %\michals{8.3}{For the current write-up we would need sth like a updatable computational special soundness however we have only computational special soundness defined (what is goof imho). Could we make a reduction to the ``standard'' computational special soundness as we defined above? }\hamid{9.3}{You're right. Corrected! Please check!}
%	By special soundness of $\ps$, we have
%	$\abs{\prob{\game{2}} - \prob{\game{3}}} \leq \epss(\secpar)$.\hamid{7.3}{If we need a proof for this, we can go as follows:\\
	To upper-bound $\abs{\prob{\game{2}} - \prob{\game{3}}}$, we build an adversary $\bdv$ against special soundness such that,
	given access to oracles $\initU$ and $\ro$, and randomness $r = (r_\adv, \rsim)$,
	internally runs $\advse^{\initU, \simulator_1, \simulator_2'} (1^\secpar; r_\adv)$ as follows:
	%and eventually when $\advse$ outputs $(\inp_\advse, \zkproof_\advse)$, $\bdv$ runs the tree building algorithm $\tdv$ and outputs the resulting $(k,n)$-tree of accepting transcripts.
	\begin{compactenum}
		\item $\bdv$ answers $\adv$ random oracle and update queries by passing the queries to the real
		oracles $\ro$ and $\initU$. When $\adv$ finalises an SRS $\srs$, $\bdv$ does the same.
		\item $\bdv$ answers simulator queries by using coins $\rsim$. $\bdv$ maintains a
		list of instance-proof pairs $Q$ consisting of all simulation queries made by
		$\adv$, and corresponding responses.
		\item Eventually when $\adv$ outputs $(\inp_\advse, \zkproof_\advse)$, $\bdv$ runs the tree building algorithm $\tdv$ and outputs the resulting $(k,n)$-tree of accepting transcripts.
	\end{compactenum}	
	If $\extse$ outputs a valid witness, then $\adv$ fails to break simulation extractability. Otherwise, in the case that $\extse$ fails, $\bdv$ breaks special soundness  of $\ps$. By assumption, this happens only with probability $\epss(\secpar)$, hence $\abs{\prob{\game{2}} - \prob{\game{3}}} \leq \epss(\secpar)$.
	
	\ncase{Conclusion} Since Game $\game{3}$ is aborted when it is impossible to
	extract a witness from $\tree$,
	the adversary $\advse$ cannot win. Thus, by the game-hopping argument,
	\[
	\abs{\prob{\game{0}} - \prob{\game{3}}} \leq 1 - \frac{\waccProb - (q
			+ 1) \epsk(\secpar)}{1 - \epsk(\secpar)} + \epsur(\secpar) +
	%q_{\ro}^{\mu} \epss +
	\epss(\secpar)\,.
	\]
	Thus the probability that extractor $\extse$ succeeds is at least
	\[
	\frac{\waccProb - (q + 1) \epsk(\secpar)}{1 - \epsk(\secpar)} -
	\epsur(\secpar)
	%- q_{\ro}^{\mu} \epss
	- \epss(\secpar)\,.
	\]
	Since $\waccProb$ is the probability of $\advse$ producing an accepting transcript
	that does not break $\ur{k}$-ness of $\ps$, then $\waccProb \geq \accProb -
	\epsur(\secpar)$, where $\accProb$ is the probability of $\advse$ outputting an accepting
	proof as defined in \cref{def:updsimext}. Thus, 
	\begin{equation}
	\label{eq:frk}
	\extProb \geq \frac{\accProb - \epsur(\secpar) - (q + 1) \epsk(\secpar)}{1 - \epsk(\secpar)} 
	- \epsur(\secpar) - \epss(\secpar)\,.
	\end{equation} 
	\qed
\end{proof}

%%% Local Variables:
%%% mode: latex
%%% TeX-master: "main"
%%% End:

