% !TEX root = main.tex
% !TEX spellcheck = en-US
\section{Preliminaries}
\label{sec:preliminaries}
\paragraph{Notation.} Let $\ppt$ denote probabilistic polynomial-time and $\secpar \in \NN$ be the
security parameter. All adversaries are stateful. For an algorithm $\adv$, let
$\image (\adv)$ be the image of $\adv$ (the set of valid outputs of $\adv$), let
$\RND{\adv}$ denote the set of random tapes of correct length for $\adv$
(assuming the given value of $\secpar$), and let $r \sample \RND{\adv}$ denote
the random choice of the randomiser $r$ from $\RND{\adv}$. We denote by $\negl$
($\poly$) an arbitrary negligible (resp.~polynomial) function.

Probability ensembles $X = \smallset{X_\secpar}_\secpar$ and
$Y = \smallset{Y_\secpar}_\secpar$, for distributions $X_\secpar, Y_\secpar$, have
\emph{statistical distance} $\SD(X_\secpar, Y_\secpar) = \epsilon(\secpar)$ if
$\sum_{a \in \supp{X_\secpar \cup Y_\secpar}} \abs{\prob{X_\secpar = a} -
  \prob{Y_\secpar = a}} = \epsilon(\secpar)$. We write $X \approx_\secpar Y$ if
$\SD(X_\secpar, Y_\secpar) \leq \negl$. For values $a(\secpar)$ and $b(\secpar)$ we
write $a(\secpar) \approx_\secpar b(\secpar)$ if
$\abs{a(\secpar) - b(\secpar)} \leq \negl$.  For a probability space
$(\samplespace, \eventspace, \probfunction)$ and event $\event{E} \in \eventspace$ we
denote by $\nevent{E}$ an event that is complementary to $\event{E}$,
i.e.~$\nevent{E} = \samplespace \setminus \event{E}$.


Denote by $\RELGEN = \smallset{\REL}$ a family of relations. We assume that if
$\REL$ comes with any auxiliary input, the latter is benign. Directly from the
description of $\REL$ one learns security parameter $\secpar$ and description of the
group $\GRP$, if the relation is a relation of group elements (as it usually the case for zkSNARKs).

\ourpar{Bilinear groups.}
A bilinear group generator $\pgen (\secparam)$ returns public parameters $ \pp =
(p, \GRP_1, \GRP_2, \GRP_T, \pair, \gone{1}, \gtwo{1})$, where $\GRP_1$,
$\GRP_2$, and $\GRP_T$ are additive cyclic groups of prime order $p = 2^{\Omega
  (\secpar)}$, $\gone{1}, \gtwo{1}$ are generators of $\GRP_1$, $\GRP_2$, resp.,
and $\pair: \GRP_1 \times \GRP_2 \to \GRP_T$ is a non-degenerate
$\ppt$-computable bilinear pairing. We assume the bilinear pairing to be Type-3,
i.e., that there is no efficient isomorphism from $\GRP_1$ to $\GRP_2$ or from
$\GRP_2$ to $\GRP_1$. We use the by now standard bracket notation, i.e., we
write $\bmap{a}{\gi}$ to denote $a g_{\gi}$ where $g_{\gi}$ is a fixed generator
of $\GRP_{\gi}$. We denote $\pair (\gone{a}, \gtwo{b})$ as $\gone{a} \bullet
\gtwo{b}$. Thus, $\gone{a} \bullet \gtwo{b} = \gtar{a b}$. We freely use the
bracket notation with matrices, e.g., if $\vec{A} \vec{B} = \vec{C}$ then
$\vec{A} \grpgi{\vec{B}} = \grpgi{\vec{C}}$ and $\gone{\vec{A}}\bullet
\gtwo{\vec{B}} = \gtar{\vec{C}}$. Since every algorithm $\adv$ takes as input
the public parameters we skip them when describing $\adv$'s input. Similarly, we
do not explicitly state that each protocol starts with generating these
parameters by $\pgen$.

\begin{lemma}[Difference lemma,~{\cite[Lemma 1]{EPRINT:Shoup04}}]
	\label{lem:difference_lemma}
	Let $\event{A}, \event{B}, \event{F}$ be events defined in some probability
	space, and suppose that $\event{A} \land \nevent{F} \iff \event{B}
		\land \nevent{F}$.  Then 
	$
		\abs{\prob{\event{A}} - \prob{\event{B}}} \leq \prob{\event{F}}\,.
	$
\end{lemma}
\subsection{Algebraic Group Model}
The algebraic group model (AGM) introduced in \cite{C:FucKilLos18} lies between the
standard model and generic bilinear group model. In the AGM it is assumed that an
adversary $\adv$ can output a group element $\gnone{y} \in \GRP$ if $\gnone{y}$ has
been computed by applying group operations to group elements given to $\adv$ as
input. It is further assumed, that $\adv$ knows how to ``build'' $\gnone{y}$ from
that elements. More precisely, the AGM requires that whenever $\adv(\gnone{\vec{x}})$
outputs a group element $\gnone{y}$ then it also outputs $\vec{c}$ such that
$\gnone{y} = \vec{c}^\top \cdot \gnone{\vec{x}}$. $\plonk$, $\sonic$ and $\marlin$
have been shown secure using the AGM. An adversary that works in the AGM is called
\emph{algebraic}.

\oursubsub{Idealised Verifier and Verification Equations.} Let
$(\kgen, \prover, \verifier, \simulator)$ be a proof system.
% or a polynomial commitment
% scheme\hamid{might be unclear as we are defining polynomial commitments as
%   $(\kgen, \com, \open, \verify)$.}.
Observe that the $\kgen$ algorithm provides an SRS which can be interpreted as a set
of group representation of polynomials evaluated at trapdoor elements. That is, for a
trapdoor $\chi$ the SRS contains $\gone{\p{p_1}(\chi), \ldots, \p{p_k}(\chi)}$, for
some polynomials $\p{p_1}(X), \ldots, \p{p_k}(X) \in \FF_p[X]$. The verifier
$\verifier$ accepts a proof $\zkproof$ for instance $\inp$ if (a set of) verification
equation $\vereq_{\inp, \zkproof}$ (which can also be interpreted as a polynomial in
$\FF_p[X]$ whose coefficients depend on messages sent by the prover) zeroes at
$\chi$. Following \cite{EPRINT:GabWilCio19} we call verifiers who check that
$\vereq_{\inp, \zkproof}(\chi) = 0$ \emph{real verifiers} as opposed to \emph{ideal
  verifiers} who accept only when $\vereq_{\inp, \zkproof}(X) = 0$. That is, while a
real verifier accepts when a polynomial \emph{evaluates} to zero, an ideal verifier
accepts only when the polynomial \emph{is} zero.

Although ideal verifiers are impractical, they are very useful in our
proofs. More precisely, we show that
\begin{compactenum}
\item the idealised verifier accepts an incorrect proof (what ``incorrect''
  means depends on the situation) with at most negligible probability (and in many
  cases---never);
\item when the real verifier accepts, but not the idealised one, then a malicious
  prover can be used to break the underlying security assumption (in our case---a
  variant of $\dlog$.)
\end{compactenum}

Analogously, idealised verifier can be defined for polynomial commitment schemes.

 
\subsection{Signatures of Knowledge}
Signatures of Knowledge (SoKs) \cite{C:ChaLys06} are closely related to
simulation-extractable SNARKs: A signer can generate a valid signature for a message
only if she has a valid witness for the statement.
%
The notion of SoKs mimics digital signatures with strong existential unforgeability:
even if the adversary has seen many signatures on arbitrary messages under arbitrary
statement, it cannot create a new signature (not seen before) without knowing the
witness for the statement in question.
%
To define Signatures of Knowledge we use a recent version \cite{C:GroMal17} that
implicitly considers only compact (succinct) signatures. \COMMENT{We omit the
  description of SoK's algorithms and security properties and defer them to
  \cref{sec:sok}.} Intuitively, a SoK behaves like a non-interactive proof system
that ``binds'' the message being signed to the proof.
     
%\subsection{Signatures of Knowledge}
\label{sec:sok}
A $\SoK$ for an efficiently decidable binary relation $\REL$ is defined as a tuple of PPT algorithms $\SoK = (\signsetup,  \sign, \allowbreak \verify,  \simsetup, \simsign)$:

\begin{description}
    \item[$\signsetup(1^\secpar, \REL) \rightarrow  \param$:]
	The setup algorithm takes a security parameter $\secpar$ and a binary relation $\REL$
	and returns public parameters $ \param$.  The input $ \param$ is implicit to all subsequent algorithms. 

    \item[$\sign(\mesage, \inp, \wit)  \rightarrow \signature$:]
The signing algorithm takes as input a message $ \mesage \in \{0,1\}^{*}$, 
a statement $\inp$, and a witness $\wit$.
	Outputs a signature $\signature$.

    \item[$\verify(\mesage, \inp, \signature) \rightarrow 1/ 0$:]
The verification algorithm takes as input 
 a message $\mesage$,  a statement $\inp$ 
 together with a signature $\signature$,
	outputs $1$ if the the signature is valid, $0$ otherwise.
	
    \item[$\simsetup(\REL) \rightarrow (\param, \td)$:]
    	A simulated setup algorithm which takes as input a relation $\REL$ and returns public parameters $\param$ and a trapdoor $\td$. 
    	
   \item[$\simsign(\td, \mesage, \inp) \rightarrow \signature'$:]
   	A simulated signing  that takes as input  a trapdoor $\td$, a message $\mesage$ and a statement $\inp$ and returns a simulated signature $\signature'$.
\end{description}
 
A SoK scheme should satisfy correctness, extractability and simulatability:
   
\begin{description}
\item[Perfect Correctness.] This guarantees that a signer with a valid witness can
  always produce a signature that will convince the verifier. For all
  $\secpar\in \NN$, for all efficiently decidable binary relation $\REL$, for all
  $(\inp, \wit) \in \REL$, and for all $ \mesage \in \{0,1\}^{*}$:
   \[
  \condprob{
	  \begin{matrix}
~\verify(\mesage, \inp, \signature) = 1   
	 \end{matrix}
}{
	  \begin{matrix}
~(\param, \td) \gets \signsetup(1^\secpar, \REL)\\
~ \signature \gets  \sign(\mesage, \inp, \wit)
 \end{matrix} }  =1. 
\]
%
\item[Simulation Extractability:] This guarantees that an adversary is not able to issue a new signature
unless it knows a witness. This should hold even if the adversary gets to see signatures on
arbitrary messages under arbitrary statements. We model this notion in a strong sense, by
letting the adversary see simulated signatures for arbitrary messages and statements, which
potentially include false statements. Even under this strong attack model, we require that
whenever the adversary outputs a valid signature not queried before, it is possible to extract a
witness for the signature. More formally,  for any PPT adversary $\adv$ there exists a PPT extractor $\ext_\adv$ such that:
   \[
  \condprob{
	  \begin{matrix}
~ \verify(\mesage, \inp, \signature) = 1   \\
\land (\inp, \wit) \notin \REL \\
\land (\mesage, \inp, \signature) \notin Q_{\simsign_\td}
	 \end{matrix}
}{
	  \begin{matrix}
~(\param, \td) \gets \simsetup(1^\secpar, \REL),\\
~ (\mesage, \inp, \signature) \gets \adv^{\simsign_\td} (\param; r),\\
\wit \gets \ext_\adv(\param, Q_{\simsign_\td}, r)
 \end{matrix}
} \leq \negl
\]
where the adversary has access to simulated signatures via the oracle
$\simsign_\td(\inp_i, \mesage_i) \gets \simsign(\td, \inp_i, \mesage_i)$ and the
extractor $\ext_\adv$ takes as input public parameters $\param$, list
$Q_{\simsign_\td}$ of all queries made by $\adv$, and $\adv$'s randomness $r$.

%\vspace{4pt}
%
\item[Perfect Simulatability:] The verifier should not learn anything about the
  witness from the signature.  The secrecy of the witness is modelled by the ability
  to simulate signatures without the witness. More precisely, we say the signatures
  of knowledge are simulatable if there is a simulator that can create public
  parameters together with an associated trapdoor that enables producing signatures
  without a witness that are indistinguishable from real ones.  More formally, for
  any number of queried $\mesage_i\in \{0,1\}^{*}$ and $(\inp_i, \wit_i) \in \REL$:
\[
\left|	\condprob{  
  \begin{matrix}
		\adv(\param) = 1 
  \end{matrix}
		}{
		\begin{matrix}
      \param \gets  \signsetup(1^\secpar, \REL),\\
      ~\signature_i \gets \sign(\mesage_i, \inp_i, \wit_i)
		\end{matrix}
  } - \condprob{
		\begin{matrix}
		\adv (\param) = 1 
		\end{matrix}
}{
		\begin{matrix}
      ~(\param,\td) \gets  \simsetup(1^\secpar, \REL), \\
      ~\signature_i \gets \simsign(\td,\mesage_i, \inp_i)
		\end{matrix}} \right| \leq \negl.
\]
\end{description}

  \subsection{Dlog assumptions}
\label{sec:dlog_assumptions}
\begin{definition}[$(q_1, q_2)\mhyph\dlog$ assumption]\label{def:dlog}
	Let $\adv$ be a $\ppt$ adversary that gets as input
  $\gone{1, \chi, \ldots, \chi^{q_1}}, \gtwo{1, \chi, \ldots, \chi^{q_2}}$, for
  some randomly picked $\chi \in \FF_p$, then
	\[
		\condprob{\chi \gets \adv(\gone{1, \chi, \ldots, \chi^{q_1}}, \gtwo{1, \chi,
        \ldots, \chi^{q_2} })}{\chi \sample \FF_p} \leq \negl.
	\]
\end{definition}

\begin{definition}[$(q_1, q_2)\mhyph\ldlog$ assumption]\label{def:ldlog}
  Let $\adv$ be a $\ppt$ adversary that gets as input
  $\gone{\chi^{-q_1}, \ldots, 1, \chi, \ldots, \chi^{q_1}}, \gtwo{\chi^{-q_2},
    \ldots, 1, \chi, \ldots, \chi^{q_2}}$, for some randomly picked
  $\chi \in \FF_p$, then
	\[
    \condprob{\chi \gets \adv(\gone{\chi^{-q_1}, \ldots, 1, \chi, \ldots,
        \chi^{q_1}}, \gtwo{\chi^{-q_2}, \ldots, 1, \chi, \ldots, \chi^{q_2}
      })}{\chi \sample \FF_p} \leq \negl.
	\]
\end{definition}

\subsection{Uber assumption}
\label{sec:uber_assumption}
\ourpar{BBG uber assumption.}  To show trapdoor-less simulatability of $\plonk$, what
is required by our reduction, we rely on the \emph{uber assumption} introduced by
Boneh et al.~\cite{EC:BonBoyGoh05} as presented by Boyen in \cite{PAIRING:Boyen08}.

Let $r, s, t, c \in \NN \setminus \smallset{0}$. Consider vectors of polynomials
$\pR \in \FF_p[X_1, \ldots, X_c]^r$, $\pS \in \FF_p[X_1, \ldots, X_c]^s$ and
$\pT \in \FF_p[X_1, \ldots, X_c]^t$. Write $\pR = \left( \p{r}_1, \ldots,
  \p{r}_r \right)$, $\pS = \left( \p{s}_1, \ldots, \p{s}_s \right)$ and $\pT =
\left( \p{t}_1, \ldots, \p{t}_t \right)$ for polynomials $\p{r}_i, \p{s}_j,
\p{t}_k$.

For a function $f$ and vector $(x_1, \ldots, x_c)$ we write $f(\pR)$ to
denote application of $f$ to each element of $\pR$, i.e.
\(
	f(\pR) = \left( f(\p{r}_1 (x_1, \ldots, x_c), \ldots, f(\p{r}_r
	(x_1, \ldots, x_c) \right).
\)
Similarly for applying $f$ to $\pS$ and $\pT$.

\begin{definition}[Independence of $\pR, \pS, \pT$]
	\label{def:independence}
	Let $\pR, \pS, \pT$ be defined as above. We say that polynomial $\p{f} \in
  \FF_p[X_1, \ldots, X_c]$ is \emph{dependent} on $\pR, \pS, \pT$ if there
  exists $rs + t$ constants $a_{i, j}, b_k$ such that $ \p{f} = \sum_{i = 1}^{r}
  \sum_{j = 1}^{s} a_{i, j} \p{r}_i \p{s}_j + \sum_{k = 1}^{t} b_k \p{t}_k. $ We
  say that $\p{f}$ is \emph{independent} if it is not dependent.
\end{definition}

To show trapdoor-less simulatability of $\plonk$ we utilize a generalization of
Boneh-Boyen-Goh's \emph{uber assumption} \cite{EC:BonBoyGoh05} stated as follows (the
changed element has been put into a \dbox{dashbox})
\begin{definition}[$(\pR, \pS, \pT, \p{F}, 1)$-uber assumption]
	\label{def:uber_assumption}
	Let $\pR, \pS, \pT$ be defined as above,
    $(x_1, \ldots, x_c, y_1, \ldots, y_{d}) \sample \FF_p^{c + d}$ and let
    $\p{F}$ be a cardinality-$d$ set of pair-wise independent polynomials which are also
    independent of $(\pR, \pS, \pT)$, cf.~\cref{def:independence}.  Then, for
    any $\ppt$ adversary $\adv$
	\begin{multline*}
      \Pr\left[\adv(\gone{\pR(x_1, \ldots x_c)}, \gtwo{\pS(x_1, \ldots, x_c)},
		\gtar{\pT(x_1, \ldots, x_c)}, \dboxed{\gone{\p{F}(x_1, \ldots, x_c)}}) = 1\right] \approx_\secpar \\
      \Pr\left[\adv(\gone{\pR(x_1, \ldots x_c)}, \gtwo{\pS(x_1, \ldots, x_c)},
        \gtar{\pT(x_1, \ldots, x_c)}, \dboxed{\gone{y_1, \ldots, y_{d}}}) =
        1\right].
	\end{multline*}
\end{definition}

Compared to the original uber assumptions, there are two major changes. First, in the
original formulation the indistinguishable (boxed) elements were in $\GRP_T$, here
they are in $\GRP_1$.
% \anca{}{do not get thie previous sentence. What are not target group elements but
% $\GRP_1$, you mean $\GRP_2$? } \michals{14.10}{Check now}
  %
Second, Boneh et al.'s assumption works for distinguishers who are given only one
challenge polynomial $\p{f}$, i.e.~$\abs{\p{F}} = 1$.
  
We show security of our version of the uber assumption using the generic group model
as introduced by Shoup \cite{EC:Shoup97},
cf.~\cref{sec:uber-assumption-security-proof}.

\subsection{Zero-Knowledge Proof Systems}\label{prelim:nizk}
In a zero-knowledge proof system, a prover convinces the verifier of the veracity of a statement
without leaking any other information. The zero-knowledge property is proven by constructing a
simulator that can simulate the view of a cheating verifier without knowing the secret
information -- witness -- of the prover. A proof system has to be sound as well, i.e.~for a
malicious prover it should be infeasible to convince a verifier of a false statement. Here, we
focus on proof systems, so-called arguments, that guarantee soundness against $\ppt$ malicious provers.

More precisely, let $\RELGEN(\secparam) = \smallset{\REL}$ be a family of universal 
relations. Let $\prover$ and $\verifier$ be $\ppt$ algorithms, the former called \emph{prover}
and the latter \emph{verifier}. We allow our proof system to have a setup, i.e.~there is a
$\kgen$ algorithm that takes as input the relation description $\REL$ and outputs a common
reference string $\srs$. We assume that the $\srs$ defines the relation, and for universal proof
systems, such as Plonk and Sonic, we treat both the reference string and the relation as
universal.

We denote by $\zkproof$ a proof created by $\prover$ with input
$(\srs, \inp, \wit)$. We say that proof is acceptable if $\verifier (\srs, \inp,
\zkproof)$ accepts it. We focus on proof systems where $\prover$ and
$\verifier$ are given oracle access to a random oracle $\ro$. The simulator
$\simulator$ is not only given access to $\ro$, but it can also \emph{program}
it. That is, it can require that for $(x, y)$ of its choice, $\ro (x) = y$.
% \changedm{This is
% formalized by dividing the simulator into two subroutines $\simulator_1,
% \simulator_2$ that share state. The first subroutine $\simulator_1$ answers random
% oracle queries, while the second one provides simulated proofs.}

A non-interactive  proof system $\proofsystem = (\kgen, \prover, \verifier, \simulator)$ for $\RELGEN$ is
required to have three properties: completeness, soundness and zero knowledge, which are
defined as follows:
%\hamid{10.2}{Should we change the definitions to the updatable setting? In that case, we should move updatable section earlier before this!}
% \begin{description}

\ourpar{Completeness.}
%\item[Completeness]
  A non-interactive proof system $\proofsystem$ is
  \emph{complete} if an honest prover always convinces an honest verifier, that
  is for all $\REL \in \RELGEN(\secparam)$ and $(\inp, \wit) \in \REL$
	\[
		\condprob{\verifier^\ro (\srs, \inp, \zkproof) = 1} {\srs \gets \kgen(\REL),
      \zkproof \gets \prover^\ro (\srs, \inp, \wit)}\,.
	\]
    % \item[Soundness]
\ourpar{Soundness.}
    We say that $\proofsystem$ for $\RELGEN$ is \emph{sound} if no
  $\ppt$ prover $\adv$ can convince an honest verifier $\verifier$ to accept a
  proof for a false statement $\inp \not\in\LANG$. More precisely, for
  all $\REL \in \RELGEN(\secparam)$
	\[
    \condprob{ \verifier^\ro (\srs, \inp, \zkproof) = 1 \land \inp \not\in
      \LANG_\REL}{\srs \gets \kgen(\REL), (\inp, \zkproof) \gets \adv^\ro(\srs)} \leq
    \negl.
	\]
%\end{description}
Sometimes a stronger notion of soundness is required -- besides requiring that the
verifier rejects proofs of statements outside the language, we request from the
prover to know a witness corresponding to the proven statement. This property is
called \emph{knowledge soundness}.%\markulf{Commented out the formal definition as we don't use it.}
 
\ourpar{Zero knowledge.}  Our definition of zero-knowledge for non-interactive
arguments is in the explicitly programmable random oracle model where the simulator
$\simulator$ can program the random oracle. We model $\simulator$ as a stateful
algorithm that runs in two modes. The first mode,
$(h, st') \gets \simulator (1, st, \srs, q)$ answers random oracle calls to $\ro$ on input
$q$. The second mode $(\zkproof, st') \gets \simulator (2, st, \srs, \inp)$ simulates the
actual argument for instance $\inp$.  Additionally, we define the
following oracles that share state as wrapper oracles around $\simulator$.
\begin{itemize}
\item $\simulator_1 (\srs,q)$ denotes an oracle that returns the first output of
  $\simulator (1, st, \srs, q)$;
\item $\simulator_2 (\srs,\inp, \wit)$ denotes an oracle that returns the first output of
  $\simulator (2, st, \srs, \inp)$ if $(\inp, \wit) \in \REL$, and returns $\bot$ otherwise;
\end{itemize}

% We call a proof system $\proofsystem$ \emph{zero-knowledge} if for any
% $\REL \in \RELGEN(\secparam)$, and adversary $\adv$ there exists a $\ppt$ simulator
% $\simulator$ such that for any $(\inp, \wit) \in \REL$, $\eps_0 \approx \eps_1$,
% where
% \[
%   \eps_b = \condprob{\adv^{\oracleo_b} (\srs)}{\srs \gets \kgen (\REL)}\,,
% \]
% \changedm{where $\oracleo_0$ on input $(1, q)$ responds with $h$ such that
%   $(h, st') \gets \simulator_1 (1, st, q)$, where $st$ and $st'$ are old and new
%   states of the simulator $\simulator$ and on input $(2, \inp, \wit)$ returns
%   $\zkproof \gets \simulator_2 (2, st, \srs, \inp)$. Alternatively, $\oracleo_1$ on
%   input $(1, q)$ responds with $h \gets \ro (q)$ and on input $(2, \inp, \wit)$
%   returns $\zkproof \gets \prover (\srs, \inp, \wit)$.}
% 	%
% We call zero knowledge \emph{perfect} if the distributions are equal and
% \emph{computational} if they are indistinguishable for any $\ppt$ distinguisher.

We call a proof system $\proofsystem$ \emph{zero-knowledge} if for any
$\REL \in \RELGEN(\secparam)$, and adversary $\adv$ there exists a $\ppt$ simulator
$\simulator$ with oracles $(\simulator_1,\simulator_2)$ such that for any $(\inp, \wit) \in \REL$, $\eps_0 \approx \eps_1$,
where
\[
  \eps_0(\secpar) = \condprob{\adv^{\simulator_1,\simulator_2} (\srs)}{\srs \gets \kgen
    (\REL)},\,  \eps_1 (\secpar) = \condprob{\adv^{\ro,\prover} (\srs)}{\srs \gets \kgen (\REL)},
\]

We call zero knowledge \emph{perfect} if the distributions are equal and
\emph{computational} if they are indistinguishable for any $\ppt$ distinguisher.


% \end{description}
Alternatively, zero-knowledge can be defined by allowing the simulator to use the
trapdoor $\td$ that is generated along the $\srs$. In this paper we distinguish
simulators that requires a trapdoor to simulate and those that do not. We call the
former \emph{SRS-simulators}. We say that a protocol is \emph{trapdoor-less
  simulatable} (TLS) if its simulator does not require the trapdoor, cf.~\cref{def:tls}.

\subsection{From interactive to non-interactive---the Fiat--Shamir transform}
Consider a $(2\mu + 1)$-message, public-coin, honest verifier zero-knowledge
interactive proof system
$\proofsystem = (\kgen, \prover, \verifier, \simulator)$ for
$\REL \in \RELGEN(\secparam)$.  Let $\zkproof$ be a proof performed by the
prover $\prover$ and verifier $\verifier$ compound of messages
$(a_1, b_1, \ldots, a_{\mu}, b_{\mu}, a_{\mu + 1})$, where $a_i$ comes from
$\prover$ and $b_i$ comes from $\verifier$.  Denote by $\ro$ a random oracle.
Let $\proofsystem_\fs = (\kgen_\fs, \prover_\fs, \verifier_\fs, \simulator_\fs)$
be a proof system such that
\begin{compactitem}
  \item $\kgen_\fs$ behaves as $\kgen$.
  \item $\prover_\fs$ behaves as $\prover$ except after sending message
    $a_i$, $i \in \range{1}{\mu}$, the prover does not wait for
    the message from the verifier but computes it locally setting $b_i
    = \ro(\zkproof[0..i])$, where $\zkproof[0..j] = (\inp, a_1, b_1, \ldots,
    a_{j - 1}, b_{j - 1}, a_j)$. (Importantly, $\zkproof[0..\mu + 1] =
    (\inp, \zkproof)$).\footnote{For Fiat--Shamir based SoK the message signed $m$ is added to $\inp$ before hashing.}
  \item $\verifier_\fs$ behaves as $\verifier$ but does not provide
    challenges to the prover's proof. Instead it computes the
    challenges locally as $\prover_\fs$ does. Then it verifies the
    resulting transcript $\zkproof$ as the verifier $\verifier$ would. 
  \item $\simulator_\fs$ behaves as $\simulator$, except when
    $\simulator$ picks challenge $b_i$ before computing message $\zkproof[0, i]$, $\simulator_\fs$ programs the
    random oracle to output $b_i$ on $\zkproof[0, i]$.
  \end{compactitem}

\noindent
The Fiat--Shamir heuristic states that $\proofsystem_\fs$ is a zero-knowledge
non-interactive proof system for $\REL \in \RELGEN(\secparam)$.

%\subsection{Non-malleability definitions for NIZKs}
%\label{sec:simext_def}
%Real life applications often require a NIZK proof system to be
%non-malleable. That is, no adversary seeing a proof $\zkproof$ for a statement
%$\inp$ should be able to provide a new proof $\zkproof'$ related to $\zkproof$.
%\emph{Simulation extractability} formalizes a strong version of non-malleability
%by requiring that no adversary can produce a valid proof without knowing the
%corresponding witness. This must hold even if the adversary is allowed to see
%polynomially many simulated proofs for any statements it wishes.
%
%%\chaya{remove reference to forking soundness. quantify for $\ext_\se$}
%\begin{definition}[Forking simulation-extractable NIZK, \cite{INDOCRYPT:FKMV12}]
%	\label{def:simext}
%  Let $\ps_\fs = (\kgen_\fs, \prover_\fs, \verifier_\fs, \simulator_\fs)$ be a
%  HVZK proof system\hamid{$\ps_\fs$ is the Fiat-Shamir variant of the underlying proof system. So maybe we mean the underlying proof system is HVZK?}. We say that $\ps_\fs$ is \emph{forking
%    simulation-extractable} with \emph{extraction error} $\nu$ if for any $\ppt$
%  adversary $\adv$ that is given oracle access to a random oracle $\ro$ and
%  simulator $\simulator_\fs$, and produces an accepting transcript of $\ps$ with
%  probability $\accProb$, where
%	\[
%		\accProb = \Pr \left[
%		\begin{aligned}
%			& \verifier_\fs(\srs, \inp_{\advse}, \zkproof_{\advse}) = 1,\\
%			& (\inp_{\advse}, \zkproof_{\advse}) \not\in Q
%		\end{aligned}
%		\, \left| \,
%		\begin{aligned}
%			& \srs \gets \kgen_\fs(\REL), r \sample \RND{\advse}, \\
%			& (\inp_{\advse}, \zkproof_{\advse}) \gets \advse^{\simulator_\fs,
%			\ro} (\srs; r)
%		\end{aligned}
%		\right.\right]\,,
%	\]
%	there exists an extractor $\extse$ such that
%	\[
%		\extProb = \Pr \left[
%		\begin{aligned}
%			& \verifier_\fs(\srs, \inp_{\advse}, \zkproof_{\advse}) = 1,\\
%			& (\inp_{\advse}, \zkproof_{\advse}) \not\in Q,\\
%			& \REL(\inp_{\advse}, \wit_{\advse}) = 1
%		\end{aligned}
%		\, \left| \,
%		\begin{aligned}
%			& \srs \gets \kgen_\fs(\REL), r \sample \RND{\advse},\\
%			& (\inp_{\advse}, \zkproof_{\advse}) \gets \advse^{\simulator_\fs,
%			\ro} (\srs; r) \\
%			& \wit_{\advse} \gets \ext_\se (\srs, \advse, r, \inp_{\advse}, \zkproof_{\advse},
%			Q, Q_\ro) 
%		\end{aligned}
%		\right.\right]
%	\]
%	is at at least 
%	\[
%		\extProb \geq \frac{1}{\poly} (\accProb - \nu)^d - \eps(\secpar)\,,
%	\]
%	for some polynomial $\poly$, constant $d$ and negligible $\eps(\secpar)$ whenever
%  $\accProb \geq \nu$. List $Q$ contains all $(\inp, \zkproof)$ pairs where
%  $\inp$ is an instance provided to the simulator by the adversary and
%  $\zkproof$ is the simulator's answer. List $Q_\ro$ contains all $\advse$'s
%  queries to $\ro$ and $\ro$'s answers.
%\end{definition}


\newcommand*{\Scale}[2][4]{\scalebox{#1}{$#2$}}% 



\subsection{Updatable SRS schemes}\label{def:upd-scheme}

We recall the definition of an updatable SRS scheme from~\cite{C:GKMMM18} which consists of the following algorithms.
\vspace{-8pt}
\begin{itemize} 
	\item
	$(\srs,\rho) \gets \kgen(\REL)$ is a PPT algorithm that takes a relation $\REL$ and outputs a SRS $\srs$, and correctness proof $\rho$.
	\item
	$ (\srs',\rho') \gets \upd(\srs, \{\rho_j \}_{j=1}^n)$ is a PPT algorithm that takes a SRS $\srs$, a list of update proofs and outputs an updated SRS together with a proof of correct update. 
	\item
	$b \gets \verifyCRS(\srs, \{\rho_j \}_{j=1}^n)$ is a DPT algorithm that takes a SRS $\srs$, a list of update proofs, and outputs a bit indicating acceptance or not.
\end{itemize}

	In all our protocols, we deal with a structured reference string that is updatable. We give the $\kgen$, $\upd$ and $\verifyCRS$ algorithms in~\cref{fig:upd-scheme}.
	
%	\begin{figure}[t!]
%		\centering
%		\fbox{
%			\begin{minipage}[t]{0.3\linewidth}
%				\procedure{$\kgen(\secparam)$}{
%				\alpha, \chi \sample \FF^2_p \\ [\myskip]
%		\pcreturn 
%\left( \gone{\smallset{\chi^i}_{i = -\dconst}^{\dconst},
%	\smallset{\alpha \chi^i}_{i = -\dconst, i \neq 0}^{\dconst}},
%\gtwo{\smallset{\chi^i, \alpha \chi^i}_{i = - \dconst}^{\dconst}},
%\gtar{\alpha} \right)
%				}
%			\end{minipage}
%			\begin{minipage}[t]{0.3\linewidth}
%				\procedure{$\verifyCRS(\srs, \{\rho_j \}_{j=1}^n)$}{
%					%
%				}
%		\end{minipage}
%	%
%\begin{minipage}[t]{0.3\linewidth}
%	\procedure{$\upd(\srs, \{\rho_j \}_{j=1}^n)$}{
%		%
%	}
%\end{minipage}}
%		\caption{Updatable SRS scheme} 
%		\label{fig:upd-scheme}
%	\end{figure}

%\newcommand{\bP}[1]{\overline{P_{#1}}}
%\newcommand{\bQ}[1]{\overline{Q_{#1}}} 
%\newcommand{\hP}[1]{\hat{P_{#1}}} 
%\newcommand{\hQ}[1]{\hat{Q_{#1}}}

%
%\begin{figure}[t!]
%	\centering
%	\fbox{
%		\begin{minipage}[t]{0.6\linewidth}
%			\procedure{$\kgen(\REL)$}{
%				\alpha, \chi \sample \FF^2_p \\ [\myskip]
%				\srs := 
%				\left( \gone{\smallset{\chi^i}_{i = -\dconst}^{\dconst},
%					\smallset{\alpha \chi^i}_{i = -\dconst, i \neq 0}^{\dconst}},
%				\gtwo{\smallset{\chi^i, \alpha \chi^i}_{i = - \dconst}^{\dconst}},
%				\gtar{\alpha} \right) \\ [\myskip]
%				\rho = 	\left( \gone{\chi, \alpha, \chi, \alpha}, \gtwo{\chi, \alpha} \right) \\ [\myskip]
%				\pcreturn (\srs, \rho) \\ [\myskip]
%			}
%			%
%			\\
%			%
%			\procedure{$\verifyCRS(\srs, \{\rho_j \}_{j=1}^n)$}{
%				\text{Parse }  \{\rho_j \}_{j=1}^n \text{ as } \left( P_j, Q_j, \bP_j, \bQ_j, \hP_j, \hQ_j \right) \\ [\myskip]
%				\text{Parse } \srs \text{ as } \left( \gone{\smallset{A_i}_{i = -\dconst}^{\dconst},
%					\smallset{B_i}_{i = -\dconst, i \neq 0}^{\dconst}},
%				\gtwo{\smallset{C_i, D_i}_{i = - \dconst}^{\dconst}},
%				\textcolor{red}{\gtar{E}} \right) \\ [\myskip]
%				\text{Verify Update proofs: } \\ [\myskip]
%				\t \bP_n = A_1 \neq 1, \bQ_n = \textcolor{red}{B_0} \neq 1 \\ [\myskip]
%				\t \bP_1 = P_1, \bQ_1 = Q_1 \\ [\myskip]
%				\t \bP_n \bullet \gtwo{1} = \gone{1} \bullet \hP_n \\ [\myskip]
%				\t \bQ_n \bullet \gtwo{1} = \gone{1} \bullet \hQ_n \\ [\myskip]
%				\t P_j \bullet \gtwo{1} = P_{j-1} \bullet \hP_j \quad \forall j \geq 2 \\ [\myskip]
%				\t Q_j \bullet \gtwo{1} = Q_{j-1} \bullet \hQ_j \quad \forall j \geq 2 \\ [\myskip]
%				\text{Verify SRS structure: } \\ [\myskip]
%				\t \gone{A_i} \bullet \gtwo{1} = \gone{A_{i-1}} \bullet \gtwo{C_1} \text{ for all } -\dconst < i \leq \dconst \\ [\myskip]
%				\t \gone{1} \bullet \gtwo{C_i} = \gone{A_{1}} \bullet \gtwo{C_{i-1}} \text{ for all } -\dconst < i \leq \dconst \\ [\myskip]
%				\t \gone{A_i} \bullet \gtwo{1} = \gone{1} \bullet \gtwo{C_i} \text{ for all } -\dconst \leq i \leq \dconst \\ [\myskip]
%				\t \gone{B_i} \bullet \gtwo{1} = \gone{B_{i-1}} \bullet \gtwo{C_1} \text{ for all } -\dconst < i \leq \dconst; \textcolor{red}{i \neq 0,1} \\ [\myskip]
%				\t \gone{1} \bullet \gtwo{D_i} = \gone{A_1} \bullet \gtwo{D_{i-1}} \text{ for all } -\dconst < i \leq \dconst \\ [\myskip]
%				\t \gone{B_i} \bullet \gtwo{1} = \gone{1} \bullet \gtwo{D_i} \text{ for all } -\dconst \leq i \leq \dconst; \textcolor{red}{i \neq 0} \\ [\myskip]
%			}
%			%
%			\\
%			%
%			\procedure{$\upd(\srs, \{\rho_j \}_{j=1}^n)$}{
%				\text{Parse } \srs \text{ as } \left( \gone{\smallset{A_i}_{i = -\dconst}^{\dconst},
%					\smallset{B_i}_{i = -\dconst, i \neq 0}^{\dconst}},
%				\gtwo{\smallset{C_i, D_i}_{i = - \dconst}^{\dconst}},
%				\gtar{E} \right) \\ [\myskip]
%				\alpha, \chi \sample \FF^2_p \\ [\myskip]
%				\srs' := 
%				\left( \gone{\smallset{\chi^i A_i}_{i = -\dconst}^{\dconst},
%					\smallset{\alpha \chi^i B_i}_{i = -\dconst, i \neq 0}^{\dconst}},
%				\gtwo{\smallset{\chi^i C_i, \alpha \chi^i D_i}_{i = - \dconst}^{\dconst}},
%				\gtar{\alpha E} \right) \\ [\myskip]
%				\rho' = 	\left( \gone{\chi A_1, \alpha \textcolor{red}{B_0}, \chi, \alpha}, \gtwo{\chi, \alpha} \right) \\ [\myskip]
%				\pcreturn (\srs', \rho')
%			}
%	\end{minipage}}
%	\caption{Updatable SRS scheme} 
%	\label{fig:upd-scheme}
%\end{figure}

\begin{figure}[t!]
	\centering
	\fbox{
		\begin{minipage}[t]{0.6\linewidth}
			\procedure{$\kgen(\REL)$}{
				\chi \sample \FF_p \\ [\myskip]
				\srs := 
				\left( \gone{\smallset{\chi^i}_{i = 0}^{\dconst}},
				\gtwo{\chi} \right); 
				\rho =  \left(\gone{\chi, \chi}, \gtwo{\chi}\right) \\ [\myskip]
				\pcreturn (\srs, \rho) \\ [\myskip]
			}
		%
		\\
		%
		\procedure{$\verifyCRS(\srs, \{\rho_j \}_{j=1}^n)$}{
			\text{Parse }  \srs \text{ as } \left( \gone{\smallset{A_i}_{i = 0}^{\dconst}},
			\gtwo{B} \right) \text{and } \{\rho_j \}_{j=1}^n \text{ as } \left\{\left( P_j, \bP_j, \hP_j \right)\right\}_{j=1}^n \\ [\myskip]
			\text{Verify Update proofs: } \\ [\myskip]
			\t \bP_1 = P_1 \\ [\myskip]
			\t P_j \bullet \gtwo{1} = P_{j-1} \bullet \hP_j \quad \forall j \geq 2 \\ [\myskip]
			\t \bP_n \bullet \gtwo{1} = \gone{1} \bullet \hP_n \\ [\myskip]
			\text{Verify SRS structure: } \\ [\myskip]
			\t \gone{A_i} \bullet \gtwo{1} = \gone{A_{i-1}} \bullet \gtwo{B} \text{ for all } 0 < i \leq \dconst \\ [\myskip]
		}
		%
		\\
		%
		\procedure{$\upd(\srs, \{\rho_j \}_{j=1}^n)$}{
			\text{Parse } \srs \text{ as } \left( \gone{\smallset{A_i}_{i = 0}^{\dconst}},
			\gtwo{B} \right) \\ [\myskip]
			\chi' \sample \FF_p  \\ [\myskip]
			\srs' := 
			\left( \gone{\smallset{{\chi'}^i A_i}_{i = 0}^{\dconst}},
			\gtwo{\chi' B} \right); 
			\rho' =	\left( \gone{\chi' A_1, \chi'}, \gtwo{\chi'}\right) \\ [\myskip]
			\pcreturn (\srs', \rho')
		}
		\end{minipage}}
	\caption{Updatable SRS scheme} 
	\label{fig:upd-scheme}
\end{figure}

In the next section, we define security notions in the updatable setting. To this
end, we define an SRS update oracle $\initU$ in~\cref{fig:upd} by which the adversary
updates the SRS.

\begin{figure}[t!]
	\centering
	\fbox{
		\begin{minipage}[t]{0.58\linewidth}
			\procedure{$\initU(\intent, \srs_n,\{\rho_j \}_{j=1}^n)$}{
				\pcif \srs \neq \bot: \pcreturn \bot \\
				\pcif (\intent = \setup): \\
				\t (\srs',\rho') \leftarrow \kgen(\REL)\\
				\t Q_\srs \gets Q_\srs \cup \{(\srs',\rho')\}\\
				\t \pcreturn (\srs',\rho')\\
				%
				\pcif (\intent = \update): \\
				\t b \gets \verifyCRS(\srs_n, \{\rho_j \}_{j=1}^{n})\\
				\t \pcif (b=0): \pcreturn \bot \\
				\t (\srs',\rho') \leftarrow \upd (\srs_n,\{\rho_j \}_{j=1}^n)\\
				\t Q_\srs \gets Q_\srs \cup \{(\srs',\rho')\}\\
		 	\Scale[0.75]{ // Q_\srs = (Q^{(1)}_\srs, Q^{(2)}_\srs) \text{ s.t. }   Q^{(2)}_\srs \text{ contains the update proofs in } Q_\srs }\\
				\t \pcreturn (\srs',\rho')\\
				%
				\pcif (\intent = \final): \\
				\t b \gets \verifyCRS(\srs_n,\{\rho_j \}_{j=1}^{n})\\
				\t \pcif (b=0) \vee Q^{(2)}_\srs \cap \{ \rho_j \}_i = \emptyset: \pcreturn \bot \\
				\t 
				\t \srs \gets \srs_n, \pcreturn \srs \\
				%
				\pcelse \pcreturn \bot
			}
		\end{minipage}
		% \begin{minipage}[t]{0.3\linewidth}
		% 	\procedure{$\simO(\inp')$}{
		% 		\pcif (\srs = \bot): \pcreturn \bot \\
		% 		% \hamid{should we formalize this extractor?}\\
		% 		\zkproof'  \gets  \simulator(\srs, \COMMENT{ \td, }\inp')\\
		% 		Q = Q \cup \{(\inp', \zkproof')\}\\
		% 		\pcreturn  \zkproof' \\	
%			}                         %
%   \end{minipage}
  }
	\caption{The oracle defines the notion of updatable SRS setup. \changedm{Here we deviate from the standard definition which requires existence of an extractor that from the final SRS and adversary's updates reveals the trapdoor that could be used by a simulator. This comes since our simulator is trapdoor-less.}}
	%\hamid{24.3}{I think the red line above can be removed as we don't need trapdoor for simulation!}\michals{25.03}{Sure, check the change. What I wrote is actually much longer than the single line with the extractor. However, I think it is important to remove the line as otherwise someone may point out that we dont show how trapdoor is extracted, which ka are used to taht end etc.}% The right oracle is the simulation oracle.} 
	\label{fig:upd}
\end{figure}

% We also define a simulator
% $\simulator = (\simualtor_1, \simulator_2')$ which provides the adversary with
% simulated queries w.r.t. the SRS finalised by the adversary using $\initU$. This
% simulation oracle will be used in the definition of \COMMENT{forking }simulation
% extractability.
% \michals{29.10}{Def for simulating oracle shouldn't be here}

\chb{\ourpar{Single adversarial updates after an honest setup.}
	One can consider a slightly different model of setup, wherein the adversary is given an initial (honestly-generated) SRS and then is allowed to maliciously update that in a one-shot fashion.
	As shown by Groth
	et al. in~\cite{C:GKMMM18}, the two definitions are equivalent for the class of SNARKs in this work and thus we use this simpler definition to show our security proofs in the updatable setting.}

\ourpar{On trusted setup.} We note that a trusted setup can be expressed by the updatable setup definition above simply by restricting the algorithm that is given access to $\initU$ to call the oracle only on $\intent = \setup$ and $\intent = \final$.

\ourpar{Updatability and zero knowledge.}
Groth et al.~\cite{C:GKMMM18} shown that if a proof system is zero-knowledge against adversaries who picks the SRS, then it is updatable zero-knowledge. We call such systems \emph{subvertible zero knowledge}, cf.~\cite{AC:BelFucSca16,AC:ABLZ17}. It was shown that Plonk, Sonic, Marlin and virtually any proof system with a SRS consisting of monomials are subvertible zero-knowledge.

\chb{\paragraph{Dlog assumptions in the updatable setting.}\label{dlog-upd} Since all our protocols and security notions are in the updatable setting, we may require to rely on Dlog assumptions that are defined in the updatable setting. That is, the adversary $\adv$ is not given as input a Dlog challenge, but access to an update oracle as defined in~\cref{fig:upd}, wherein an honestly generated SRS is defined as a Dlog challenge and the update algorithm $\upd$ works similarly as in the updatable schemes by re-randomizing the challenge. Here we define these assumptions and then show a reduction between the assumptions and their variant in the updatable setting. \textcolor{blue}{Note that for ease of clarity, with a slight abuse of notation, we here denote the SRS by $\Ch$. Further, to avoid cluttering notation, we do not make the proofs of correctly generating or updating the Dlog challenges explicit as they can be generated in a similar manner to the proofs in~\cref{fig:upd-scheme}.}
	
%The reduction $\reduction$ proceeds as follows: given the input dlog instance, $\reduction$ answers adversary's queries for the dlog updates and sets the honest update to be the input dlog instance. Once the dlog challenge in the updatable setting is finalized, it runs the adversary and obtains the answer $\chi'$. Let $\chi_1, \ldots, \chi_\ell$ be the partial discrete logarithms of dlog instances corresponding to the adversary's dlog updates. These values can be computed by $\reduction$ by extracting from the update proofs given by the adversary. $\reduction$ now returns  $\chi = \chi' (\chi_1 \chi_2 \ldots \chi_\ell)^{-1}$. The same argument holds for the $(q_1, q_2)\mhyph\ldlog$ assumption, ~\cref{def:ldlog}.

\subsection{Dlog assumptions in the updatable setting}
\label{sec:udlog_assumptions}
\begin{definition}[$(q_1, q_2)\mhyph\udlog$ assumption]\label{def:udlog}
	Let $\adv$ be a $\ppt$ adversary that gets oracle access to $\initU$ with internal algorithms $(\kgen_{\dlog}, \upd_{\dlog}, \verifyCRS)$, where $\kgen_{\dlog}$ and $\upd_{\dlog}$ are defined as follows:
	\begin{itemize}
	\item $\kgen_\dlog(\secpar)$ samples $\chi \sample \FF_p$ and defines 
		$\Ch:=(\gone{1, \chi, \ldots,
			\chi^{q_1}}, \gtwo{1, \chi, \ldots, \chi^{q_2}
		})$.
	\item $\upd_\dlog(\Ch, \{\rho_j \}_{j=1}^n)$ 
	parses $\Ch$ as $\left( \gone{\smallset{A_i}_{i = 0}^{q_1}},
	\gtwo{\smallset{B_i}_{i = 0}^{q_2}} \right)$, samples
	$\widetilde{\chi} \sample \FF_p$, and defines
	$\widetilde{\Ch} := 
	\left( \gone{\smallset{\widetilde{\chi}^i A_i}_{i = 0}^{q_1}},
	\gtwo{\smallset{\widetilde{\chi}^i B_i}_{i = 0}^{q_2}} \right)$.
	\end{itemize}
	Then
	\[
	\prob{\bar{\chi} \gets \adv^{\initU}(\secpar)} \leq \negl,
	\]
	where $\left( \gone{\smallset{\bar{\chi}^i}_{i = 0}^{q_1}},
	\gtwo{\smallset{\bar{\chi}^i}_{i = 0}^{q_2}} \right)$ is the finalized challenge.
\end{definition}

\begin{definition}[$(q_1, q_2)\mhyph\uldlog$ assumption]\label{def:uldlog}
	Let $\adv$ be a $\ppt$ adversary that gets oracle access to $\initU$ with internal algorithms $(\kgen_{\ldlog}, \upd_{\ldlog}, \verifyCRS)$, where $\kgen_{\ldlog}$ and $\upd_{\ldlog}$ are defined as follows:
	\begin{itemize}
		\item $\kgen_\ldlog(\secpar)$ samples $\chi \sample \FF_p$ and defines 
		$\Ch:=(\gone{\chi^{-q_1}, \ldots, 1, \chi, \ldots,
			\chi^{q_1}}, \gtwo{\chi^{-q_2}, \ldots, 1, \chi, \ldots, \chi^{q_2}
		})$.
		\item $\upd_\ldlog(\Ch, \{\rho_j \}_{j=1}^n)$ 
		parses $\Ch$ as $\left( \gone{\smallset{A_i}_{i = -q_1}^{q_1}},
		\gtwo{\smallset{B_i}_{i = -q_2}^{q_2}} \right)$, samples
		$\widetilde{\chi} \sample \FF_p$, and defines
		$\widetilde{\Ch} := 
		\left( \gone{\smallset{\widetilde{\chi}^i A_i}_{i = -q_1}^{q_1}},
		\gtwo{\smallset{\widetilde{\chi}^i B_i}_{i = -q_2}^{q_2}} \right)$.
	\end{itemize}
	Then
	\[
	\prob{\bar{\chi} \gets \adv^{\initU}(\secpar)} \leq \negl,
	\]
	where $\left( \gone{\smallset{\bar{\chi}^i}_{i = -q_1}^{q_1}},
	\gtwo{\smallset{\bar{\chi}^i}_{i = -q_2}^{q_2}} \right)$ is the finalized challenge.
\end{definition}

We show a reduction from $(q_1, q_2)\mhyph\dlog$ assumption to its variant in the updatable setting, i.e.,~\cref{def:udlog}. We omit showing the reduction $(q_1, q_2)\mhyph\ldlog \Rightarrow (q_1, q_2)\mhyph\uldlog$ as it can be done similarly in a straightforward manner.
\begin{lemma}
	$(q_1, q_2)\mhyph\dlog \Rightarrow (q_1, q_2)\mhyph\udlog$.
	\end{lemma}
\begin{proof}
	We construct a reduction algorithm $\reduction$ which uses an adversary $\adv$ on the $(q_1, q_2)\mhyph\udlog$ and construct an adversary on the $(q_1, q_2)\mhyph\dlog$. Specifically, $\reduction$ proceeds as follows: given a dlog instance $\Ch$ as input, $\reduction$ answers adversary's queries for the dlog updates and sets the honest update to be $\Ch$. Once the dlog challenge in the updatable setting is finalized, it runs $\adv$ and obtains the answer $\chi'$. Let $\chi_1, \ldots, \chi_\ell$ be the partial discrete logarithms of dlog instances corresponding to the adversary's dlog updates. These values can be computed by $\reduction$ by extracting from the update proofs given by $\adv$. $\reduction$ now returns  $\chi = \chi' (\chi_1 \chi_2 \ldots \chi_\ell)^{-1}$.\qed %The same argument holds for the $(q_1, q_2)\mhyph\ldlog$ assumption, ~\cref{def:ldlog}.
	\end{proof}
}
  %\michals{24.03}{I think we should write a formal proof for the reduction}
\ourpar{Remark on universality of the SRS.} The proof systems we consider in this work are universal. This means the reference string $\srs$ allows to prove arithmetic constraint defined over a particular field up to some size bound. Thus $\REL$ is universal and the public instance $\inp$ must determine the constraints. We elide public preprocessing of constraint specific proving and verification keys. While important for performance, this modeling is not critical for security.



%\hamid{TODO: add a paragraph here about the universality!}
%\markulf{30/09, how about this}{The proof systems we consider are universal. This means the reference string $\srs$ allows to prove arithmetic constraint defined over a particular field up to some size bound. Thus $\REL$ is universal and the public instance $\inp$ must determine the constraints. We ellide public preprocessing of constraint specific proving and verification keys. While important for performance, this modeling is not critical for security.}
% Consider a sigma protocol $\sigmaprot = (\prover, \verifier, \simulator)$ that
% is special-sound and has a unique response property. Let $\sigmaprot_\fs =
% (\prover_\fs, \verifier_\fs, \simulator_\fs)$ be a NIZK obtained by applying the
% Fiat--Shamir transform to $\sigmaprot$. Faust et al.~\cite{INDOCRYPT:FKMV12}
% show that every such $\sigmaprot_\fs$ is forking simulation-extractable. This result is
% presented in \cref{sec:forking_lemma} along with the instrumental forking lemma,
% cf.~\cite{CCS:BelNev06}.

\iffalse
\noindent \textbf{Simulation sound NIZKs.}
Another notion for non-malleable NIZKs is \emph{simulation soundness}. It allows the adversary to see simulated proof, however, in contrast to simulation
extractability it does not require an extractor to provide a witness for the
proven statement. Instead, it is only necessary, that an adversary who sees
simulated proofs cannot make the verifier accept a proof of an incorrect
statement. More precisely,
\chaya{this definition will go}
\begin{definition}[Simulation soundness]
  	\label{def:simsnd}
    Let $\ps = (\kgen, \prover, \verifier, \simulator)$ be a NIZK proof and
    $\ps_\fs = (\kgen_\fs, \prover_\fs, \verifier_\fs, \simulator_\fs)$ be $\ps$
    transformed by the Fiat--Shamir transform. We say that $\ps_\fs$ is
    \emph{simulation-sound}
    for any $\ppt$ adversary $\adv$ that is given oracle access to a random
    oracle $\ro$ and simulator $\simulator_\fs$, probability
    \[
      \ssndProb =
      \Pr\left[
        \begin{aligned}
          & \verifier_\fs(\srs, \inp_{\adv}, \zkproof_{\adv}) = 1,\\
          & (\inp_{\advse}, \zkproof_{\advse}) \not\in Q,\\
          & \neg \exists \wit_{\adv}: \REL(\inp_{\adv}, \wit_{\adv}) = 1
        \end{aligned}
        \, \left| \,
          \vphantom{\begin{aligned}
          & \verifier_\fs(\srs, \inp_{\adv}, \zkproof_{\adv}) = 1,\\
          & (\inp_{\advse}, \zkproof_{\advse}) \not\in Q,\\
          & \neg \exists \wit_{\adv}: \REL(\inp_{\adv}, \wit_{\adv}) = 1
        \end{aligned}}
      \begin{aligned}
        & \srs \gets \kgen(\REL), r \sample \RND{\advse},\\
        & (\inp_{\advse}, \zkproof_{\advse}) \gets \advse^{\simulator_\fs,
          \ro} (\srs; r)
      \end{aligned}
		\right.  \right]
    \]
    is at most negligible.  List $Q$ contains all $(\inp, \zkproof)$ pairs where
  $\inp$ is an instance provided to the simulator by the adversary and
  $\zkproof$ is the simulator's answer. 
\end{definition}

  \label{rem:simext_to_simsnd}
  We note that the probability $\ssndProb$ \cref{def:simsnd} can be expressed in
  terms of simulation-extractability. More precisely, the
  condition $\neg \exists \wit: \REL(\inp_\adv, \wit_\adv) = 1$ can be substituted with
  $\REL(\inp_\adv, \wit_\adv) = 0$, where $\wit_\adv$, returned by a possibly unbounded
  extractor, is either a witness to $\inp_\adv$ (if there exists any) or $\bot$ (if
  there is none). More precisely,
\[
      \ssndProb =
      \Pr\left[
        \begin{aligned}
          & \verifier_\fs(\srs, \inp_{\adv}, \zkproof_{\adv}) = 1,\\
          & (\inp_{\advse}, \zkproof_{\advse}) \not\in Q,\\
          & \REL(\inp_{\adv}, \wit_{\adv}) = 0
        \end{aligned}
        \, \left| \,
      \begin{aligned}
        & \srs \gets \kgen(\REL), r \sample \RND{\advse},\\
        & (\inp_{\advse}, \zkproof_{\advse}) \gets \advse^{\simulator_\fs,
          \ro} (\srs; r)\\
        & \wit_{\adv} \gets \ext(\srs, \advse, r, \inp_{\advse}, \zkproof_{\advse},
			Q, Q_\ro,) 
      \end{aligned}
		\right.  \right].
\]
The only necessary input to the unbounded extractor $\ext$ is the instance
$\inp_\adv$ (the rest is given for the consistency with the simulation extractability
definition). 
%
With the probabilities in \cref{def:simext} holding regardless of whether the extractor
is unbounded or not, we obtain the following equality
$ \ssndProb = \accProb - \extProb$.

% In \cref{cor:simext_to_ssnd} we show that (under some mild conditions) this is enough
% to conjecture that probability $\ssndProb$ is not only at most negligible, but
% also, in some parameters, exponentially smaller than $(1 - \extProb)$
% (probability of extraction failure in \cref{def:simext}).

\fi

%%% Local Variables:
%%% mode: latex
%%% TeX-master: "main"
%%% End:
%%% Local Variables:
%%% mode: latex
%%% TeX-master: "main"
%%% End:
