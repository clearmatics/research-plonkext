% !TEX root = main.tex
% !TEX spellcheck = en-US
\section{\COMMENT{Forking s}Simulation-extractability---the general result}
\label{sec:general}
Equipped with definitional framework of \cref{sec:se_definitions}, we now present the main result of this paper---a proof of
\COMMENT{forking }simulation extractability of Fiat-Shamir compiled multi-round protocols.

The proof proceeds by game hopping. The games are controlled by an environment $\env$
that internally runs a simulation extractability adversary $\advse$, provides it
with access to a random oracle and simulator, and when necessary, rewinds it. The
games differ by various breaking points, i.e.~points where the environment
decides to abort the game.

Denote by $\zkproof_{\advse}, \zkproof_{\simulator}$ proofs returned by the
adversary and the simulator respectively. We use $\zkproof[i]$ to denote
prover's message in the $i$-th round of the proof (counting from 1),
i.e.~$(2i - 1)$-th message exchanged in the protocol. $\zkproof[i].\ch$ denotes
the challenge that is given to the prover after $\zkproof[i]$, and
$\zkproof[i..j]$ to denote all messages of the proof including challenges
between rounds $i$ and $j$, but not challenge $\zkproof[j].\ch$. When it is not
explicitly stated, we denote the proven instance $\inp$ by $\zkproof[0]$
(however, there is no following challenge $\zkproof[0].\ch$).

Without loss of generality, we assume that whenever the accepting proof contains
a response to a challenge from a random oracle, then the adversary queried the
oracle to get it. It is straightforward to transform any adversary that violates
this condition into an adversary that makes these additional queries to the
random oracle and wins with the same probability.


\begin{theorem}[\COMMENT{Forking s}Simulation-extractable multi-message protocols]
  \label{thm:se}
  Let $\ps = (\kgen, \prover, \verifier, \simulator)$ be an interactive $(2 \mu + 1)$-message
  zero-knowledge proof system for $\RELGEN(\secparam)$, which is \emph{$k$-programmable trapdoor-less simulatable}, has
the $k$-\emph{unique response property} with security loss $\epsur(\secpar)$, and is $(k, n)$-\emph{forking special
  sound} with security loss $\epss(\secpar)$.  Let $\ro\colon \bin^{*} \to \bin^{\secpar}$ be a random oracle.  Then $\psfs$ is \COMMENT{
  forking }\emph{simulation-extractable} with extraction error $\epsur(\secpar)$ against $\ppt$
  adversaries that makes up to $q$ random oracle queries and returns an acceptable
  proof with probability at least $\accProb$.  The extraction probability $\extProb$ is at
  least
  \( \extProb \geq \frac{1}{q^{n - 1}} (\accProb - \epsur(\secpar))^{n} -\eps(\secpar)\,, \)
  for some negligible $\eps(\secpar)$.
\end{theorem}
\begin{proof}		

  \ngame{0} This is the simulation-extractability game played between an adversary
  $\advse$ who is given access to an oracle $\initU$ that defines an updatable SRS setup, a random oracle $\ro$ and a simulation oracle
  $\simO$. There is an extractor $\ext$ that, from a proof
  $\zkproof_{\advse}$ for instance $\inp_{\advse}$ output by the adversary and from
   transcripts of $\advse$'s operations is tasked to extract a witness
  $\wit_{\advse}$ such that $\REL(\inp_{\advse}, \wit_{\advse})$ holds. $\advse$ wins
  if it manages to produce an acceptable proof and the extractor fails to output a witness. In the following game hops we upper-bound the
  probability that this happens. Note that $\srs$ is with respect to the finalised SRS with respect to which $\adv$ is allowed to make simulation queries.

  \ngame{1} This is identical to $\game{0}$ except that now the game is aborted
  if there is $(\inp_\advse, \zkproof_{\simulator}) \in Q$ such that $\zkproof_\simulator[1..k] = \zkproof_{\advse}[1..k]$. That is, the adversary in its final proof
  reuses at least $k$ messages from a simulated proof, and the proof is accepting.
  Denote this event by $\event{\errur}$.

   \ncase{Game 0 to Game 1} We have, \( \prob{\game{0} \land
  	\nevent{\errur}} = \prob{\game{1} \land \nevent{\errur}} \) and, from the
  difference lemma, cf.~\cref{lem:difference_lemma},
  $ \abs{\prob{\game{0}} - \prob{\game{1}}} \leq \prob{\event{\errur}}\,$.
  Thus, to show that the transition from one game to another introduces only
  minor change in probability of $\adv$ winning it should be shown that
  $\prob{\event{\errur}}$ is small.
  
  We can assume that $\adv$ queried the simulator on the instance it wishes to
  output, i.e.~$\inp_{\adv}$. We show a reduction $\rdvur$ that utilises $\adv$
  to break the $\ur{k}$ property of $\ps$. Let $\rdvur$ run $\advse$ internally
  as a black-box:
  \begin{compactitem}
  	\item The reduction answers $\adv$ update queries by asking the same query from the update oracle in the unique response experiment. The reduction finalises the same SRS $\srs$ as the one $\adv$ does.
  	\item The reduction answers both queries to the simulator $\psfs.\simulator$
  	and to the random oracle.  It also keeps lists $Q$, for the simulated
  	proofs, and $Q_\ro$ for the random oracle queries.
  	\item When $\adv$ makes a fake proof $\zkproof_{\adv}$ for $\inp_{\adv}$,
  	$\rdvur$ looks through lists $Q$ and $Q_\ro$ until it finds
  	$\zkproof_{\simulator}[0..k]$ such that
  	$\zkproof_{\adv}[0..k] = \zkproof_{\simulator}[0..k]$ and a random oracle
  	query $\zkproof_{\simulator}[k].\ch$ on $\zkproof_{\simulator}[0..k]$.
  	\item $\rdvur$ returns two proofs for $\inp_{\adv}$:
  	\begin{align*}
  	\zkproof_1 = (\zkproof_{\simulator}[1..k],
  	\zkproof_{\simulator}[k].\ch, \zkproof_{\simulator}[k + 1..\mu + 1])\\
  	\zkproof_2 = (\zkproof_{\simulator}[1..k],
  	\zkproof_{\simulator}[k].\ch, \zkproof_{\adv}[k + 1..\mu + 1])
  	\end{align*}
  \end{compactitem}  
  If $\zkproof_1 = \zkproof_2$, then $\adv$ fails to break simulation soundness,
  as $\zkproof_2 \in Q$. On the other hand, if the proofs are not equal, then
  $\rdvur$ breaks $\ur{k}$-ness of $\ps$. This happens only with negligible
  probability $\epsur(\secpar)$, hence
  \( \prob{\event{\errur}} \leq \epsur(\secpar)\,. \)

  \COMMENT{We have,
    \( \prob{\game{0} \land \nevent{\errur}} = \prob{\game{1} \land
      \nevent{\errur}} \) and, from the difference lemma,
    cf.~\cref{lem:difference_lemma},
  \[ \abs{\prob{\game{0}} - \prob{\game{1}}} \leq \prob{\event{\errur}}\,. \]
  Thus, to show that the transition from one game to another introduces only
  minor change in probability of $\advse$ winning it should be shown that
  $\prob{\event{\errur}}$ is small.

  We can assume that $\advse$ queried the simulator on the instance it wishes to
  output---$\inp_{\advse}$. We show a reduction $\rdvur$ that utilises $\advse$,
  who outputs a valid proof for $\inp_{\advse}$, to break the $\ur{k}$ property of
  $\ps$. Let $\rdvur$ run $\advse$ internally as a black-box:
\begin{itemize}
	\item The reduction answers both queries to the simulator $\psfs.\simulator$ and to the random oracle. 
	It also keeps lists $Q$, for the simulated proofs, and $Q_\ro$ for the random oracle queries. 
\item When $\advse$ makes a fake proof $\zkproof_{\advse}$ for $\inp_{\advse}$,
  $\rdvur$ looks through lists $Q$ and $Q_\ro$ until it finds
  $\zkproof_{\simulator}[0..k]$ such that
  $\zkproof_{\advse}[0..k] = \zkproof_{\simulator}[0..k]$
  and a random oracle query $\zkproof_{\simulator}[k].\ch$ on
  $\zkproof_{\simulator}[0..k]$.
	\item $\rdvur$ returns two proofs for $\inp_{\advse}$:
	\begin{align*}
		\zkproof_1 = (\zkproof_{\simulator}[1..k],
		\zkproof_{\simulator}[k].\ch, \zkproof_{\simulator}[k + 1..\mu + 1])\\
		\zkproof_2 = (\zkproof_{\simulator}[1..k],
		\zkproof_{\simulator}[k].\ch, \zkproof_{\advse}[k + 1..\mu + 1])
	\end{align*}
	\end{itemize}  
	If $\zkproof_1 = \zkproof_2$, then $\advse$ fails to break simulation
  extractability, as $\zkproof_2 \in Q$. On the other hand, if the proofs are
  not equal, then $\rdvur$ breaks $\ur{k}$-ness of $\ps$. This happens only with
  negligible probability $\epsur(\secpar)$, hence \( \prob{\event{\errur}} \leq
  \epsur(\secpar)\,. \)
}
%
		\ngame{2} Define an adversary $\bdv$ against forking special soundness such that, given access to oracles $\initU$ and $\ro$, and randomness $r_\bdv$, it internally runs $\advse^{\initU, \simO, \ro} (1^\secpar; r_\adv)$, where
		\begin{compactenum}
			\item $r_\bdv$ is split into two substrings $r_\adv$ and $r_\simulator$;
			\item $\bdv$ answers $\adv$ update queries by asking the same query from its own update oracle. When $\adv$ finalises an SRS $\srs$, $\bdv$ does the same.
			\item $\bdv$ answers $\adv$'s simulator queries by programming the random oracle locally by using coins from $r_\simulator$.
			$\bdv$ maintains a list of instance-proof pairs $Q$ consisting of of all simulation queries made by $\adv$, and corresponding responses.
			\item Eventually when $\adv$ outputs $(\inp_\advse, \zkproof_\advse)$, $\bdv$ outputs the same; $(\inp_\advse, \zkproof_\advse)$.

		\end{compactenum}
		
		In this game, the environment aborts also when it fails to build a
		$(1, \ldots, 1, n, 1, \ldots, 1)$-tree of accepting transcripts $\tree$ by rewinding
		$\bdv$. Denote that event by $\event{\errfrk}$.
	
	\ncase{Game 1 to Game 2}
		Note that for every accepting proof $\zkproof_{\advse}$, we may
		assume that whenever $\advse$ outputs a round $k$ message $\zkproof_{\advse}[k]$, then the
		$(\inp_{\advse}, \zkproof_{\advse}[1..k])$ random oracle query was made by the adversary, not
		the simulator\footnote{\cite{INDOCRYPT:FKMV12} calls these queries \emph{fresh}.}, i.e.~there
		is no simulated proof $\zkproof_\simulator$ on $\inp_\simulator$ such that
		$(\inp_{\advse}, \zkproof_{\advse} [1..k]) = (\inp_\simulator,
		\zkproof_\simulator[1..k])$. Otherwise, the game would be already interrupted by the error
		event in Game $\game{1}$.  As previously,
		\( \abs{\prob{\game{1}} - \prob{\game{2}}} \leq \prob{\event{\errfrk}}\,.  \)
		
		We describe our extractor $\ext$ here. The extractor takes as input relation $\REL$, SRS
		$\srs$, $\bdv$'s code, its randomness $r_\bdv$, the output instance $\inp_{\advse}$ and proof
		$\zkproof_{\advse}$, and the list of random oracle queries and responses $Q_\ro$. Then, $\ext$
		starts a forking algorithm $\genforking^{n}_\zdv(y,h_1, \ldots, h_q)$ for
		$y = (\srs, \bdv, r_\bdv, \inp_{\advse}, \zkproof_{\advse})$ where we set $h_1, \ldots, h_q$ to
		be the consecutive queries from list $Q_\ro$. We run $\bdv$ internally in $\zdv$.
		
		To assure that in the first execution of $\zdv$ the adversary $\bdv$ produces the same
		$(\inp_{\advse}, \zkproof_{\advse})$ as in the extraction game, $\zdv$ provides $\bdv$ with
		the same randomness $r_\bdv$ and answers queries to the random oracle with
		pre-recorded responses in $Q_\ro$.
		%
		Note, that since the view of the adversary when run inside $\zdv$ is the same as its view with
		access to the real random oracle, it produces exactly the same output. After the
		first run, $\zdv$ outputs the index $i$ of a random oracle query that was used by $\bdv$ to
		compute the challenge $\zkproof[k].\ch = \ro(\zkproof_{\advse}[0..k])$ it had to answer in the
		$(k + 1)$-th round and adversary's transcript, denoted by $s_1$ in $\genforking$'s
		description. If no such query took place $\zdv$ outputs $i = 0$.
		
		Then, new random oracle responses are picked for queries indexed by
		$i, \ldots, q$ and the adversary is rewound to the point just prior to when it gets the
		response to RO query $\zkproof_{\advse}[0..k]$. The adversary gets a random
		oracle response from a new set of responses $h^2_i, \ldots, h^2_q$. If the
		adversary requests a simulated proof after seeing $h^2_i$, then $\zdv$ computes
		the simulated proof on its own. Eventually, $\zdv$ outputs index $i'$ of a query
		that was used by the adversary to compute $\ro(\zkproof_{\advse}[0..k])$, and a
		new transcript $s_2$. $\zdv$ is run $n$ times with different random oracle
		responses. If a tree $\tree$ of $n$ transcripts is built, then $\ext$
		internally runs the tree extractor $\extt(\tree)$ and outputs what it returns.
		
		We emphasize here the importance of the unique response property. If it does not hold then in
		some $j$-th execution of $\zdv$ the adversary $\adv$ (run internally in $\bdv$) could reuse a
		challenge that it learned from observing proofs in $Q$. In that case, $\bdv$ would output a
		proof that would make $\zdv$ output $i = 0$, making the extractor fail. Fortunately, the case
		that the adversary breaks the unique response property has already been covered by the abort
		condition in $\game{1}$.
		
		Denote by $\waccProb$ the probability that $\advse$ outputs a proof that is accepted and does
		not break $\ur{k}$-ness of $\ps$. With the same probability, an accepting proof is returned by
		$\bdv$. Denote by $\waccProb'$ the probability that algorithm $\zdv$, defined in the general forking lemma,
		produces an accepting proof with a fresh challenge after round $k$. From the above argument,
		we have that $\waccProb = \waccProb'$.
		
		Next, from the generalised forking lemma, cf.~\cref{lem:generalised_forking_lemma}, we get that
		\begin{equation}
		\begin{split}
		\prob{\event{\errfrk}} \leq 1 - \waccProb \cdot \left(\infrac{\waccProb^{n -
				1}}{q^{n - 1}} + \infrac{(2^\secpar) !}{((2^\secpar - n)! \cdot
			(2^\secpar)^{n})} - 1\right).
		\end{split}
		\end{equation}
	
	\ngame{3}
		This game is identical to $\game{2}$ except that it aborts if
		$\extt(\tree)$ run by $\ext$ fails to extract a witness. 
	
	\ncase{Game 2 to Game 3}	
		Since $\ps$ is forking-sound the probability that $\extt(\tree)$
		fails is upper-bounded by $\epsss(\secpar)$.
		
		Since Game $\game{3}$ is aborted when it is impossible to extract a witness from
		$\tree$ and $\bdv$ only passes proofs produced by $\adv$, the adversary $\advse$ cannot
		win. Thus, by the game-hopping argument,
		\[
		\abs{\prob{\game{0}} - \prob{\game{4}}} \leq 1 -
		\left(\frac{\waccProb^{n}}{q^{n - 1}} + \waccProb \cdot \frac{(2^\secpar)
			!}{(2^\secpar - n)! \cdot (2^\secpar)^{n}} - \waccProb\right) + \epsur(\secpar) +
		%q_{\ro}^{\mu} \epss +
		\epsss(\secpar)\,.
		\]
		Thus the probability that extractor $\extss$ succeeds is at least
		\[
		\frac{\waccProb^{n}}{q^{n - 1}} + \waccProb \cdot \frac{(2^\secpar)
			!}{(2^\secpar - n)! \cdot (2^\secpar)^{n}} - \waccProb - \epsur(\secpar) 
		%- q_{\ro}^{\mu} \epss
		- \epsss(\secpar)\,.
		\]
		Since $\waccProb$ is the probability of $\advse$ producing an accepting transcript
		that does not break $\ur{k}$-ness of $\ps$, then $\waccProb \geq \accProb -
		\epsur(\secpar)$, where $\accProb$ is the probability of $\advse$ outputting an accepting
		proof as defined in \cref{def:updsimext}. Thus, 
		\begin{equation}
		\label{eq:frk}
		\extProb \geq \frac{(\accProb - \epsur(\secpar))^{n}}{q^{n - 1}} -
		\underbrace{(\accProb - \epsur(\secpar)) \cdot \left( 1 - \frac{(2^\secpar)
				!}{(2^\secpar - n)! \cdot (2^\secpar)^{n}}\right) - \epsur(\secpar) -
			% q_{\ro}^{\mu} \epss -
			\epsss(\secpar)}_{\eps(\secpar)}\,.
		\end{equation}
		Note that the part of \cref{eq:frk} denoted by $\eps(\secpar)$ is negligible as
		$\epsur(\secpar), \epsss(\secpar)$ are negligible, and
		$\frac{(2^\secpar) !}{(2^\secpar - n)! \cdot (2^\secpar)^{n}} \geq
		\left(\infrac{(2^\secpar - n)}{2^\secpar}\right)^{n}$ is overwhelming.  Therefore,
		\[
		\extProb \geq q^{-(n - 1)} (\accProb - \epsur(\secpar))^{n} -\eps(\secpar)\,.
		\] 
		and $\psfs$ is \COMMENT{forking }simulation extractable with extraction error $\epsur(\secpar)$.
	\qed
\end{proof}


We conjecture that based on the recent results on state restoration soundness~\cite{C:GhoTes21}, which effectively allows to query the verifier multiple times on different overlapping transcripts, the $q^{\mu}$ loss could be avoided. However, this would reduce the class of protocols covered by our results. 

%%% Local Variables:
%%% mode: latex
%%% TeX-master: "main"
%%% End:
