% !TEX root = main.tex
% !TEX spellcheck = en-US

\section{Simulation Soundness }
\noindent \textbf{Simulation sound NIZKs.}
Another notion for non-malleable NIZKs is \emph{simulation soundness}. It allows the adversary to see simulated proof, however, in contrast to simulation
extractability it does not require an extractor to provide a witness for the
proven statement. Instead, it is only necessary, that an adversary who sees
simulated proofs cannot make the verifier accept a proof of an incorrect
statement. More precisely,
\chaya{this definition will go}
\begin{definition}[Simulation soundness]
	\label{def:simsnd}
	Let $\ps = (\kgen, \prover, \verifier, \simulator)$ be a NIZK proof and
	$\ps_\fs = (\kgen_\fs, \prover_\fs, \verifier_\fs, \simulator_\fs)$ be $\ps$
	transformed by the Fiat--Shamir transform. We say that $\ps_\fs$ is
	\emph{simulation-sound}
	for any $\ppt$ adversary $\adv$ that is given oracle access to a random
	oracle $\ro$ and simulator $\simulator_\fs$, probability
	\[
	\ssndProb =
	\Pr\left[
	\begin{aligned}
	& \verifier_\fs(\srs, \inp_{\adv}, \zkproof_{\adv}) = 1,\\
	& (\inp_{\advse}, \zkproof_{\advse}) \not\in Q,\\
	& \neg \exists \wit_{\adv}: \REL(\inp_{\adv}, \wit_{\adv}) = 1
	\end{aligned}
	\, \left| \,
	\vphantom{\begin{aligned}
		& \verifier_\fs(\srs, \inp_{\adv}, \zkproof_{\adv}) = 1,\\
		& (\inp_{\advse}, \zkproof_{\advse}) \not\in Q,\\
		& \neg \exists \wit_{\adv}: \REL(\inp_{\adv}, \wit_{\adv}) = 1
		\end{aligned}}
	\begin{aligned}
	& \srs \gets \kgen(\REL), r \sample \RND{\advse},\\
	& (\inp_{\advse}, \zkproof_{\advse}) \gets \advse^{\simulator_\fs,
		\ro} (\srs; r)
	\end{aligned}
	\right.  \right]
	\]
	is at most negligible.  List $Q$ contains all $(\inp, \zkproof)$ pairs where
	$\inp$ is an instance provided to the simulator by the adversary and
	$\zkproof$ is the simulator's answer. 
\end{definition}

\label{rem:simext_to_simsnd}
We note that the probability $\ssndProb$ \cref{def:simsnd} can be expressed in
terms of simulation-extractability. More precisely, the
condition $\neg \exists \wit: \REL(\inp_\adv, \wit_\adv) = 1$ can be substituted with
$\REL(\inp_\adv, \wit_\adv) = 0$, where $\wit_\adv$, returned by a possibly unbounded
extractor, is either a witness to $\inp_\adv$ (if there exists any) or $\bot$ (if
there is none). More precisely,
\[
\ssndProb =
\Pr\left[
\begin{aligned}
& \verifier_\fs(\srs, \inp_{\adv}, \zkproof_{\adv}) = 1,\\
& (\inp_{\advse}, \zkproof_{\advse}) \not\in Q,\\
& \REL(\inp_{\adv}, \wit_{\adv}) = 0
\end{aligned}
\, \left| \,
\begin{aligned}
& \srs \gets \kgen(\REL), r \sample \RND{\advse},\\
& (\inp_{\advse}, \zkproof_{\advse}) \gets \advse^{\simulator_\fs,
	\ro} (\srs; r)\\
& \wit_{\adv} \gets \ext(\srs, \advse, r, \inp_{\advse}, \zkproof_{\advse},
Q, Q_\ro,) 
\end{aligned}
\right.  \right].
\]
The only necessary input to the unbounded extractor $\ext$ is the instance
$\inp_\adv$ (the rest is given for the consistency with the simulation extractability
definition). 
%
With the probabilities in \cref{def:simext} holding regardless of whether the extractor
is unbounded or not, we obtain the following equality
$ \ssndProb = \accProb - \extProb$.


\section{Simulation soundness---the general result}
\label{sec:general}
Equipped with definitional framework of \cref{sec:se_definitions} we are ready
to present the main result of this paper---a proof of simulation soundness and
forking simulation extractability of Fiat-Shamir NIZK based on multi-round protocols.

The proofs go by game hopping. The games are controlled by an environment $\env$
that internally runs a simulation extractability adversary $\advse$, provides it
with access to a random oracle and simulator, and when necessary, rewinds it. The
games differ by various breaking points, i.e.~points where the environment
decides to abort the game.

Denote by $\zkproof_{\advse}, \zkproof_{\simulator}$ proofs returned by the
adversary and the simulator respectively. We use $\zkproof[i]$ to denote
prover's message in the $i$-th round of the proof (counting from 1),
i.e.~$(2i - 1)$-th message exchanged in the protocol. $\zkproof[i].\ch$ denotes
the challenge that is given to the prover after $\zkproof[i]$, and
$\zkproof[i..j]$ to denote all messages of the proof including challenges
between rounds $i$ and $j$, but not challenge $\zkproof[j].\ch$. When it is not
explicitly stated, we denote the proven instance $\inp$ by $\zkproof[0]$
(however, there is no following challenge $\zkproof[0].\ch$).

Without loss of generality, we assume that whenever the accepting proof contains
a response to a challenge from a random oracle, then the adversary queried the
oracle to get it. It is straightforward to transform any adversary that violates
this condition into an adversary that makes these additional queries to the
random oracle and wins with the same probability.

\begin{theorem}[Simulation soundness]
	\label{thm:simsnd}
	Assume that $\ps$ is $k$-programmable HVZK in the standard model, that is
	$\epss(\secpar)$-sound and $\ur{k}$ with security $\epsur(\secpar)$. Then, the
	probability that a $\ppt$ adversary $\adv$ breaks simulation soundness of
	$\ps_{\fs}$ is upper-bounded by
	\(
	\epsur(\secpar) + q_\ro^\mu  \epss(\secpar)\,,
	\)
	where $q$ is the total number of queries made by the adversary $\adv$ to a
	random oracle $\ro\colon \bin^{*} \to \bin^{\secpar}$.
\end{theorem}

\begin{proof}
	\ngame{0} This is a simulation soundness game played between an adversary
	$\adv$ who is given access to a random oracle $\ro$ and simulator
	$\psfs.\simulator$. $\adv$ wins if it manages to produce an accepting proof
	for a false statement. In the following game hops, we upper-bound the
	probability that this happens.
	
	\ngame{1} This is identical to $\game{0}$ except that the game is aborted if
	there is a simulated proof $\zkproof_\simulator$ for $\inp_{\adv}$ such that
	$(\inp_{\adv}, \zkproof_\simulator[1..k]) = (\inp_{\adv},
	\zkproof_{\adv}[1..k])$. That is, the adversary in its final proof reuses at
	least $k$ messages from a simulated proof it saw before and the proof is
	accepting.  Denote this event by $\event{\errur}$.
	
	\ncase{Game 0 to Game 1} We have, \( \prob{\game{0} \land
		\nevent{\errur}} = \prob{\game{1} \land \nevent{\errur}} \) and, from the
	difference lemma, cf.~\cref{lem:difference_lemma},
	$ \abs{\prob{\game{0}} - \prob{\game{1}}} \leq \prob{\event{\errur}}\,$.
	Thus, to show that the transition from one game to another introduces only
	minor change in probability of $\adv$ winning it should be shown that
	$\prob{\event{\errur}}$ is small.
	
	We can assume that $\adv$ queried the simulator on the instance it wishes to
	output, i.e.~$\inp_{\adv}$. We show a reduction $\rdvur$ that utilises $\adv$
	to break the $\ur{k}$ property of $\ps$. Let $\rdvur$ run $\advse$ internally
	as a black-box:
	\begin{compactitem}
		\item The reduction answers both queries to the simulator $\psfs.\simulator$
		and to the random oracle.  It also keeps lists $Q$, for the simulated
		proofs, and $Q_\ro$ for the random oracle queries.
		\item When $\adv$ makes a fake proof $\zkproof_{\adv}$ for $\inp_{\adv}$,
		$\rdvur$ looks through lists $Q$ and $Q_\ro$ until it finds
		$\zkproof_{\simulator}[0..k]$ such that
		$\zkproof_{\adv}[0..k] = \zkproof_{\simulator}[0..k]$ and a random oracle
		query $\zkproof_{\simulator}[k].\ch$ on $\zkproof_{\simulator}[0..k]$.
		\item $\rdvur$ returns two proofs for $\inp_{\adv}$:
		\begin{align*}
		\zkproof_1 = (\zkproof_{\simulator}[1..k],
		\zkproof_{\simulator}[k].\ch, \zkproof_{\simulator}[k + 1..\mu + 1])\\
		\zkproof_2 = (\zkproof_{\simulator}[1..k],
		\zkproof_{\simulator}[k].\ch, \zkproof_{\adv}[k + 1..\mu + 1])
		\end{align*}
	\end{compactitem}  
	If $\zkproof_1 = \zkproof_2$, then $\adv$ fails to break simulation soundness,
	as $\zkproof_2 \in Q$. On the other hand, if the proofs are not equal, then
	$\rdvur$ breaks $\ur{k}$-ness of $\ps$. This happens only with negligible
	probability $\epsur(\secpar)$, hence
	\( \prob{\event{\errur}} \leq \epsur(\secpar)\,. \)
	
	\ngame{2} This is identical to $\game{1}$ except that now the environment
	aborts if the instance the adversary proves is not in the language.
	
	\ncase{Game 1 to Game 2} 
	% REDUCTION TO INTERACTIVE SOUNDNESS:
	We show that
	$\abs{\prob{\game{1}} - \prob{\game{2}}} \leq q^{\mu} \cdot \epss(\secpar)$,
	where $\epss(\secpar)$ is the probability of breaking soundness of the underlying
	\emph{interactive} protocol $\ps$. Note that
	$\abs{\prob{\game{1}} - \prob{\game{2}}}$ is the probability that $\adv$
	outputs an acceptable proof for a false statement which does not break the
	unique response property (such proofs have been excluded by
	$\game{1}$). Consider a soundness adversary $\adv'$ who initiates a proof with
	$\ps$'s verifier $\ps.\verifier$, internally runs $\adv$ and proceeds as
	follows:
	\begin{compactitem}
		\item It guesses indices $i_1, \ldots, i_\mu$ such that random oracle queries
		$h_{i_1}, \ldots, h_{i_\mu}$ are the queries used in the $\zkproof_\adv$
		proof eventually output by $\adv$. This is done with probability at least
		$1/q^\mu$ (since there are $\mu$ challenges from the verifier in
		$\ps$).
		\item On input $h$ for the $i$-th,
		$i \not\in \smallset{{i_1}, \ldots, {i_\mu}}$, random oracle query, $\adv'$
		returns randomly picked $y$, sets $\ro(h) = y $ and stores $(h, y)$ in
		$Q_\ro$ if $h$ is sent to $\ro$ the first time. If that is not the case,
		$\adv$ finds $h$ in $Q_\ro$ and returns the corresponding $y$.
		\item On input $h_{i_j}$ for the $i_j$-th,
		$i_j \in \smallset{{i_1}, \ldots, {i_\mu}}$, random oracle query, $\adv'$
		parses $h_{i_j}$ as a partial proof transcript $\zkproof_\adv[1..j]$ and
		runs $\ps$ using $\zkproof_\adv[j]$ as a $\ps.\prover$'s $j$-th message to
		$\ps.\verifier$. The verifier responds with a challenge
		$\zkproof_\adv[j].\ch$. $\adv'$ sets $\ro(h_{i_j}) =
		\zkproof_\adv[j].\ch$. If we guessed the indices correctly we have that
		$h_{i_{j'}}$, for $j' \leq j$, parsed as $\zkproof_\adv[1..j']$ is a prefix
		of $\zkproof_\adv[1..j]$.
		\item On query $\inp_\simulator$ to $\simulator$, $\adv'$ runs the simulator
		$\ps.\simulator$ internally. Note that we require a simulator that only
		programs the random oracle for $j \geq k$.  If the simulator makes a
		previously unanswered random oracle query with input
		$\zkproof_\simulator[1..j]$, $1 \leq j < k$, and this is the $i_j$-th query,
		it generates $\zkproof_\simulator[j].\ch$ by invoking $\ps.\verifier$ on
		$\zkproof_\simulator[j]$ and programs
		$\ro(h_{i_j}) = \zkproof_\simulator[j].\ch$.  It returns
		$\zkproof_\simulator$.
		\item Answers $\ps.\verifier$'s final challenge $\zkproof_\adv[\mu].\ch$ using the
		answer given by $\adv$, i.e.~$\zkproof_\adv[\mu]$.
	\end{compactitem}
	That is, $\adv'$ manages to break soundness of $\ps$ if $\adv$ manages to
	break simulation soundness without breaking the unique response property and
	$\adv'$ correctly guesses the indices of $\adv$ random oracle queries. This
	happens with probability upper-bounded by $\abs{\prob{\game{1}} -
		\prob{\game{2}}} \cdot \infrac{1}{q^{\mu}}$. Hence $\abs{\prob{\game{1}} -
		\prob{\game{2}}} \leq q^{\mu} \cdot \epss(\secpar)$.
	
	Note that in $\game{2}$ the adversary cannot win. Thus the probability
	that $\advss$ is successful is upper-bounded by
	$\epsur(\secpar) + q^{\mu} \cdot \epss(\secpar)$.  \qed
\end{proof}


We conjecture that based on the recent results on state restoration soundness~\cite{cryptoeprint:2020:1351}, which effectively allows to query the verifier multiple times on different overlapping transcripts, the $q^{\mu}$ loss could be avoided. However, this would reduce the class of protocols covered by our results. 


\subsection{Simulation soundness and simulation
	extractability of~$\plonkprotfs$}
Since \cref{lem:plonkprot_ur,lem:plonkprot_ss} hold, $\plonkprot$ is $\ur{2}$
and forking sound. We now make use of \cref{thm:simsnd} and \cref{thm:se} and show that
$\plonkprot_\fs$ is simulation sound and forking simulation-extractable as defined in
\cref{sec:simext_def}.

\begin{corollary}[Forking simulation extractability of $\plonkprot_\fs$]
	\label{thm:plonkprotfs_se}
	Assume an idealised $\plonkprot$ verifier fails at most with probability
	$\epsid(\secpar)$, the discrete logarithm advantage is bounded by
	$\epsdlog(\secpar)$ and the $\PCOMp$ is a commitment of knowledge with security
	$\epsk(\secpar)$, binding security $\epsbind(\secpar)$ and has unique opening
	property with security $\epsop(\secpar)$. Let
	$\ro\colon \bin^* \to \bin^\secpar$ be a random oracle. Let $\advse$ be an
	algebraic adversary that can make up to $q$ random oracle queries, up to $S$
	simulation oracle queries, and outputs an acceptable proof for $\plonkprotfs$
	with probability at least $\accProb$. Then $\plonkprotfs$ is forking
	simulation-extractable with extraction error $\eta = \epsur(\secpar)$. The
	extraction probability $\extProb$ is at least
	\[
	\extProb \geq \frac{1}{q^{3 (\epsid(\secpar)+\epsdlog(\secpar))}} (\accProb - \epsk(\secpar) - 2\cdot\epsbind(\secpar) -
	\epsop(\secpar))^{3\noofc + 1} -\eps(\secpar)\,,
	\]
	for some negligible $\eps(\secpar)$ and $\noofc$ being the number of
	constrains in the proven circuit.
\end{corollary}

 \begin{corollary}[Simulation soundness of $\plonkprot_\fs$]
   \label{thm:simsnd}
   Assume that $\plonkprot$ is $2$-programmable HVZK in the standard model, that
   is $\epss(\secpar)$-sound and the $\PCOMp$ is a commitment of knowledge with
   security $\epsk(\secpar)$, binding security $\epsbind(\secpar)$ and has unique
   opening property with security $\epsop(\secpar)$. Then the probability that a
   $\ppt$ adversary $\adv$ breaks simulation soundness of $\ps_{\fs}$ is
   upper-bounded by
   \( \epsk(\secpar) + 2\cdot\epsbind(\secpar) + \epsop(\secpar) + q_\ro^4
   \epss(\secpar)\,, \) where $q$ is the total number of queries made by the
   adversary $\adv$ to a random oracle $\ro\colon \bin^{*} \to \bin^{\secpar}$.
 \end{corollary}